\upaper{154}{Последние дни в Капернауме}
\author{Комиссия срединников}
\vs p154 0:1 В богатую событиями субботнюю ночь 30 апреля, когда Иисус говорил слова утешения и ободрения своим удрученным и сбитым с толку ученикам, в Тивериаде проходил совет между Иродом Антипой и группой особых уполномоченных, представлявших иерусалимский синедрион. Эти книжники и фарисеи призывали Ирода арестовать Иисуса и сделали все возможное, чтобы убедить его в том, что Иисус подстрекал народ к смуте и даже к восстанию. Однако Ирод отказался предпринимать какие\hyp{}либо действия против Иисуса как политического преступника. Советники Ирода верно описали случившееся на противоположном берегу озера, и то, как народ пытался провозгласить Иисуса царем, и то, как он отверг это предложение.
\vs p154 0:2 Один из членов официальной семьи Ирода Хуза, чья жена состояла в отряде женщин\hyp{}служителей, сообщил ему, что Иисус не намеревался вмешиваться в дела земной власти; что его заботило лишь установление духовного братства его верующих, каковое братство он называл царством небесным. Ирод доверял сообщениям Хузы настолько, что отказался вмешиваться в дела Иисуса. В это время на отношение Ирода к Иисусу влиял также его суеверный страх перед Иоанном Крестителем. Ирод был одним из тех евреев\hyp{}отступников, которые, ни во что не веря, боялись всего. На его совести было убийство Иоанна, и он не хотел впутываться в интриги против Иисуса. По\hyp{}видимому, зная о многочисленных больных, исцеленных Иисусом, он считал его либо пророком, либо относительно безобидным религиозным фанатиком.
\vs p154 0:3 Когда евреи пригрозили Ироду донести кесарю о том, что он покрывает смутьяна, Ирод приказал им покинуть палаты, где происходил совет. Сложившаяся ситуация сохранялась целую неделю; в течение этого времени Иисус готовил своих последователей к предстоящему рассеянию.
\usection{1. Неделя совета}
\vs p154 1:1 С 1 мая по 7 мая Иисус в доме Зеведея обсуждал со своими последователями самые сокровенные вопросы. К этим беседам были допущены только испытанные и надежные ученики. В это время было лишь около ста учеников, обладавших внутренней смелостью для того, чтобы противостоять фарисеям, и готовых открыто заявить о своей приверженности Иисусу. С ними он совещался утром, днем и вечером. Каждый день после полудня у берега моря собирались небольшие группы интересующихся, где с ними беседовали или кто\hyp{}нибудь из евангелистов или апостолы. Эти группы редко насчитывали более пятидесяти человек.
\vs p154 1:2 В пятницу этой недели управителями капернаумской синагоги была предпринята официальная акция, закрывшая дом Божий для Иисуса и всех его последователей. Эта акция была предпринята по наущению иерусалимских фарисеев. Иаир сложил с себя обязанности старшего управителя и открыто примкнул к Иисусу.
\vs p154 1:3 Последняя встреча у берега моря произошла в субботу 7 мая после полудня. На этот раз Иисус говорил менее чем со ста пятьюдесятью собравшимися. В эту субботнюю ночь популярность Иисуса и его учений достигла низшей точки. Однако впоследствии, начиная с этого момента, возникло постепенное, медленное, но более здравое и надежное усиление сочувственных настроений; возникло новое движение последователей, которое в значительной степени основывалось на духовной вере и истинном религиозном опыте. Относительно сложный и компромиссный этап перехода от материалистических представлений о царстве, которых придерживались последователи Учителя, к более идеалистическим и духовным представлениям, которым учил Иисус, теперь практически закончился. Отныне евангелие царства провозглашалось более открыто, в гораздо большей полноте и с широчайшим охватом его духовного смысла.
\usection{2. Неделя отдыха}
\vs p154 2:1 В воскресенье 8 мая 29 года в Иерусалиме синедрион издал указ, который закрыл для Иисуса и его последователей все синагоги Палестины. Это была новая и беспрецедентная узурпация власти иерусалимским синедрионом. До сих пор все синагоги существовали и действовали как независимые собрания почитателей Бога и находились во власти и под управлением своих собственных советов управляющих. Власти синедриона подчинялись только синагоги Иерусалима. За этой резкой акцией синедриона последовал уход в отставку пятерых его членов. Сто вестников были немедленно отправлены объявить и привести в действие этот указ. В течение двух недель все синагоги в Палестине, кроме синагоги в Хевроне, подчинились этому манифесту синедриона. Управители хевронской синагоги отказались признать право синедриона вторгаться в сферу полномочий их собрания. Этот отказ подчиниться указу синедриона проистекал из их стремления к автономии их общины, а вовсе не из сочувствия делу Иисуса. Вскоре после этого хевронская синагога была уничтожена пожаром.
\vs p154 2:2 \pc В то же самое воскресное утро Иисус объявил недельный отдых, советуя ученикам вернуться к своим домам или друзьям, чтобы успокоить свои встревоженные души и ободрить своих близких. Он сказал: «Ступайте к домам вашим отдыхать или ловить рыбу и молитесь о расширении царства».
\vs p154 2:3 Эта неделя отдыха позволила Иисусу встретиться со многими семьями и группами, жившими на побережье. Он также несколько раз рыбачил с Давидом Зеведеевым, и хотя большую часть времени провел в одиночестве, его всегда скрытно сопровождали два или три самых надежных вестника Давида, получивших от своего начальника недвусмысленные приказания относительно охраны Иисуса. Всю эту неделю отдыха не было никаких публичных проповедей.
\vs p154 2:4 \pc В эту неделю Нафанаил и Иаков Зеведеев страдали от тяжкой болезни. Три дня и три ночи они сильно мучились от болезненного расстройства желудка. В третью ночь Иисус отправил мать Иакова Соломею отдыхать, а сам заботился о своих страдающих апостолах. Разумеется, Иисус мог моментально исцелить этих людей, но не таким образом Отец или Сын разрешают подобные обыденные трудности и лечат такие недуги детей человеческих в эволюционирующих мирах времени и пространства. Ни разу за всю свою богатую событиями жизнь во плоти Иисус не прибегал к какому\hyp{}либо сверхъестественному способу служения любому члену своей семьи либо на благо любого из своих ближайших последователей.
\vs p154 2:5 Одоление вселенских проблем и столкновение с планетарными препятствиями должны быть составной частью воспитания посредством опыта, которое служит росту и развитию, последовательному совершенствованию развивающихся душ смертных созданий. Одухотворение человеческой души зависит от развития личного опыта разрешения широкого спектра реальных вселенских проблем. Животная природа человека и низшие формы волевых созданий в тепличных условиях не развиваются в нужном направлении. Сложные ситуации, в сочетании с побуждающими к действию стимулами, вместе вызывают такую деятельность ума, души и духа, которая чрезвычайно способствует достижению достойных целей совершенствования смертных и обретению ими более высоких уровней духовного предназначения.
\usection{3. Второе совещание в Тивериаде}
\vs p154 3:1 6 мая в Тивериаде было созвано второе совещание между властями Иерусалима и Иродом Антипой. На нем присутствовали как религиозные, так и политические деятели из Иерусалима. Они сообщили Ироду, что практически все синагоги как в Галилее, так и в Иудее уже закрыты для учений Иисуса. Была предпринята новая попытка убедить Ирода взять Иисуса под арест, но он отказался последовать этой рекомендации. Однако 18 мая Ирод согласился с предложением позволить властям синедриона схватить Иисуса и доставить его в Иерусалим для суда по обвинению в религиозных преступлениях при условии, что римский правитель Иудеи даст согласие на эти действия. Тем временем враги Иисуса усердно распространяли по Галилее слух, будто Ирод ожесточился против Иисуса и намеревается истребить всех верующих в его учения.
\vs p154 3:2 1 мая в субботу ночью до Тивериады дошла весть о том, что гражданские власти в Иерусалиме не имеют ничего против соглашения между Иродом и фарисеями схватить Иисуса, доставить в Иерусалим и предать суду синедриона по обвинению в глумлении над священными законами еврейского народа. Соответственно, в этот же день, около полуночи, Ирод подписал указ, наделявший представителей синедриона полномочиями схватить Иисуса во владениях Ирода и силой доставить его в Иерусалим для предания суду. Прежде чем Ирод согласился дать такое разрешение, на него с разных сторон было оказано сильное давление; он ведь прекрасно знал, что Иисус не мог ожидать от своих злейших врагов в Иерусалиме честного суда.
\usection{4. В субботу ночью в Капернауме}
\vs p154 4:1 В ту же субботнюю ночь в капернаумской синагоге собрались пятьдесят наиболее влиятельных жителей города, чтобы обсудить важный вопрос: «Что делать с Иисусом?» Обсуждение и споры затянулись далеко за полночь, но они так и не смогли прийти к общему соглашению. Не считая нескольких человек, склонных верить, что Иисус мог быть Мессией, по крайней мере святым, или возможно пророком, собрание разделилось на четыре приблизительно равные группы, которые, соответственно, считали, что Иисус:
\vs p154 4:2 \ublistelem{1.}\bibnobreakspace Заблуждавшийся и безобидный религиозный фанатик.
\vs p154 4:3 \ublistelem{2.}\bibnobreakspace Опасный и коварный агитатор, который может разжечь восстание.
\vs p154 4:4 \ublistelem{3.}\bibnobreakspace В сговоре с бесами и даже, может быть, бесовский князь.
\vs p154 4:5 \ublistelem{4.}\bibnobreakspace Безумен и психически неуравновешен, не в себе.
\vs p154 4:6 \pc Много говорили и об учениях, которые проповедовал Иисус и которые так волновали простой люд; враги Иисуса утверждали, что его учения неприменимы на практике, что все разрушилось бы, если бы каждый честно попытался жить согласно его идеям. И люди во многих последующих поколениях говорили то же самое. Многие разумные и доброжелательные люди даже в более просвещенный век этих откровений утверждают, что современная цивилизация не могла быть создана в соответствии с учениями Иисуса, --- и они отчасти правы. Однако все подобные сомневающиеся забывают, что на его учениях могла бы быть построена намного лучшая цивилизация и что когда\hyp{}нибудь так и будет. Несмотря на частые попытки руководствоваться доктринами так называемого христианства, этот мир никогда не стремился по\hyp{}настоящему следовать учениям Иисуса во всей полноте.
\usection{5. Богатое событиями воскресное утро}
\vs p154 5:1 2 мая в жизни Иисуса было днем, полным событий. В это воскресное утро еще до рассвета из Тивериады чрезвычайно спешно прибыл один из вестников Давида и принес весть о том, что Ирод наделил, или готовится наделить, представителей синедриона полномочиями арестовать Иисуса. Известие о нависшей опасности заставило Давида Зеведеева поднять на ноги своих вестников и отправить их ко всем местным группам учеников, чтобы созвать тех к семи часам утра на чрезвычайный совет. Услышав эту тревожную весть, свояченица Иуды (брата Иисуса) поспешила сообщить ее всем членам семьи Иисуса, жившим неподалеку, и призвала их собраться в доме Зеведея. В ответ на этот поспешный зов там вскоре собрались Мария, Иаков, Иосиф, Иуда и Руфь.
\vs p154 5:2 На встрече, происходившей в это раннее утро, Иисус дал прощальные наставления собравшимся ученикам; то есть попрощался с ними на какое\hyp{}то время, хорошо зная, что вскоре им придется покинуть Капернаум. Он повелел им искать Божьего водительства и, невзирая на последствия, продолжать дело царства. Евангелисты должны были трудиться, как сочтут нужным, до тех пор, пока не наступит время, когда они будут призваны. Он избрал двенадцать евангелистов, которые должны были сопровождать его; двенадцати же апостолам велел оставаться с ним, что бы ни происходило. Иисус распорядился, чтобы двенадцать женщин остались в доме Зеведея и в доме Петра до тех пор, пока он не пошлет за ними.
\vs p154 5:3 Иисус согласился с тем, чтобы Давид Зеведеев продолжал содержать по всей стране свою службу вестников, и Давид, прощаясь с Учителем, сказал: «Иди и делай свое дело, Учитель. Не дай фанатикам поймать себя и никогда не сомневайся в том, что вестники будут следовать за тобой. Мои люди никогда не потеряют связь с тобой, и через них ты будешь узнавать о делах царства в других местах, а мы с их помощью будем все знать о тебе. Что бы ни случилось со мной, это не повлияет на эту службу, ибо я назначил первого, второго и даже третьего начальников. Я не учитель и не проповедник, но всем сердцем поддерживаю тебя и ничто не остановит меня».
\vs p154 5:4 В это утро около 7:30 Иисус обратился с прощальными словами к примерно сотне верующих, столпившихся в доме послушать его. Для всех присутствовавших это было торжественным событием, но Иисус, казалось, был необыкновенно весел; он снова стал таким, каким был всегда. Серьезность, присущая ему в последние несколько недель, прошла, и он вдохновил всех своими словами веры, надежды и отваги.
\usection{6. Прибытие семьи Иисуса}
\vs p154 6:1 В то воскресенье около восьми часов утра к месту событий прибыли пять членов земной семьи Иисуса, срочно вызванных свояченицей Иуды. Из всех членов его земной семьи только одна Руфь всем сердцем и без колебаний верила в божественность его миссии на земле. Иуда же, Иаков и даже Иосиф, хотя в основном и продолжали верить в Иисуса, все же позволили гордыне повлиять на их здравое суждение и подлинные духовные устремления. Так же и Мария разрывалась между любовью и страхом, между материнской любовью и семейной гордостью. Но хотя ее и терзали сомнения, она не могла до конца забыть, как посетил ее Гавриил до рождения Иисуса. Фарисеи старались убедить Марию, что Иисус не в себе, что он безумен. Они призывали ее пойти и вместе с сыновьями попытаться отговорить его от дальнейших публичных проповедей. Они уверяли Марию, что здоровье Иисуса вскоре пошатнется и что, если позволить ему продолжать свои труды, позор и бесчестие падут на всю семью. Поэтому, получив известие от свояченицы Иуды, все пятеро сразу же отправились из дома Марии, где они предыдущим вечером встречались с фарисеями, к дому Зеведея. С иерусалимскими лидерами они беседовали до поздней ночи, и все были так или иначе убеждены, что Иисус ведет себя странно, что он уже какое\hyp{}то время вел себя странно. Хотя Руфь и не могла объяснить все в его поведении, она, тем не менее настаивала, что он всегда хорошо относился к своей семье и никак не соглашалась с идеей попытаться отговорить его от дальнейшей работы.
\vs p154 6:2 По пути к дому Зеведея они еще раз обсудили это и договорились попытаться убедить Иисуса вернуться вместе с ними домой, ибо Мария сказала: «Я знаю, что смогу повлиять на моего сына, только если он придет домой и выслушает меня». До Иакова и Иуды дошли слухи о планах арестовать Иисуса и доставить его в Иерусалим для предания суду. Они опасались и за свою собственную безопасность. Пока в глазах людей Иисус был популярной фигурой, его семья допускала, чтобы события развивались своей чередой, однако теперь, когда жители Капернаума и иерусалимские лидеры неожиданно ополчились на него, они начали остро ощущать тяжесть мнимого бесчестия их сложного положения.
\vs p154 6:3 Они надеялись встретиться с Иисусом, отвести его в сторону и уговорить идти вместе с ними домой. Они думали, что сумеют убедить Иисуса, что забудут его небрежение ими --- простят и забудут, --- если только он оставит глупые попытки проповедовать новую религию, которые могут навлечь на него самого лишь беды, а на его семью --- бесчестье. На все это Руфь лишь сказала: «Я скажу своему брату, что считаю его Божьим человеком и надеюсь на то, что он будет готов скорее умереть, чем позволить этим нечестивым фарисеям остановить его проповедь». Иосиф пообещал, что заставит Руфь молчать, пока другие будут уговаривать Иисуса.
\vs p154 6:4 Когда они прибыли к дому Зеведея, Иисус еще продолжал говорить прощальные слова ученикам. Они попытались войти в дом, но дом был забит битком. Наконец они нашли себе место на заднем крыльце и по цепочке от одного человека к другому передали Иисусу весть, ее в конце концов прошептал ему Петр, который ради этого прервал речь Иисуса, сказав: «Вот матерь твоя и братья твои стоят снаружи, желая говорить с тобой». Его мать не понимала, сколь важно было это прощальное послание для последователей Иисуса, и не знала, что его обращение в любой момент могло быть прервано приходом тех, кто собирался его схватить. После столь долгого кажущегося отчуждения и считая, что она и его братья проявили доброжелательность, придя к нему, она думала, что Иисус прервет свою речь и придет к ним, как только услышит о том, что они его ждут.
\vs p154 6:5 Это был еще один из тех случаев, когда земная семья Иисуса не смогла понять, что он должен быть в том, что принадлежит Отцу. Поэтому и Мария, и его братья были глубоко обижены тем, что, хотя он и сделал паузу в своей речи, чтобы выслушать послание, он не поспешил наружу приветствовать их, а услышали они, как его мелодичный голос громче обычного произнес: «Скажите матери моей и братьям моим, что они не должны бояться за меня. Отец, пославший меня в этот мир, не оставит меня; не случится никакой беды и с моей семьей. Вели им держаться смело и уповать на Отца царства. Но в конце концов, кто матерь моя, и кто братья мои?» И протянув руки свои ко всем своим ученикам, собравшимся в зале, он сказал: «Нет у меня матери; нет у меня братьев. Вот матерь моя и вот братья мои! Ибо, кто исполняет волю Отца моего Небесного, тот мне матерь, брат и сестра».
\vs p154 6:6 Услышав эти слова, Мария упала на руки Иуде. Чтобы привести Марию в чувство, ее отнесли в сад; Иисус же тем временем произносил заключительные слова своего прощального послания. Он наверняка вышел бы переговорить со своей матерью и своими братьями, но из Тивериады спешно прибыл вестник и принес сообщение о том, что служители синедриона, наделенные полномочиями арестовать Иисуса и доставить его в Иерусалим, уже в пути. Андрей принял это известие и, прервав Иисуса, сообщил его ему.
\vs p154 6:7 Андрей упустил из виду, что Давид выставил двадцать пять часовых вокруг дома Зеведея, и никто не мог застать их врасплох; поэтому он спросил Иисуса, что делать. Учитель стоял в молчании, в то время как в саду его мать, услышавшая слова: «Нет у меня матери», приходила в себя от потрясения. Именно в это время одна из женщин, находившихся в зале, встала и воскликнула: «Блаженно чрево, носившее тебя, и сосцы, тебя питавшие». Иисус на мгновение оторвался от разговора с Андреем и ответил этой женщине, сказав: «Нет, блаженны слышащие слово Бога и нашедшие в себе смелость повиноваться ему».
\vs p154 6:8 \pc Мария и братья Иисуса считали, что Иисус не понимал их, что он утратил к ним интерес, и совершенно не сознавали, что это они не сумели понять Иисуса. Иисус отчетливо понимал, как трудно людям порвать со своим прошлым. Он знал, как воздействует на человеческие существа красноречие проповедника и что сознание реагирует на эмоциональный призыв так же, как разум --- на логику и аргументы, но он также знал, насколько сложнее убедить человека \bibemph{отречься от прошлого.}
\vs p154 6:9 Вечная истина состоит в том, что все, считающие себя непонятыми и неоцененными, имеют в Иисусе сочувствующего друга и понимающего советчика. Он предупредил своих апостолов, что врагами человека могут быть его домашние, но едва ли догадывался о том, насколько близким к истине окажется это предсказание в отношении его самого. Не Иисус оставил свою земную семью ради дела Отца --- это она оставила его. Позднее, после смерти и воскресения Учителя, Иаков присоединился к движению первых христиан и безмерно страдал, поскольку ранее не сумел насладиться общением с Иисусом и его учениками.
\vs p154 6:10 \pc Переживая эти события, Иисус решил действовать согласно ограниченным знаниям своего человеческого ума. Он хотел как обычный человек подвергнуться испытанию вместе со своими соратниками. И перед тем, как уйти, Иисус\hyp{}человек намеревался повидаться со своей семьей. но он не хотел прерывать свою речь и тем самым превращать их первую после столь долгой разлуки встречу в публичное зрелище. Он собирался закончить свое обращение, и только потом, перед уходом, пообщаться с ними, но этому помешало стечение обстоятельств.
\vs p154 6:11 Поспешный уход ускорило прибытие к задним дверям дома Зеведея отряда вестников Давида. Смятение, которое произвели вестники, испугало апостолов: и они подумали, что вновь прибывшие могли быть теми, кто должен был их схватить, и, опасаясь немедленного ареста, поспешили через передние двери к ожидавшей их лодке. Все это и объясняет, почему Иисус не повидался со своей семьей, ожидавшей его у задних дверей.
\vs p154 6:12 Но в момент поспешного бегства, уже садясь в лодку, он сказал Давиду Зеведееву: «Передай матери моей и братьям моим: я ценю, что они пришли, и собирался увидеться с ними. Убеди их не искать обиды во мне, но искать знания воли Бога, а также милости и мужества исполнять эту волю».
\usection{7. Поспешное бегство}
\vs p154 7:1 Итак, в это воскресное утро, двадцать второго мая 29 года н.э. Иисус со своими двенадцатью апостолами и двенадцатью евангелистами, поспешно спасались бегством от служителей синедриона, которые уже подходили к Вифсаиде с полномочиями от Ирода Антипы арестовать его и доставить в Иерусалим для суда по обвинению в богохульстве и других нарушениях священных законов евреев. Была почти половина восьмого прекрасного утра, когда двадцать пять человек сели за весла и направились к восточному берегу Галилейского моря.
\vs p154 7:2 За лодкой Учителя следовало еще одно судно поменьше, в котором были шесть вестников Давида, получивших приказ держать связь с Иисусом и его соратниками и следить за тем, чтобы известия о местонахождении и безопасности Иисуса регулярно передавались в дом Зеведея в Вифсаиде, служивший уже какое\hyp{}то время главным центром управления делами царства. Однако Иисусу больше не довелось останавливаться в доме Зеведея. Отныне на протяжении всей своей оставшейся земной жизни Учитель истинно «не имел, где приклонить голову». У него больше никогда не было даже подобия постоянного жилища.
\vs p154 7:3 Приплыв к месту неподалеку от селения Хересы, они передали лодку на хранение друзьям и отправились в странствие, длившееся весь богатый событиями последний год жизни Учителя на земле. Какое\hyp{}то время они оставались во владениях Филиппа, идя от Хересы к Кесарии Филипповой, а оттуда направились к финикийскому берегу.
\vs p154 7:4 \pc Толпа около дома Зеведея не расходилась и смотрела, как эти две лодки плыли к восточному берегу озера, и люди были сильно встревожены, когда к ним подбежали иерусалимские служители и стали искать Иисуса. Они отказывались верить, что Иисус бежал, и пока тот со своими спутниками шел на север по Батании, фарисеи и их помощники почти целую неделю напрасно искали его в окрестностях Капернаума.
\vs p154 7:5 Семья Иисуса вернулась домой в Капернаум и почти неделю провела в беседах, спорах и молитвах. Они были смятены и растеряны. И успокоились только тогда, когда в четверг после полудня возвратилась Руфь, навещавшая дом Зеведея, где от Давида она узнала, что ее отец\hyp{}брат в безопасности и в добром здравии держит путь к финикийскому берегу.
