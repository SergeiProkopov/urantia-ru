\upaper{26}{Духи\hyp{}служители центральной вселенной}
\author{Совершенствователь Мудрости}
\vs p026 0:1 Супернафимы --- это духи\hyp{}служители Рая и центральной вселенной; это высший чин низшей группы детей Бесконечного Духа --- ангельского сонма. Таких духов\hyp{}служителей можно встретить повсюду от Райского Острова до миров со временем и пространством. Ни одна крупная часть формированного и обитаемого мироздания не остается без их службы.
\usection{1. Духи\hyp{}служители}
\vs p026 1:1 Ангелы --- это духи\hyp{}служители, помогающие эволюционным и восходящим волевым созданиям всего пространства; они являются также соратниками и сподвижниками более высоких сонмов божественных личностей миров. Ангелы всех чинов --- это, несомненно, личности с четко выраженной индивидуальностью. Все они способны испытывать чувство большой признательности за помощь руководителей восстановления. Вместе с Сонмами Вестников Пространства духи\hyp{}служители наслаждаются периодами отдыха и сменой обстановки; это очень общительные натуры, и по своей способности к общению они намного превосходят людей.
\vs p026 1:2 \pc Духов\hyp{}служителей великой вселенной классифицируют:
\vs p026 1:3 \ublistelem{1.}\bibnobreakspace Супернафимы.
\vs p026 1:4 \ublistelem{2.}\bibnobreakspace Секонафимы.
\vs p026 1:5 \ublistelem{3.}\bibnobreakspace Терциафимы.
\vs p026 1:6 \ublistelem{4.}\bibnobreakspace Омниафимы.
\vs p026 1:7 \ublistelem{5.}\bibnobreakspace Серафимы.
\vs p026 1:8 \ublistelem{6.}\bibnobreakspace Херувимы и сановимы.
\vs p026 1:9 \ublistelem{7.}\bibnobreakspace Срединные создания.
\vs p026 1:10 \pc Личный статус конкретных существ, входящих в ангельские чины, не является совершенно неизменным. Ангелы некоторых чинов могут на время становиться Райскими Компаньонами; кто\hyp{}то становится Небесным Протоколистом; другие поднимаются до ранга Технических Советчиков. Отдельные херувимы могут подниматься до статуса и предназначения серафимов, а эволюционные серафимы могут достигать духовных уровней восходящих Сынов Бога.
\vs p026 1:11 \pc Семь чинов духов\hyp{}хранителей, о которых сказано выше, в этом повествовании сгруппированы в соответствии с их функциями, имеющими величайшее значение для восходящих созданий:
\vs p026 1:12 \ublistelem{1.}\bibnobreakspace \bibemph{Духи\hyp{}служители центральной вселенной.} Три чина \bibemph{супернафимов} служат в системе Рая\hyp{}Хавоны. Первичные, или Райские супернафимы созданы Бесконечным Духом. Чины вторичных и третичных супернафимов, служащие в Хавоне, являются творениями, соответственно, Духов\hyp{}Мастеров и Духов Контуров.
\vs p026 1:13 \ublistelem{2.}\bibnobreakspace \bibemph{Духи\hyp{}служители сверхвселенных ---} секонафимы, терциафимы и омниафимы. \bibemph{Секонафимы,} дети Отражательных Духов, выполняют разные службы в семи сверхвселенных. \bibemph{Терциафимы,} созданные Бесконечным Духом, в конечном счете, предназначены для службы связи Сынов\hyp{}Творцов и Древних Дней. \bibemph{Омниафимы} созданы совместно Бесконечным Духом и Семью Верховными Распорядителями и служат исключительно последним. Описание этих трех чинов будет темой последующего рассказа в этом выпуске.
\vs p026 1:14 \ublistelem{3.}\bibnobreakspace \bibemph{Духи\hyp{}служители локальных вселенных ---} это \bibemph{серафимы} и их помощники \bibemph{херувимы.} С этими потомками Духа\hyp{}Матери Вселенной восходящие смертные вступают в контакт раньше всего. \bibemph{Срединные создания,} уроженцы обитаемых миров, вообще говоря, не принадлежат к собственно ангельским чинам, хотя функционально их часто объединяют с духами\hyp{}служителями. Повествование о них и рассказ о серафимах и херувимах представлены в текстах, посвященных делам вашей локальной вселенной.
\vs p026 1:15 \pc Все чины ангельских сонмов заняты выполнением различных служб во вселенных и тем или иным образом помогают более высоким чинам небесных существ; но именно множество супернафимов, секонафимов и серафимов занято поддержкой в процессе восхождения и совершенствования детей времени. Действуя в центральной вселенной, в сверхвселенной и в локальных вселенных, они образуют неразрывную цепь духовных служителей, созданную Бесконечным Духом для того, чтобы помогать всем тем, кто стремится достичь Отца Всего Сущего через Вечного Сына, и направлять их.
\vs p026 1:16 Супернафимы имеют ограниченную «духовную полярность», только к одному аспекту деятельности --- тому, что связан с Отцом Всего Сущего. Они могут работать в одиночку, кроме случаев, когда прямо используют исключительные контуры Отца. Когда супернафимы воспринимают мощь для прямого служения Отца, они должны по собственному выбору объединяться в пары, чтобы быть способными действовать. Точно так же ограничены и секонафимы, которые к тому же должны действовать парами, чтобы синхронизироваться с контурами Вечного Сына. Серафимы могут действовать поодиночке как конкретные и локализованные личности, но они могут быть объятыми контуром, только когда соединены в пары с созданием противоположной полярности. Когда эти духовные существа объединяются в пары, то один называется дополнителем другого. Дополнители могут быть временными; их объединение не обязательно имеет постоянный характер.
\vs p026 1:17 Эти блестящие создания света поддерживаются непосредственно за счет потребления духовной энергии из первичных контуров вселенной. Смертные Урантии должны получать энергию света через посредство растений, но ангельские сонмы объяты контуром; они «имеют пищу, которой вы не знаете». Они также получают циркулирующие учения изумительных Сынов\hyp{}Учителей Троицы; их восприятие знаний и впитывание мудрости очень похоже на то, как они впитывают жизненную энергию.
\usection{2. Могучие супернафимы}
\vs p026 2:1 Супернафимы --- искусные служители всех типов существ, пребывающих в Раю и в центральной вселенной. Сотворено три главных чина этих высоких ангелов: первичные, вторичные и третичные.
\vs p026 2:2 \pc \bibemph{Первичные супернафимы} являются исключительным потомством Объединенного Творца. Их служение примерно в равной степени посвящается определенным группам Граждан Рая и постоянно увеличивающемуся отряду восходящих пилигримов. Эти ангелы вечного Острова чрезвычайно эффективно содействуют необходимому обучению этих групп обитателей Рая. Они значительно способствуют взаимопониманию между этими двумя уникальными чинами вселенских созданий --- один из которых является высшим типом божественного и совершенного волевого создания, а другой --- усовершенствованным эволюцией низшим типом волевого создания во всей вселенной вселенных.
\vs p026 2:3 \pc Работа первичных супернафимов настолько уникальна и своеобразна, что она будет отдельно рассмотрена в последующем изложении.
\vs p026 2:4 \pc \bibemph{Вторичные супернафимы} управляют делами восходящих существ в семи контурах Хавоны. Они в равной степени участвуют также и в служении обучению многочисленных чинов Граждан Рая, подолгу пребывающих в контурах миров центральной вселенной, но нам не позволено обсуждать этот аспект их службы.
\vs p026 2:5 \pc Существует семь типов этих высоких ангелов, каждый из которых происходит от одного из Семи Духов\hyp{}Мастеров и создан по его подобию. Семь Духов\hyp{}Мастеров совместно создают много разных групп уникальных существ и сущностей, и все существа одного чина по природе относительно похожи друг на друга. Но когда те же самые Семь Духов творят поодиночке, то создаваемые чины всегда семеричны по своей природе; дети каждого из Духов\hyp{}Мастеров разделяют природу своего творца и, соответственно, отличаются от других. Таково происхождение вторичных супернафимов, и ангелы всех семи созданных типов функционируют во всех сферах деятельности, доступных всему их чину, главным образом, в семи контурах центральной и божественной вселенной.
\vs p026 2:6 \pc Каждый из семи планетарных контуров Хавоны напрямую подчиняется одному из Семи Духов Контуров, которые в свою очередь являются коллективным --- а потому единообразным --- творением Семи Духов\hyp{}Мастеров. Хотя они и разделяют природу Третьего Источника и Центра, эти семь вспомогательных Духов Хавоны не были частью паттерна изначальной вселенной. Они начали функционировать уже после изначального (вечного) творения, но задолго до времен Грандфанды. Несомненно, они появились как творческий отклик Духов\hyp{}Мастеров на возникающую цель Верховного Существа, и они были обнаружены по функционированию при организации великой вселенной. Бесконечный Дух и все его сподвижники\hyp{}творцы, являющиеся вселенскими координаторами, очевидно, в избытке наделены способностью адекватно реагировать созидательными действиями на процессы развития, имеющие место одновременно в Божествах опыта и в развивающихся вселенных.
\vs p026 2:7 \pc \bibemph{Третичные Супернафимы} ведут происхождение от этих Семи Духов Контуров. Каждый из них в отдельном контуре Хавоны уполномочен Бесконечным Духом сотворить необходимое количество высоких служителей --- третичных супернафимов, чтобы удовлетворить потребности центральной вселенной. Если Духи Контуров до прибытия пилигримов времени в Хавону создали сравнительно немного этих ангельских служителей, то Семь Духов\hyp{}Мастеров до прибытия Грандфанды даже не начинали творить вторичных супернафимов. Поскольку из этих двух чинов третичные супернафимы созданы первыми, сначала рассмотрим их.
\usection{3. Третичные Супернафимы}
\vs p026 3:1 Эти служители Семи Духов\hyp{}Мастеров являются специализированными ангелами в разных контурах Хавоны, и их служение распространяется как на восходящих пилигримов времени, так и на нисходящих пилигримов вечности. В миллиарде учебных миров совершенной центральной вселенной ваши сподвижники\hyp{}супернафимы станут полностью видимыми для вас. Там между вами установятся братские отношения и взаимопонимание в самом высоком смысле этих слов, взаимный контакт и симпатия. Вы также близко познакомитесь и установите братские отношения с нисходящими пилигримами, Гражданами Рая, пересекающими эти контуры изнутри, входящими в Хавону через путеводный мир первого контура и следующими из него далее в седьмой контур.
\vs p026 3:2 Восходящие пилигримы из семи сверхвселенных проходят через Хавону в противоположном направлении, входя через путеводный мир седьмого контура и следуя дальше внутрь. Не существует никаких ограничений, налагаемых на время продвижения восходящих созданий от мира к миру и от контура к контуру, как нет и никакого фиксированного периода времени, произвольно установленного для пребывания в моронтийных мирах. Но если от пребывания в одном или более учебных мирах локальных вселенных достаточно развитые индивидуумы могут быть освобождены, то прохождения через все семь контуров возрастающего одухотворения Хавоны не может избежать ни один пилигрим.
\vs p026 3:3 \pc В этот отряд третичных супернафимов, предназначенных, главным образом, служить пилигримам времени, входят:
\vs p026 3:4 \ublistelem{1.}\bibnobreakspace \bibemph{Руководители Гармонии.} Абсолютно ясно, что даже в совершенной Хавоне необходимо определенное координирующее влияние для того, чтобы поддерживать систему и обеспечивать гармонию в работе по подготовке пилигримов времени к их последующим райским достижениям. Такова реальная миссия руководителей гармонии --- обеспечивать, чтобы все шло гладко и быстро. Они возникают в первом контуре, служат повсюду в Хавоне, и их присутствие в контурах исключает возможность каких\hyp{}либо сбоев. Высокая способность координировать разнообразные виды деятельности, которыми занимаются личности разных чинов и даже разных уровней, позволяет этим супернафимам оказывать помощь везде и всегда, где и когда она требуется. Они в огромной степени содействуют взаимопониманию между пилигримами времени и пилигримами вечности.
\vs p026 3:5 \ublistelem{2.}\bibnobreakspace \bibemph{Главные Протоколисты.} Эти ангелы созданы во втором контуре, но действуют повсюду в центральной вселенной. Они ведут протоколы в трех экземплярах, делая один экземпляр для текстовых архивов Хавоны, для духовных архивов их чина и для официальных протоколов Рая. Кроме того, они автоматически передают труды, содержащие особо важные аспекты истинного знания, в живые библиотеки Рая --- хранителям знаний чина первичных супернафимов.
\vs p026 3:6 \ublistelem{3.}\bibnobreakspace \bibemph{Возвестники.} Эти дети третьего Духа Контура действуют повсюду в Хавоне, хотя их официальное местопребывание --- планета номер семьдесят в самом внешнем контуре. Эти мастера\hyp{}техники принимают и передают вести в центральной вселенной и являются управителями космических сообщений о всех Божественных явлениях в Раю. Они могут управлять всеми основными контурами космоса.
\vs p026 3:7 \ublistelem{4.}\bibnobreakspace \bibemph{Вестники} возникли в контуре номер четыре. Они перемещаются по всей системе Рая\hyp{}Хавоны, доставляя те известия, которые требуют личной передачи. Они служат своим собратьям --- небесным личностям, райским пилигримам и даже восходящим душам времени.
\vs p026 3:8 \ublistelem{5.}\bibnobreakspace \bibemph{Координаторы информации.} Эти третичные супернафимы, дети пятого Духа Контура, занимаются тем, что всегда мудро и благожелательно способствуют установлению братских отношений между восходящими и нисходящими пилигримами. Они содействуют всем обитателям Хавоны, и особенно --- восходящим, постоянно держа их в курсе дел вселенной вселенных. Благодаря личным контактам с возвестниками и отражателями, эти «живые газеты» Хавоны мгновенно знакомятся с информацией обо всем проходящем по обширным контурам новостей центральной вселенной. Они получают сведения хавонографическим методом, что позволяет за один час урантийского времени автоматически усвоить столько информации, что на ее передачу самым быстрым вашим телеграфом потребовалась бы тысяча лет.
\vs p026 3:9 \ublistelem{6.}\bibnobreakspace \bibemph{Личности Перемещения.} Эти существа, возникшие в контуре номер шесть, обычно действуют с планеты номер сорок в самом внешнем контуре. Именно они уносят разочарованных кандидатов, временно потерпевших неудачу на божественном пути. Они постоянно готовы служить всем, кто должен отправиться куда\hyp{}нибудь по служебным делам Хавоны, но не способны перемещаться в пространстве.
\vs p026 3:10 \ublistelem{7.}\bibnobreakspace \bibemph{Резервный Отряд.} Непредсказуемость работы с восходящими существами, райскими пилигримами и существами других чинов, пребывающими в Хавоне, вызывает необходимость содержать в резерве супернафимов в путеводном мире седьмого контура --- там же, где они и были созданы. Они создавались без специального предназначения и способны брать на себя выполнение менее ответственных задач в любых видах деятельности своих товарищей --- третичных супернафимов.
\usection{4. Вторичные супернафимы}
\vs p026 4:1 Вторичные супернафимы исполняют служение в семи планетарных контурах центральной вселенной. Часть из них занимается обслуживанием пилигримов времени, а половина всего их чина назначена обучать райских пилигримов вечности. Этим Гражданам Рая во время их странствия по контурам Хавоны уделяют внимание также добровольцы из Отряда Смертных Финалитов, и это получило широкое распространение со времени завершения формирования первой группы финалитов.
\vs p026 4:2 \pc Вторичные супернафимы действуют в составе следующих семи групп, которые выделены в соответствии с периодически получаемыми ими заданиями, связанными со служением восходящим пилигримам:
\vs p026 4:3 \ublistelem{1.}\bibnobreakspace Помощники пилигримов.
\vs p026 4:4 \ublistelem{2.}\bibnobreakspace Проводники Верховенства.
\vs p026 4:5 \ublistelem{3.}\bibnobreakspace Проводники Троицы.
\vs p026 4:6 \ublistelem{4.}\bibnobreakspace Искатели Сына.
\vs p026 4:7 \ublistelem{5.}\bibnobreakspace Проводники Отца.
\vs p026 4:8 \ublistelem{6.}\bibnobreakspace Советники и советчики
\vs p026 4:9 \ublistelem{7.}\bibnobreakspace Дополнители отдыха.
\vs p026 4:10 \pc В каждую из этих рабочих групп входят ангелы всех семи созданных типов, и пилигрима пространства всегда обучает вторичный супернафим, созданный Духом\hyp{}Мастером, руководителем именно той сверхвселенной, из которой родом этот пилигрим. Когда вы, смертные Урантии, достигнете Хавоны, то вас обязательно будет сопровождать супернафим, сотворенная природа которого --- как и ваша собственная эволюционная природа --- происходит от Духа\hyp{}Мастера Орвонтона. И поскольку ваши наставники происходят от Духа\hyp{}Мастера вашей собственной сверхвселенной, они особенно подходят для того, чтобы понять, поддержать вас и помочь во всех ваших усилиях достичь райского совершенства.
\vs p026 4:11 Пилигримов времени переносят мимо темных гравитационных тел Хавоны во внешний планетарный контур личности перемещения из чина первичных секонафимов, действующие из центра семи сверхвселенных. Большинство, но не все серафимы планетарного и локально\hyp{}вселенского служения, которые были определены для Райского восхождения, расстанутся со своими смертными сподвижниками перед долгим перелетом в Хавону и сразу же начнут продолжительную и интенсивную подготовку к возвышенному назначению, надеясь достичь --- как серафимы --- совершенства существования и верховенства служения. И они делают это в надежде вновь соединиться с пилигримами времени, чтобы войти в число вечно следующих по стопам тех смертных, которые достигли Отца Всего Сущего и получили возвышенное назначение на нераскрытое еще служение в Отряде Финалитов.
\vs p026 4:12 Пилигрим прибывает на принимающую планету Хавоны, в путеводный мир седьмого контура только с одним даром совершенства --- с совершенством намерения. Отец Всего Сущего повелел: «Будьте совершенны, как и я совершенен». Это поразительное приглашение\hyp{}указание, переданное конечным детям пространственных миров. Возвещение этого предписания привело в движение все творение, и небесные существа стали прилагать совместные усилия, чтобы помочь выполнению и осуществлению этого грандиозного указания Великого Первоисточника и Центра.
\vs p026 4:13 Когда благодаря служению всех сонмов помощников вселенского плана продолжения существования ты, наконец, вступишь в принимающий мир Хавоны, то ты прибываешь туда только с одним видом совершенства --- с \bibemph{совершенством намерения.} Ты доказал серьезность своего намерения; твоя вера испытана. Установлено и проверено, что ты можешь противостоять разочарованию. Даже если не удается различить Отца Всего Сущего, это не может поколебать веру или серьезно подорвать доверие восходящего смертного, прошедшего через опыт, через который должны пройти все, чтобы достичь совершенных сфер Хавоны. К тому времени, когда ты достиг Хавоны, ты обрел высочайшую искренность. Совершенство намерений и божественность желаний при непоколебимости веры обеспечили тебе вход в установленное обиталище вечности; твое избавление от неопределенности времени --- полное и окончательное; и теперь тебе предстоит вплотную столкнуться с проблемами Хавоны и безмерностью Рая --- то, к чему ты так долго готовился в опытные эпохи времени в школах\hyp{}мирах пространства.
\vs p026 4:14 Вера дала восходящему пилигриму совершенство намерения, которое позволяет детям времени достичь врат вечности. Теперь помощники пилигримов должны начать заниматься развитием того совершенства понимания и той техники постижения, которые так необходимы для райского совершенства личности.
\vs p026 4:15 \bibemph{Способность понимать --- это пропуск в Рай для человека.} Готовность верить --- это ключ к дверям Хавоны. Признание сыновства, взаимодействие с внутренним Настройщиком --- это цена продолжения существования эволюционного существа в посмертии.
\usection{5. Помощники пилигримов}
\vs p026 5:1 Первая из семи групп вторичных супернафимов, с которыми предстоит встретиться, --- это помощники пилигримов, те сообразительные и доброжелательные существа, которые радушно приветствуют восходящих пространства, проделавших долгий путь к устойчивым мирам и уравновешенной системе центральной вселенной. Одновременно эти высокие служители начинают трудиться на благо райских пилигримов вечности, первый из которых прибыл в путеводный мир внутреннего контура Хавоны одновременно с прибытием Грандфанды в путеводный мир внешнего контура. Еще в те очень далекие дни пилигримы из Рая и пилигримы времени встретились в принимающем мире контура номер четыре.
\vs p026 5:2 Эти помощники пилигримов, действующие в седьмом круге миров Хавоны, проводят работу с восходящими смертными по трем основным направлениям: во\hyp{}первых, верховное понимание Райской Троицы; во\hyp{}вторых, духовное постижение партнерства Отца и Сына; и в\hyp{}третьих, интеллектуальное осознание Бесконечного Духа. Каждая из этих фаз обучения делится на семь разделов, состоящих из двенадцати подразделов, в которые входит по семьдесят групп; а в каждую из этих групп входит по тысяче тем. Более детальное обучение проводится в последующих кругах, но помощники пилигримов учат основам всего, что требуется знать в Раю.
\vs p026 5:3 Так что этот начальный, или элементарный, курс предстоит пройти обладающим испытанной верой и проделавшим долгий путь пилигримов пространства. Но еще задолго до прибытия в Хавону эти восходящие дети времени научились питаться неопределенностью, кормиться разочарованиями, испытывать воодушевление при очевидных поражениях, радоваться трудностям, проявлять неукротимое мужество перед лицом беспредельности и несокрушимую веру при столкновении с необъяснимым. С давних пор боевым кличем этих пилигримов стало: «Во взаимосвязи с Богом не существует ничего, абсолютно ничего --- невозможного».
\vs p026 5:4 В каждом из кругов Хавоны к пилигримам времени предъявляются определенные требования; и, хотя каждый пилигрим продолжает проходить курс, который должен быть освоен под опекой супернафима, по природе своей подходящего для помощи именно данному конкретному типу восходящих созданий, требования эти почти одинаковы для всех восходящих, достигающих центральной вселенной. Этот курс подразумевает достижения количественные, качественные и практические --- интеллектуальные, духовные и верховные.
\vs p026 5:5 Время не имеет большого значения в кругах Хавоны. В ограниченной степени оно учитывается при оценке успеваемости, но решающим и верховным критерием являются результаты. Как только ваш сподвижник\hyp{}супернафим сочтет, что вы достаточно подготовлены, чтобы перейти глубже на следующий круг, вы предстанете перед двенадцатью помощниками седьмого Духа Контура. Здесь от вас потребуется сдать экзамены данного круга, утвержденные той сверхвселенной и той системой, откуда вы родом. Божественное достижение этого круга происходит в путеводном мире и заключается в духовном осознании и постижении Духа\hyp{}Мастера сверхвселенной данного восходящего пилигрима.
\vs p026 5:6 Когда работа во внешнем круге Хавоны завершена и преподаваемый курс освоен, помощники пилигримов препровождают своих подопечных в путеводный мир следующего круга и поручают их заботам проводников верховенства. Помощники пилигримов всегда остаются еще на некоторое время, чтобы помочь сделать этот переход приятным и полезным.
\usection{6. Проводники Верховенства}
\vs p026 6:1 Когда восходящие личности пространства переносятся из седьмого круга в шестой, они называются «духовными выпускниками» и переходят под непосредственное руководство проводников верховенства. Этих проводников не следует путать с Проводниками Выпускников, принадлежащими к числу Высших Личностей Бесконечного Духа, которые вместе со своими сподвижниками\hyp{}сервиталами осуществляют служение как восходящим, так и нисходящим пилигримам во всех контурах Хавоны. Проводники верховенства действуют только на шестом круге центральной вселенной.
\vs p026 6:2 Именно в этом круге восходящие достигают нового понимания Верховной Божественности. На протяжении долгого пути в эволюционных вселенных пилигримы времени испытывали растущее осознание реальности всемогущего сверхконтроля над пространственно\hyp{}временными творениями. Здесь, в этом круге Хавоны, они приближаются к встрече с находящимся в центральной вселенной источником пространственно\hyp{}временного единства --- с духовной реальностью Бога Верховного.
\vs p026 6:3 Я затрудняюсь объяснить, что именно происходит в этом круге. Никакое персонализированное присутствие Верховенства не ощутимо для восходящих. В какой\hyp{}то степени новые взаимоотношения с Седьмым Духом\hyp{}Мастером компенсируют это отсутствие контакта с Верховным Существом. Но, несмотря на нашу неспособность понять, как именно это происходит, каждое восходящее создание, очевидно, испытывает преобразующий рост, новый уровень сознания, новое одухотворение намерений, новую чувствительность к божественному, что едва ли можно удовлетворительно объяснить, если не иметь в виду нераскрытую активность Верховного Существа. Тем из нас, кто наблюдал эти таинственные процессы, представляется, будто Бог Верховный любовно дарует своим детям опыта\hyp{}в полном соответствии с их способностью обретать этот опыт --- то повышение уровня интеллектуального понимания, духовной проницательности и стремления личности, которые так будут им необходимы при всех их усилиях проникнуть на божественный уровень Троицы Верховенства, достичь вечных и экзистенциальных Божеств Рая.
\vs p026 6:4 Когда проводники верховенства решают, что их ученики подготовлены к переходу на следующий уровень, те предстают перед комиссией семидесяти --- смешанной группой существ, служащих экзаменаторами в путеводном мире контура номер шесть. После того, как эту комиссию удовлетворит уровень понимания пилигримами Верховного Существа и Троицы Верховенства, их аттестуют для перевода в пятый контур.
\usection{7. Проводники Троицы}
\vs p026 7:1 Проводники Троицы --- это неутомимые служители пятого круга обучения восходящих пилигримов времени и пространства в Хавоне. Духовные выпускники называются здесь «кандидатами в божественное путешествие», поскольку именно в этом круге под руководством проводников Троицы пилигримы получают углубленные знания относительно божественной Троицы, что подготавливает их к попытке достичь осознания личности Бесконечного Духа. И здесь восходящим пилигримам открывается значение истинного учения и настоящих умственных усилий, они начинают понимать природу тех еще более сильных и гораздо более напряженных духовных усилий, которые необходимы, чтобы соответствовать требованиям высокой цели, которой им предстоит достигнуть в мирах этого контура.
\vs p026 7:2 Проводники Троицы чрезвычайно добросовестны и умелы; и каждый пилигрим пользуется безраздельным вниманием и всецелой любовью вторичного супернафима, принадлежащего к этому чину. Пилигрим времени никогда бы не нашел первую из личностей Райской Троицы, к которой можно приблизиться, если бы не помощь и содействие этих проводников и сонма других духовных существ, которые информируют восходящих личностей о характере и особенностях предстоящего пути к Божеству.
\vs p026 7:3 После завершения курса обучения в этом контуре проводники Троицы препровождают своих учеников в путеводный мир этого контура, где они предстают перед одной из многих триединых экзаменационных комиссий, которая аттестует кандидатов для продолжения пути к Божеству. В эту комиссию входят один представитель финалитов, один руководитель поведения из чина первичных супернафимов и один Одиночный Вестник пространства или же Тринитизированный Сын Рая.
\vs p026 7:4 Когда восходящая душа действительно отправляется в Рай, то перемещается только с сопровождающей тройкой, состоящей из его сподвижника из супернафимов на кругах Хавоны, Проводника Выпускников и всегда присутствующего сервитала\hyp{}сподвижника последнего. Эти путешествия из Хавоны в Рай носят пробный характер; восходящие еще не имеют райского статуса. Они не приобретут статус постоянных обитателей Рая, пока не пройдут последний период отдыха, достигнув Отца Всего Сущего и получив окончательное разрешение из контуров Хавоны. Только после божественного отдыха вкусят они «божественной сущности» и «духа верховенства» и, таким образом, действительно начнут функционировать в кругу вечности и в присутствии Троицы.
\vs p026 7:5 \pc От спутников из сопровождающей тройки требуется не наделить восходящего способностью определить географическое присутствие духовного свечения Троицы, а оказать пилигриму всю возможную помощь в его трудной задаче осознать, разглядеть и постичь Бесконечный Дух на уровне, позволяющем осознать личность. Любой восходящий пилигрим в Раю может различить географическое присутствие Троицы, подавляющее большинство способны контактировать с интеллектуальной реальностью Божеств, особенно Третьего Лица, но не все в состоянии постичь или даже частично понять реальность духовного присутствия Отца и Сына. Еще труднее даже минимальное духовное постижение Отца Всего Сущего.
\vs p026 7:6 Поиски Бесконечного Духа редко не достигают цели, и когда подопечные успешно завершили эту фазу пути к Божеству, проводники Троицы готовятся перепоручить их служению искателей Сына в четвертом круге Хавоны.
\usection{8. Искатели Сына}
\vs p026 8:1 Четвертый контур Хавоны иногда называют «контуром Сынов». Из миров этого контура восходящие пилигримы отправляются в Рай, чтобы обрести осмысленный контакт с Вечным Сыном, и в мирах этого контура нисходящие пилигримы достигают нового понимания природы и миссии Сынов\hyp{}Творцов времени и пространства. В этом контуре есть семь миров, в которых резервный отряд Райских Михаилов организует особые школы для взаимного служения как восходящим, так и нисходящим пилигримам; и именно в этих мирах Сынов\hyp{}Михаилов впервые достигается подлинное взаимопонимание между пилигримами времени и пилигримами вечности. Во многих отношениях опыт, обретаемый в этом контуре, самый увлекательный за весь период пребывания в Хавоне.
\vs p026 8:2 Искатели Сына --- это супернафимы, занимающиеся служением восходящим смертным в четвертом контуре. В дополнение к обыденной деятельности по подготовке своих учеников к постижению взаимоотношений Сына в Троице эти Искатели Сына должны так хорошо научить своих подопечных, чтобы те достигли полного успеха: во\hyp{}первых, в соответствующем духовном понимании Сына; во\hyp{}вторых, в удовлетворительном опознании личности Сына и, в\hyp{}третьих, в точном различении Сына от личности Бесконечного Духа.
\vs p026 8:3 После постижения Бесконечного Духа экзамены больше не проводятся. Испытания во внутренних кругах --- это поведение пилигримов\hyp{}кандидатов в объятиях Божеств. Продвижение вперед определяется только духовностью индивидуума, и никто, кроме Богов, не осмеливается выносить решение по этому вопросу. В случае неудачи никогда не выясняются причины, и ни самих кандидатов, ни различных их наставников и проводников никогда не упрекают и не критикуют. В Раю разочарование никогда не считается поражением; отсрочка никогда не рассматривается как что\hyp{}то постыдное; очевидные неудачи во времени никогда не смешиваются со значительными промедлениями в вечности.
\vs p026 8:4 \pc Лишь у немногих пилигримов возникают задержки, связанные с мнимыми неудачами на пути к Божеству. Почти все достигают Бесконечного Духа, хотя иногда пилигриму из сверхвселенной номер один это удается не с первой попытки. Редко случается, чтобы пилигримам, достигшим Духа, не удалось найти Сына; почти все, кто потерпел неудачу в первой попытке, --- родом из сверхвселенных номер три и пять. Подавляющее большинство терпящих неудачу в достижении Отца с первой попытки после того, как нашли и Духа, и Сына, --- уроженцы сверхвселенной номер шесть, хотя некоторые из них --- уроженцы второй и третьей сверхвселенных. И все это, по\hyp{}видимому, ясно указывает на то, что существует некая веская и достаточная причина для этих очевидных неудач; в сущности, просто неизбежных задержек.
\vs p026 8:5 Непрошедшие кандидаты в путешествие к Божеству передаются в ведение глав назначения --- группы первичных супернафимов и отсылаются обратно трудиться в пространственных мирах на период не менее тысячи лет. Они никогда не возвращаются в ту сверхвселенную, в которой родились, а всегда в ту, которая лучше всего подходит для дополнительного обучения и подготовки ко второму путешествию к Божеству. После этой службы они по собственному побуждению возвращаются во внешний круг Хавоны, их немедленно провожают в тот круг, где прервался их путь, и они сразу же возобновляют подготовку к путешествию к Богу. При второй попытке вторичные супернафимы всегда успешно проводят своих подопечных, и во время этого второго путешествия этим кандидатам всегда уделяют внимание те же самые супернафимы\hyp{}служители и другие проводники.
\usection{9. Проводники Отца}
\vs p026 9:1 Когда душа\hyp{}пилигрим достигает третьего круга Хавоны, она переходит под попечительство проводников Отца, более старых, чрезвычайно умелых и самых опытных из супернафимов\hyp{}служителей. В мирах этого контура проводники Отца содержат школы мудрости и колледжи методики, в которых учителями служат все существа, обитающие в центральной вселенной. Не упускается из виду ничто из того, что может принести пользу созданию времени на этом трансцендентном пути достижения вечности.
\vs p026 9:2 Достижение Отца Всего Сущего --- это пропуск в вечность, несмотря на оставшиеся контуры, которые предстоит пересечь. Поэтому наступает важнейшее событие в путеводном мире круга номер три, когда сопровождающая тройка объявляет, что вот\hyp{}вот произойдет последнее событие во времени; что еще одно создание пространства пытается войти в Рай через врата вечности.
\vs p026 9:3 \pc Испытание во времени почти завершено; путь к вечности почти пройден. Дни неуверенности заканчиваются; искушение усомниться исчезает; предписание быть \bibemph{совершенным} исполнено. Создание времени и материальная личность поднялась по эволюционным мирам пространства от самого низшего уровня разумного существования, подтверждая, таким образом, схему восхождения и вечно демонстрируя справедливость и праведность указания Отца Всего Сущего его низшим созданиям в мирах: «Будьте совершенны, как и я совершенен».
\vs p026 9:4 Шаг за шагом, жизнь за жизнью, мир за миром преодолен восходящий путь, и божественная цель достигнута. Продолжение существования завершается в совершенстве, а совершенство изобильно в верховенстве божественности. Время исчезает в вечности; пространство растворяется в благоговейной идентичности и гармонии с Отцом Всего Сущего. В передачах из Хавоны сообщаются триумфальные космические известия, благая весть, что, воистину, добросовестные создания, имевшие животную природу и материальное происхождение, посредством эволюционного восхождения действительно и навеки стали достигшими совершенства сынами Бога.
\usection{10. Советники и советчики}
\vs p026 10:1 Супернафимы советники и советчики второго круга являются учителями детей времени, наставляющими их относительно пути вечности. Достижение Рая приводит к обязанности нового и более высокого порядка, и пребывание во втором круге дает широкие возможности получить полезный совет от этих любящих супернафимов.
\vs p026 10:2 \pc Потерпевшие неудачу в достижении Божества в первой попытке переводятся из того круга, где они потерпели неудачу, прямо во второй круг прежде, чем их отправляют обратно на службу в сверхвселенные. Таким образом, советники и советчики служат также советниками и утешителями для этих потерпевших неудачу пилигримов. Их только что постигло величайшее разочарование, которое, по сути, ничем кроме значительности не отличается от длинной череды подобных же переживаний, постигавших их там, откуда они взбирались вверх --- как по лестнице, ведущей от хаоса к триумфу. Это те, кто до дна испили чашу опыта; и я замечал, что они временно возвращаются на службу в сверхвселенные как исполненные любви служители высочайшего типа, помогающие детям времени, испытавшим преходящие разочарования.
\vs p026 10:3 После долгого пребывания в контуре номер два потерпевших неудачу экзаменует совет совершенства, заседающий в путеводном мире этого круга, и аттестует их как сдавших экзамен в Хавоне; и это, если говорить не о духовном статусе, дает им такое же положение во вселенных времени, как если бы они действительно успешно завершили божественное путешествие. Дух этих кандидатов был вполне приемлемым; их неудача была вызвана каким\hyp{}то техническим аспектом их подхода или какой\hyp{}то частью их прошлого опыта.
\vs p026 10:4 Тогда советники данного круга приводят их к главам назначений в Раю, и те отправляют их обратно на службу времени в миры пространства; и они с радостью и удовольствием принимаются за дела прежних дней и эпох. В какой\hyp{}то другой день они вернутся в круг, где испытали величайшее разочарование, и снова попытаются завершить божественное путешествие.
\vs p026 10:5 Для пилигримов, достигших успеха во втором контуре, стимулирующее влияние эволюционной неуверенности уже позади, а увлекательное выполнение вечного задания еще не началось; и хотя пребывание в этом круге чрезвычайно приятно и очень полезно, здесь отчасти не хватает той страсти предвкушения, которая присутствовала в предыдущих кругах. Есть много таких пилигримов, которые в это время с просветленной завистью вспоминают долгую, долгую борьбу и действительно желали бы каким\hyp{}то образом вернуться в миры времени и начать все сначала, --- подобно тому, как вы, люди, с приближением преклонного возраста иногда вспоминаете о той борьбе, которую приходилось вести в юности и в молодые годы, и вам действительно хочется, чтобы можно было прожить жизнь заново еще раз.
\vs p026 10:6 Но предстоит пересечение самого внутреннего круга, и вскоре после этого закончится последний сон перемещения и начнется новое предприятие --- путь вечности. Во втором круге советники и советчики начинают подготовку своих подопечных к этому великому и последнему покою, неизбежному сну, который всегда наступает между эпохальными этапами восходящего пути.
\vs p026 10:7 Когда восходящие пилигримы, достигшие Отца Всего Сущего, завершают свой опыт во втором круге, то постоянно сопровождающие их Проводники Выпускников отдают приказ допустить их в последний круг. Эти проводники лично проводят своих подопечных во внутренний круг и там передают их на попечение дополнителей отдыха --- последнего из тех чинов вторичных супернафимов, которые назначены служить пилигримам времени в контурах миров Хавоны.
\usection{11. Дополнители отдыха}
\vs p026 11:1 Большую часть времени восходящий пилигрим в последнем контуре посвящает продолжению изучения связанных с пребыванием в Раю проблем, с которыми предстоит столкнуться. Обширный и разнообразный сонм существ, большинство из которых не открыты, постоянно или временно пребывают в этом внутреннем кольце миров Хавоны. И смешение этих многообразных типов дает супернафимам\hyp{}дополнителям отдыха богатую ситуативную среду, которую они эффективно используют для продолжения обучения восходящих пилигримов, особенно в связи с проблемами адаптации к многочисленным группам существ, с которыми скоро предстоит встретиться в Раю.
\vs p026 11:2 \pc К числу тех, кто обитает в этом внутреннем контуре, относятся сыны, тринитизированные созданиями. Первичные и вторичные супернафимы являются общими опекунами объединенного отряда этих сынов, включающего тринитизированных потомков смертных финалитов и таких же потомков Граждан Рая. Некоторые из этих сынов объяты Троицей и назначены в правительства сверхвселенных, другие получили разные назначения, но подавляющее большинство собраны вместе в объединенный отряд в совершенных мирах внутреннего контура Хавоны. Здесь под руководством супернафимов их готовит к некой будущей деятельности особый и безымянный отряд высоких Граждан Рая, которые еще до времен Грандфанды были первыми распорядительными помощниками Вечных Дней. Есть много оснований полагать, что эти две уникальные группы тринитизированных существ будут трудиться вместе в отдаленном будущем, и не последним из этих оснований является их общее предназначение в резервах Райского отряда тринитизированных финалитов.
\vs p026 11:3 В самом внутреннем контуре восходящие и нисходящие пилигримы общаются между собой и с сынами, тринитизированными созданиями. Подобно их родителям, эти сыны извлекают огромную пользу из взаимного общения, и особая миссия супернафимов заключается в том, чтобы облегчать и обеспечивать установление братских отношений между тринитизированными сынами смертных финалитов и тринитизированными сынами Граждан Рая. Супернафимы\hyp{}дополнители отдыха заняты не столько обучением, сколько содействием взаимопониманию и сближению различных групп существ.
\vs p026 11:4 Смертные получили райское веление: «Будьте совершенны, как и совершенен ваш Райский Отец». Руководящий супернафим не перестает взывать к этим тринитизированным сынам из объединенного отряда: «Относитесь с пониманием к вашим восходящим братьям так же, как и Райские Сыны\hyp{}Творцы понимают и любят их».
\vs p026 11:5 \pc Человеческое создание должно найти Бога. Сын\hyp{}Творец никогда не останавливается, пока не найдет человека --- самое низшее создание, обладающее свободой воли. Без всякого сомнения, Сыны\hyp{}Творцы и их человеческие дети готовятся к некоему будущему и неизвестному служению. И те, и другие проходят через все многообразие опыта опытной вселенной и таким образом обучаются и подготавливаются к своей вечной миссии. Повсюду во вселенных происходит это уникальное слияние человеческого и божественного, соединение творения и Творца. Не думающие люди рассматривали проявление божественного милосердия и доброты, особенно по отношению к слабым и бедствующим, как свидетельство антропоморфности Бога. Какая ошибка! Скорее, такие проявления милосердия и терпимости у людей следует воспринимать как свидетельство того, что в смертном человеке пребывает дух живого Бога; того, что создание, в конечном счете, движимо божественными мотивами.
\vs p026 11:6 \pc Незадолго до конца пребывания в первом круге восходящие пилигримы впервые встречаются с побуждающими к отдыху из чина первичных супернафимов. Это ангелы Рая, выходящие приветствовать тех, кто стоит у порога вечности, и завершить их подготовку к переходному сну перед последним воскресением. Вы не станете настоящими детьми Рая, пока не пересечете внутренний круг и не испытаете воскресение в вечности после заключительного сна во времени. Усовершенствованные пилигримы начинают этот отдых, засыпая в первом круге Хавоны, просыпаются же они на берегах Рая. Из всех взошедших на вечный Остров лишь приходящие таким путем становятся детьми вечности; прочие же приходят как посетители, как гости, без статуса постоянных обитателей.
\vs p026 11:7 И теперь, в кульминационный момент пути через Хавону, когда вы, люди, засыпаете в путеводном мире внутреннего контура, вы засыпаете не одни, как засыпали в мирах, где родились, когда вы закрывали глаза в естественном сне человеческой смерти или когда входили в продолжительный транс, предшествующий пути в Хавону. Теперь, когда вы готовитесь к отдыху, предшествующему достижению Рая, рядом с вами находится ваш долговременный помощник в первом круге, дополнитель отдыха, который готовится впасть в сон как единое целое вместе с вами в знак обещания Хавоны, что путь завершен и вам предстоит добавить лишь последние штрихи совершенства.
\vs p026 11:8 Ваш первый переход действительно был смертью, второй --- идеальным сном, а теперь третья метаморфоза --- это истинный отдых, отдых эпох.
\vsetoff
\vs p026 11:9 [Представлено Совершенствователем Мудрости из Уверсы.]
