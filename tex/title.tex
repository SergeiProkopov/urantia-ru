\vspace*{\stretch{0.1}}
\begin{center}
\bibcovertitlefont\urantiabook\\[3ex]
\huge\tunemarkup{pghanlin}{\Large}\tunemarkup{pgnexus7}{\large} ПЯТОЕ ЭПОХАЛЬНОЕ ОТКРОВЕНИЕ\\[2ex]
\LARGE\tunemarkup{pgnexus7}{\small} ТРЕТЬЕ ИЗДАНИЕ\\
\vspace*{\stretch{0.1}}
\ifmultivol
\Large
\ifvoli ТОМ I: ПРЕДИСЛОВИЕ, ТЕКСТЫ 1--94\\\fi
\ifvolii ТОМ II: ТЕКСТЫ 95--196\\\fi
\fi
\vspace*{\stretch{0.6}}
\titlesepbig\\
\vspace*{\stretch{0.1}}
\end{center}

\titleframe

\newpage

\begin{center}
\vspace*{\stretch{0.3}}
\begin{center}\shadowbox{\strut\parbox{\tunemarkuptwo{pgnexus7}{6cm}{11cm}}{\large\tunemarkup{pgcrownq}{\Large}\tunemarkup{pgafour}{\LARGE}\tunemarkup{pgnexus7}{\normalsize}\bfseries\itshape ``В разуме Бога существует план, объемлющий собой каждое создание всех его необъятных владений, и план этот есть вечный замысел безграничной возможности, неограниченного совершенствования и бесконечной жизни. И эти бесконечные сокровища столь несравненного пути принадлежат вам, стремящимся к ним!'' \bibref[(32:5.7)]{p032 5:7}}}\end{center}
\vspace*{\stretch{0.7}}
\itshape
\tunemarkup{pgafour}{\fontsize{18}{22}\selectfont}
\tunemarkup{pgletter}{\fontsize{18}{22}\selectfont}
\tunemarkup{pgluluhb}{\fontsize{12}{15}\selectfont}
\tunemarkup{pghanlin}{\fontsize{9}{12}\selectfont}
\tunemarkup{pgcrownq}{\fontsize{10}{15}\selectfont}
\tunemarkup{pgveligor}{\fontsize{10}{15}\selectfont}
\tunemarkup{pgnexus7}{\fontsize{8}{10}\selectfont}
Copyright \textcopyright\ The Urantia Book Fellowship, {\upshape\bfseries www.urantiabook.org}.\\
\tux\ Вёрстка {\upshape\bfseries Bibles.org.uk}, используя \TeX\ Live 2023 в системе Linux.\\
Электронная версия данной книги доступна бесплатно на сайте:\\
{\upshape\bfseries http://www.bibles.org.uk/books.html}\\
Исследователь Книги Урантии: {\upshape\bfseries http://urantiaexplorer.org}\\
Телеграм канал <<Урантийцы>>: {\upshape\bfseries http://t.me/urantia\_ru}\\
Символом \pc\ отмечен первый параграф группы.\\
В 1-ом изд. 1955~г. группы параграфов отделяла пустая строка.\\
Текст набран шрифтом \textbf{\urantiamainfont} \urantiamainfontsize pt.\\[4pt]
\upshape\normalsize\tunemarkup{pgnexus7}{\small}\bfseries\mytoday{}\\
\end{center}

\titleframe
