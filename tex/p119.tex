\upaper{119}{Пришествия Христа Михаила}
\author{Глава Вечерних Звезд}
\vs p119 0:1 Я --- глава Вечерних Звезд Небадона и послан Гавриилом на Урантию с поручением открыть историю семи пришествий Владыки вселенной, Михаила Небадонского, а имя мое --- Гавалия. В этом повествовании я буду действовать строго в пределах моих полномочий.
\vs p119 0:2 \pc Совершение пришествий неотъемлемо присуще Райским Сынам Отца Всего Сущего. В желании Райских Сынов различных чинов приблизиться к жизненному опыту подчиненных им живых созданий отражается божественная природа их Райских родителей. Занимает ведущее положение на этом поприще Вечный Сын Райской Троицы, который за период восхождения Грандфанды и первых пилигримов времени и пространства совершил семь пришествий в семь контуров Хавоны. И Вечный Сын продолжает совершать пришествия в локальные вселенные пространства в лице своих представителей --- Сынов\hyp{}Михаилов и Сынов\hyp{}Авоналов.
\vs p119 0:3 Когда Вечный Сын посылает Сына\hyp{}Творца в задуманную локальную вселенную, то этот Сын\hyp{}Творец берет на себя всю ответственность за завершение ее создания, управление и спокойствие этой новой вселенной и дает вечной Троице торжественную клятву не обретать полного владычества над новым творением вплоть до успешного завершения семи его пришествий в облике созданий, каждое из которых будет официально одобрено Древними Дней той сверхвселенной, к юрисдикции которой это творение относится. Это обязательство берет на себя каждый Сын\hyp{}Михаил, который добровольно отправляется из Рая заниматься формированием и созданием вселенных.
\vs p119 0:4 Эти пришествия в облике созданий имеют целью дать таким Творцам возможность стать мудрыми, сострадательными, справедливыми и понимающими владыками. Эти божественные Сыны врожденно справедливы, но сострадательно\hyp{}милосердными они становятся в результате опыта, получаемого во время этой череды пришествий; они милосердны по своей природе, но этот опыт делает их милосердными новым и дополнительным образом. Эти пришествия --- последние этапы в их обучении и подготовке к грандиозным задачам управления локальными вселенными божественно\hyp{}праведно и беспристрастно\hyp{}справедливо.
\vs p119 0:5 Хотя различные миры, системы и созвездия, равно как и разные чины вселенских разумных существ, которых эти пришествия затрагивают и на которых оказывают благотворное воздействие, получают многочисленные косвенные преимущества, тем не менее цель пришествия --- прежде всего, завершение личной подготовки и вселенского обучения самого Сына\hyp{}Творца. Эти пришествия не столь существенны для мудрого, справедливого и эффективного управления локальной вселенной, но они совершенно необходимы для беспристрастного, милосердного и чуткого руководства таким творением, наполненным разнообразными формами жизни и мириадами разумных, но несовершенных созданий.
\vs p119 0:6 Сыны\hyp{}Михаилы начинают свою деятельность по организации вселенной, испытывая полную и справедливую симпатию к созданным ими существам разных чинов. Они преисполнены милосердия ко всем этим разнообразным созданиям и даже жалости к тем, кто ошибается и вязнет в созданной ими самими трясине эгоизма. Но такого дара справедливости и праведности не достаточно, с точки зрения Древних Дней. Эти триединые правители сверхвселенных никогда не утвердят Сына\hyp{}Творца Владыкой вселенной, пока он в результате подлинного опыта, обретенного в среде обитания своих собственных созданий действительно не встанет на их точку зрения, воплотившись в сами эти создания. Таким образом эти Сыны становятся умными и понимающими правителями; они обретают \bibemph{знание} различных групп, которыми управляют и над которыми имеют вселенскую власть. Через живой опыт они овладевают практическим милосердием, справедливым суждением и терпением, которое рождается из опыта существования в качестве создания.
\vs p119 0:7 \pc Локальная вселенная Небадон управляется сейчас Сыном\hyp{}Творцом, завершившим все свои пришествия; он справедливо и милосердно и верховно правит над всеми обширными сферами своей эволюционирующей и совершенствующейся вселенной. Михаил Небадонский --- это 611\,121\hyp{}е пришествие Вечного Сына во вселенные времени и пространства, и он начал формировать вашу локальную вселенную примерно четыреста миллиардов лет назад. Михаил приготовился совершить свое первое пришествие примерно в то время, когда Урантия обретала свою сегодняшнюю форму, миллиард лет назад. Его пришествия происходили с интервалом примерно в сто пятьдесят миллионов лет, и последнее из них на Урантию произошло тысяча девятьсот лет назад. Сейчас я перехожу к откровению природы и характера этих пришествий с той мерой полноты, какую допускают мои полномочия.
\usection{1. Первое пришествие}
\vs p119 1:1 Почти миллиард лет назад в Спасограде произошло важное событие, когда Михаил возвестил собравшимся руководителям и главам вселенной Небадон, что его старший брат Иммануил вскоре примет на себя власть в Небадоне, пока он (Михаил) будет отсутствовать по некоему делу, суть которого он не объяснил. Об этом деле больше ничего не было объявлено кроме того, что в прощальном послании, переданном Отцам Созвездия, наряду с прочими указаниями говорилось: «И на этот период я вверяю вас попечению и заботам Иммануила, пока я отправлюсь исполнять веление моего Райского Отца».
\vs p119 1:2 Послав это прощальное сообщение, Михаил появился на поле отбытия в Спасограде точно так же, как и много раз до этого, когда он готовился отправиться в Уверсу или в Рай, но с той лишь разницей, что он пришел один. Свою речь при отбытии он закончил такими словами: «Я покидаю вас лишь на короткое время. Многие из вас, я знаю, отправились бы со мной, но туда, куда я отправляюсь, вы пойти не можете. И то, что я собираюсь сделать, вы сделать не можете. Я отправляюсь исполнять волю Райских Божеств, а когда исполню свою миссию и обрету этот опыт, то вернусь и займу свое место среди вас». И сказав это, Михаил Небадонский скрылся из виду всех собравшихся и не появлялся на протяжении двадцати лет стандартного времени. Во всем Спасограде только Божественная Служительница и Иммануил знали, что происходит, и Объединяющий Дней поделился этой тайной только с главным распорядителем вселенной Гавриилом, Яркой и Утренней Звездой.
\vs p119 1:3 Все обитатели Спасограда и жители центральных миров созвездий и систем собрались вокруг своих станций приема вселенской информации в надежде получить какое\hyp{}нибудь известие о миссии и местонахождении Сына\hyp{}Творца. Два дня со времени отбытия Михаила не было никаких сколько\hyp{}нибудь важных известий. На третий день в Спасограде получили сообщение из сферы Мелхиседеков, из центра этого чина в Небадоне, в котором просто фиксировалось это дотоле неслыханное событие: «Сегодня в полдень на поле прибытия этого мира появился незнакомый Сын\hyp{}Мелхиседек не из нашего числа, но полностью подобный нашему чину. Его сопровождал единственный омниафим, который имел при себе мандат из Уверсы и передал адресованные нашему главе распоряжения, отданные Древними Дней при согласии Иммануила из Спасограда и предписывающие, чтобы этот новый Сын\hyp{}Мелхиседек был принят в наш чин и зачислен в спасительную службу Мелхиседеков Небадона. И так было предписано; и так было сделано.»
\vs p119 1:4 И это почти все, что появилось в записях Спасограда относительно первого пришествия Михаила. Новое известие появилось лишь через сто лет по урантийскому времени, когда был зарегистрирован факт, что Михаил вернулся и без предварительного уведомления продолжил руководство делами вселенной. Но в мире Мелхиседеков обнаруживается странная запись --- рассказ о служении в ту эру этого необыкновенного Сына\hyp{}Мелхиседека в спасательном отряде. Эта запись хранится в простом храме, который сейчас находится перед домом Отца\hyp{}Мелхиседека, и она содержит повествование о служении этого временного Сына\hyp{}Мелхиседека, связанном с его назначением в двадцать четыре вселенские спасательные миссии. И эта запись, которую я совсем недавно просмотрел, кончается так:
\vs p119 1:5 «И в полдень в этот день без предварительного уведомления этот пришедший Сын нашего чина исчез из нашего мира так же, как и появился, --- в сопровождении лишь единственного омниафима, чему свидетелями были лишь трое из нашего братства; и эта запись сейчас удостоверяет, что этот пришелец жил как Мелхиседек, в облике Мелхиседека, трудился как Мелхиседек и добросовестно выполнял все свои задания в качестве чрезвычайного Сына нашего чина. Со всеобщего согласия он стал главой Мелхиседеков, заслужив нашу любовь и восхищение благодаря своей несравненной мудрости, величайшей любви и возвышенной преданности долгу. Он любил нас, понимал нас и служил вместе с нами, и мы навсегда стали его преданными и любящими собратьями\hyp{}Мелхиседеками, ибо этот пришелец в наш мир теперь навеки стал вселенским служителем, имеющим природу Мелхиседека».
\vs p119 1:6 И это все, что мне позволено сообщить вам о первом пришествии Михаила. Мы, конечно, вполне понимаем, что этот необычный Мелхиседек, так таинственно служивший вместе с Мелхиседеками миллиард лет назад, был не кто иной, как воплощенный Михаил, исполнявший миссию своего первого пришествия. В записях конкретно не утверждается, что этот необыкновенный и эффективно действовавший Мелхиседек был Михаилом, но все верят, что это был именно он. Вероятно, точную формулировку этого события можно обнаружить только лишь в записях Сынограда, а записи этого тайного мира закрыты для нас. Только в этом священном мире божественных Сынов полностью известны тайны воплощений и пришествий. Все мы знаем о самих фактах пришествий Михаила, но не понимаем, как они осуществляются. Мы не знаем, как правитель вселенной, творец Мелхиседеков может таким внезапным и таинственным образом стать вдруг одним из них и в качестве такового в течение ста лет жить среди них и трудиться как Сын\hyp{}Мелхиседек. Но это было именно так.
\usection{2. Второе пришествие}
\vs p119 2:1 Во вселенной Небадон все шло хорошо на протяжении почти ста пятидесяти миллионов лет со времени пришествия Михаила в облике Мелхиседека, как вдруг над системой 11 созвездия 37 нависла беда. Осложнения были вызваны неверным пониманием со стороны Сына\hyp{}Ланонандека, Владыки Системы; Отцы созвездия вынесли по этому делу решение, которое было одобрено Верным Дней, Райским советником этого созвездия, но протестующий Владыка Системы не совсем примирился с вынесенным вердиктом. После более чем столетнего недовольства он привел своих сподвижников к одному из самых широкомасштабных и катастрофичных бунтов против владычества Сына\hyp{}Творца, когда\hyp{}либо разжигавшихся во вселенной Небадон, по этому бунту давно вынесено решение и ему положен конец действием Древних Дней в Уверсе.
\vs p119 2:2 Лютенция, мятежный владыка системы, безраздельно властвовал на своей центральной планете более двадцати лет стандартного небадонского времени; после чего Всевышние с одобрения Уверсы отдали приказ о его изоляции и затребовали от правителей Спасограда назначить нового Владыку Системы, который принял бы на себя руководство этой раздираемой распрями и приведенной в смятение системой обитаемых миров.
\vs p119 2:3 \pc Одновременно с получением в Спасограде этого требования Михаил сделал второе из этих необыкновенных заявлений о намерении отбыть из центра вселенной с целью «исполнить веление моего Райского Отца», пообещав «вернуться в должное время» и сосредоточив всю власть в руках своего Райского брата Иммануила, Объединяющего Дней.
\vs p119 2:4 И затем точно так же, как и во время своего отбытия, связанного с пришествием в облике Мелхиседека, Михаил снова удалился со своей центральной планеты. Через три дня после этого необъясненного ухода в резервном отряде первичных Сынов\hyp{}Ланонандеков Небадона появился его новый и неизвестный член. Этот новый Сын появился в полдень без предварительного уведомления и в сопровождении единственного терциафима, который имел при себе мандат от Древних Дней Уверсы, подтвержденный Иммануилом из Спасограда и предписывающий, чтобы этого нового Сына назначили в систему номер 11 созвездия 37 преемником смещенного Лютенции и облекли полной властью временно исполняющего обязанности Владыки Системы впредь до назначения нового владыки.
\vs p119 2:5 Более семнадцати лет вселенского времени этот необычный и неизвестный временный правитель управлял делами и мудро выносил решения по поводу разногласий в этой пришедшей в смятение и деморализованной локальной системе. Ни один Владыка Системы никогда не пользовался такой горячей любовью, таким повсеместным почитанием и уважением. Этот новый правитель справедливо и милосердно навел порядок в неспокойной системе, усердно служа всем своим подданным, и даже предложил своему мятежному предшественнику честь разделить с ним престол власти над системой, если только он принесет извинения Иммануилу за свои опрометчивые поступки. Но Лютенция отверг эти милосердные шаги, хорошо зная, что этим новым и необычным Владыкой Системы был не кто иной, как Михаил, сам правитель вселенной, которому он совсем недавно отказывался повиноваться. Но миллионы его сбитых с толку и обманутых последователей приняли прощение от этого нового правителя, известного в ту пору как Владыка\hyp{}Спаситель системы Палония.
\vs p119 2:6 \pc А затем наступил знаменательный день, когда прибыл новоназначенный Владыка Системы, утвержденный властями вселенной в качестве постоянного преемника отстраненного Лютенции, и вся Палония сокрушалась по поводу ухода самого благородного и самого милостивого правителя, какого когда\hyp{}либо знал Небадон. Он был любим всей системой, и его обожали собратья из всех групп Сынов\hyp{}Ланонандеков. Его отбытие не прошло без церемоний; когда он удалялся из центра системы, было устроено огромное празднество. Даже его сбившийся с пути истинного предшественник послал такое сообщение: «Справедлив и праведен ты во всех твоих деяниях. По\hyp{}прежнему не принимая Райское правление, я вынужден признать, что ты справедливый и милосердный руководитель».
\vs p119 2:7 И затем этот временный правитель мятежной системы отбыл с планеты, которой недолгое время руководил, а на третий день после этого Михаил появился в Спасограде и вернулся к управлению вселенной Небадон. Вскоре последовало третье объявление Уверсы о расширении власти и полномочий Михаила. Первое объявление было сделано во время его прибытия в Небадон, второе --- вскоре после завершения его пришествия в облике Мелхиседека, а теперь третье последовало за окончанием второго пришествия --- в облике Ланонандека.
\usection{3. Третье пришествие}
\vs p119 3:1 Верховный совет в Спасограде только что закончил рассмотрение призыва Носителей Жизни планеты 217 в системе 87 созвездия 61 об отправке им на помощь Материального Сына. Эта планета находилась в системе населенных миров, где другой Владыка Системы сбился с истинного пути, и к тому времени это был второй подобный бунт во всем Небадоне.
\vs p119 3:2 По просьбе Михаила действия в связи с прошением от Носителей Жизни этой планеты были отложены до времени рассмотрения этого прошения Иммануилом и его доклада по этому поводу. Это был непривычный образ действий, и я хорошо помню, как мы все предчувствовали что\hyp{}то необычное, и нас недолго держали в состоянии напряженного ожидания. Далее Михаил передал руководство вселенной Иммануилу, а командование небесными силами поручил Гавриилу и, сложив с себя таким образом обязанности по управлению, попрощался с Духом\hyp{}Матерью Вселенной и исчез с поля отбытия Спасограда точно так же, как делал это в двух предыдущих случаях.
\vs p119 3:3 И, как и можно было ожидать, на третий день после этого в центральном мире системы 87 созвездия 61 появился необычный Материальный Сын в сопровождении единственного секонафима, который имел полномочия от Древних Дней Уверсы, подтвержденные Иммануилом из Спасограда. Исполняющий обязанности Владыки Системы немедленно назначил этого нового и таинственного Материального Сына исполняющим обязанности Планетарного Принца мира 217, и это назначение было сразу же утверждено Всевышними созвездия 61.
\vs p119 3:4 Таким образом этот необыкновенный Материальный Сын начал свой трудный путь в подвергнутом изоляции мире раскола и мятежа, находящемся в осажденной системе, не имеющей никакой прямой связи с остальной вселенной, функционирующей в одиночестве на протяжении одного поколения планетарного времени. Этот Материальный Сын\hyp{}спасатель добился, чтобы совершивший проступок Планетарный Принц и весь его штат покаялись и исправились, и стал свидетелем возвращения планеты к преданному служению Райской власти, установленной в локальных вселенных. В соответствующее время в этот восстановленный и исправленный мир прибыли Материальные Сын и Дочь, и когда они были должным образом введены в должность зримых планетарных правителей, временный и выполнявший роль спасателя Планетарный Принц в один прекрасный день официально попрощался и в полдень исчез. На третий день после этого Михаил появился в своем привычном месте в Спасограде и очень скоро в сверхвселенной были переданы сообщения, в которых Древние Дней в четвертый раз объявили о расширении владычества Михаила в Небадоне.
\vs p119 3:5 Очень жаль, что у меня нет разрешения рассказывать о терпении, стойкости и искусности, которые этот Материальный Сын проявлял в трудных ситуациях на этой сбитой с толку планете. Исправление этого изолированного мира --- одна из самых прекрасных и трогательных глав в летописи спасения во всем Небадоне. К концу этой миссии для всего Небадона стало очевидно, почему их любимый правитель решал совершить эти повторяющиеся пришествия в облике разумного существа более низкого плана.
\vs p119 3:6 \pc Пришествия Михаила как Сына\hyp{}Мелхиседека, потом как Сына\hyp{}Ланонандека и затем как Материального Сына --- все в равной степени загадочны и необъяснимы. В каждом случае он появлялся \bibemph{внезапно} и в виде полностью развившегося индивидуума соответствующей группы. Тайна таких воплощений никогда не станет известна никому, кроме имеющих доступ к внутреннему кругу записей в священной сфере Сынограда.
\vs p119 3:7 \pc Никогда со времени этого удивительного пришествия в облике Планетарного Принца мира, пребывавшего в изоляции и бунте, ни один Материальный Сын или Дочь в Небадоне не впадали в искушение пожаловаться на свои назначения или посетовать на трудности своих планетарных миссий. Во все времена Материальные Сыны знают, что в лице Сына\hyp{}Творца вселенной они имеют понимающего владыку и участливого друга, во «всех отношениях проверенного и испытанного», точно так же, как и они должны быть проверены и испытаны.
\vs p119 3:8 За каждой из этих миссий следовал период приумножения служения и верности всех небесных разумных существ вселенского происхождения, а каждый последующий период пришествия характеризовался прогрессом и совершенствованием всех методов управления вселенной и всех способов правления. Со времени этого пришествия ни один Материальный Сын или Дочь не принял намеренно участия в мятеже против Михаила; они слишком сильно любят и чтят его, чтобы когда\hyp{}либо сознательно его отвергнуть. Только путем обмана и ухищрений мятежные личности более высоких типов сбили с пути истинного Адамов более поздних времен.
\usection{4. Четвертое пришествие}
\vs p119 4:1 В конце одной из поверок, регулярно раз в тысячелетие происходящих в Уверсе, Михаил передал правление Небадоном в руки Иммануила и Гавриила; и конечно же помня, что случалось после таких действий, все мы приготовились стать свидетелями исчезновения Михаила и ухода его в четвертое пришествие, и нам не пришлось долго ждать, ибо вскоре он отправился на поле отбытия и мы потеряли его из виду.
\vs p119 4:2 На третий день после этого исчезновения мы обнаружили во вселенских сообщениях Уверсы такую важную новость из центра серафимов Небадона: «Сообщаем о не объявленном заранее прибытии неизвестного серафима в сопровождении единственного супернафима и Гавриила из Спасограда. Этот незарегистрированный серафим правомочен войти в чин серафимов Небадона и имеет при себе мандат от Древних Дней Уверсы, подтвержденный Иммануилом из Спасограда. Этот серафим был проверен и выяснилось, что он принадлежит к верховному чину ангелов локальной вселенной и уже назначен в отряд обучающих советников».
\vs p119 4:3 Во время этого пришествия в качестве серафима Михаил отсутствовал в Спасограде более сорока стандартных вселенских лет. В течение этого времени он назначался обучающим советником\hyp{}серафимом --- вы, возможно, назвали бы это личным секретарем --- к двадцати шести разным учителям\hyp{}мастерам, действовавшим в двадцати двух разных мирах. Последним и завершающим было его назначение на должность советника и помощника при миссии пришествия Сына\hyp{}Учителя Троицы в мире номер 462 системы 84 созвездия 3 вселенной Небадон.
\vs p119 4:4 Никогда на протяжении всех семи лет, пока выполнялось это задание, этот Сын\hyp{}Учитель Троицы не имел полной уверенности относительно личности своего сподвижника\hyp{}серафима. Правда, в течение этого периода ко всем серафимам проявлялся особый интерес и пристальное внимание. Все мы отлично знали, что наш любимый владыка находится где\hyp{}то во вселенной в облике серафима, но мы никогда не могли с уверенностью идентифицировать его. Он ни разу не был с определенностью идентифицирован до того времени, когда его прикомандировали к миссии пришествия этого Сына\hyp{}Учителя Троицы. Но постоянно на протяжении этой эпохи к верховным серафимам относились с особым вниманием, чтобы кто\hyp{}то из нас вдруг не обнаружил, что, сам того не подозревая, принимал в качестве гостя Владыку вселенной в ходе его миссии пришествия в облике создания. Итак, для ангелов навеки стало истиной, что их Создатель и Правитель «во всех отношениях проверен и испытан в облике ангельской личности».
\vs p119 4:5 По мере того, как одно за другим совершались эти пришествия в виде все более низких форм вселенской жизни, Гавриил все больше становился сподвижником этих воплощений, выступая в роли вселенского связующего звена между совершающим пришествие Михаилом и исполняющим обязанности правителя вселенной Иммануилом.
\vs p119 4:6 \pc Теперь Михаил прошел через опыт пришествий в облике трех типов своих сотворенных вселенских Сынов: Мелхиседеков, Ланонандеков и Материальных Сынов. Затем он соизволил принять ангельский облик в качестве верховного серафима прежде, чем сосредоточить внимание на различных фазах пути восхождения своих обладающих волей творений низшей формы, эволюционных смертных времени и пространства.
\usection{5. Пятое пришествие}
\vs p119 5:1 Чуть более трехсот миллионов лет назад по летоисчислению, принятому на Урантии, мы стали свидетелями еще одной из этих передач власти над вселенной Иммануилу и наблюдали приготовления Михаила к отбытию. Этот случай отличался от предыдущих тем, что он объявил, что отправляется в Уверсу, центр сверхвселенной Орвонтон. В соответствующее время наш Владыка отбыл, но в сообщениях сверхвселенной ни разу не упоминалось о прибытии Михаила ко двору Древних Дней. Вскоре после его отбытия из Спасограда в сообщениях из Уверсы появилось такое важное известие: «Сегодня прибыл без предварительного уведомления не имеющий номера восходящий пилигрим, происходящий из смертных из вселенной Небадон с полномочиями от Иммануила из Спасограда и в сопровождении Гавриила из Небадона. Это неидентифицированное существо проявляет качества истинного духа и принято в наше братство».
\vs p119 5:2 Если бы вы посетили Уверсу сегодня, то услышали бы рассказ о тех днях, когда там пребывал Эвентод, этот особенный и неведомый пилигрим времени и пространства, известный в Уверсе под таким именем. И этот восходящий смертный, по меньшей мере, величественная личность, точно соответствующая духовной стадии восходящих смертных, жил и действовал в Уверсе на протяжении одиннадцати лет стандартного времени Орвонтона. Это существо получало задания и выполняло обязанности духовного смертного вместе со своими собратьями из разных локальных вселенных Орвонтона. Во «всех отношениях он был проверен и испытан, равно как и его товарищи», и во всех случаях он оказался достойным доверия своих руководителей, в то же время неизменно вызывал уважение и преданное восхищение своих духовных собратьев.
\vs p119 5:3 В Спасограде мы с величайшим интересом следили за судьбой этого духовного пилигрима, отлично понимая по присутствию Гавриила, что этим скромным и не имеющим номера духовным пилигримом был не кто иной, как совершающий пришествие правитель нашей локальной вселенной. Это первое появление Михаила воплощенным в существо одной из стадий человеческой эволюции явилось событием, взволновавшим и приковавшим внимание всего Небадона. Раньше мы слышали о подобных вещах, но теперь мы их видели. Он появился на Уверсе в облике полностью развившегося и прекрасно подготовленного духовного смертного и в качестве такового продолжал свой жизненный путь вплоть до продвижения группы восходящих смертных в Хавону; в это время он имел беседу с Древними Дней и сразу же в сопровождении Гавриила внезапно и без официальных церемоний отбыл из Уверсы, появившись вскоре после этого в своем обычном месте в Спасограде.
\vs p119 5:4 \pc Лишь после завершения этого пришествия нас, наконец, осенило, что Михаил, вероятно, собирается воплощаться в разные чины вселенских личностей, от самых высших --- Мелхиседеков и вплоть до низших --- смертных из плоти и крови пространственно\hyp{}временных эволюционных миров. Примерно в это время в учебных заведениях Мелхиседеков стали учить, что когда\hyp{}нибудь вероятно воплощение Михаила в смертного во плоти, и было много предположений о возможном методе такого необъяснимого пришествия. То, что Михаил лично выступил в роли восходящего смертного, вызвало у всех новый и дополнительный интерес к схеме движения творения вперед на протяжении всего пути как по локальной вселенной, так и по сверхвселенной.
\vs p119 5:5 И все\hyp{}таки метод этих следовавших одно за другим пришествий оставался тайной. Даже Гавриил признавался, что не понимает, каким образом этот Райский Сын и Творец вселенной мог по своему желанию обретать личность одного из своих собственных подчиненных ему творений и жить его жизнью.
\usection{6. Шестое пришествие}
\vs p119 6:1 Теперь, когда весь Спасоград был знаком с тем, что обычно предшествует близящемуся пришествию, Михаил созвал тех, кто присутствовал в то время на планете, и впервые раскрыл остальные планы воплощений, объявив, что скоро ему предстоит покинуть Спасоград, чтобы ступить на путь моронтийного смертного при дворе Всевышних Отцов на центральной планете пятого созвездия. И тогда мы впервые услышали сообщение о том, что его седьмое и последнее пришествие будет совершено в какой\hyp{}нибудь эволюционный мир в облике человека и в подобии смертной плоти.
\vs p119 6:2 Прежде, чем покинуть Спасоград и отправиться в шестое пришествие, Михаил обратился к собравшимся обитателям сферы с речью и отбыл на виду у всех в сопровождении единственного серафима и Яркой Утренней Звезды Небадона. Хотя управление вселенной снова было вверено Иммануилу, но были шире распределены административные обязанности.
\vs p119 6:3 Михаил появился в центре созвездия номер пять в облике полностью развившегося моронтийного человека восходящего статуса. Жаль, что мне запрещено раскрывать подробности жизненного пути этого не имеющего номера моронтийного смертного, ибо это была одна из самых необыкновенных и изумительных эпох в опыте Михаила, связанном с пришествиями, не исключая даже его драматического и трагического пребывания на Урантии. Но к числу многих ограничений, которые наложили на меня, давая это поручение, относится и запрет браться за раскрытие подробностей этой удивительной жизни Михаила в качестве моронтийного смертного Эндантума.
\vs p119 6:4 Когда Михаил вернулся из этого моронтийного пришествия, всем нам было очевидно, что наш Творец стал собратом своих созданий, что Владыка Вселенной является также другом и сочувствующим помощником сотворенных разумных существ даже самой низшей формы, обитающих в его владениях. Мы и до этого замечали это растущее понимание точки зрения созданий при руководстве вселенной, ибо оно проявлялось постепенно, но стало более явным после завершения пришествия в облике моронтийного смертного и еще более очевидным после возвращения Михаила по завершении жизненного пути сына плотника на Урантии.
\vs p119 6:5 Мы были заранее проинформированы Гавриилом о времени окончания пришествия Михаила в моронтийной форме и, соответственно, устроили подобающий прием в Спасограде. Миллионы и миллионы существ из центральных миров созвездий Небадона и большинство обитателей соседних со Спасоградом миров собрались все вместе, чтобы приветствовать его возвращение к управлению его вселенной. В ответ на наши многочисленные приветственные обращения и выражения признательности Владыке, который так живо интересуется своими созданиями, он лишь ответил: «Я просто был в том, что принадлежит моему Отцу. Я только выполняю приятную обязанность Райских Сынов, любящих свои создания и жаждущих их понять».
\vs p119 6:6 Но с того дня и вплоть до часа, когда Михаил отправился в свое пришествие на Урантию в качестве Сына Человеческого, весь Небадон продолжал обсуждать многочисленные подвиги, совершенные их Верховным Правителем в период деятельности на Эндантуме, когда он был воплощен в моронтийного смертного эволюционного восхождения и прошел всевозможные испытания, как и его собратья, собравшиеся из материальных миров всего созвездия, в котором он пребывал.
\usection{7. Седьмое и последнее пришествие}
\vs p119 7:1 Десятки тысяч лет все мы с нетерпением ждали седьмого и последнего пришествия Михаила. Гавриил учил нас, что его завершающее пришествие будет в облике смертного во плоти, но мы совершенно не были осведомлены о времени, месте и характере этого кульминационного события.
\vs p119 7:2 О том, что Михаил избрал местом своего последнего пришествия Урантию, было публично объявлено вскоре после того, как мы узнали о срыве Адама и Евы. И, таким образом, в течение более тридцати пяти тысяч лет ваш мир занимал очень важное место в советах всей вселенной. Ни один шаг в ходе пришествия на Урантию не был окутан завесой секретности (если не считать тайны воплощения). От начала и до конца, вплоть до завершающего и триумфального возвращения Михаила в Спасоград в качестве верховного Владыки Вселенной, во вселенной была полнейшая гласность относительно всего, что происходило в вашем маленьком, но высоко чтимом мире.
\vs p119 7:3 \pc Хотя мы и верили, что все будет происходить именно таким образом, но пока не произошло само это событие, мы не знали, что Михаил появится на земле в облике беспомощного младенца этой планеты. До сих пор он всегда появлялся в виде полностью развившегося индивидуума той группы существ, которая была избрана им для пришествия, и всех взволновало переданное из Спасограда сообщение о том, что на Урантии родилось дитя из Вифлеема.
\vs p119 7:4 Тогда мы не только осознали, что наш Творец и товарищ предпринял самый опасный шаг за все время своей деятельности, явно рискуя своим положением и властью, начав это пришествие в виде беспомощного дитя, но и поняли, что опыт этого завершающего пришествия в облике человека навеки возведет его на престол в качестве бесспорного и верховного владыки вселенной Небадон. На протяжении трети столетия земного времени все взоры во всех частях этой локальной вселенной были прикованы к Урантии. Все разумы понимали, что происходит последнее пришествие, и поскольку мы давно знали о бунте Люцифера в Сатании и о нелояльности Калигастии на Урантии, то хорошо понимали, какая напряженная борьба начнется, когда наш правитель соизволит воплотиться на Урантии в скромном образе и подобии человеческом.
\vs p119 7:5 Иешуа бен Иосеф, еврейский младенец, был зачат и рожден на свет точно так же, как все прочие дети до и после него, \bibemph{за исключением} того, что этот конкретный ребенок был воплощением Михаила из Небадона, божественного Райского Сына и творца всей этой локальной вселенной и всего, что есть в ней. И эта тайна воплощения Божества в человеческую форму Иисуса, который в этом мире в прочих отношениях был естественного происхождения, навеки останется неразгаданной. Даже в вечности вы никогда не узнаете технику и метод воплощения Творца в форму и подобие своих созданий. Это секрет Сынограда, и знанием таких тайн обладают исключительно лишь те божественные Сыны, которые прошли через опыт пришествий.
\vs p119 7:6 Некоторые посвященные люди земли знали о близящемся прибытии Михаила. Через контакты одного мира с другим эти мудрецы, обладающие способностью к духовному озарению, узнали о предстоящем пришествии Михаила на Урантию. И серафимы через срединников возвестили об этом группе халдейских жрецов во главе с Ардноном. Эти божьи люди навестили новорожденного ребенка. Единственным сверхъестественным событием, связанным с рождением Иисуса, было это сообщение, переданное Арднону и его товарищам серафимом, который ранее был прикреплен к Адаму и Еве в первом саду.
\vs p119 7:7 Человеческие родители Иисуса были обычными людьми своего времени и поколения, и этот воплощенный Сын Бога был, таким образом, рожден женщиной и выращен обычным для детей той расы и периода образом.
\vs p119 7:8 \pc История пребывания Михаила на Урантии, повествование о пришествии Сына\hyp{}Творца в ваш мир в облике смертного выходит за рамки темы и цели данного повествования.
\usection{8. Статус Михаила после пришествия}
\vs p119 8:1 После завершающего и успешного пришествия Михаила на Урантию не только Древние Дней признали его полновластным правителем Небадона, но и Отец Всего Сущего утвердил его в качестве руководителя им самим созданной локальной вселенной. По возвращении в Спасоград этот Михаил, Сын Человеческий и Сын Божий, был провозглашен постоянным правителем Небадона. Из Уверсы пришло восьмое объявление о владычестве Михаила, а из Рая поступило совместное заявление Отца Всего Сущего и Вечного Сына, утверждающее это двуединство Бога и человека единственным главой вселенной и предписывающее Объединяющему Дней, находящемуся в Спасограде, выразить его намерение удалиться в Рай. Верным Дней, находящимся в центре созвездия, также было дано указание удалиться из советов Всевышних. Но Михаил не согласился на уход Сынов Троицы, осуществляющих совет и сотрудничество. Он собрал их в Спасограде и лично попросил их навечно остаться исполнять свои обязанности в Небадоне. Они выразили своим руководителям в Раю желание исполнить эту просьбу, и вскоре после этого были отданы те распоряжения об отделении от Рая, которые навсегда назначили этих Сынов центральной вселенной ко двору Михаила, владыки Небадона.
\vs p119 8:2 \pc Потребовался почти миллиард лет урантийского времени, чтобы завершить путь пришествий Михаила и окончательно утвердить его верховную власть в им самим созданной вселенной. Михаил был рожден творцом, получил управленческое образование, был подготовлен для осуществления исполнительной власти, но от него требовалось заслужить свое владычество путем обретения опыта. И, таким образом, ваш маленький мир стал известен во всем Небадоне как место, где Михаил завершил обретение опыта, необходимого каждому Райскому Сыну\hyp{}Творцу прежде, чем ему будет доверено руководство и дан неограниченный контроль над созданной им самим вселенной. По мере восхождения в локальной вселенной вы будете больше узнавать об идеалах тех личностей, которые были связаны с предыдущими пришествиями Михаила.
\vs p119 8:3 \pc Совершая свои пришествия в облике созданий, Михаил не только устанавливал свое собственное владычество, но и усугублял эволюционирующее владычество Бога Верховного. В ходе этих пришествий Сын\hyp{}Творец не только занимался исследованием сотворенных личностей различной природы от высших до низших, но и постиг откровение разнообразной воли Райских Божеств, синтетическое единство которой, открываемое Верховными Творцами, является откровением воли Верховного Существа.
\vs p119 8:4 Эти различные аспекты воли Божеств вечно персонализированы в разных природах Семи Духов\hyp{}Мастеров, и каждое из пришествий Михаила конкретно раскрывало одно из этих проявлений божественности. Во время пришествия в качестве Мелхиседека он продемонстрировал объединенную волю Отца, Сына и Духа, во время пришествия в качестве Ланонандека --- волю Отца и Сына; в адамическом пришествии открыл волю Отца и Духа, в серафическом пришествии --- волю Сына и Духа; в человеческом пришествии в Уверсу представил волю Носителя Объединенных Действий, в моронтийно\hyp{}человеческом пришествии --- волю Вечного Сына; а в материальном пришествии на Урантию он жил как смертный из плоти и крови в соответствии с волей Отца Всего Сущего.
\vs p119 8:5 Результатом завершения этих семи пришествий стало суверенное верховное владычество Михаила, а также создание возможности владычества Верховного в Небадоне. Ни в одном из своих пришествий он не открыл Бога Верховного, но общий итог всех семи пришествий --- это новое откровение Верховного Существа в Небадоне.
\vs p119 8:6 Опыту нисхождения Михаила от Бога к человеку сопутствовал опыт восхождения от частичности проявления к верховенству конечного действия и законченности освобождения его потенциала для абсонитного действия. Михаил как Сын\hyp{}Творец является творцом во времени и пространстве, но Михаил как семеричный Сын\hyp{}Мастер является членом одного из божественных отрядов, составляющих Предельную Троицу.
\vs p119 8:7 Испытав опыт откровения воли Семи Духов\hyp{}Мастеров Троицы, Сын\hyp{}Творец прошел через опыт откровения воли Верховного. Действуя как раскрыватель воли Верховенства, Михаил вместе со всеми прочими Сынами\hyp{}Мастерами навеки отождествил себя с Верховным. В этот вселенский период он раскрывает Верховного и участвует в осуществлении владычества Верховенства. Но мы верим, что в следующем вселенском периоде он будет сотрудничать с Верховным Существом в первой опытной Троице во вселенных внешнего пространства и ради них.
\vs p119 8:8 \pc Урантия --- это почитаемая святыня всего Небадона, это главный из десяти миллионов обитаемых миров, это человеческий дом Христа\hyp{}Михаила, владыки всего Небадона, Мелхиседека\hyp{}служителя сфер, спасителя системы, адамического избавителя, собрата серафимов, сподвижника восходящих духов, моронтийного прогрессора, Сына Человеческого в подобии смертной плоти и Планетарного Принца Урантии. И ваши предания говорят правду, сообщая, что этот самый Иисус обещал когда\hyp{}нибудь вернуться в мир своего последнего пришествия, в Мир Креста.
\separatorline
\vsetoff
\vs p119 8:9 [Этот текст, описывающий семь пришествий Христа Михаила --- шестьдесят третий в серии повествований, представленных многочисленными личностями, излагавшими историю Урантии до времени появления на земле Михаила в подобии смертной плоти. Эти тексты были одобрены небадонской комиссией двенадцати под руководством Мантутии Мелхиседека. Мы записали эти повествования и изложили их на английском языке, пользуясь методикой, одобренной нашими руководителями, в 1935 году н.э. по урантийскому летоисчислению.]
