\upaper{94}{Учение Мелхиседека на Востоке}
\author{Мелхиседек}
\vs p094 0:1 Древние учителя салимской религии проникли в самые отдаленные племена Африки и Евразии, проповедуя благую весть Махивенты о вере и доверии человека единому всеобщему Богу, которые являются единственной платой за обретение божественного благоволения. Завет Мелхиседека Аврааму был образцом для всей первоначальной пропаганды, которая исходила из Салима и других центров. Ни у одной религии на Урантии никогда не было более воодушевленных и энергичных миссионеров, чем эти благородные мужчины и женщины, которые несли учение Мелхиседека по всему Восточному полушарию. Эти миссионеры были призваны из многих народов и рас, и они распространяли свое учение через посредство обращенных в веру местных жителей. Они основали центры обучения в различных частях мира, где они проповедовали салимскую религию местным жителям, а затем посылали этих своих учеников работать в качестве учителей среди своего собственного народа.
\usection{1. Салимское учение в ведической Индии}
\vs p094 1:1 Во дни Мелхиседека Индия была космополитической страной, в которой недавно установилось политическое и религиозное господство арийско\hyp{}андитских завоевателей с севера и запада. К этому времени арийцы овладели только северной и западной частями полуострова. Эти ведические пришельцы принесли с собой множество своих племенных богов. Их религиозные церемонии богослужения во многом следовали обрядовой практике их древних андитских предков: отец все еще исполнял роль жреца, мать --- жрицы, а домашний очаг использовался в качестве алтаря.
\vs p094 1:2 Ведический культ развивался и изменялся под руководством касты брахманов --- жрецов\hyp{}учителей, которые постепенно брали под свой контроль расширяющийся ритуал богослужения. Когда салимские миссионеры проникли на север Индии, процесс объединения прежних тридцати трех арийских богов шел полным ходом.
\vs p094 1:3 Политеизм этих арийцев отражал деградацию их более раннего монотеизма, обусловленную их разделением на племенные группы, где каждое племя поклонялось своему богу. Процесс распада первоначального монотеизма и тринитаризма андитов Месопотамии в первые века второго тысячелетия до Христа вновь перешел в стадию соединения. Многочисленные боги были объединены в пантеон с триединым верховенством Дьяус питар, повелителя неба, Индры, буйного повелителя воздуха, и Агни, трехглавого бога огня, повелителя земли, который когда\hyp{}то был символом древнего представления о Троице.
\vs p094 1:4 Явные проявления генотеизма проложили путь для развития монотеизма. Агни, наиболее древнее божество, часто превозносился как отец\hyp{}глава всего пантеона. Понятие божество\hyp{}отец, иногда называемое Праджапати, а иногда --- Брахмой, было погребено в теологических битвах, которые жрецы\hyp{}брахманы позднее вели с салимскими проповедниками. \bibemph{Брахман} понимался как божественно\hyp{}энергетическое начало, дающее жизнь всему ведическому пантеону.
\vs p094 1:5 \pc Салимские миссионеры проповедовали единого Бога Мелхиседека, Всевышнего Бога неба. Такое изображение, вообще говоря, не шло вразрез с нарождающимся представлением об Отце\hyp{}Брахме как источнике всех богов, но у салимского учения не было системы ритуала и поэтому противоречило всем догмам, обычаям и учениям жрецов\hyp{}брахманов. Жрецы\hyp{}брахманы никогда не смогли бы принять салимское учение о спасении через веру, получении благоволения Бога без соблюдения ритуалов и обрядов жертвоприношения.
\vs p094 1:6 \pc Непринятие евангелия Мелхиседека об уповании на Бога и спасении через веру обозначило жизненно важный поворотный пункт для Индии. Салимские миссионеры во многом способствовали утрате веры во всех древних ведических богов, но вожди, жрецы ведического культа, отказались принять учение Мелхиседека о едином Боге и о единой истинной вере.
\vs p094 1:7 В попытке противостоять салимским проповедникам брахманы собрали священные тексты своего времени, и эти тексты, отредактированные позднее, дошли до наших дней как Ригведа, одна из самых древних священных книг. Вторая, третья и четвертая Веды появлялись по мере того, как брахманы пытались сформировать, формализовать и зафиксировать свои обряды богослужения и жертвоприношения в обычаях народов тех дней. В лучших своих частях эти писания по красоте идеи и истине проникновения равнозначны любым другим подобным собраниям. Но так как эта высшая религия впитала в себя тысячи и тысячи суеверий, культов и ритуалов южной Индии, она постепенно превратилась в самую пеструю теологическую систему, когда\hyp{}либо выработанную смертными людьми. Исследование Вед обнаружит в них и некоторые из самых величайших, и некоторые из наиболее низменных представлений о Божестве, которые когда\hyp{}либо выдвигались.
\usection{2. Брахманизм}
\vs p094 2:1 По мере того, как салимские миссионеры продвигались на юг Деканского плоскогорья, населенного дравидами, они сталкивались со все усиливающейся кастовой системой, системой, с помощью которой арийцы пытались предотвратить потерю расовой индивидуальности перед лицом нарастающего притока вторичных сангикских народов. Поскольку каста жрецов\hyp{}брахманов составляла самую суть этой системы, такое устройство общества препятствовало продвижению салимских проповедников. Эта кастовая система была не в состоянии спасти арийскую расу, но смогла сохранить существование брахманов, которые, между прочим, не утратили в Индии свою религиозную гегемонию до настоящего времени.
\vs p094 2:2 И тогда, наряду с ослаблением ведического учения в результате отказа от высшей истины, культ арийцев стал подвергаться опасности из\hyp{}за все увеличивающегося проникновения обычаев и верований с Деканского плоскогорья. В отчаянной попытке сдержать волну расового и религиозного уничтожения, каста брахманов стремилась возвысить себя над всеми остальными людьми. Они учили, что жертвоприношение божеству само по себе всегда действенно и всегда неотразимо в своем могуществе. Они провозглашали, что из двух основных божественных начал вселенной одним является божество Брахман, а другим --- жреческая каста брахманов. Нет ни одного другого народа на Урантии, священники которого позволяли бы себе превозносить себя даже выше своих богов и присваивать себе почести, предназначенные своим богам. Но они в этих своих самонадеянных утверждениях зашли так далеко, так что вся их непрочная система рухнула, встретившись с приливом низменных культов соседних менее развитых цивилизаций. Обширное ведическое духовенство запуталось и потонуло в черным потоке инертности и пессимизма, который их собственная эгоистическая и неразумная самонадеянность навлекла на всю Индию.
\vs p094 2:3 Чрезмерная сосредоточенность на себе, естественно, вела к страху перед неэволюционирующим существованием в бесконечном круге последовательных воплощений, например, в человека, животное или траву. И изо всех этих пагубных верований, которые могли ополчиться на то, что привело бы к нарождающемуся монотеизму, ни одно не было столь удушающим, как эта вера в переселение --- доктрина перевоплощения душ, --- которая пришла от дравидского Декана. Это верование в утомительный и монотонный круговорот повторяющихся переселений лишало смертных долго лелеемой надежды на обретение в смерти того освобождения и духовного прогресса, которые были частью древней ведической веры.
\vs p094 2:4 За этим философски расслабляющим учением вскоре последовала доктрина вечного избавления от собственного «я» в результате погружения во всеобщий покой и мир абсолютного единения с Брахманом, сверхдушой всего мироздания. Смертная страсть и человеческое честолюбие насильственно подавлялись и практически были сведены к нулю. Более двух тысяч лет лучшие умы Индии пытались освободиться от всех желаний, тем самым широко открыв дверь тем более поздним культам и учениям, которые фактически надели цепи духовной безнадежности на души многих индийских народов. За свой отказ от благой вести Салима ведическая арийская цивилизация заплатила из всех цивилизаций самую ужасную цену.
\vs p094 2:5 \pc Каста как таковая не могла поддерживать существование арийской религиозно\hyp{}культурной системы, и когда низшие религии Декана проникли на север, наступила эра отчаянья и безнадежности. Именно в эти мрачные времена возник культ, согласно которому никакую жизнь нельзя отнимать, и с тех пор он продолжает существовать. Многие из новых культов были откровенно атеистическими, утверждающими, что спасения можно достичь только в результате собственных усилий человека --- без посторонней помощи. Но во всей этой неудачной философии в значительной степени видно, хоть и искаженное, влияние учений Мелхиседека и даже Адама.
\vs p094 2:6 \pc Это было время собирания более поздних священных писаний индуистской веры --- Брахманов и Упанишад. Отвергнув учения личной религии, основывающейся на опыте личной веры в Бога, и погрязнув в массе низменных и вредоносных культов и верований Декана с их антропоморфизмами и перевоплощениями, духовенство брахманов встретилось с яростным противодействием этим порочным верованиям; налицо была явная попытка искать и найти \bibemph{истинную реальность.} Брахманы намеревались деантропоморфизировать индийское понятие божества, но, поступая таким образом, они впали в тяжкую ошибку деперсонализации понятия Бога и в результате оказались не с возвышенным духовным идеалом Райского Отца, а с холодным метафизическим представлением об обнимающем все Абсолюте.
\vs p094 2:7 В своих попытках самосохранения брахманы отвергли единого Бога Мелхиседека, и теперь они остались с гипотезой о Брахмане, неопределенной и иллюзорной философской «самостью», неличностным и бессильным \bibemph{оно,} которое привело духовную жизнь Индии в состояние беспомощности и прострации, длящееся с тех несчастливых времен до двадцатого столетия.
\vs p094 2:8 \pc Именно во время написания Упанишад в Индии возник буддизм. Но несмотря на тысячелетний успех буддизма, он не мог конкурировать с возникшим позднее индуизмом; несмотря на более высокую мораль, его раннее представление о Боге было еще менее ясным, чем в индуизме, который предусматривал существование личных богов, хотя и меньшего масштаба. В конце концов, в северной Индии буддизм уступил ожесточенным атакам воинственного ислама с его отчетливым представлением об Аллахе как о верховном Боге вселенной.
\usection{3. Философия брахманизма}
\vs p094 3:1 Хотя брахманизм в высшей фазе своего развития трудно назвать религией, это, на самом деле, одно из самых благородных достижений смертного ума в области философии и метафизики. Пытаясь познать реальность во всей ее безмерности, индийский разум не остановился, пока не рассмотрел почти каждый аспект теологии, за исключением весьма важного двуединого догмата: понятия о существовании Отца Всего Сущего, всех созданий во вселенной и понятия о реальности опыта восхождения во вселенной этих самых созданий по мере того, как они стремятся достичь вечного Отца, который внушил им быть совершенными, даже столь же совершенными, как он сам.
\vs p094 3:2 В представлении о Брахмане мыслители той эпохи правильно усмотрели идею некоего всеобъемлющего Абсолюта, ибо это понятие в одно и то же время было отождествлено и с творческой энергией, и с космической реакцией. Брахман должен был восприниматься помимо всех определений, его можно было постичь только в результате последовательного отрицания всех конечных качеств. Безусловно, это была вера в абсолютное и даже бесконечное существо, но такое представление было, в значительной степени, лишено личностных атрибутов и, следовательно, недоступно личному опыту верующих.
\vs p094 3:3 Брахман\hyp{}Нараяна понимался как Абсолют, как бесконечное ТО, ЧТО ЕСТЬ, изначальное творческое могущество потенциального космоса, Вселенском «Я», извечно существующее\hyp{}статическое и потенциальное. Если бы философы того времени были способны сделать следующий шаг в представлении о божестве, если бы они смогли воспринять Брахмана как связующее и творческое начало, как личность, которая может быть доступна сотворенным эволюционирующим существам, тогда такое учение могло бы стать самым передовым на Урантии представлением о Божестве, поскольку оно бы охватывало первые пять уровней всей божественной деятельности и, возможно, могло предвидеть существование двух остальных.
\vs p094 3:4 В определенных аспектах представление о Единой Вселенской Сверхдуше как об общем итоге, представляющем сумму существования всех сотворенных существ, привело индийских философов очень близко к истине о Верховном Существе, но эта истина оказалась для них бесполезной, потому что они не сумели выдвинуть никакого разумного или рационального личного подхода для достижения их теоретической монотеистической цели Брахмана\hyp{}Нараяны.
\vs p094 3:5 Кармический принцип причинно\hyp{}следственной неразрывности также очень близок к истине о синтезе последствий отражения всех действий в пространстве\hyp{}времени в Божественном присутствии Верховного; но этот постулат никогда не предусматривал личного достижения Божества отдельным верующим, а только окончательное поглощение всех личностей Вселенской Сверхдушой.
\vs p094 3:6 Философия брахманизма также очень близко подошла к осознанию пребывания внутри людей Настройщиков Мысли, но оно было извращено из\hyp{}за неправильного представления об истине. Учение, говорящее, что душа есть пребывание Брахмана, могло бы проложить дорогу для передовой религии, если бы это представление не было бы полностью испорчено убеждением, что не существует человеческой индивидуальности помимо этого Всеобщего, пребывающего в человеке.
\vs p094 3:7 В доктрине о слиянии индивидуальной души со Сверхдушой индийские теологи не смогли предусмотреть выживание чего\hyp{}то человеческого, чего\hyp{}то нового и уникального, чего\hyp{}то порожденного объединением воли человека и воли Бога. Учение о возвращении души в Брахмана во многом сходно с истиной, говорящей, что Настройщик возвращается в лоно Отца Всего Сущего, но существует нечто отличное от Настройщика, которое тоже выживает, --- моронтийный двойник смертной личности. И это жизненно важное представление, роковым образом, отсутствует в философии брахманизма.
\vs p094 3:8 Философия брахманизма почти подошла к объяснению многих фактов во вселенной, приблизилась к многочисленным космическим истинам, но она слишком часто становилась жертвой ошибки из\hyp{}за неспособности провести различие между несколькими уровнями реальности: абсолютной, трансцендентальной и конечной. Она не смогла принять во внимание, что то, что может быть конечно\hyp{}иллюзорным на абсолютном уровне, может быть абсолютно реальным на конечном уровне. И она также не обратила внимание на очень важную личность Отца Всего Сущего, с которым личности могут вступать в контакт на всех уровнях --- от ограниченного общения эволюционирующих созданий с Богом до не имеющего пределов общения Вечного Сына с Райским Отцом.
\usection{4. Религия индуизма}
\vs p094 4:1 С течением времени, население в Индии отчасти возвратилось к древним ведическим ритуалам в том виде, в каком они были модифицированы под влиянием учения миссионеров Мелхиседека и впоследствии сформированы духовенством брахманов. Эта наиболее древняя и самая космополитическая из мировых религий подверглась дальнейшим изменениям под влиянием буддизма и джайнизма и затем магометанства и христианства. Однако ко времени появления учений Христа, они настолько изменились под влиянием Западной культуры, что стали «религией белого человека» и потому непривычной и чуждой уму индуса.
\vs p094 4:2 \pc В настоящее время в теологии индуизма представлены четыре нисходящих уровня богов и божеств.
\vs p094 4:3 \ublistelem{1.}\bibnobreakspace \bibemph{Брахман,} Абсолют, Бесконечный, ТО, ЧТО ЕСТЬ
\vs p094 4:4 \ublistelem{2.}\bibnobreakspace \bibemph{Тримурти,} верховная троица индуизма. В этой связи первый член, \bibemph{Брахма,} понимается как существо, создавшее само себя из Брахмана --- бесконечности. Если бы не близкое отождествление с пантеистским Бесконечным, Брахма мог бы составить основу для представления об Отце Всего Сущего. Брахма отождествляется также с судьбой.
\vs p094 4:5 Почитание второго и третьего члена троицы, Шивы и Вишну, возникло в первом тысячелетии после Христа. \bibemph{Шива ---} повелитель жизни и смерти, бог плодородия и великий разрушитель. \bibemph{Вишну} чрезвычайно популярен из\hyp{}за поверья, что он периодически воплощается в человеческом облике. Таким образом, Вишну становится реальным и живым в воображении индийцев. Некоторые рассматривают Шиву, а другие Вишну как верховных над всем сущим.
\vs p094 4:6 \ublistelem{3.}\bibnobreakspace \bibemph{Ведические и постведические божества.} Многие из древних богов арийцев, такие как Агни, Индра, Сома, сохранились в качестве второстепенных по сравнению с тремя членами Тримурти. Другие многочисленные боги появились с ранних времен ведической Индии, и они также были включены в пантеон индуизма.
\vs p094 4:7 \ublistelem{4.}\bibnobreakspace \bibemph{Полубоги:} сверхлюди, полубоги, герои, демоны, призраки, духи зла, эльфы, монстры, гоблины и святые позднейших культов.
\vs p094 4:8 \pc Хотя индуизму долго не удавалось возродить индийский народ, он в то же время обычно был терпимой религией. Его основная сила заключается в том, что он оказался наиболее адаптивной, аморфной религией из всех, появившихся на Урантии. Он обладает способностью почти неограниченно изменяться и всесторонне приспосабливаться --- от высоких полумонотеистических представлений брахмана\hyp{}интеллектуала до откровенного фетишизма и практики примитивных культов униженных и угнетенных классов невежественных верующих.
\vs p094 4:9 Индуизм выжил, потому что он --- неотъемлемая часть основной социальной структуры Индии. Он не обладает высокоразвитой иерархией, которая может быть нарушена или уничтожена; он неразрывно связан с жизнью народа. Он умеет приспосабливаться к изменяющимся условиям намного лучше, чем все другие культы, и терпимо воспринимает многие другие религии: Гаутама Будда и даже сам Христос объявлены воплощениями Вишну.
\vs p094 4:10 Сегодня крайне необходимо представить в Индии евангелие Иисуса --- благой вести об Отцовстве Бога и сыновстве и последующем братстве всех людей, которое реализуется каждой личностью в исполненном любви служении и служении обществу. В Индии существует философская основа, имеется культовая структура; все, что необходимо, --- это живительная искра действенной любви, отраженная в подлинном евангелии Сына Человеческого, лишенного западных догм и доктрин, которые стремились сделать пришествие жизни Михаила религией белого человека.
\usection{5. Борьба за истину в Китае}
\vs p094 5:1 По мере того, как салимские миссионеры шли через Азию, распространяя учение о Боге Всевышнем и спасении через веру, они впитали много философских и религиозных представлений различных стран, через которые они прошли. Но учителя, направленные Мелхиседеком и его последователями, оправдали оказанное им доверие; они проникли ко всем народам евразийского континента, и в середине второго тысячелетия до Христа прибыли в Китай. У озера Фу они устроили свой центр и поддерживали его более ста лет, там они обучали китайских учителей, которые проповедовали во всех областях обитания желтой расы.
\vs p094 5:2 Прямым следствием этих учений было то, что в Китае возникла первоначальная форма даосизма, религия, весьма отличная от той, которая сегодня носит это название. Ранний или протодаосизм был соединением следующих факторов:
\vs p094 5:3 \ublistelem{1.}\bibnobreakspace Уцелевшее учение Синглангтона, которое сохранилось в представлении о Шаньди, Боге Неба. Во времена Синглангтона китайский народ стал, в сущности, монотеистическим, он сосредоточился на поклонении Единой Истине, известной позже как Дух Неба, правитель вселенной. И желтая раса никогда полностью не утрачивала это ранее представление о Божестве, хотя в последующие столетия в ее религию коварно прокрались многие второстепенные боги и духи.
\vs p094 5:4 \ublistelem{2.}\bibnobreakspace Салимская религия Всевышнего Бога\hyp{}Творца, который дарует человечеству свое благоволение в ответ на человеческую веру. Но правда также и то, что к тому времени, когда миссионеры Мелхиседека проникли в страны желтой расы, их первоначальная весть значительно отличалась от ясных доктрин Салима в дни Мелхиседека.
\vs p094 5:5 \ublistelem{3.}\bibnobreakspace Представление индийских философов о Брахмане\hyp{}Абсолюте, соединенное с желанием избежать всякого зла. Возможно, самое большое внешнее влияние на распространение салимской религии на восток было оказано индийскими проповедниками ведической веры, которые внесли свое представление о Брахмане --- Абсолюте --- в идею спасения салимитов.
\vs p094 5:6 \pc Эта смешанная вера распространялась по странам желтой и коричневой рас, оказывая определяющее влияние на религиозно\hyp{}философскую мысль. В Японии этот протодаосизм известен как синтоизм, и в этой стране, находящейся так далеко от палестинского Салима, народы узнали о воплощении Махивенты Мелхиседека, который жил на земле для того, чтобы имя Бога не было забыто человечеством.
\vs p094 5:7 В Китае все эти верования позднее смешались и добавились к набирающему силу культу почитания предков. Но никогда со времен Синглангтона китайцы не становились беспомощными рабами духовенства. Желтая раса первой освободилась от варварского рабства и создала организованную цивилизацию, потому что она первой добилась хоть какого\hyp{}то освобождения от презренного страха перед богами, причем они даже не боялись духов мертвых, как их боялись другие расы. Китай потерпел поражение потому, что ему не удалось продвинуться дальше, за пределы своей ранней эмансипации от священников; он впал в почти столь же гибельное заблуждение --- поклонение предкам.
\vs p094 5:8 \pc Но салимиты трудились не напрасно. Именно на основе их евангелия великие философы Китая шестого столетия создали свои учения. Атмосфера морали и духовный настрой времен Лао\hyp{}цзы и Конфуция выросли из учения салимских миссионеров более ранней эпохи.
\usection{6. Лао\hyp{}цзы и Конфуций}
\vs p094 6:1 Примерно за шестьсот лет до прибытия Михаила Мелхиседеку, уже давно покинувшему плотский облик, казалось, что на земле чистоте его учения чрезмерно угрожает опасность быть полностью поглощенной древними урантийскими верованиями. Какое\hyp{}то время представлялось, что его миссии, как предвестника Михаила, грозит опасность провала. И в шестом веке до Христа, благодаря необычному взаимодействию духовных сил, не все из которых были понятны даже планетарным руководителям, Урантия стала свидетельницей самого необыкновенного представления многоликой религиозной истины. Благодаря посредничеству нескольких учителей, которые были людьми, салимская благая весть была объявлена вновь и ей была дана новая жизнь, и из того, что было тогда предъявлено, многое сохранилось до времени написания этих строк.
\vs p094 6:2 Этот замечательный век духовного прогресса отмечен для всего цивилизованного мира появлением великих учителей религии, философии и морали. В Китае это были два выдающихся учителя, Лао\hyp{}цзы и Конфуций.
\vs p094 6:3 \pc \bibemph{Лао\hyp{}цзы} основывался непосредственно на идеях салимских традиций, когда он заявил, что Дао есть Первопричина всего творения. Лао был человеком большой духовной дальновидности. Он учил, что «неизменный удел человека заключается в вечном союзе с Дао, Верховным Богом и Царем Всего Сущего». В своем понимании изначальной причины он был в высшей степени проницателен, ибо он писал: «Единство возникает из Абсолюта Дао, и из Единства появляется космическая Двойственность, а из этой Двойственности возникает Троица, и Троица есть первоисточник всей реальности». «Всякая реальность всегда есть равновесие между потенциальностью и действительностью космоса, и они извечно приводятся духом божественности в состояние гармонии».
\vs p094 6:4 Лао\hyp{}цзы был также одним из самых первых, кто выдвинул учение о воздаянии добром за зло: «Доброта порождает доброту, но для того, кто истинно добр, зло также порождает доброту».
\vs p094 6:5 Он учил о возвращении создания к своему Творцу и изображал жизнь как возникновение личности из космической потенциальности, в то время как смерть похожа на возвращение домой этой сотворенной личности. Его представление об истинной вере было необычным, и он также уподоблял ее «отношению малого ребенка».
\vs p094 6:6 Его понимание вечной цели Бога было ясным, ибо он говорил: «Абсолютное божество не стремится к победе, но всегда является победоносным; оно не принуждает человечество, но всегда готово ответить на его истинные желания; воля Бога неизменна в терпении и вечна в неизбежности своего проявления». А об истинно верующем он сказал, выражая истину, что более благословенно давать, чем получать: «Хороший человек не стремится сохранить истину для самого себя, но, скорее, пытается одарить ее сокровищами своих ближних, ибо это и есть осознание истины. Воля Абсолютного Бога всегда приносит благо, никогда разрушение; цель истинно верующего всегда состоит в том, чтобы действовать, но никогда не принуждать».
\vs p094 6:7 Учение Лао о непротивлении и различие, которое он проводил между \bibemph{действием} и \bibemph{принуждением,} позднее было извращено и превратилось в убеждение, что следует «ничего не видеть, ничего не делать и ни о чем не думать». Но Лао никогда не учил такому заблуждению, хотя его представление о непротивлении было составляющей дальнейшего развития пацифистских наклонностей народов Китая.
\vs p094 6:8 Но популярный даосизм Урантии двадцатого века имеет очень мало общего с возвышенными чувствами и космическими представлениями древнего философа, который учил, что истина, как он ее понимает, --- это вера в Абсолютного Бога --- источник той божественной энергии, которая переделает мир и с помощью которой человек возвышается до духовного союза с Дао, Вечным Божеством и Абсолютом\hyp{}Творцом вселенных.
\vs p094 6:9 \pc \bibemph{Конфуций} (Кун Фу\hyp{}цзы) был младшим современником Лао в Китае шестого века. Конфуций основывал свои доктрины на лучших моральных традициях длительной истории желтой расы и в какой\hyp{}то степени испытал влияние сохранившихся традиций салимских миссионеров. Главный его труд представляет собой компиляцию мудрых высказываний древних философов. В течение всей своей жизни Конфуций не был признан как учитель, но с той поры его сочинения и учения оказали огромное влияние на Китай и Японию. Конфуций сделал новый шаг в шаманизме, а именно, поставил мораль на место магии. Но он перестарался; он создал новый фетиш из \bibemph{порядка} и установил традицию уважения к образу жизни предков, которая все еще почитаема китайцами во время написания этих строк.
\vs p094 6:10 Конфуцианская проповедь морали была основана на теории, что земной путь есть искаженная тень небесного пути; что истинный образец мирской цивилизации есть зеркальное отражение вечного порядка неба. Потенциальное представление о существовании Бога в конфуцианстве было подчиненным по отношению к понятию Пути Неба, модели космоса, которой придавалось особое значение.
\vs p094 6:11 Учение Лао было утрачено для всех, за исключением небольшого числа людей на Востоке, однако, сочинения Конфуция составили с той поры основу нравственных принципов культуры почти трети жителей Урантии. Эти конфуцианские предписания, хотя и сохраняли лучшее из прошлой жизни, были в какой\hyp{}то степени враждебны самому духу исследования китайского народа, породившему те достижения, которые столь почитались. В борьбе за влияние с этим учением безуспешно сражались имперские начинания Чэн Ши Гуан Ди и учения Мо Ди, который провозгласил братство, основанное не на этическом долге, а на любви Бога. Он пытался вновь возбудить интерес к древним поискам новой истины, но его учение потерпело поражение из\hyp{}за мощного противодействия учеников Конфуция.
\vs p094 6:12 Как многие другие учителя духовности и морали, и Конфуций, и Лао\hyp{}цзы были, в конце концов, обожествлены их последователями в мрачные для духовности Китая времена, которые приходятся на период между упадком и извращением даосской веры и приходом из Индии буддийских миссионеров. В эти века духовного упадка религия желтой расы выродилась в жалкую теологию, которую заполонили дьяволы, драконы и духи зла, что указывает на постоянно возвращающийся страх в непросвещенном сознании смертных. И Китай, стоявший когда\hyp{}то благодаря передовой религии во главе человеческого общества, оказался теперь позади вследствие временной неспособности идти по истинному пути развития того представления о Боге, которое абсолютно необходимо для истинного прогресса не только отдельных смертных, но и сложных цивилизаций, которые и являются определяющими для улучшения культуры и общества на планете, эволюционирующей в пространстве и времени.
\usection{7. Гаутама Сиддхартха}
\vs p094 7:1 В одно время с Лао\hyp{}цзы и Конфуцием в Китае другой великий учитель истины появился в Индии. Гаутама Сиддхартха родился в шестом веке до Христа в Непале, северной провинции Индии. Позднее его последователи пытались представить дело так, будто он был сыном сказочно богатого правителя, но в действительности он был наследником местного вождя, который, со всеобщего согласия, правил маленькой уединенной долиной в южных Гималаях.
\vs p094 7:2 После шести лет безуспешных занятий йогой Гаутама сформулировал те самые теории, которые развились в философию буддизма. Сиддхартха вел решительную, но тщетную борьбу против распространяющейся кастовой системы. В этом юном принце\hyp{}пророке была возвышенная искренность и удивительная бескорыстность, которые были очень привлекательны для людей того времени. Он отошел от обычая искать личного спасения через физическую боль и личное страдание. И он призвал своих последователей нести его евангелие по всему миру.
\vs p094 7:3 Среди смятения и крайних культов Индии более здоровое и более умеренное учение Гаутамы пришло как живительное облегчение. Он осудил богов, жрецов и их жертвоприношения, но и он не сумел осознать \bibemph{личность} Единого Вселенского. Не веря в существование личной человеческой души, Гаутама вел, конечно, героическую борьбу против освященной временем веры в переселение душ. Он прилагал благородные усилия, чтобы избавить человека от страха, чтобы тот мог чувствовать себя во вселенной непринужденно, как дома, но ему не удалось указать людям дорогу к тому истинному и возвышенному дому восходящих смертных --- Раю --- и к расширяющемуся служению вечного бытия.
\vs p094 7:4 Гаутама был настоящим пророком, но если бы он последовал наставлениям отшельника Годада, он, может быть, пробудил бы всю Индию вдохновляющим возрождением салимского евангелия о спасении через веру. Годад происходил из рода, который никогда не утрачивал традиций миссионеров Мелхиседека.
\vs p094 7:5 В Бенаресе Гаутама основал свою школу, и на втором году ее существования один из учеников, Ботан сообщил своему учителю предания салимских миссионеров о завете Мелхиседека Аврааму; и хотя Сиддхартха не имел достаточно ясного представления об Отце Всего Сущего, он принял передовое положение о спасении через веру --- простое доверие. Он изложил свою позицию своим последователям и стал посылать своих учеников группами по шестьдесят человек, чтобы те возвестили всей Индии «счастливую весть о всем доступном спасении; что все люди, высокие и низкие, могут достигнуть блаженства через веру в праведность и справедливость».
\vs p094 7:6 Жена Гаутамы поверила в евангелие своего мужа и основала орден монахинь. Сын Гаутамы стал его преемником и значительно расширил влияние культа; он воспринял новую идею о спасении через веру, но в более поздние годы своей жизни колебался относительно представления о божественном благоволении, получаемом, согласно салимскому евангелию, только одной верой. И в конце жизни последними его словами были: «Добивайтесь своего собственного спасения».
\vs p094 7:7 \pc Лучшее, что содержалось в евангелии Гаутамы о всеобщем спасении, не связанном с жертвоприношениями, муками, ритуалами и жрецами, представляло собой революционную и поразительную для своего времени доктрину. И она удивительно близко подошла к тому, чтобы возродить благую весть Салима. Она пришла на помощь отчаявшимся душам и, несмотря на нелепые извращения последних столетий, все еще остается надеждой миллионов людей.
\vs p094 7:8 Сиддхартха учил истине, которая была гораздо больше того, что уцелело в современном учении, носящем его имя. Современный буддизм является учением Гаутамы Сиддхартхи не в большей степени, чем христианство --- учением Иисуса из Назарета.
\usection{8. Буддийская вера}
\vs p094 8:1 Чтобы стать буддистом, надо просто дать публичный обет веры, произнося формулу Прибежища: «Я нахожу свое прибежище в Будде; я нахожу свое прибежище в Учении; я нахожу свое прибежище в Братстве».
\vs p094 8:2 Буддизм порожден исторической личностью, а не мифом. Последователи Гаутамы называли его Саста, что значит господин или учитель. Хотя он и не приписывал ни себе, ни своему учению никаких сверхчеловеческих черт, его ученики давно начали называть его \bibemph{просветленный,} Будда, впоследствии --- Сакьямуни Будда.
\vs p094 8:3 \pc Подлинное евангелие Гаутамы основывалось на четырех высших истинах:
\vs p094 8:4 \ublistelem{1.}\bibnobreakspace Высшие истины страдания.
\vs p094 8:5 \ublistelem{2.}\bibnobreakspace Истоки страдания.
\vs p094 8:6 \ublistelem{3.}\bibnobreakspace Уничтожение страдания.
\vs p094 8:7 \ublistelem{4.}\bibnobreakspace Путь к уничтожению страдания.
\vs p094 8:8 \pc Близко связанной с учением о страдании и избавлении от него была философия Восьмиричного Пути: правильные взгляды, стремления, речь, поведение, средства к жизни, усилие, осознание и созерцание. В намерение Гаутамы не входило уничтожить все усилия, стремления и привязанности для того, чтобы избежать страдания, его учение, скорее, было предназначено показать смертному человеку тщетность попыток связывать все надежды и помыслы исключительно с временными целями и материальными стремлениями. Это не значит, что надо остерегаться любви своих ближних, но истинно верующий должен также заглядывать за пределы этого материального мира, стремиться к реалиям вечного будущего.
\vs p094 8:9 \pc Заповедей морали в вероучении Гаутамы было пять:
\vs p094 8:10 \ublistelem{1.}\bibnobreakspace Не убий.
\vs p094 8:11 \ublistelem{2.}\bibnobreakspace Не укради.
\vs p094 8:12 \ublistelem{3.}\bibnobreakspace Не прелюбодействуй.
\vs p094 8:13 \ublistelem{4.}\bibnobreakspace Не лги.
\vs p094 8:14 \ublistelem{5.}\bibnobreakspace Не пей дурманящих напитков.
\vs p094 8:15 \pc Существовало несколько добавочных или второстепенных заповедей, соблюдение которых осуществлялось по выбору верующих.
\vs p094 8:16 \pc Сиддхартха вряд ли верил в бессмертие человеческой личности; его философия обеспечивала только некую функциональную непрерывность. Он никогда четко не определял, что он мыслит включенным в доктрину Нирваны. Тот факт, что теоретически возможно испытать состояние нирваны во время смертного существования, показывает, что оно не рассматривается как состояние полной аннигиляции. Оно подразумевает условие высшей просвещенности и божественного блаженства, когда разрушены все узы, связывающие человека с материальным миром; наступает свобода от желаний смертной жизни и избавление от всех дальнейших воплощений.
\vs p094 8:17 Согласно первоначальному учению Гаутамы, спасение достигается человеческим усилием, без божественной помощи; в нем нет места спасительной вере или молитвам, обращенным к сверхчеловеческим силам. Гаутама в своем стремлении свести к минимуму суеверия Индии пытался отвратить людей от громогласных притязаний на магическое спасение. И, делая такую попытку, он оставлял своим преемникам дверь широко открытой для неправильного истолкования своего учения и для утверждения, что все человеческие стремления достичь цели отвратительны и мучительны. Его последователи проглядели тот факт, что высшее счастье связано с разумным и воодушевленным стремлением к достижению достойных целей и что такие достижения составляют истинный прогресс в космической самореализации.
\vs p094 8:18 Великая истина учения Сиддхартхи состоит в его возвещении о вселенской абсолютной справедливости. Он учил лучшей безбожной философии, которая когда\hyp{}либо была изобретена смертным человеком; это был идеал гуманизма и он самым действенным образом устранил все основания для появления суеверий, магических обрядов и страха перед призраками и демонами.
\vs p094 8:19 Огромной слабостью первоначального буддийского благовествования было то, что оно не породило религию бескорыстного общественного служения. Буддийское братство в течение длительного времени было не братством верующих, а, скорее, сообществом обучающихся учителей. Гаутама запретил им получать деньги, пытаясь таким образом предотвратить рост иерархических тенденций. Сам Гаутама был в высшей степени общественным человеком; действительно, его жизнь была гораздо замечательнее, чем его проповедь.
\usection{9. Распространение буддизма}
\vs p094 9:1 Буддизм процветал, потому что он предлагал спасение через веру в Будду, в просветленного. Он лучше представлял истины Мелхиседека, чем любая другая религиозная система, существовавшая во всей восточной Азии. Но буддизм не стал широко распространенной религией, пока его не поддержал в целях самозащиты царь Ашока, который принадлежал к низкой касте и который, вслед за Эхнатоном в Египте, был одним из наиболее выдающихся светских правителей в период между Мелхиседеком и Михаилом. Ашока создал огромную Индийскую империю благодаря пропаганде своих буддийских миссионеров. В течение двадцати пяти лет он обучил и разослал свыше семнадцати тысяч миссионеров в самые отдаленные края всего известного тогда мира. В течение одного поколения он сделал буддизм господствующей религией половины мира. Вскоре буддизм установился в Тибете, Кашмире, Цейлоне, Бирме, на Яве, в Сиаме, Корее, Китае и Японии. Вообще говоря, это была религия, значительно превосходящая те, которые буддизм вытеснил или улучшил.
\vs p094 9:2 Распространение буддизма из своей родины в Индии по всей Азии --- одна из захватывающих историй духовной преданности и миссионерского упорства истинно верующих. Учителя евангелия Гаутамы не только бросили вызов опасностям сухопутных караванных троп, неся весть своей веры всем народам, они столкнулись с опасностями Китайских морей при выполнении своей миссии на азиатском континенте. Но этот буддизм больше не был простым и ясным учением Гаутамы, он стал восприниматься как чудотворное евангелие, которое сделало его богом. И чем дальше от родных высокогорий в Индии распространялся буддизм, тем более непохожим на учения Гаутамы он становился, и все более начинал походить на религии, которые вытеснил.
\vs p094 9:3 Впоследствии буддизм испытал значительное влияние даосизма в Китае, синтоизма --- в Японии и христианства --- в Тибете. По прошествии тысячи лет буддизм в Индии просто увял и угас. Он стал брахманизованным и позднее униженно подчинился исламу, тогда как на большей части всего остального Востока он выродился в ритуал, который Гаутама Сиддхартха никогда бы не узнал.
\vs p094 9:4 Фундаменталистское стереотипное представление об учениях Сиддхартхи продолжало существовать на юге --- на Цейлоне, в Бирме и на полуострове Индокитай. Это хинаянская ветвь буддизма, которая придерживается ранней или асоциальной доктрины.
\vs p094 9:5 Но еще до упадка буддизма в Индии китайские и северно\hyp{}индийские группы последователей Гаутамы стали развивать учение махаяны, «Широкого Пути» спасения, в противоположность пуристам на юге, которые придерживались хинаяны, или «Узкого Пути». Эти приверженцы махаяны были свободны от социальных ограничений, присущих буддийской доктрине, и с той поры этот северный вариант буддизма продолжал развиваться в Китае и Японии.
\vs p094 9:6 Сегодня буддизм --- живая, развивающаяся религия, потому что она продолжает соблюдать многие из высших нравственных ценностей своих приверженцев. Она помогает сохранять хладнокровие и самоконтроль, способствует достижению безмятежности и счастья и делает многое для того, чтобы предотвратить горе и скорбь. Те, кто верят этой философии, живут лучшей жизнью, чем многие, которые ей не верят.
\usection{10. Религия в Тибете}
\vs p094 10:1 В Тибете можно обнаружить самое удивительное сочетание учений Мелхиседека, соединенных с буддизмом, индуизмом, даосизмом и христианством. Когда буддийские миссионеры вступили в Тибет, они встретились со страной первобытной дикости, очень похожей на то, что христианские миссионеры увидели среди северных племен Европы.
\vs p094 10:2 Эти простодушные тибетцы не желали полностью отказываться от своей древней магии и амулетов. Изучение религиозных обрядов сегодняшнего Тибета обнаруживает существование чрезмерно разросшегося братства бритоголовых священников, выполняющих сложные обряды, в которых используются колокола, монотонные песнопения, воскурения, процессии, четки, идолы, амулеты, изображения, святая вода, великолепные облачения и изощренные хоры. У них есть строгие догмы и выкристаллизовавшиеся убеждения, мистические ритуалы и особые посты. Их иерархия включает монахов, монахинь, настоятелей и Великого Ламу. Они молятся ангелам, святым, Святой Матери и богам. Они практикуют исповедь и верят в чистилище. Их монастыри обширны, а храмы отличаются великолепием. Они придерживаются нескончаемого повторения священных обрядов и верят, что такие ритуалы даруют спасение. Молитвы прикрепляются к колесу, и они верят, что при вращении колеса эти прошения осуществляются. В настоящее время нет другого народа, который соблюдал бы такое множество обычаев, заимствованных из стольких религий; а ритуал богослужения, составленный из множества компонентов, неизбежно оказывается чрезмерно громоздким и невыносимо обременительным.
\vs p094 10:3 У тибетцев есть понемногу ото всех ведущих мировых религий, за исключением простых истин евангелия Иисуса: сыновство по отношению к Богу, братство по отношению к человеку и все восходящее гражданство в вечной вселенной.
\usection{11. Буддийская философия}
\vs p094 11:1 Буддизм пришел в Китай в первом тысячелетии после Христа и хорошо вписался в религиозные обычаи желтой расы. В своем почитании предков они издавна молились мертвым, теперь они также могли молиться за них. Буддизм скоро слился со все еще существующими религиозными обычаями распадающегося даосизма. Эта новая синтетическая религия с ее богослужебными храмами и четким религиозным ритуалом скоро превратилась для народов Китая, Кореи и Японии в общепринятый культ.
\vs p094 11:2 \pc Хотя в некоторых отношениях и не вполне удачно, что буддизм не распространился по миру до того, как последователи Гаутамы настолько извратили традиции и учения этого культа, что сделали его божественным существом, и все же этот миф о его человеческой жизни, приукрашенный массой чудес, оказался весьма привлекательным для последователей махаяны или северного варианта буддизма.
\vs p094 11:3 Некоторые из его более поздних последователей учили, что дух Сакьямуни Будды периодически возвращается на землю как живой Будда, открывая таким образом путь для бесконечного увековечивания образов Будды, его храмов, ритуалов и самозванных «живых Будд». Таким образом, религия великого индийского протестанта, в конце концов, оказалась в плену тех самых церемониальных обычаев и ритуальных заклинаний, против которых он так бесстрашно боролся и которые он так доблестно отвергал.
\vs p094 11:4 \pc Большой прогресс, сделанный в буддийской философии, состоял в осознании ею того факта, что всякая истина относительна. Благодаря этой гипотезе буддисты смогли примирить и согласовать противоречия в своих собственных религиозных писаниях, а также --- различия между своими собственными и многими другими священными писаниями. Людей учили, что для малых умов существует малая истина, а для великих умов --- большая.
\vs p094 11:5 Эта философия придерживалась также взгляда, что (божественная) природа Будды пребывает во всех людях, что человек при помощи собственных усилий может достичь реализации этой внутренней божественности. И это учение является одним из самых ясных представлений о пребывающем Настройщике, которые когда\hyp{}либо существовали в урантийской религии.
\vs p094 11:6 Но особым недостатком первоначального евангелия Сиддхартхи, как оно интерпретировалось его последователями, было то, что оно пыталось полностью освободить человеческое «я» от всех ограничений смертной природы путем изоляции собственного «я» от объективной реальности. Истинная космическая самореализация проистекает из отождествления с космической реальностью и с конечным космосом энергии, разума и духа, ограниченным пространством и обусловленным временем.
\vs p094 11:7 Но хотя ритуалы и форма обрядов буддизма были сильно испорчены влиянием обычаев тех стран, в которых он распространялся, это вырождение не в полной мере относилось к философской жизни великих мыслителей, которые время от времени избирали эту систему мысли и веры. В продолжение более двух тысяч лет многие лучшие умы Азии сосредоточивались на проблеме установления абсолютной истины и истины Абсолюта.
\vs p094 11:8 Эволюция высокой идеи Абсолюта происходила по многим каналам мысли и окольными путями рассуждения. Развитие этой доктрины бесконечности не было так ясно и определено, как эволюция понятия Бога в иудейской теологии. Тем не менее существовали определенные основные уровни, которые разум буддистов достигал, останавливался и проходил на своем пути постижения Первоисточника вселенных:
\vs p094 11:9 \ublistelem{1.}\bibnobreakspace \bibemph{Легенда о Гаутаме.} В основе этого представления лежал исторический факт жизни и учения Сиддхартхи, принца\hyp{}пророка Индии. По мере того, как эта легенда превращалась в миф, шла через столетия и через обширные страны Азии, она переросла представления о Гаутаме как о просветленном и начала приобретать дополнительные атрибуты.
\vs p094 11:10 \ublistelem{2.}\bibnobreakspace \bibemph{Множество Будд.} Люди рассудили, что, если Гаутама пришел к народам Индии, то в далеком прошлом и в отдаленном будущем человеческие расы должны были и, несомненно, будут благословлены приходом других учителей истины. Это дало начало учению, что существует много Будд, число которых неопределенно и бесконечно, и даже любой человек может надеяться стать одним из них --- достичь божественности Будды.
\vs p094 11:11 \ublistelem{3.}\bibnobreakspace \bibemph{Абсолютный} \bibemph{Будда.} Со временем число Будд приблизилось к бесконечности, и для умов той поры стало необходимым заново унифицировать это громоздкое понятие. Соответственно, стали учить, что все Будды есть не что иное, как выражение некоей высшей сущности, некоего Вечного, имеющего бесконечное и неограниченное существование, некоего Абсолютного Источника всякой реальности. С этого момента понятие Божества в буддизме, в своей высшей форме, становится не связанным с человеческой личностью Гаутамы Сиддхартхи и лишается антропоморфных ограничений, которые его сдерживали. Это окончательное представление о Будде Вечном может вполне быть отождествлено с Абсолютом, а иногда --- даже с бесконечным Я ЕСТЬ.
\vs p094 11:12 \pc Хотя эта идея Абсолютного Божества никогда не пользовалась популярностью среди народов Азии, она дала возможность интеллектуалам этих стран унифицировать свою философию и гармонизировать свою космологию. Представление о Будде\hyp{}Абсолюте является иногда квазиличностным, а иногда полностью неличностным --- понимаемым даже как бесконечная творческая сила. Такие представления, хотя и полезные в философии, не существенны для развития религии. Даже антропоморфный Яхве имеет большую религиозную ценность, чем бесконечно далекий Абсолют буддизма или брахманизма.
\vs p094 11:13 Временами Абсолют мыслился даже как содержащийся внутри бесконечного Я ЕСТЬ. Но эти спекуляции были слабым утешением множеству страждущих, страстно желавших услышать слово обещания, услышать простое евангелие Салима, что вера в Бога гарантирует божественное благоволение и вечную жизнь.
\usection{12. Понятие Бога в буддизме}
\vs p094 12:1 Большая слабость космологии буддизма определяется двумя моментами: она смешана со многими суевериями Индии и Китая и ее представления о Гаутаме сведены --- сначала к представлению о просветленном, а затем к представлению о Вечном Будде. Как христианство страдало от того, что впитало в себя многие ошибочные представления человеческой философии, так и буддизм отмечен этим родимым пятном человеческой принадлежности. Но в течение последних двух с половиной тысяч лет учения Гаутамы продолжали развиваться. Представление о Будде для просвещенного буддиста определяется личностью Гаутамы не в большей степени, чем представление о Яхве для просвещенного христианина связывается с духом демона Хорива. Скудость терминологии вместе с сентиментальной приверженностью к прежней системе терминов часто являются причиной неспособности понять истинную значимость эволюции религиозных понятий.
\vs p094 12:2 \pc Постепенно в буддизме в противоположность понятию Абсолюта стало появляться понятие Бога. Его истоки восходят к давним временам, когда произошло разделение на сторонников Узкого Пути и Широкого Пути. Именно среди адептов Широкого Пути и сформировалась окончательно двойственная концепция Бога и Абсолюта. Шаг за шагом, столетие за столетием развивалась идея Бога, пока, наконец, смешавшись с учениями Рионина, Хонэна Сонина и Синрана в Японии, эта идея не привела к возникновению веры в Амиду Будду.
\vs p094 12:3 В среде этих верующих учат, что душа, встретившись со смертью, может по своему выбору насладиться жизнью в Раю до того, как погрузиться в Нирвану --- конечный пункт бытия. Утверждается, что это новое спасение приобретается через веру в божественное милосердие и любовную заботу Амиды, Бога Рая на западе. В своей философии амидисты придерживаются представления о Бесконечной Реальности, которая находится за пределами всякого ограниченного понимания смертных; в своей религии они основываются на вере во всемилостивого Амиду, который настолько любит мир, что не позволит ни одному смертному, который взывает к его имени с истинной верой и чистым сердцем, потерпеть неудачу в достижении высшего счастья Рая.
\vs p094 12:4 Очень сильной стороной буддизма является то, что его приверженцы свободны выбирать истину из всех религий; такая свобода выбора редко свойственна урантийской вере. В этом отношении японская секта Син стала одной из наиболее прогрессивных религиозных групп в мире; она возродила древний миссионерский дух последователей Гаутамы и стала посылать учителей к другим народам. Эта готовность добывать истину из любого источника --- конечно, похвальное стремление, которое возникло среди религиозных верующих в первой половине двадцатого века после Христа.
\vs p094 12:5 Сам буддизм в двадцатом веке возрождается. Благодаря контакту с христианством, социальные аспекты буддизма значительно усилились. В сердцах священников\hyp{}монахов этого братства вновь поселилось желание узнавать новое, и распространение образования во всей вере, несомненно, будет стимулировать новые успехи в эволюции религии.
\vs p094 12:6 Большая часть Азии ко времени написания этих строк свою надежду возлагает на буддизм. Приобретет ли снова эта благородная вера, которая так отважно прошла сквозь темные века прошлого, истину возросших космических реалий, подобно тому, как когда\hyp{}то в Индии ученики великого учителя слушали его возвещение новой истины? Откликнется ли еще раз эта древняя вера на воодушевляющий призыв представления новых концепций Бога и Абсолюта, которых она так долго искала?
\vs p094 12:7 \pc Вся Урантия ожидает провозглашения одухотворяющего послания Михаила, не обремененного доктринами и догмами, накопленными за девятнадцать веков взаимодействия с религиями эволюционного происхождения. Пробил час представления буддизму, христианству, индуизму и всем людям всех вероисповеданий не евангелия о Христе, а живой, духовной реальности евангелия Иисуса.
\vsetoff
\vs p094 12:8 [Представлено Мелхиседеком Небадона.]
