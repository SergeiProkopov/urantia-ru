\upaper{194}{Дарование Духа Истины}
\author{Комиссия срединников}
\vs p194 0:1 Около часа дня, когда сто двадцать верующих молились, все они почувствовали присутствие чего\hyp{}то необычного в комнате. В один и тот же момент вдруг все эти последователи испытали новое и глубокое чувство духовной радости, уверенности и защищенности. За этим новым ощущением духовной силы сразу последовало неодолимое желание пойти и публично провозгласить евангелие царства и благую весть о том, что Иисус воскрес из мертвых.
\vs p194 0:2 Петр встал и объявил, что это, должно быть, пришествие Духа Истины, обещанное им Учителем, и предложил всем идти в храм и начать провозглашение благой вести, вверенной им. И они поступили так, как предложил Петр.
\vs p194 0:3 \pc Иисус воспитывал и учил этих людей, что евангелие, которое они должны проповедовать, состоит в отцовстве Бога и сыновстве человека, однако в этот момент духовного экстаза и личного триумфа самая благая весть, самая великая новость, о которой эти люди только и могли думать, заключалась в \bibemph{факте} воскресения Учителя. Поэтому наделенные силой с неба, они пошли, проповедуя народу благую весть --- и даже спасение через Иисуса, --- но невольно пошли по ложному пути, подменив отдельными связанными с евангелием фактами само евангельское послание. Петр, сам того не желая, первый совершил эту ошибку, а за ним последовали остальные, включая и Павла, который из новой версии благой вести создал новую религию.
\vs p194 0:4 Евангелие царства --- это факт отцовства Бога в сочетании с вытекающей из него истиной братства\hyp{}сыновства людей. Христианство же, как оно развивалось с того дня, есть факт Бога как Отца Господа Иисуса Христа в сочетании с опытом братства верующих с воскресшим и прославленным Христом.
\vs p194 0:5 Неудивительно, что эти исполнившиеся духом люди должны были ухватиться за саму возможность выразить свои чувства торжества над силами, которые пытались погубить их Учителя и положить конец влиянию его учений. В такое время, как это, самым простым для них было помнить их личную связь с Иисусом и воодушевиться уверенностью, что Учитель по\hyp{}прежнему жив, что их дружба не закончилась и что дух, действительно, снизошел на них, как и обещал Иисус.
\vs p194 0:6 Эти верующие внезапно почувствовали себя перенесенными в иной мир, в новое бытие, полное радости, силы и славы. Учитель говорил им, что царство придет в силе, и некоторые из них подумали, что начинают понимать, что тот имел в виду.
\vs p194 0:7 И учитывая все это, нетрудно понять, как получилось, что эти люди стали проповедовать \bibemph{новое евангелие об Иисусе} вместо своего прежнего послания об отцовстве Бога и братстве людей.
\usection{1. Проповедь в день Пятидесятницы}
\vs p194 1:1 Апостолы скрывались сорок дней. Этот же день пришелся на еврейский праздник Пятидесятницы, и тысячи паломников со всех концов света были в Иерусалиме. Многие пришли специально на праздник, но большинство оставалось в городе еще с Пасхи. И вот эти испуганные апостолы вышли из продолжавшегося несколько недель уединения, чтобы открыто явиться во храме, где и начали проповедовать новое послание о воскресшем Мессии. И все ученики точно так же ощутили, что и они получили некий новый духовный дар понимания и силы.
\vs p194 1:2 Около двух часов Петр встал на то же место, где последний раз в храме проповедовал Учитель, и выступил столь страстно, что привел в царство более двух тысяч душ. Учитель ушел, но они неожиданно открыли, что этот рассказ о нем имеет великую власть над людьми. Неудивительно, что они увлеклись и стали продолжать провозглашение того, что подтверждало их прежнюю преданность Иисусу и одновременно побуждало людей с такой силой верить в него. В этом собрании участвовало шесть апостолов: Петр, Андрей, Иаков, Иоанн, Филипп и Матфей. Они говорили более полутора часов и произносили послания на греческом, древнееврейском и арамейском языках, и сказали несколько слов на других наречиях, которые были им знакомы.
\vs p194 1:3 Еврейских лидеров поразила смелость апостолов, но они побоялись мешать им из\hyp{}за большого числа поверивших рассказу апостолов.
\vs p194 1:4 К половине пятого больше двух тысяч уверовавших последовало за апостолами к Силоамской купели, где Петр, Андрей, Иаков и Иоанн крестили их во имя Учителя. Когда же апостолы закончили крестить эту толпу, уже стемнело.
\vs p194 1:5 Пятидесятница была великим праздником крещения, днем усыновления прозелитов врат --- тех неевреев, которые хотели служить Яхве. Поэтому в этот день окрестить большое число и евреев, и верующих неевреев было проще. Поступая так, они никоим образом не отделяли себя от еврейской веры. Впоследствии в течение некоторого времени верующие в Иисуса являлись сектой в иудаизме. Все они, включая апостолов, по\hyp{}прежнему были верны основным требованиям еврейской системы обрядов.
\usection{2. Значение Пятидесятницы}
\vs p194 2:1 Иисус жил на земле и учил евангелию, которое избавляло человека от суеверия, будто он, человек, был порождением дьявола, и возводило человека в высокое достоинство верующего сына Бога. Во время жизни Иисуса его послание, каким он его проповедовал и согласно которому жил, эффективно умеряло духовные трудности человека. Теперь же, когда сам Иисус покинул мир, он вместо себя посылает свой Дух Истины, который должен жить в человеке и для каждого нового поколения заново формулировать послание Иисуса, так что каждое следующее поколение смертных, живущих на земле, будет обладать новой и современной версией евангелия, таким личным озарением и таким коллективным водительством, которое эффективно развеет любые вечно новые духовные трудности.
\vs p194 2:2 \pc Первая цель этого духа состоит конечно же в том, чтобы пестовать и персонализировать истину, ибо понимание истины и есть высшая форма человеческой свободы. Другое назначение этого духа --- разрушение чувства сиротства у верующих. При том, что в прошлом Иисус жил среди людей, все верующие испытывали бы чувство одиночества, если бы Дух Истины не стал жить в человеческих сердцах.
\vs p194 2:3 Это дарование духа Сына фактически подготовило умы всех нормальных людей к последующему всеобщему дарованию духа Отца (Настройщика) всему человечеству. В известном смысле сей Дух Истины является духом и Отца Всего Сущего, и Сына\hyp{}Творца.
\vs p194 2:4 Не делайте ошибку, ожидая, что вы интеллектуально ощутите излившийся Дух Истины. Дух никогда не создает ощущение присутствия самого себя, но лишь ощущение присутствия Михаила, то есть Сына. С самого начала Иисус учил, что дух не будет говорить от себя. Следовательно, доказательства вашего родства с Духом Истины нужно искать не в вашем ощущении этого духа, а в вашем опыте близкого родства с Михаилом.
\vs p194 2:5 Дух пришел и затем, чтобы помогать людям вспоминать и понимать слова Учителя, равно как и освещать и заново толковать его жизнь на земле.
\vs p194 2:6 Кроме того, Дух Истины пришел помогать верующим свидетельствовать о реальностях учений Иисуса и его жизни, как он прожил ее во плоти, и как теперь опять заново и снова живет ей в каждом отдельно взятом верующем каждого последующего поколения исполненных духом сынов Бога.
\vs p194 2:7 Таким образом, представляется, что Дух Истины действительно приходит затем, чтобы привести всех верующих к полноте истины, к расширенному знанию опыта живого и растущего духовного осознания реальности вечного и восходящего сыновства по отношению к Богу.
\vs p194 2:8 \pc Иисус жил жизнью, которая суть откровение о человеке, подчиненном воле Отца, а не пример, которому должен буквально следовать человек. Эта жизнь во плоти, наряду с его смертью на кресте и последующим воскресением, вскоре стали новым евангелием о жертве, принесенной, дабы искупить человека из когтей нечистого --- от осуждения оскорбленного Бога. Несмотря на то, что евангелие и претерпело сильное искажение, фактом остается и то, что это новое послание об Иисусе несло в себе множество фундаментальных истин и положений его изначального евангелия царства. И рано или поздно эти сокрытые истины об отцовстве Бога и братстве людей откроются и действительно преобразуют всю человеческую цивилизацию.
\vs p194 2:9 Однако эти ошибки интеллектуального плана никоим образом не повлияли на великий духовный прогресс верующих. После дарования Духа Истины менее чем за месяц апостолы достигли большего индивидуального духовного развития, чем за почти четыре года личного и полного любви общения с Учителем. Совсем не мешала быстрому распространению их учений и эта подмена \bibemph{фактом} воскресения Иисуса \bibemph{истины} спасительного евангелия о сыновстве по отношению к Богу; напротив, то, что послание Иисуса было вытеснено новыми учениями о его личности и воскресении, казалось бы, крайне способствовало проповеди благой вести.
\vs p194 2:10 \pc Понятие о «крещении духом», столь глубоко и повсеместно вошедшее в обиход приблизительно в это время, просто означало сознательное принятие этого дара Духа Истины и личное признание этой новой духовной силы, усугубляющей все духовные влияния, ранее испытываемые душами, познавшими Бога.
\vs p194 2:11 \pc С момента дарования Духа Истины человек подчинен учению и водительству троичного духовного дара: духа Отца --- Настройщика Мысли, духа Сына --- Духа Истины, духа Духа --- Святого Духа.
\vs p194 2:12 В известном смысле человечество подвержено двойному влиянию семеричного призыва духовных сил вселенной. Первые эволюционные расы смертных подчинены совершенствующейся связи семи духов\hyp{}помощников разума Духа Матери локальной вселенной. По мере продвижения человека вверх по ступеням разума и духовного восприятия, семь высших духовных влияний в конце концов сближаются, чтобы пребывать над ним и быть в нем. И эти семь духов все более совершенных миров таковы:
\vs p194 2:13 \ublistelem{1.}\bibnobreakspace Дарованный дух Отца Всего Сущего --- Настройщики Мысли.
\vs p194 2:14 \ublistelem{2.}\bibnobreakspace Духовное присутствие Вечного Сына --- духовная гравитация вселенной вселенных и определенный канал всякого духовного общения.
\vs p194 2:15 \ublistelem{3.}\bibnobreakspace Духовное присутствие Бесконечного Духа --- вселенский дух\hyp{}разум всего творения, духовный источник интеллектуального родства всех совершенствующихся разумных существ.
\vs p194 2:16 \ublistelem{4.}\bibnobreakspace Дух Отца Всего Сущего и Сына\hyp{}Творца --- Дух Истины, обычно рассматриваемый как дух Вселенского Сына.
\vs p194 2:17 \ublistelem{5.}\bibnobreakspace Дух Бесконечного Духа и Вселенской Духа\hyp{}Матери --- Святой Дух, обычно рассматриваемый как дух Вселенского Духа.
\vs p194 2:18 \ublistelem{6.}\bibnobreakspace Дух\hyp{}разум Вселенской Духа\hyp{}Матери --- семь духов\hyp{}помощников разума локальной вселенной.
\vs p194 2:19 \ublistelem{7.}\bibnobreakspace Дух Отца, Сынов и Духов --- новонареченный дух идущих по пути восхождения смертных после слияния рожденной от духа души смертного с Райским Настройщиком Мысли и последующего достижения божественности и прославления на уровне Райского Отряда Финальности.
\vs p194 2:20 \pc Итак, дарование Духа Истины принесло в мир и всем народам последние из духовных даров, предназначенные оказать помощь в восходящих исканиях Бога.
\usection{3. Что произошло в день Пятидесятницы}
\vs p194 3:1 С первыми повествованиями о дне Пятидесятницы стали связывать многие странные и необычные учения. Впоследствии события этого дня, в который Дух Истины, новый учитель, сошел пребывать в человечестве, стали путать с безрассудными всплесками бурных эмоций. Основная миссия этого излившегося духа Отца и Сына --- учить людей истинам о любви Отца и милосердии Сына. Таковы истины божественности, которые человек может понять полнее, нежели все остальные божественные черты характера. Главная задача Духа Истины --- открывать духовную природу Отца и нравственную сущность Сына. Сын\hyp{}Творец, живя во плоти, открыл людям Бога; Дух Истины, живущий в сердце, открывает людям Сына\hyp{}Творца. Принося «плоды духа» в своей жизни, человек просто проявляет черты, явленные Учителем в его земной жизни. Когда Иисус был на земле, то жил своей жизнью как цельная личность --- Иисус из Назарета. Как дух «нового учителя», пребывающий в человеке, Учитель после Пятидесятницы получил возможность прожить свою жизнь заново в опыте каждого наставленного истиной верующего.
\vs p194 3:2 Многое из происходящего в человеческой жизни сложно понять, трудно совместить с мыслью о том, что такова вселенная, где преобладает истина, а праведность побеждает. Ведь так часто кажется, что клевета, ложь, нечестность и неправедность --- грех --- торжествуют. Неужели вера все\hyp{}таки побеждает зло, грех и порочность? Да, это так. Жизнь и смерть Иисуса --- вот вечное доказательство того, что истина добродетели и вера ведомого духом творения всегда будут оправданы. Над Иисусом, распятом на кресте, глумились, говоря: «Посмотрим, придет ли и спасет ли его Бог». День распятия был мрачен, но утро воскресения было светло и прекрасно, а день Пятидесятницы еще радостнее и светлее. Религии пессимистического отчаяния стремятся освободиться от тягот жизни, пытаясь найти успокоение в бесконечном сне и покое. Это религии примитивного страха и ужаса. Религия же Иисуса --- это новое евангелие веры, которое должно провозглашаться борющемуся человечеству. Сия новая религия основана на вере, надежде и любви.
\vs p194 3:3 Жизнь в облике смертного нанесла Иисусу самые тяжелые, самые жестокие и самые страшные удары; и этот человек встретил эти горести с верой, отвагой и непоколебимой решимостью исполнять волю своего Отца. Иисус претерпел все ужасные реалии жизни и вышел победителем, даже в смерти. Он не использовал религию как способ ухода от жизни. Религия Иисуса вовсе не старается уйти от жизни, чтобы насладиться блаженством ожидания иного бытия. Религия Иисуса дает радость и мир другого, духовного бытия, чтобы обогатить и одухотворить жизнь, которой сейчас живут люди во плоти.
\vs p194 3:4 Если религия --- опиум для народа, значит, это не религия Иисуса. На кресте он отказался пить одурманивающее снадобье, и его дух, излившийся на всякую плоть, есть могучая мировая сила, увлекающая человека вверх, ведущая его вперед. Духовное побуждение к совершенствованию --- вот самая мощная движущая сила, присутствующая в этом мире; верующий, познающий истину, --- вот совершенствующаяся и активная душа на земле.
\vs p194 3:5 В день Пятидесятницы религия Иисуса разрушила все национальные ограничения и расовые барьеры. «Где дух Господень, там свобода» --- вот слова, истинные навсегда. В этот день Дух Истины стал личным даром Учителя каждому смертному. Этот дух был дарован, чтобы подготовить верующих к более высокому уровню проповедования евангелия царства, но они ошибочно поняли опыт принятия излившегося духа за часть нового евангелия, которое они неосознанно создавали.
\vs p194 3:6 \pc Не упускайте из вида, что Дух Истины был дарован всем искренне верующим; этот дар духа предназначался не только апостолам. Сто двадцать мужчин и женщин, собравшихся в комнате наверху, приняли нового учителя, как сделали это во всем мире все, у кого было чистое сердце. Этот новый учитель был дарован всему человечеству, и каждая душа приняла его в меру своей любви к истине и своей способности понимать и постигать духовные реалии. Наконец, истинная религия освобождена от опеки первосвященников и всех священных каст и находит свое подлинное воплощение в душах конкретных людей.
\vs p194 3:7 \pc Религия Иисуса способствует развитию высшего типа человеческой цивилизации, ибо она создает высочайший тип человеческой личности и провозглашает священность такого человека.
\vs p194 3:8 Пришествие Духа Истины в день Пятидесятницы сделало возможной религию, которая не радикальна, не консервативна; она не нова и не стара; в ней не должны главенствовать ни старые, ни молодые. Сам факт земной жизни Иисуса дает фиксированную точку отсчета времени, тогда как дарование Духа Истины обеспечивает вечное развитие и бесконечный рост религии, по которой он жил, и евангелия, которое он возвестил. Дух наставляет во \bibemph{всякой} истине; он --- учитель развивающейся и вечно растущей религии бесконечного совершенствования и божественных открытий. Сей новый учитель будет всегда открывать верующему, ищущему истину, то, что столь божественно сокрыто в личности и природе Сына Человеческого.
\vs p194 3:9 События, связанные с дарованием «нового учителя», и принятие проповеди апостолов представителями различных рас и народов, собравшимися в Иерусалиме, свидетельствуют об универсальности религии Иисуса. Евангелие царства не должно было быть связано с какой\hyp{}либо одной расой, культурой или языком. И этот день Пятидесятницы стал свидетелем великой попытки духа освободить религию Иисуса от унаследованных ею еврейских оков. Но даже после этого излияния духа на всякую плоть апостолы сначала пытались навязать требования иудаизма тем, кого они обратили. Даже у Павла были трения со своими иерусалимскими братьями, потому что он отказался подчинять неевреев этим еврейским обычаям. Ни одна богооткровенная религия не может распространиться по всему миру, если допускает серьезную ошибку, связывая себя с какой\hyp{}либо национальной культурой или с установившимися расовыми, социальными или экономическими обычаями.
\vs p194 3:10 Дарование Духа Истины не зависело от каких бы то ни было форм, обрядов, священных мест и особого поведения тех, кто принял полноту его проявления. Когда дух снизошел на собравшихся в комнате наверху, они там просто сидели и молча молились. Дарование духа за пределами города происходило так же, как в нем самом. Чтобы принять дух, апостолам вовсе не требовалось удаляться в уединенное место и годами предаваться уединенным размышлениям. Пятидесятница навсегда отделила идею духовного переживания от понятия особо благоприятных мест.
\vs p194 3:11 \pc Пятидесятница с ее духовным дарованием была предназначена для того, чтобы навсегда освободить религию Учителя от всякой зависимости от физической силы; учителя этой новой религии теперь вооружены духовным оружием. Они должны идти и покорять мир неизменным прощением, несравненным благоволением и бесконечной любовью. Они вооружены, чтобы преодолевать добром зло, побеждать ненависть любовью, разрушать страх смелой и живой верой в истину. Иисус уже научил своих последователей, что его религия никогда не была пассивной; его ученики всегда должны были быть активными и уверенными в своем милосердном служении и в проявлениях своей любви. Эти верующие больше не смотрели на Яхве как на «Господа Воинств». Теперь они смотрели на вечное Божество как на «Бога и Отца Господа Иисуса Христа». И, по крайней мере, сделали этот шаг вперед, даже если в известной мере и не сумели полностью осознать истину, что Бог в равной мере есть духовный Отец каждого человека.
\vs p194 3:12 Пятидесятница одарила смертного человека силой прощать личные обиды, сохранять доброту среди величайшей несправедливости, оставаться непоколебимым перед лицом страшной опасности и любовью и терпением бросать вызов ненависти и злобе. В своей истории Урантия прошла через бури великих и разрушительных войн. Все участники этих ужасных битв потерпели поражение. Победителем был только один; только один вышел из этих жестоких сражений с возросшей славой --- это был Иисус из Назарета и его евангелие победы добра над злом. Секрет лучшей цивилизации кроется в учениях Учителя о братстве людей, благоволении любви и взаимном доверии.
\vs p194 3:13 Вплоть до Пятидесятницы религия открывала лишь человека, ищущего Бога; после Пятидесятницы человек по\hyp{}прежнему ищет Бога, но над миром сияет зрелище Бога, также ищущего человека и посылающего свой дух пребывать в человеке, когда находит его.
\vs p194 3:14 \pc До появления учений Иисуса, достигших высшего проявления в день Пятидесятницы, женщина в догматах старых религий в духовном смысле занимала скромное положение или вовсе не имела его. После Пятидесятницы в братстве царства женщина перед Богом обрела равное положение с мужчиной. Среди ста двадцати принявших это особое нисхождение духа было много учеников\hyp{}женщин, и они поровну делили эти благословения с верующими мужчинами. Мужчина более не смеет монополизировать исполнения религиозной службы. Фарисей мог продолжать благодарить Бога за то, что он «не родился женщиной, прокаженным или неевреем», но среди последователей Иисуса женщина была навсегда освобождена от всякой религиозной дискриминации по половому признаку. Пятидесятница уничтожила всякую религиозную дискриминацию, основанную на расовых отличиях, культурных различиях, принадлежности к определенной касте общества или предубежденности к женщинам. Неудивительно, что верующие в новую религию воскликнут: «Где дух Господень, там свобода».
\vs p194 3:15 \pc И мать и брат Иисуса присутствовали среди ста двадцати верующих и как принадлежащие к этой группе учеников также приняли излившийся дух. Благого дара они получили отнюдь не больше, чем их собратья. Членам земной семьи Иисуса не было даровано какого\hyp{}либо особого дара. Пятидесятница ознаменовала конец особого священства и всякой веры в священные семьи.
\vs p194 3:16 \pc До Пятидесятницы апостолы от многого отказывались ради Иисуса. Они пожертвовали своими домами, семьями, друзьями, мирскими благами и занимаемым положением. В день Пятидесятницы они отдали себя Богу, и Отец и Сын в ответ отдали себя человеку --- ниспослав свои духи пребывать в людях. Этот опыт отказа от собственного «я» и обретения духа не был эмоциональным переживанием, а был актом сознательной самоотдачи и полного посвящения.
\vs p194 3:17 Пятидесятница была призывом к духовному единству верующих в евангелие. Когда дух снизошел на учеников в Иерусалиме, то же самое произошло в Филадельфии, Александрии и во всех остальных местах, где жили истинно верующие. То, что «у множества верующих было одно сердце и одна душа», было истинно буквально. Религия Иисуса --- самая мощная объединяющая сила, которую когда\hyp{}либо знал мир.
\vs p194 3:18 \pc Пятидесятница должна была служить снижению стремления отдельных людей, групп, наций и рас отстаивать свои притязания. Этот дух отстаивания своих притязаний увеличивает напряженность настолько, что она периодически переходит в разрушительные войны. Человечество можно объединить лишь на духовной основе, и Дух Истины --- это та мировая сила, которая универсальна.
\vs p194 3:19 Пришествие Духа Истины очищает человеческое сердце и ведет того, кто его принял, к формированию единственной цели в жизни --- исполнять волю Отца и заботиться о благополучии людей. Материальный дух себялюбия был поглощен этим новым духовным даром самоотвержения. И тогда и теперь Пятидесятница означает, что Иисус как историческая личность стал божественным Сыном живого опыта. Радость, которую дает этот излившийся дух, сознательно испытываемый в человеческой жизни, укрепляет здоровье, стимулирует ум и неизменно наполняет душу энергией.
\vs p194 3:20 \pc В день Пятидесятницы не молитва принесла дух, но она в значительной степени определила способность к его восприятию, которая и была свойственна отдельным верующим. Вовсе не молитва побуждает божественное сердце к щедрости дарования, однако она так часто расширяет и углубляет каналы, по которым божественные дары могут проникать в сердца и души тех, кто таким образом помнит о поддержании неразрывной связи со своим Творцом через искреннюю молитву и истинное почитание.
\usection{4. Начала христианской церкви}
\vs p194 4:1 Когда Иисус был столь неожиданно схвачен своими врагами и так быстро распят между двумя разбойниками, его апостолы и ученики были совершенно деморализованы. Мысль об Учителе, арестованном, связанном, подвергнутом бичеванию и распятом, была невыносимой даже для апостолов. Они забыли его учение и его предостережения. Он действительно мог быть «пророком в деле и славе пред Богом и всем народом», но едва ли мог быть Мессией, который, они надеялись, восстановит царство Израиля.
\vs p194 4:2 И вот приходит воскресение с его избавлением от отчаяния и возвращением к вере в божественность Учителя. Они снова и снова видят его и беседуют с ним, а он ведет их на Масличную гору, где прощается с ними и говорит им, что возвращается к Отцу. Он велит им оставаться в Иерусалиме до времени, когда им будет дарована сила --- до времени пришествия Духа Истины. И вот в день Пятидесятницы сей новый учитель приходит, и они сразу идут с новой силой проповедовать свое евангелие. Они --- смелые и отважные последователи живого Господа, а не мертвого и поверженного вождя. Учитель живет в сердцах этих евангелистов; Бог --- это не интеллектуальное учение; он стал живым присутствием в их душах.
\vs p194 4:3 «День за днем они единодушно пребывали во храме и преломляли хлеб дома. И вкушали свою пищу в радости и простоте сердца, хваля Бога и находясь в любви у всего народа. И исполнились духа и говорили слово Божье с дерзновением. У множества же уверовавших было одно сердце и одна душа; и никто ничего из имения своего не называл своим, и все у них было общее».
\vs p194 4:4 \pc Что же случилось с этими людьми, кого Иисус посвятил идти, проповедуя евангелие царства, отцовство Бога и братство людей? У них есть новое евангелие; они возгорелись новыми чувствами и исполнились новой духовной энергии. Их послание неожиданно сменилось провозглашением воскресшего Христа: «Иисуса из Назарета, человека, которого Бог отметил великими делами и знамениями; его, по определенному замыслу и предназначению Бога преданного, вы распяли и убили. Что Бог предвозвестил устами всех пророков, то он тем исполнил. Сего Иисуса Бог воскресил. Бог сделал его и Господом, и Христом. Быв вознесен десницею Божьей и приняв от Отца обетование духа, он излил то, что вы видите и слышите. Покайтесь, чтобы загладились грехи ваши; чтобы Отец мог послать Христа, предназначенного вам, и Иисуса, которого небеса должны принять до времени возрождения всего сущего».
\vs p194 4:5 Евангелие царства, послание Иисуса, было внезапно превращено в евангелие Господа Иисуса Христа. Теперь они провозглашали факты, касающиеся его жизни, смерти и воскресения, и проповедовали надежду на его скорое возвращение в этот мир и завершение дела, им начатого. Таким образом, послание первых верующих являло собой проповедь фактов, касающихся его первого пришествия, и учение о надежде на его второе пришествие, события, которое они считали делом ближайшего будущего.
\vs p194 4:6 Христос очень скоро превратился в символ веры быстро формировавшейся церкви. Иисус жив; он умер за людей; он даровал свой дух; он возвращается снова. Иисус занимал все их мысли и определял все в их новом понятии Бога и во всем остальном. Они так восторгались новым учением, что «Бог есть Отец Господа Иисуса», что забыли прежнее послание о том, что «Бог --- любящий Отец всех людей» и каждого человека в отдельности. Правда, в этих первых общинах верующих возникали чудесные проявления братской любви и беспримерной добродетели. Однако то была общность верующих в Иисуса, а не общность братьев в царстве семьи Отца Небесного. Их доброта проистекала из любви, рожденной представлением о пришествии Иисуса, а не признанием братства смертных людей. Тем не менее их переполняла радость и жили они столь новой и особенной жизнью, что всех людей привлекали их проповеди об Иисусе. Они совершили великую ошибку, используя живой и наглядный комментарий к евангелию вместо евангелия, но даже и этого было достаточно, чтобы создать; величайшую религию, которую когда\hyp{}либо знало человечество.
\vs p194 4:7 Очевидно, что в мире возникала новая общность. «Множество уверовавших постоянно слушало проповеди и общалось с апостолами, преломляло с ними хлеб и молилось». Они называли друг друга братьями и сестрами, приветствовали друг друга святым поцелуем, служили бедным. Это была общность жизни, общность поклонения. Они были общиной не по указу, но благодаря желанию разделить свое имущество со своими собратьями\hyp{}верующими. Они с уверенностью ожидали, что Иисус вернется, дабы совершить установление царства Отца еще при жизни их поколения. Это добровольное разделение земного имущества не проистекало из учения Иисуса. Оно произошло потому, что эти мужчины и женщины так искренне и непоколебимо верили, что он должен в ближайшее время вернуться, чтобы увенчать свое дело --- установить царство. Однако конечные результаты проявлений бездумной братской любви, исполненных из лучших побуждений, были катастрофическими и плачевными. Тысячи искренне верующих распродали свое имущество и избавились от всех орудий труда и других материальных ценностей. Со временем сокращавшиеся ресурсы христианского «равного распределения» подошли \bibemph{к концу ---} но мир не прекратил своего существования. И очень скоро верующие Антиохии собирали пожертвования, дабы спасти от голода своих собратьев\hyp{}верующих в Иерусалиме.
\vs p194 4:8 \pc В эти дни они праздновали Вечерю Господню так же, как она была учреждена; то есть собирались на общую дружескую трапезу и в завершении трапезы причащались таинствам.
\vs p194 4:9 \pc Сначала они крестили во имя Иисуса и лишь через двадцать лет начали крестить во имя Отца, Сына и Святого Духа. Крещение --- это все, что требовалось для принятия в братство верующих. У них еще не было организации; это было просто братство Иисуса.
\vs p194 4:10 \pc Эта секта последователей Иисуса быстро росла, и саддукеи снова обратили на них внимание. Фарисеев же возникшая ситуация почти не беспокоила, ведь ни одно из учений секты никоим образом не сказывалось на соблюдении еврейских законов. А саддукеи начали сажать в тюрьмы вождей секты последователей Иисуса, но перестали делать это, приняв совет Гамалиила, одного из выдающихся раввинов, который сказал им: «Отстаньте от людей сих и оставьте их, ибо если это дело от людей, то оно разрушится; а если от Бога, то вы не сможете сокрушить их; и может случиться, что вы боретесь с Богом». Саддукеи решили последовать совету Гамалиила, и в Иерусалиме наступило время мира и спокойствия, в течение которого новое евангелие об Иисусе быстро распространялось.
\vs p194 4:11 Итак, в Иерусалиме все было благополучно до тех пор, пока из Александрии не прибыло большое число греков. В Иерусалим прибыли два ученика Родана и обратили многих эллинов. Среди первых обращенных были Стефан и Варнава. Эти способные греки мало придерживались еврейской точки зрения и не во всем соблюдали еврейские обряды поклонения и другие ритуальные обычаи. Деяния этих греков и положили конец мирным отношениям братства Иисуса с фарисеями и саддукеями. Стефан и его товарищ\hyp{}грек начали в большей степени проповедовать, как учил Иисус, и это втянуло их в конфликт непосредственно с еврейскими правителями. Во время одной из публичных проповедей Стефана, когда в своей речи он коснулся вопроса, который вызвал возражения, те, обошлись без всяких судебных формальностей и его прямо на месте до смерти забили камнями.
\vs p194 4:12 Таким образом, Стефан, глава греческой колонии иерусалимских верующих, стал первым мучеником, пострадавшим за новую веру, и это послужило конкретным поводом для последователей Иисуса, чтобы создать формальную организацию церкви первых христиан. Этот новый кризис был вызван осознанием того, что верующие не могут оставаться сектой в среде еврейской веры. Все они согласились, что им необходимо отделиться от неверующих; и через месяц после смерти Стефана в Иерусалиме под руководством Петра была основана церковь, титулярным главой которой был назначен Иаков, брат Иисуса.
\vs p194 4:13 Тогда и начались новые и безжалостные преследования со стороны евреев, которые привели к тому, что активные проповедники новой религии об Иисусе, которая впоследствии в Антиохии была названа христианством, пошли в отдаленные концы империи, провозглашая Иисуса. До времени Павла руководство в распространении этой вести находилось в руках у греков; и эти первые миссионеры так же, как и их более поздние последователи, пошли тем же путем, что и войско Александра в былые времена, --- через Газу и Тир в Антиохию, затем через Малую Азию в Македонию, а потом в Рим и самые отдаленные окраины империи.
