\upaper{189}{Воскресение}
\author{Комиссия срединников}
\vs p189 0:1 Вскоре после погребения Иисуса, состоявшегося в пятницу после полудня, глава архангелов Небадона, которые в то время присутствовали на Урантии, созвал свой совет по воскрешению спящих созданий, наделенных волей, и приступил к рассмотрению вопроса о возможном способе восстановления Иисуса. Эти сыновья локальной вселенной, создания Михаила, собрались по своей собственной инициативе; Гавриил их не созывал. К полуночи они пришли к заключению, что создание не может ничего сделать, дабы способствовать воскресению Творца. И склонились к тому, чтобы последовать совету Гавриила, который наставлял их, что, поскольку Михаил «отдал свою жизнь по собственной воле, в его власти взять ее снова по своему собственному решению». Вскоре после завершения совета архангелов, Носителей Жизни и их разнообразных сподвижников в деле восстановления создания и моронтийного творения Персонализированный Настройщик Иисуса, который лично командовал небесными воинствами, собравшимися в то время на Урантии, сказал нетерпеливо ожидающим наблюдателям такие слова:
\vs p189 0:2 «Ни один из вас не может ничего сделать, чтобы помочь вашему Отцу\hyp{}творцу вернуться к жизни. Как смертный планеты он претерпел смерть, которой умирают смертные, но как Владыка вселенной он по\hyp{}прежнему жив. То, что вы наблюдаете, есть переход из смертного состояния Иисуса из Назарета, переход из жизни во плоти в жизнь в моронтии. Достижение же духовного состояния Иисусом завершилось в момент, когда я отделился от его личности и стал вашим временным руководителем. Ваш Творец\hyp{}отец решил пройти весь путь своих смертных созданий, от рождения в материальных мирах до естественной смерти, моронтийного воскресения и далее до состояния истинно духовного бытия. Определенную фазу этого пути вы вскоре увидите, но участвовать в ней не сможете. Что вы обычно делаете для создания, того не можете делать для Творца. Во власти Сына\hyp{}Творца совершить пришествие в подобии любого из сотворенных им сыновей; в его власти отдать свою видимую жизнь и взять ее снова, причем имеет эту власть по прямому указанию Райского Отца; я же знаю, о чем говорю».
\vs p189 0:3 Услышав сказанное Персонализированным Настройщиком, все они, от Гавриила до смиреннейшего херувима, замерли в тревожном ожидании. Они видели смертное тело Иисуса в гробнице, различали признаки вселенской деятельности своего возлюбленного Владыки и, не понимая подобных явлений, терпеливо ждали развития событий.
\usection{1. Переход в моронтийное состояние}
\vs p189 1:1 В воскресенье в два часа сорок пять минут ночи Райская комиссия по воплощению, состоявшая из семи неидентифицированных Райских личностей, прибыла к месту событий и тотчас же расположилась возле гробницы. Без десяти минут три из новой гробницы Иосифа начали исходить интенсивные сотрясения, вызванные процессами материальной и моронтийной деятельности, и в три часа две минуты воскресного утра 9 апреля 30 года н.э. воскресшее моронтийное тело и личность Иисуса из Назарета вышли из гробницы.
\vs p189 1:2 После того, как воскресший Иисус вышел из своей гробницы, тело из плоти, в котором он жил и трудился на земле почти тридцать шесть лет, оставалось лежать в нише гробницы, завернутое в полотняную пелену в том же положении, в каком его оставили Иосиф со своими товарищами в пятницу после полудня. Ни на йоту не был сдвинут и камень у входа в гробницу; печать Пилата оставалась неповрежденной, а солдаты по\hyp{}прежнему несли караул. Храмовые стражи несли непрерывное дежурство, а римского часового сменили в полночь. Ни один из стражей не заподозрил, что объект, из\hyp{}за которого они не спали, вступил в новую и более высокую форму бытия и что тело, которое они охраняли, теперь было сброшенной внешней оболочкой, более не имевшей никакого отношения к освобожденной и воскресшей моронтийной личности Иисуса.
\vs p189 1:3 \pc Человечеству трудно понять, что во всем, что является личным, материя --- это остов моронтии и обе они --- отраженная тень бессмертной духовной реальности. Когда же наконец вы будете рассматривать время как движущееся отражение вечности, а пространство --- как скоротечную тень Райских реальностей?
\vs p189 1:4 Насколько мы можем судить, ни одно создание этой вселенной и ни одна личность из другой вселенной не имела никакого отношения к этому моронтийному воскресению Иисуса из Назарета. В пятницу он отдал свою жизнь как смертный царства, а в воскресенье утром взял ее снова как моронтийное существо системы Сатания в Норлатиадеке. В воскресении Иисуса есть много такого, чего мы понять не можем. Однако мы знаем, что оно произошло так, как мы утверждаем, и приблизительно в указанное время. Мы можем также записать, что все известные явления, связанные с этим переходом из смертного состояния, или моронтийным воскресением, произошли непосредственно здесь, в новой гробнице Иосифа, где прах Иисуса лежал завернутый в погребальные одежды.
\vs p189 1:5 Мы знаем, что ни одно создание локальной вселенной не участвовало в этом моронтийном пробуждении. Мы ощущали, что гробницу окружали семь Райских личностей, но не видели, чтобы они что\hyp{}либо предпринимали для пробуждения Учителя. Как только Иисус появился рядом с Гавриилом, семь Райских личностей, находившихся над самой гробницей, выразили намерение немедленно отправиться в Уверсу.
\vs p189 1:6 Позвольте нам на веки вечные разъяснить понятие о воскресении Иисуса, заявив следующее:
\vs p189 1:7 \ublistelem{1.}\bibnobreakspace Материальное или физическое тело Иисуса не было частью воскресшей личности. Когда Иисус вышел из гробницы, его тело из плоти оставалось на месте своего погребения нетронутым. Из гробницы он вышел, не сдвинув камень перед входом и не повредив печатей Пилата.
\vs p189 1:8 \ublistelem{2.}\bibnobreakspace Иисус не выходил из гробницы ни как дух, ни как Михаил из Небадона; не явился он и в обличии Владыки\hyp{}Творца, которым он обладал до своего воплощения в подобии смертной плоти на Урантии.
\vs p189 1:9 \ublistelem{3.}\bibnobreakspace Иисус покинул гробницу Иосифа в точном подобии моронтийных личностей тех, кто как воскрешенные моронтийные существа, идущие по пути восхождения, выходит из залов воскрешения первого мира\hyp{}обители локальной вселенной Сатания. И присутствие памятника Михаилу в центре огромного двора залов воскрешения мира\hyp{}обители номер один наводит нас на мысль, что воскресению Учителя на Урантии некоторым образом благоприятствовал этот первый из миров\hyp{}обителей системы.
\vs p189 1:10 \pc Первое, что сделал Иисус, восстав из гробницы, --- поприветствовал Гавриила и велел ему продолжать руководить делами вселенной под началом Иммануила, а затем дал указание главе Мелхиседеков передать Иммануилу свои братские приветствия. Вслед за этим он попросил Всевышнего Эдентии о свидетельстве Древних Дней, подтверждающем его прохождение смертной жизни; и, повернувшись к собравшимся моронтийным группам из семи миров\hyp{}обителей, которые сошлись здесь, дабы встретить и приветствовать своего Творца как создание их чина, произнес первые слова своего посмертного пути. Перешедший в моронтийное состояние Иисус сказал: «Завершив свою жизнь во плоти, я задержусь здесь на короткое время в переходной форме, чтобы полнее узнать жизнь моих восходящих созданий и еще больше раскрыть волю моего Отца в Раю».
\vs p189 1:11 Окончив говорить, Иисус подал знак Персонализированному Настройщику, и все разумные существа вселенной, собранные на Урантии, чтобы стать свидетелями воскресения, были тотчас отправлены исполнять свои обязанности во вселенной.
\vs p189 1:12 Познакомившись как создание с требованиями жизни, которой он решил прожить небольшое время на Урантии, Иисус теперь начал сообщаться на уровне моронтии. Это вступление в мир моронтии потребовало более часа земного времени и дважды прерывалось его желанием пообщаться с бывшими сотоварищами во плоти, когда те приходили из Иерусалима, чтобы с изумлением заглянуть в пустую гробницу и обнаружить то, что они сочли доказательством его воскресения.
\vs p189 1:13 Теперь переход Иисуса из смертного состояния в моронтийное воскресение Сына Человеческого --- свершился. Переходный опыт Учителя как личности, занимавшей промежуточное положение между материальным и духовным, начался. И все это он свершил силой, сокрытой в нем самом; ни одна личность не оказала ему никакой помощи. Теперь Иисус существует как перешедший в моронтийное состояние, и в то время как он начинает эту моронтийную жизнь, физическое тело его плоти лежит в гробнице нетронутым. Солдаты по\hyp{}прежнему несут караул, а печать правителя на камнях еще не повреждена.
\usection{2. Физическое тело Иисуса}
\vs p189 2:1 В три часа десять минут, когда воскресший Иисус общался с собравшимися моронтийными личностями из семи миров\hyp{}обителей Сатании, глава архангелов --- ангелов воскресения --- приблизился к Гавриилу и попросил отдать им смертное тело Иисуса. Глава архангелов сказал: «Мы не можем участвовать в моронтийном воскресении опыта пришествия Михаила, владыки нашего, но мы бы хотели взять на себя заботу о его смертных останках, подвергнув их немедленному распаду. Мы не предлагаем использовать наш метод дематериализации, а просто хотим прибегнуть к процессу ускорения времени. Достаточно того, что мы видели, как Владыка жил и умер на Урантии; воинства небесные будут избавлены от воспоминаний о том, что они претерпели, наблюдая картину медленного разложения человеческого тела Творца и Вседержителя вселенной. Именем всех небесных разумных существ всего Небадона я прошу указа дать мне право позаботиться о смертном теле Иисуса из Назарета и уполномочить нас приступить к его немедленному уничтожению».
\vs p189 2:2 Когда Гавриил провел переговоры со старшим Всевышним Эдентии, архангелу\hyp{}представителю небесных воинств было дано разрешение осуществить такую ликвидацию физических останков Иисуса, какую он найдет нужной.
\vs p189 2:3 После того, как просьба главы архангелов была удовлетворена, он призвал на помощь множество своих собратьев, многочисленное воинство представителей всех чинов небесных личностей и затем с помощью срединников Урантии завладел физическим телом Иисуса. Это смертное тело было сугубо материальным творением; это было в буквальном смысле физическое тело и оно не могло быть изъято из гробницы, как было способно выйти из запечатанного гроба воскресшее моронтийное тело. С помощью определенных моронтийных вспомогательных личностей моронтийное тело в какой\hyp{}то момент может становиться духом, так что оказывается нейтральным по отношению к обычной материи, а в другой --- видимым и осязаемым для материальных существ, таких, как смертные Урантии.
\vs p189 2:4 Когда они приготовились забрать тело Иисуса из гробницы, то перед тем, как величественно и почтительно уничтожить путем почти моментального распада, урантийским срединникам второго рода было поручено откатить камни от входа в гробницу. Больший из них --- огромный камень круглой формы был очень похож на жернов, который двигался по желобу, выдолбленному в скале, так что мог кататься по нему в двух направлениях, открывая и закрывая гробницу. Когда бодрствовавшие еврейские стражи и римские солдаты в слабом утреннем свете увидели, что этот огромный камень, казалось бы, сам по себе --- без каких бы то ни было видимых причин, вызывающих подобное движение, --- стал откатываться от входа в гробницу, их охватил ужас и паника, и они поспешно бежали с места событий. Евреи разбежались по своим домам, после чего вернулись в храм доложить о случившемся своему командиру. Римляне же убежали в крепость Антонию и доложили об увиденном центуриону, как только тот заступил на дежурство.
\vs p189 2:5 Еврейские лидеры начали грязное дело устранения, как они думали, Иисуса, дав взятку предателю Иуде, и теперь, столкнувшись с трудной ситуацией, вместо того, чтобы подумать о наказании стражей, бросивших свой пост, прибегли к подкупу этих стражей и римских солдат. Уплатив каждому из этих двадцати человек некоторую сумму денег, они им велели всем говорить: «Ночью, пока мы спали, его ученики напали на нас и унесли тело». Еврейские управители также дали солдатам торжественное обещание защитить их перед Пилатом в случае, если прокуратору когда\hyp{}нибудь станет известно, что они получили взятку.
\vs p189 2:6 \pc Христианская вера в воскресение Иисуса была основана на факте «пустой гробницы». То, что гробница была пуста, действительно, \bibemph{факт,} однако \bibemph{истина} воскресения отнюдь не в этом. Гробница была поистине пуста, когда к ней пришли первые верующие, и это обстоятельство в сочетании с фактом несомненного воскресения Учителя, привели к формированию веры, которая истинной не была, --- к учению, согласно которому физическое и смертное тело Иисуса восстало из могилы. Истину, имеющую отношение к духовным реальностям и вечным ценностям, не всегда можно установить с помощью комбинации очевидных фактов. Хотя отдельные факты и могут быть истинны в материальном плане, это еще не означает, что сочетание группы фактов должно обязательно приводить к истинно духовным заключениям.
\vs p189 2:7 Гробница Иисуса была пуста не потому, что тело Иисуса ожило или воскресло, но потому, что была удовлетворена просьба небесных воинств предоставить им возможность осуществить особый и уникальный распад, возможность возвращения «праха во прах» без задержек во времени и без действия обычных и видимых процессов смертного тления и материального разложения.
\vs p189 2:8 Смертный прах Иисуса подвергся тому же естественному процессу стихийного распада, что характерен для всех человеческих тел на земле, за исключением того обстоятельства, что время этого естественного процесса разложения было чрезвычайно сокращено и доведено до состояния, когда процесс прошел почти мгновенно.
\vs p189 2:9 Истинные доказательства воскресения Михаила по своей природе духовны, хотя это учение и подтверждается свидетельством многих смертных, которые встретили, узнали Учителя, перешедшего в моронтийное состояние, и общались с ним. Перед тем, как окончательно покинуть Урантию, он стал частью личного опыта почти тысячи человеческих существ.
\usection{3. Диспенсационное воскресение}
\vs p189 3:1 В это воскресное утро после половины пятого Гавриил собрал возле себя архангелов и приготовился торжественно объявить общее воскрешение --- окончание адамической диспенсации на Урантии. Когда огромное воинство серафимов и херувимов, которые участвовали в этом великом событии, было выстроено надлежащим порядком, Михаил, перешедший в моронтийное состояние, явился к Гавриилу и сказал: «Как Отец мой имеет жизнь в себе, так дано им и Сыну иметь в себе жизнь. Хоть я еще не окончательно вернулся к управлению делами вселенной, это добровольное самоограничение никоим образом не ограничивает дарование жизни моим спящим сыновьям; пусть начнется поверка планетарного воскрешения».
\vs p189 3:2 Тогда контур архангелов впервые действовал с Урантии. Гавриил и воинства архангелов переместились к месту духовной полярности планеты, и по сигналу Гавриила к первому из миров\hyp{}обителей системы вознесся глас Гавриила, глаголющий: «Согласно указу Михаила, да восстанут мертвые урантийской диспенсации!» Тогда все спасенные из человеческих родов Урантии, уснувшие со дней Адама и еще не ушедшие на суд, явились в залы воскрешения миров\hyp{}обителей, готовые к переходу в моронтийное состояние. И в одно мгновение серафимы и их сподвижники приготовились отправиться в миры\hyp{}обители. При обычных обстоятельствах эти серафимы\hyp{}хранительницы, однажды получившие назначение совместно охранять сих спасаемых смертных, присутствовали бы в момент их пробуждения в залах воскрешения миров\hyp{}обителей, однако в это время они находились в этом мире, что было вызвано необходимостью присутствия здесь Гавриила в связи с моронтийным воскресением Иисуса.
\vs p189 3:3 Несмотря на то, что бессчетное число смертных, имевших личных серафимов\hyp{}хранительниц, и тех, кто достиг необходимого уровня духовного совершенствования личности, пришли в миры\hyp{}обители за века со времен Адама и Евы, и хотя уже произошло много особых и тысячелетних воскрешений сыновей Урантии, это была третья планетарная поверка, или полное диспенсационное воскрешение. Первое произошло во время прибытия Планетарного Принца, второе --- во времена Адама, а это, третье, ознаменовало моронтийное воскресение Иисуса из Назарета, его переход из смертного состояния.
\vs p189 3:4 \pc Когда сигнал планетарного воскрешения был принят главой архангелов, Персонализированный Настройщик Сына Человеческого сложил свои полномочия над небесными воинствами, собравшимися на Урантии, вернув всех этих сыновей локальной вселенной под эгиду соответствующих начальников. Сделав же это, он отбыл в Спасоград, чтобы сообщить Иммануилу о завершении Михаилом перехода из смертного состояния. И тотчас же за ним последовало все небесное воинство, чье присутствие на Урантии не требовалось. Гавриил же остался на Урантии с перешедшим в моронтийное состояние Иисусом.
\vs p189 3:5 \pc Так живописуют события воскресения Иисуса те, кто видел их, когда они происходили, описание свободно от недостатков частичного и ограниченного человеческого восприятия.
\usection{4. Обнаружение пустой гробницы}
\vs p189 4:1 Говоря о времени воскресения Иисуса, произошедшего в воскресенье ранним утром, следует помнить, что десять апостолов находились в доме Илии и Марии Марк, где они спали в верхней комнате, отдыхая на тех же ложах, где они возлежали во время последней вечери со своим Учителем. В это воскресное утро они собрались здесь все, исключая Фому. Фома пробыл с ними только несколько минут поздно вечером в субботу, однако видеть апостолов и к тому же думать о том, что произошло с Иисусом, было выше его сил. Бросив взгляд на своих товарищей, он немедленно покинул комнату и пошел в дом Симона в Беф\hyp{}Фаге, где и хотел пережить свое горе в одиночестве. Все апостолы страдали не столько от сомнений и отчаяния, сколько от страха, горя и стыда.
\vs p189 4:2 \pc В доме Никодима вместе с Давидом Зеведеевым и Иосифом Аримафейским собрались то ли двенадцать, то ли пятнадцать наиболее выдающихся иерусалимских учеников Иисуса. В доме же Иосифа Аримафейского находилось пятнадцать --- двадцать наиболее выдающихся верующих женщин. В доме Иосифа Аримафейского присутствовали только эти женщины; днем и вечером в субботу они не выходили из дома, так что ничего не знали о вооруженной охране гробницы; не знали они и о том, что у входа в гробницу накатили второй камень и что на оба камня наложена печать Пилата.
\vs p189 4:3 В это воскресное утро на исходе второго часа, когда на востоке стала разгораться заря, пять женщин отправились к гробнице Иисуса. Они приготовили большое количество особых бальзамирующих помад и взяли с собой множество льняных бандажей. Они намеревались более тщательно забальзамировать мертвое тело Иисуса и обернуть его в новые бандажи.
\vs p189 4:4 Для помазания тела Иисуса шли: Мария Магдалина, Мария, мать близнецов Алфеевых, Соломея, мать братьев Зеведеевых, Иоанна, жена Хузы, и Сусанна, дочь Эзры Александрийского.
\vs p189 4:5 Было около половины четвертого, когда пять женщин с ношей подошли к пустой гробнице. Выходя из Дамасских ворот, они столкнулись с несколькими солдатами, бегущими в город в состоянии, близком к паническому, что вынудило их задержаться на несколько минут; когда же за этим ничего не последовало, женщины продолжили свой путь.
\vs p189 4:6 Увидев, что камень у входа в гробницу отодвинут, они очень удивились, поскольку, оправляясь в путь, гадали между собой: «Кто поможет нам откатить камень?» Положив свои ноши, они посмотрели друг на друга в страхе и с великим изумлением. Так они и стояли, дрожа от страха, пока Мария Магдалина не обошла меньший камень и не решилась войти в открытую усыпальницу. Гробница Иосифа находилась в его саду на склоне горы у восточной стороны дороги и также была обращена на восток. К этому часу света зари нового дня было достаточно, чтобы Мария смогла оглядеть место, где лежало тело Иисуса, и увидеть, что тело исчезло. В каменной нише, куда они положили Иисуса, Мария увидела лишь сложенный платок, который был на голове его, и бандажи, в которые он был завернут; они лежали нетронутыми, и в том же виде, в каком были оставлены на камне, когда небесные воинства забирали тело. Покрывальная пелена же лежала у подножия погребальной ниши.
\vs p189 4:7 Простояв несколько минут у входа в гробницу (она не видела отчетливо, когда впервые вошла в гробницу), она увидела, что тело Иисуса исчезло, а на его месте лежат лишь эти погребальные одежды, и в смятении отчаянно вскрикнула. Все женщины были чрезвычайно взволнованы; после встречи у городским ворот с охваченными паникой солдатами они нервничали, а когда Мария издала сей отчаянный вопль, исполнились ужаса и поспешно бежали. Они не останавливались до тех пор, пока не достигли Дамасских ворот. К этому моменту Иоанна почувствовала угрызения совести, вызванные тем, что они бросили Марию; она собрала своих спутниц, и они отправились обратно к гробнице.
\vs p189 4:8 Когда они приблизились к гробнице, испуганная Магдалина, которая испытала еще больший ужас, когда не застала своих сестер, ожидавших ее выхода из гробницы, теперь бросилась к ним, взволнованно восклицая: «Его там нет --- они забрали его!» Магдалина отвела их обратно к гробнице, куда они все вошли и увидели, что она пуста.
\vs p189 4:9 Тогда пять женщин сели на камень у входа и обсудили сложившееся положение. Им еще не пришло в голову, что Иисус воскрес. Всю субботу они провели одни и пришли к выводу, что тело перенесли в другую могилу. Однако предполагая такое решение возникшей перед ними проблемы, они не могли объяснить, почему были аккуратно сложены погребальные одежды; как могло быть убрано тело, если те самые бандажи, в которые оно было завернуто, были оставлены в том же положении и явно нетронутыми на погребальном ложе?
\vs p189 4:10 \pc Сидя там на заре этого нового дня, женщины посмотрели по сторонам и увидели молчаливого и неподвижного незнакомца. На мгновение они снова испугались, но Мария Магдалина бросилась к нему и, обратившись к нему так, как если бы перед ней был садовник, спросила: «Куда вы унесли Учителя? Где они положили его? Скажи нам, чтобы мы могли найти и взять его». Когда же незнакомец не ответил Марии, она заплакала. Тогда Иисус, обращаясь к ним, сказал: «Кого вы ищите?» Мария сказала: «Мы ищем Иисуса, которого похоронили в гробнице Иосифа, но его там нет. Знаешь ли ты, куда унесли его?» Тогда Иисус сказал: «Не говорил ли вам этот Иисус еще в Галилее, что он умрет, но снова воскреснет?» Эти слова поразили женщин, но Учитель так изменился, что они еще не узнавали его, к тому же в этот момент он стоял спиной к слабому свету. Пока же они обдумывали его слова, он обратился к Магдалине знакомым голосом и сказал: «Мария». Когда же та услышала это слово, произнесенное с хорошо знакомым участием и ласковой приветливостью, то поняла, что это был голос Учителя, и поспешила пасть на колени к его ногам, восклицая: «Господь мой и Учитель мой!» И остальные женщины поняли, что перед ними стоит Учитель во славе, и быстро опустились перед ним на колени.
\vs p189 4:11 Эти человеческие глаза получили возможность увидеть моронтийное тело Иисуса благодаря особому служению преобразователей и срединников, которые действовали в сотрудничестве с определенными моронтийными личностями, в то время сопровождавшими Иисуса.
\vs p189 4:12 \pc Когда же Мария попыталась обнять ноги Иисуса, он сказал: «Не прикасайся ко мне, Мария, ибо я не тот, каким ты знала меня во плоти. В этом теле я останусь с вами еще на какое\hyp{}то время прежде, чем вознесусь к Отцу. Теперь же ступайте и расскажите моим апостолам --- и Петру --- что я воскрес, и вы говорили со мной».
\vs p189 4:13 Оправившись от шока и изумления, женщины поспешили обратно в город к дому Илии Марка, где и рассказали десяти апостолам все, что случилось с ними; однако апостолы не хотели им верить. Сначала они подумали, что женщинам было видение; но когда Мария Магдалина повторила слова, которые Иисус сказал им, и Петр услышал свое имя, то с великой поспешностью бросился вон из верхней комнаты, а следом за ним и Иоанн, чтобы добраться до гробницы и увидеть все своими глазами.
\vs p189 4:14 Женщины повторили рассказ о разговоре с Иисусом остальным апостолам, но они им не поверили и не захотели пойти и самим убедиться во всем, как это сделали Петр и Иоанн.
\usection{5. Петр и Иоанн у гробницы}
\vs p189 5:1 Пока два апостола бежали на Голгофу к гробнице Иосифа, в мыслях Петра надежда сменялась страхом; он боялся встретить учителя, но рассказ о том, что Иисус передал ему особое послание, пробуждал в нем надежду. Он был почти убежден, что Иисус действительно жив, и вспомнил его обещание воскреснуть на третий день. Как ни странно, после распятия эти слова не приходили ему в голову до этой минуты, когда он бежал на север через Иерусалим. Когда Иоанн выбежал из города, его душа наполнилась каким\hyp{}то радостным восторгом и надеждой. Он был почти уверен, что женщины действительно видели воскресшего Учителя.
\vs p189 5:2 Иоанн, будучи моложе Петра, обогнал его и прибежал к гробнице первым. Иоанн остановился у входа и осмотрел гробницу; она была в том же состоянии, как описала ее Мария. Очень скоро подоспел и Петр и, войдя, увидел ту же пустую гробницу и странно уложенные погребальные пелены. Когда же Петр вышел наружу, Иоанн тоже вошел внутрь и все увидел своими глазами; тогда апостолы сели на камень, чтобы обдумать значение того, что они увидели и услышали. И, сидя там, мысленно возвращались к тому, что было сказано им об Иисусе, но не могли ясно понять, что же произошло.
\vs p189 5:3 Петр сначала предположил, что могила была ограблена, что враги выкрали тело, возможно, подкупив стражей. Но Иоанн рассудил, что вряд ли могилу оставили бы в таком порядке, если бы тело украли, и задался вопросом, как могло случиться, что бандажи были оставлены и притом совершенно нетронутыми. Тогда апостолы снова вошли в гробницу, чтобы внимательно осмотреть погребальные одежды. Выйдя из гробницы во второй раз, они встретили Марию Магдалину, которая вернулась и плакала у входа. Мария пришла к апостолам, веря, что Иисус воскрес из могилы, но когда все они отказались верить ее рассказу, опечалилась и впала в отчаяние. Ей хотелось вернуться к гробнице, где, как ей казалось, она слышала знакомый голос Иисуса.
\vs p189 5:4 После того, как Петр и Иоанн ушли, Мария задержалась и Учитель снова явился ей и сказал: «Не сомневайся; имей мужество и верь тому, что ты слышала и видела. Возвращайся к моим апостолам и снова возвести им, что я воскрес, что я явлюсь им и что скоро предварю их в Галилее, как и обещал».
\vs p189 5:5 Мария поспешила обратно к дому Марков и сказала апостолам, что она снова говорила с Иисусом, но те не хотели ей верить. Однако когда вернулись Петр и Иоанн, они прекратили насмешки и исполнились страха и опасений.
