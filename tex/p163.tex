\upaper{163}{Посвящение семидесяти в Магадане}
\author{Комиссия срединников}
\vs p163 0:1 Через несколько дней после возвращения Иисуса и двенадцати апостолов из Иерусалима в Магадан из Вифлеема прибыли Авенир с примерно пятьюдесятью учениками. В это же время в Магаданском лагере уже находились евангелисты, женский отряд и около ста пятидесяти других преданных и испытанных учеников со всех концов Палестины. Несколько дней было посвящено встречам и реорганизации лагеря, затем Иисус и двенадцать апостолов приступили к интенсивному обучению этой особой группы верующих, и уже из этих хорошо обученных и опытных учеников Учитель впоследствии выбрал семьдесят учителей и послал их возвещать евангелие царства. Это обучение началось в пятницу 4 ноября и продолжалось до субботы 19 ноября.
\vs p163 0:2 Каждое утро Иисус проводил беседы с этими учениками. Петр обучал их методам публичного проповедования; Нафанаил наставлял в искусстве обучения; Фома объяснял, как отвечать на вопросы; а Матфей руководил организацией финансовых дел группы. Другие апостолы тоже принимали участие в обучении в соответствии со своим специфическим опытом и природными талантами.
\usection{1. Посвящение семидесяти}
\vs p163 1:1 Днем в субботу 19 ноября в Магаданском лагере семьдесят человек были посвящены Иисусом, и во главе этих проповедников и учителей евангелия был поставлен Авенир. В числе этих семидесяти человек были Авенир, десять бывших апостолов Иоанна, пятьдесят один старый евангелист и восемь учеников, отличившихся в служении царству.
\vs p163 1:2 В эту же субботу около двух часов дня дождь на берегу Галилейского озера на короткое время прекратился и толпа верующих, сильно возросшая с прибытием Давида и большинства его вестников и насчитывающая уже к тому моменту больше четырехсот человек, собралась там, чтобы присутствовать при посвящении семидесяти.
\vs p163 1:3 Прежде, чем возложить руки на голову каждого из семидесяти, посвящая их таким образом в вестники евангелия, Иисус обратился к ним и сказал: «Урожай действительно обилен, но работники малочисленны; поэтому я призываю вас всех молить Хозяина урожая послать еще работников на сбор его урожая. Я собираюсь избрать вас в качестве вестников царства; я собираюсь послать вас к евреям и неевреям, как агнцев в волчьи стаи. Когда вы попарно пойдете каждый своим путем, я наказываю вам не брать с собой ни кошелька, ни лишней одежды, ибо вы отправляетесь в эту первую миссию лишь на непродолжительное время. Не задерживайтесь по дороге ни с кем, занимайтесь только своим делом. Всякий раз, намереваясь остановиться в чьем\hyp{}то доме, скажите сначала: мир этому дому. Если в нем живут любящие мир, вы остановитесь там; если нет, вы уйдете. А выбрав дом, оставайтесь в нем в течение всего пребывания в этом городе, ешьте и пейте все, что поставят перед вами. И делайте так потому, что труженик заслуживает пропитания. Не переселяйтесь из дома в дом, даже если вам предложат лучшее жилище. Помните, что, когда вы будете шествовать, провозглашая мир на земле и добрую волю между людьми, вам придется сталкиваться с ожесточенными и впавшими в самообман врагами; поэтому будьте мудры, как змеи, и при этом кротки, как голуби.
\vs p163 1:4 И куда бы вы ни пошли, проповедуйте словами: „Царство небесное рядом“, и служите всем, кто болен умом или телом. Безвозмездно вы получили благодать царства; безвозмездно и давайте. Если жители какого\hyp{}либо города примут вас, они получат неограниченный доступ в царство Отца; но если жители какого\hyp{}то города откажутся принять это евангелие, то все равно провозглашайте свою весть и, даже уходя из этого города неверующих, говорите его жителям, отвергающим ваше учение: „Несмотря на то, что вы отвергаете истину, все равно царство Божье приблизилось к вам“. Тот, кто слышит вас, слышит меня. И тот, кто слышит меня, слышит Того, кто послал меня. Тот, кто отвергает вашу благую весть, отвергает меня. А тот, кто отвергает меня, отвергает Того, кто послал меня».
\vs p163 1:5 После того, как Иисус произнес эти слова перед семьюдесятью, они встали на колени вокруг него, и он, начав с Авенира, возложил руки на голову каждого из них.
\vs p163 1:6 На следующий день рано утром Авенир отправил семьдесят вестников евангелия во все города Галилеи, Самарии и Иудеи. И эти тридцать пять пар ушли примерно на шесть недель проповедовать и учить, а в пятницу 30 декабря все они собрались в новом лагере возле Пеллы в Перее.
\usection{2. Богатый молодой человек и другие}
\vs p163 2:1 Более пятидесяти учеников, стремившихся быть включенными в число семидесяти посвященных, были отвергнуты комиссией, назначенной Иисусом для отбора кандидатов. В комиссию входили Андрей, Авенир и исполняющий обязанности главы корпуса евангелистов. Во всех случаях, когда эта комиссия из трех человек не могла принять единодушного решения, кандидата приводили к Иисусу, и, хотя Учитель никогда не отвергал ни одного человека, жаждущего посвящения в вестники евангелия, более дюжины из них, поговорив с Иисусом, уже не захотели становиться вестниками.
\vs p163 2:2 \pc Один ревностный ученик пришел к Иисусу и сказал: «Учитель, я был бы одним из твоих новых апостолов, но мой отец очень стар и близок к смерти; мог бы я получить разрешение вернуться домой, чтобы похоронить его?» Иисус сказал этому человеку: «Сын мой, лисы имеют норы, и птицы небесные имеют гнезда, но Сын Человеческий не имеет где преклонить голову. Ты преданный ученик, и ты можешь им остаться, если вернешься домой, чтобы служить тем, кого любишь, но с моими вестниками евангелия дело обстоит иначе. Они оставили все, чтобы следовать за мной и возвещать царство. Если бы ты был посвящен в учителя, ты должен был бы предоставить другим хоронить мертвых, когда идешь возвещать благую весть». И этот человек ушел сильно огорченный.
\vs p163 2:3 Еще один ученик пришел к Учителю и сказал: «Я стал бы посвященным вестником, но я хотел бы ненадолго сходить домой, чтобы успокоить свою семью». И Иисус ответил: «Если бы ты был посвящен, то должен был бы с готовностью отказаться от всего. Вестники евангелия не могут иметь привязанностей, мешающих делу. Ни один человек, если он, взявшись за плуг, повернул назад, не достоин стать вестником царства».
\vs p163 2:4 \pc Позже Андрей привел к Иисусу некоего богатого молодого человека, который искренне веровал и желал посвящения. Этот молодой человек по имени Матадорм был членом Иерусалимского Синедриона; он слушал учение Иисуса, а затем Петр и другие апостолы наставляли его в евангелии царства. Иисус обсудил с Матадормом условия, необходимые для посвящения, и предложил ему отложить принятие решения до тех пор, пока тот не обдумает все более тщательно. На следующий день рано утром, когда Иисус прогуливался, этот молодой человек подошел к нему и сказал: «Учитель, я бы хотел узнать у тебя, что гарантирует вечную жизнь. Так как я с юности соблюдал все заповеди, я бы хотел знать, что еще я должен сделать, чтобы обрести вечную жизнь?» В ответ на эти вопросы Иисус сказал: «Если ты соблюдаешь все заповеди --- не прелюбодействуй, не убивай, не кради, не произноси ложного свидетельства, не обманывай, почитай отца твоего и мать твою --- ты поступаешь хорошо, но спасение --- это вознаграждение за веру, а не только за поступки. Ты веришь в это евангелие царства?» И Матадорм ответил: «Да, Учитель, я действительно верю во все, чему учили меня ты и твои апостолы». И Иисус сказал: «Тогда ты действительно мой ученик и сын царства».
\vs p163 2:5 Затем молодой человек произнес: «Но, Учитель, мне недостаточно быть твоим учеником; я хотел бы быть одним из твоих новых вестников». Услышав это, Иисус с большой любовью посмотрел на него и сказал: «Я сделаю тебя одним из своих вестников, если ты готов заплатить за это соответствующую цену, если ты доставишь единственное, чего тебе не хватает». Матадорм ответил: «Учитель, я сделаю что угодно, если мне может быть позволено следовать за тобой». Иисус поцеловал в лоб стоящего на коленях молодого человека и сказал: «Если ты хочешь быть моим вестником, пойди и продай все, что имеешь, и когда раздашь полученные деньги бедным или своим братьям, приходи и следуй за мной, и ты обретешь сокровище в царствие небесном».
\vs p163 2:6 Услышав это, Матадорм изменился в лице. Он поднялся и опечаленный пошел прочь, потому что владел огромными богатствами. Этот молодой фарисей был воспитан в убеждении, что богатство является признаком благосклонности Бога. Иисус знал, что он не освободился от любви к себе и к своим сокровищам. Учитель же хотел избавить его от \bibemph{любви} к богатству а вовсе не от самого богатства. Хотя ученики Иисуса и не расстались со всем своим мирским имуществом, апостолы и семьдесят избранных с ним расстались. Матадорм желал быть одним из семидесяти новых вестников, и по этой причине Иисус потребовал, чтобы он расстался со всей его мирской собственностью.
\vs p163 2:7 \pc Почти каждый человек сохраняет какой\hyp{}нибудь особо дорогой ему порок, отказ от которого необходим как часть платы за вход в царство. Если бы Матадорм расстался со своим богатством, оно, вероятно, было бы передано обратно в его же руки, чтобы он распоряжался им как казначей семидесяти вестников. Потому что позже, после основания церкви в Иерусалиме, он поступил согласно предписанию Учителя, хотя тогда это было уже слишком поздно, чтобы стать одним из семидесяти, и он сделался казначеем Иерусалимской церкви, главой которой был Иаков, родной брат Господа.
\vs p163 2:8 Таким образом, так было всегда и будет вовеки: люди должны сами принимать свои решения. В определенной степени существует свобода выбора, которой смертные могут пользоваться. Силы духовного мира не принуждают человека; они позволяют ему следовать по пути, избранному им самим.
\vs p163 2:9 Иисус понимал, что Матадорм с его богатствами не может стать посвященным сподвижником людей, которые ради евангелия отказались от всего; и в то же время он видел, что лишенный своих богатств, он стал бы лидером их всех. Но, подобно собственным братьям Иисуса, он так и не обрел величия в царстве, потому что лишил себя той тесной личной связи с Учителем, которая могла бы стать частью его опыта, если бы в этот раз он с готовностью согласился сделать то, о чем его попросил Иисус и что он действительно совершил впоследствии, через несколько лет.
\vs p163 2:10 Богатство само по себе никак напрямую не влияет на возможность войти в царство небесное, но \bibemph{влияет любовь к богатству.} Духовная преданность царству несовместима со служением мамоне. Человек не может сочетать высочайшую приверженность духовному идеалу со страстью к материальным благам.
\vs p163 2:11 Иисус никогда не учил, что нехорошо иметь богатство. Лишь от двенадцати апостолов и семидесяти вестников он требовал, чтобы они пожертвовали всю свою мирскую собственность на общее дело. Но даже и в этом случае он предусматривал, чтобы они с прибылью избавлялись от своей собственности, как это было в случае с апостолом Матфеем. Иисус много раз советовал своим состоятельным ученикам то же, что и богатому римлянину. Учитель считал, что разумно вложить излишние доходы --- значит должным образом подстраховаться на случай неизбежных трудностей в будущем. Когда казна апостолов была полна, Иуда вкладывал деньги под проценты, чтобы использовать их впоследствии, когда возникнут серьезные трудности, связанные с сокращением денежных поступлений. Иуда делал это, советуясь с Андреем. Иисус лично никогда не занимался вопросами финансов апостолов, кроме как при раздаче милостыни. Но было одно экономическое злоупотребление, которое он много раз осуждал, а именно --- несправедливая эксплуатация слабых, необразованных и менее удачливых людей их сильными, энергичными и более умными согражданами. Иисус заявлял, что такое бесчеловечное отношение к мужчинам, женщинам и детям несовместимо с идеалами братства царства небесного.
\usection{3. Дискуссия о богатстве}
\vs p163 3:1 К тому времени, когда Иисус закончил беседовать с Матадормом, вокруг него собрались Петр и несколько апостолов, и когда этот богатый молодой человек собрался уходить, Иисус повернулся к апостолам и сказал: «Вы видите, как трудно имеющим богатства в полной мере войти в царство Божье! Поклонение духовному невозможно сочетать с приверженностью материальному; ни один человек не может служить двум господам. Есть поговорка, что „легче верблюду пролезть в игольное ушко, чем язычнику обрести вечную жизнь“. И я заявляю, что самодовольным богачам не легче войти в царство небесное, чем верблюду пролезть в игольное ушко».
\vs p163 3:2 Когда Петр и апостолы услышали эти слова, они были чрезвычайно удивлены, до такой степени, что Петр спросил: «Кто же тогда, Господи, может быть спасен? Все ли, кто имеют богатства, не попадут в царство?» И Иисус ответил: «Нет, Петр, но все те, кто уповают на богатство, едва ли достигнут духовной жизни, ведущей к вечному развитию. Но даже и в этом случае, многое, что невозможно для человека, доступно для Отца Небесного; прежде всего мы должны осознать, что с Богом возможно все».
\vs p163 3:3 Когда же они ушли, Иисус был опечален тем, что Матадорм не остался с ними, потому что он очень полюбил его. Спустившись к озеру, они сели у воды, и Петр сказал от имени двенадцати апостолов (которые к этому времени собрались здесь): «Мы в замешательстве от твоих слов, сказанных этому богатому молодому человеку. Следует ли и нам требовать от тех, кто желал бы последовать за тобой, чтобы они отказались от всего своего мирского имущества?» И Иисус ответил: «Нет, Петр, лишь от тех, кто пожелал бы стать апостолом и, подобно вам, жить со мной одной семьей. Но Отец требует, чтобы чувства и устремления его детей были чистыми и не противоречивыми. Вы должны отрешиться от всех вещей и людей, стоящих между вами и любовью к истинам царства. Если богатство не затрагивает душу, тогда оно не имеет значения для последующей духовной жизни тех, кто желал бы вступить в царство».
\vs p163 3:4 И тогда Петр спросил: «Но, Учитель, мы оставили все и последовали за тобою, что же мы получим?» И Иисус сказал в ответ двенадцати апостолам: «Истинно, истинно говорю я вам, нет никого, кто оставил бы богатство, дом, жену, братьев, родителей или детей ради меня и ради царства небесного, который не получит во много крат больше в этом мире, возможно --- наряду с гонениями, а в веке грядущем жизнь вечную. Многие же будут первые последними, и последние первыми. Отец поступает со своими созданиями в соответствии с их потребностями и согласно его справедливым законам, исполненным милосердия и любви и служащим на благо вселенной.
\vs p163 3:5 Царство небесное подобно хозяину дома, который вышел рано поутру нанять работников в виноградник свой. И, договорившись с работниками по динарию на день, послал их в виноградник свой. Потом он вышел около девяти часов и, увидев других людей, стоящих на торжище праздно, сказал им: „Идите и вы в виноградник мой, и что следовать будет, дам вам“. И они пошли. Опять он выходил еще около двенадцати и около трех часов и поступал точно так же. А придя на торжище около пяти часов дня, он обнаружил еще других людей, стоящих праздно, и спросил у них: „Почему вы весь день стоите здесь праздно?“ И эти люди отвечали: „Потому что никто нас не нанял“. Тогда хозяин сказал: „Идите и вы в виноградник мой, и что следовать будет, получите“.
\vs p163 3:6 Когда же наступил вечер, господин виноградника сказал управителю своему: „Позови работников и отдай им плату, начав с последних до первых“. Когда пришли те, кто были наняты около пяти часов, они получили по динарию, и столько же получил каждый из остальных работников. Когда люди, нанятые в начале дня, увидели, сколько заплатили пришедшим позже, они стали ожидать, что получат больше обещанной платы. Но получили и они по динарию. И получивши стали роптать на хозяина дома: „Эти люди, которые были наняты последними, проработали всего лишь час, и ты сравнял их с нами, перенесшими тягость дня и зной“.
\vs p163 3:7 Он же в ответ сказал: „Друзья, я не обижаю вас; не за динарий ли вы договорились со мною? Возьмите свое и пойдите; я же хочу дать этим последним то же, что и вам. Разве я не властен в своем делать, что хочу? или вы жалеете о моей щедрости потому, что я желаю быть добрым и проявлять милосердие?“»
\usection{4. Прощание с семьюдесятью вестниками}
\vs p163 4:1 В тот день, когда семьдесят вестников отправились в свою первую миссию, в Магаданском лагере царило оживление. Рано утром в этот день в своей последней беседе с семьюдесятью вестниками евангелия Иисус особо подчеркнул следующее:
\vs p163 4:2 \ublistelem{1.}\bibnobreakspace Евангелие царства должно возвещаться всему миру, и неевреям, и евреям.
\vs p163 4:3 \ublistelem{2.}\bibnobreakspace При служении больным не учите их ожидать чудес.
\vs p163 4:4 \ublistelem{3.}\bibnobreakspace Провозглашайте духовное братство сыновей Бога, а не зримое царство с мирской властью и материальным великолепием.
\vs p163 4:5 \ublistelem{4.}\bibnobreakspace Избегайте тратить время на излишние визиты и другие несущественные дела, которые могут помешать всецело посвятить себя проповеди царства.
\vs p163 4:6 \ublistelem{5.}\bibnobreakspace Если первый же дом, выбранный пристанищем, оказался подходящим, живите там все время своего пребывания в этом городе.
\vs p163 4:7 \ublistelem{6.}\bibnobreakspace Давайте ясно понять всем преданным верующим, что настало время открыто порвать с религиозными предводителями евреев в Иерусалиме.
\vs p163 4:8 \ublistelem{7.}\bibnobreakspace Учите, что долг человека весь заключен в одной заповеди: Возлюби Господа Бога своего всем своим разумом и всей душой, и возлюби ближнего своего, как самого себя. (Они должны были учить этой самой главной заповеди вместо 613 правил жизни, провозглашаемых фарисеями.)
\vs p163 4:9 \pc После того, как Иисус в присутствии всех апостолов и учеников обратился к семидесяти вестникам, Симон Петр отвел их уже одних в сторону и произнес перед ними проповедь по поводу их посвящения, в которой развил наказ Учителя, данный им в момент, когда он возложил на них руки и избрал их вестниками царства. Петр призвал семьдесят вестников в своем опыте подвижничества соблюдать следующие добродетели:
\vs p163 4:10 \ublistelem{1.}\bibnobreakspace \bibemph{Посвященная преданность.} Всегда молиться, чтобы больше тружеников было ниспослано на ниву евангелия. Он объяснил, что молящийся об этом, скорее всего, скажет: «Вот я; пошли меня». Он велел им не забывать ежедневно молиться.
\vs p163 4:11 \ublistelem{2.}\bibnobreakspace \bibemph{Истинное мужество.} Он предупредил их, что их встретят враждебно и, наверняка, подвергнут преследованиям. Петр сказал им, что их миссия --- не для малодушных, и посоветовал тем, кто боится, отступить, пока они еще не отправились. Но никто не отступил.
\vs p163 4:12 \ublistelem{3.}\bibnobreakspace \bibemph{Вера и упование.} Они должны отправится в эту непродолжительную миссию без ничего; они должны уповать на то, что Отец даст им пищу и кров и все необходимое.
\vs p163 4:13 \ublistelem{4.}\bibnobreakspace \bibemph{Рвение и предприимчивость.} Они должны быть одержимы рвением и разумным энтузиазмом; они должны неукоснительно выполнять дело, порученное им Учителем. Восточное приветствие представляло собой длинную и весьма тонкую церемонию; поэтому они получили наставление «не приветствовать по пути ни одного человека», что было обычной формой настоятельного совета заниматься своим делом, не тратя времени попусту. Речь не шла о дружеских приветствиях.
\vs p163 4:14 \ublistelem{5.}\bibnobreakspace \bibemph{Доброта и учтивость.} Учитель велел им избегать ненужной траты времени на светские церемонии, но повелел быть учтивыми со всеми, с кем доведется общаться. Они должны проявлять доброту к тем, кто будет принимать их в своих домах. Они получили строгий наказ не покидать скромный дом, даже если их пригласят в более удобный или богатый.
\vs p163 4:15 \ublistelem{6.}\bibnobreakspace \bibemph{Служение больным.} Петр поручил семидесяти вестникам разыскивать больных разумом и телом и делать все, что в их силах, чтобы облегчить или исцелить их недуги.
\vs p163 4:16 \pc И выслушав эти поручения и наставления, они по двое отправились в путь исполнять свою миссию в Галилее, Самарии и Иудее.
\vs p163 4:17 Хотя у евреев было особое отношение к числу семьдесят и иногда они считали, что семьдесят --- это число народов языческого мира, и хотя семьдесят вестников должны были отправиться с евангелием ко всем народам, тем не менее, насколько мы можем судить, лишь по чистой случайности в этой группе оказалось именно семьдесят человек. Несомненно, Иисус принял бы еще не меньше полдюжины других, но они не были склонны платить цену за отказ от благосостояния и семьи.
\usection{5. Перенос лагеря в Пеллу}
\vs p163 5:1 Иисус и двенадцать апостолов теперь решили устроить свое последнее пристанище в Перее, возле Пеллы, где Учитель принял крещение в Иордане. Последние десять дней ноября прошли в совещаниях в Магадане, а во вторник 6 декабря вся группа, почти триста человек, со всем своим имуществом отправились на рассвете в путь и остановилась на ночь у реки неподалеку от Пеллы. Это было то самое место у источника, где несколько лет тому назад располагался лагерь Иоанна Крестителя.
\vs p163 5:2 После того, как Магаданский лагерь прекратил существование, Давид Зеведей вернулся в Вифсаиду и сразу же начал сокращать службу своих вестников. Царство уже вступало в новую фазу. Ежедневно со всех концов Палестины и даже из отдаленных районов Римской империи прибывали паломники. Верующие иногда приходили из Месопотамии и из земель к востоку от Тигра. Таким образом, в воскресенье 18 декабря Давид со своими вестниками погрузил на вьючных животных лагерное снаряжение, которое хранил в доме отца и с чьей помощью раньше уже обустраивал лагерь у озера около Вифсаиды. Распрощавшись на время с Вифсаидой, он проследовал вдоль берега озера и затем вдоль Иордана до места, расположенного примерно в полумиле к северу от лагеря апостолов; и менее чем через неделю он уже был готов оказать радушный прием почти полутора тысячам прибывающих паломников. Лагерь апостолов мог вместить около пятисот человек. В Палестине в это время был сезон дождей, и лагеря эти были нужны для размещения постоянно растущего числа любознательных посетителей, в большинстве своем искренних людей, приходивших в Перею, чтобы увидеть Иисуса и услышать его учение.
\vs p163 5:3 Все это Давид делал по своей собственной инициативе, хотя он и предварительно советовался в Магадане с Филиппом и Матфеем. Большинство его бывших вестников помогали ему в обустройстве этого лагеря; теперь в качестве вестников он постоянно использовал не более двадцати человек. Ближе к концу декабря, перед возвращением семидесяти, вокруг Учителя собралось уже почти восемьсот человек и они нашли пристанище в лагере Давида.
\usection{6. Возвращение семидесяти}
\vs p163 6:1 В пятницу 30 декабря, пока Иисус вместе с Петром, Иаковом и Иоанном находился в близлежащих горах, в лагерь около Пеллы парами начали приходить семьдесят вестников в сопровождении многочисленных верующих. Около пяти часов, когда Иисус вернулся в лагерь, все семьдесят собрались в том месте, где Иисус обычно учил. Вечерняя трапеза была отложена почти на час, пока подвижники евангелия царства рассказывали каждый о своей деятельности. В течение предыдущих недель вестники Давида уже принесли апостолам значительную часть этих новостей, но действительно было интересно услышать из собственных уст этих недавно посвященных учителей евангелия вдохновленный рассказ о том, как изголодавшиеся евреи и неевреи принимали их весть. Наконец\hyp{}то Иисус мог увидеть людей, самостоятельно, без него отправившихся распространять благую весть. Учитель теперь знал, что может покинуть этот мир, и это не причинит серьезного ущерба делу распространения царства.
\vs p163 6:2 Когда семьдесят вестников евангелия рассказывали, как «даже дьяволы подчинялись» им, они имели в виду совершенные ими чудесные исцеления людей, страдающих нервными расстройствами. Однако и в самом деле было несколько случаев одержимости духами, которых эти пастыри изгнали, и в связи с этим Иисус сказал: «Нет ничего странного в том, что вам подчинились эти непокорные второстепенные духи, поскольку я видел, как Сатана, подобно молнии, низвергнулся с неба. Но не сильно радуйтесь по этому поводу, ибо я заявляю вам, что, как только вернусь к моему Отцу, мы ниспошлем наш дух прямо в умы людей, чтобы немногочисленные заблудшие духи не могли больше вселяться в умы несчастных смертных. Я радуюсь вместе с вами, что у вас есть влияние в мире людей, но больше радуйтесь не этому опыту, а тому, что ваши имена занесены в небесные скрижали и что вам, таким образом, предстоит продвигаться вперед по бесконечному пути духовных свершений».
\vs p163 6:3 Именно тогда, перед вечерней трапезой, Иисус пережил один из тех редких моментов эмоционального экстаза, свидетелями которых иногда оказывались его последователи. Он воскликнул: «Благодарю тебя, Отец мой, Господь неба и земли, за то, что, дух открыл этим детям царства духовное великолепие чудесного евангелия, хотя оно и было скрыто от мудрых и самодовольных. Да, Отец мой, тебе, должно быть, приятно было свершить это, и я радуюсь, что благая весть будет распространяться по всему миру даже после того, как я вернусь к тебе, и к тому делу, которое ты поручил мне. Я счастлив сознавать, что ты собираешься передать в мои руки всю власть, что только ты действительно знаешь, кто я, и что только я действительно знаю тебя и тех, кому я открыл тебя. И открыв тебя моим земным братьям, я продолжу открывать тебя твоим созданиям на небесах».
\vs p163 6:4 Поговорив так с Отцом, Иисус повернулся и обратился к своим апостолам и пастырям: «Благословенны глаза, которые видят, и уши, которые слышат это. Говорю вам, что многие пророки и многие из великих мужей прошлого желали узреть то, что вы сейчас видите, но им это не было даровано. И многие поколения детей света в будущем, слушая обо всем этом, будут завидовать вам, слышавшим и видевшим это».
\vs p163 6:5 Затем, обращаясь ко всем ученикам, он сказал: «Вы слышали, сколько городов и селений приняли благую весть царства и как евреи и неевреи принимали моих пастырей и учителей. И воистину благословенны эти города и селения, решившие поверить в евангелие царства. Но горе отвергающим свет жителям Хоразина, Вифсаиды\hyp{}Юлии и Капернаума, городов, в которых не приняли моих вестников хорошо. Я заявляю, что если бы великие деяния, совершенные в этих городах, случились бы в Тире и Сидоне, жители этих, так называемых, языческих городов, надев власяницы и посыпав себя пеплом, давно бы раскаялись. Во истину, в судный день я буду более терпим к Тиру и Сидону».
\vs p163 6:6 \pc На следующий день, в субботу, Иисус удалился только с семьюдесятью вестниками и сказал им: «Воистину, я возрадовался вместе с вами, когда вы вернулись и принесли добрые вести о том, сколько людей по всей Галилее, Самарии и Иудее приняли евангелие царства. Но удивительно, почему вы так ликовали? Разве вы не ожидали, что ваша весть покажет свою силу, когда вы ее возвестите? Или вы, когда отправлялись, настолько мало верили в это евангелие, что вернулись пораженные его воздействием? И теперь, хоть я и не хотел бы портить ваше радостное настроение, я хотел бы сурово предостеречь вас о коварности гордости, духовной гордыни. Если бы вы только могли понять глубину падения Люцифера, его беззаконности, вы бы серьезно остерегались всех форм духовной гордыни.
\vs p163 6:7 «Вы взялись за великое дело, цель которого --- научить смертного человека, что он есть сын Бога. Я указал вам путь; идите же вперед, исполняйте свой долг и неустанно делайте это хорошо. Вам и всем, кто последует по вашему пути во все века, я говорю: я всегда буду рядом, и мой призыв есть и всегда будет: придите ко мне все трудящиеся и обремененные, и я дам вам утешение. Возложите на себя мое бремя и познайте меня, ибо я верный и преданный, и вы найдете духовное утешение для своих душ».
\vs p163 6:8 \pc И они убедились в истинности слов Учителя на собственном опыте. И с того дня бесчисленное множество людей сами удостоверялись в верности этих самых слов.
\usection{7. Подготовка к последней миссии}
\vs p163 7:1 Следующие несколько дней в лагере около Пеллы царило оживление; завершалась подготовка к миссии в Перею. Иисус и его сподвижники готовы были приступить к своей последней миссии --- трехмесячному путешествию по всей Перее, которое закончилось только, когда Учитель вступил в Иерусалим, чтобы совершить свое последнее деяние на земле. На протяжении всего этого периода центром Иисуса и двенадцати апостолов продолжал оставаться этот лагерь возле Пеллы.
\vs p163 7:2 Для Иисуса больше уже не было необходимости отправляться в иные земли учить людей. Теперь все больше и больше народа приходило к нему каждую неделю отовсюду, не только из Палестины, но из всех концов Римской империи и с Ближнего Востока. Хотя Учитель вместе с семьюдесятью сподвижниками и путешествовал по Перее, однако большую часть своего времени он проводил в лагере возле Пеллы, уча толпы народа и наставляя двенадцать апостолов. В течение этого трехмесячного периода, рядом с Иисусом всегда было не меньше десяти апостолов.
\vs p163 7:3 Группа женщин тоже готовилась по двое отправиться вместе с семьюдесятью вестниками евангелия трудиться в крупных городах Переи. Эти двенадцать женщин, первые примкнувшие к Иисусу, незадолго до того подготовили больше пятидесяти женщин, обучив их навещать в домах больных и страждущих и служить им. В этом новом отряде женщин была и Перпетуя, жена Симона Петра, ей было поручено возглавить деятельность женщин при Авенире. После Пятидесятницы она постоянно была рядом со своим знаменитым мужем, сопровождая его во всех его миссионерских поездках; и в тот день, когда Петр был распят в Риме, она была брошена на арену на съедение диким зверям. В этом же отряде женщин были и жены Филиппа и Матфея, и мать Иакова и Иоанна.
\vs p163 7:4 Теперь под личным руководством Иисуса миссия, связанная с царством, была готова вступить в свою последнюю фазу. И эта предстоящая фаза отмечена необыкновенной духовной глубиной, тогда как ранее толпами, следовавшими за Учителем в дни его особой популярности в Галилее, двигал прежде всего жгучий интерес к чудесам. Однако у него все же было немало материалистически мыслящих последователей, которые так и не смогли осознать истину, что царство небесное --- это духовное братство людей, основанное на вечном факте вселенского отцовства Бога.
