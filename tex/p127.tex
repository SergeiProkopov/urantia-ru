\upaper{127}{Годы юности}
\author{Комиссия срединников}
\vs p127 0:1 Иисус вступил в годы юности, будучи главой и единственной опорой большой семьи. За несколько лет, прошедших со дня смерти отца, семье пришлось расстаться со всей ее собственностью. В течение этих лет Иисус постепенно осознавал свое предшествующее этой жизни существование, а также начал отчетливее понимать, что его воплощение на земле в плотском образе имеет совершенно определенную цель: откровение Райского Отца детям человеческим.
\vs p127 0:2 Ни одному юному человеку, который жил или когда\hyp{}нибудь будет жить в этом или каком\hyp{}либо ином мире, не приходилось и не придется никогда иметь дело с более многотрудными проблемами, с более сложными обстоятельствами. Ни один юноша на Урантии не проходил когда\hyp{}либо через более серьезные испытания или более напряженные ситуации, чем те, которые пережил Иисус в эти трудные годы между пятнадцатью и двадцатью.
\vs p127 0:3 Таким образом, Сын Человеческий, изведав в юные годы опыт реальной жизни в мире, наводненном злом и пораженным грехом, Сын Человеческий получил исчерпывающее знание о жизненных переживаниях периода юности во всех сферах обитания Небадона и с тех пор навечно стал отзывчивым утешителем для растерянной и страдающей молодежи всех времен во всех мирах локальной вселенной.
\vs p127 0:4 Так медленно, но неуклонно и на реальном опыте этот божественный Сын \bibemph{зарабатывает} право стать владыкой своей вселенной --- неоспоримым и верховным правителем для всех сотворенных разумных существ во всех мирах локальной вселенной, чутким прибежищем для существ всякого возраста, личного опыта и личной одаренности.
\usection{1. Шестнадцатый год (10 г. н.э.)}
\vs p127 1:1 Воплощенный Сын вкусил младенчество, прожил спокойное детство. Теперь он преодолел трудный и напряженный переходный период между детством и юной зрелостью --- стал юношей Иисусом.
\vs p127 1:2 В этом году Иисус полностью развился физически. Он возмужал и был привлекательным юношей. Он становился все более сдержанным и серьезным, но был добросердечен и полон сочувствия. Взгляд его был добрым, но испытующим; улыбка --- всегда располагающей и ободряющей. Голос был мелодичным, но повелительным; приветствие --- дружелюбным, но не притворным. Всегда, даже в самом обыденном общении давала себя знать его двойственная природа, божественная и человеческая. Всегда проявлял он себя и благожелательным другом, и авторитетным учителем. Эти личностные черты начали обнаруживаться рано, уже в эти юношеские годы.
\vs p127 1:3 Этот физически сильный и крепкий юноша достиг также полной зрелости своего человеческого интеллекта --- не полного опыта человеческого мышления, а полноты способностей к такому интеллектуальному развитию. Он обладал здоровым и хорошо сложенным телом, острым и аналитическим умом, добрым и милосердным характером, энергичным, хотя несколько переменчивым темпераментом, и из всего этого постепенно складывалась сильная, замечательная и привлекательная личность.
\vs p127 1:4 \pc Со временем матери, братьям и сестрам все труднее становилось понимать его; они приходили в недоумение от его речей и неправильно истолковывали его поступки. Все они не могли постичь жизнь своего старшего брата, так как мать дала понять им, что ему суждено стать освободителем еврейского народа. После того, как они узнали от матери эту семейную тайну, --- представьте себе их замешательство, когда они слышали от Иисуса искреннее отрицание подобных идей и намерений.
\vs p127 1:5 \pc В этом году Симон пошел в школу, и они были вынуждены продать еще один дом. Иаков теперь взял на себя обучение своих трех сестер, две из которых уже достигли такого возраста, чтобы начать серьезное учение. Как только подросла Руфь, с нею стали заниматься Мириам и Марфа. Обычно девочки в еврейских семьях почти не получали образования, но Иисус придерживался того мнения (и мать была согласна с ним), что девочки должны учиться так же, как мальчики, а поскольку синагогальная школа не могла их принять, оставалось только устроить специально для них домашнюю школу.
\vs p127 1:6 Весь этот год Иисус очень много работал за верстаком. К счастью, у него было в избытке заказов. Он был мастером такого высокого уровня, что никогда не сидел без дела, как бы плохо ни было с работой в округе. Иногда заказов было так много, что Иаков должен был помогать ему.
\vs p127 1:7 К концу этого года Иисус был близок к решению начать свое общественное служение в качестве учителя истины и возвестителя миру откровения о Небесном Отце, как только он вырастит своих братьев и сестер и они женятся и выйдут замуж. Он знал, что не станет ожидаемым еврейским Мессией, и пришел к выводу, что едва ли стоит обсуждать это с матерью; он решил дать ей возможность оставаться при своих мыслях, поскольку все, что он говорил ей прежде, оказывало на нее мало или практически никакого влияния, и он вспоминал, что отец тоже никогда не мог уговорить ее изменить свою позицию. С этого года он все меньше и меньше говорил с матерью, как и с кем\hyp{}либо еще, об этих проблемах. Столь особой была его миссия, что никто из живущих на земле не смог бы ему посоветовать, как ее исполнить.
\vs p127 1:8 Несмотря на молодость, он был настоящим отцом для своей семьи; каждый свободный час проводил с детьми, и они искренне любили его. Его матери горько было видеть, что он так тяжко трудится; она печалилась оттого, что он день за днем проводит за плотницким верстаком, зарабатывая на жизнь для семьи, вместо того, чтобы, как они любовно планировали, учиться в Иерусалиме у раввинов. Хотя Мария многое в своем сыне не понимала, она по\hyp{}настоящему его любила и глубоко ценила готовность, с которой он взвалил на свои плечи ответственность за семью.
\usection{2. Семнадцатый год (11 г. н.э.)}
\vs p127 2:1 В то время проводилась значительная агитация, особенно в Иерусалиме и в Иудее, призывающая восстать против платы налогов Риму. Формировалась сильная националистическая партия, членов которой стали называть зилотами. В отличие от фарисеев, зилоты не собирались ожидать пришествия Мессии. Они намеревались навести порядок путем политического переворота.
\vs p127 2:2 В Галилею прибыла группа организаторов из Иерусалима, которая действовала успешно, пока не достигла Назарета. Когда они пришли к Иисусу, он внимательно их выслушал и задал множество вопросов, но отказался присоединиться к партии. Он никак не объяснил причины своего неприсоединения, и его отказ подействовал на многих его сверстников в Назарете, удержав их от вступления в партию.
\vs p127 2:3 Мария сделала все возможное, чтобы уговорить его присоединиться, но не смогла поколебать его. Она зашла далеко и даже дала ему понять, что его отказ выполнить ее волю и поддержать дело националистов является неподчинением, нарушением обещания быть послушным родительской воле, которое он дал, вернувшись из Иерусалима; но Иисус в ответ на это обвинение лишь нежно положил ей руку на плечо и, глядя в глаза, спросил: «Мама, как ты можешь?» И Мария взяла свои слова назад.
\vs p127 2:4 Один из дядей Иисуса (брат Марии Симон) уже присоединился к этой группе и позднее стал офицером галилейского подразделения. И в течение нескольких лет между Симоном и Иисусом было определенное отчуждение.
\vs p127 2:5 Между тем в Назарете назревали волнения. Позиция Иисуса в этих вопросах привела к расколу среди еврейской молодежи города. Примерно половина вступила в организацию националистов, вторая половина начала создавать оппозиционную группу более умеренных патриотов, ожидая, что Иисус примет руководство ею. Они были изумлены, когда он отказался от предложенной чести, объяснив это тяжелыми семейными обязанностями, и это объяснение было всеми принято. Но ситуация вскоре еще более осложнилась, когда богатый еврей по имени Исаак, который ссужал деньги неевреям, выступил с предложением содержать семью Иисуса, если тот отложит свои плотницкие инструменты и встанет во главе этих назаретских патриотов.
\vs p127 2:6 Иисус, которому только исполнилось семнадцать лет, должен был разрешить одну из самых сложных и деликатных ситуаций своей юности. Духовным лидерам всегда бывает трудно иметь дело с патриотическими вопросами, особенно если они усугублены налогами в пользу чужеземных угнетателей, а в данном случае положение осложнялось вдвойне тем, что в выступление против Рима была вовлечена религия евреев.
\vs p127 2:7 Положение Иисуса было еще более затруднено тем, что мать Иисуса, его дядя и даже младший брат Иаков --- все убеждали его присоединиться к националистам. Все уважаемые евреи Назарета уже примкнули к патриотам, а те молодые люди, которые еще не включились в их движение, ожидали только момента, когда Иисус изменит свою позицию. Во всем Назарете у него был единственный мудрый советчик --- его старый учитель, хазан, с которым он обсудил выступление перед представителями жителей Назарета, просивших дать им ответ на сделанное публично предложение. Впервые в своей короткой жизни Иисус сознательно прибег к публичной стратегии. До сих пор он всегда, чтобы прояснить ситуацию, был откровенен и правдив, но сейчас он не мог сказать всю правду. Не мог дать понять, что он больше чем человек; не мог открыть им свое представление о собственной миссии, которая ожидала его, когда он достигнет полной зрелости. Несмотря на эти ограничения, это был прямой вызов его религиозной верности и национальной преданности. В семье его царило смятение, молодые друзья раскололись на два лагеря, все еврейское население Назарета волновалось. И подумать только, что все из\hyp{}за него! Он не имел совершенно никаких намерений создавать какие\hyp{}либо беспорядки, тем более --- такого рода.
\vs p127 2:8 Нужно было что\hyp{}то предпринять. Иисус должен был определить свою позицию, и он сделал это мужественно и дипломатично, к удовлетворению многих, хотя не всех. Он сохранил первоначальный мотив своего оправдания, сказав, что его главный долг --- перед семьей, что овдовевшая мать и восемь братьев и сестер нуждаются в большем, чем то, что можно купить за деньги, --- удовлетворение физических жизненных нужд; что они вправе рассчитывать на отцовскую опеку и руководство, и он не может с чистой совестью снять с себя обязанности, которые на него возложил жестокий случай. Он похвалил мать и старшего из братьев за их готовность освободить его от семейных забот, но повторил, что верность покойному отцу запрещает ему покинуть семью, как бы материально обеспеченна она ни была; при этом он произнес незабываемые слова: «деньги не могут любить». Несколько раз в этой речи Иисус намекал на свою «жизненную миссию», но объяснил, что, независимо от того, совместима или нет она с вооруженной борьбой, он пожертвовал ею, как и всем остальным в жизни, ради добросовестного выполнения обязанностей перед семьей. Все в Назарете знали, что он хороший отец для своей семьи, и это так трогало сердце каждого благородного еврея, что оправдание Иисуса нашло одобрительный отклик в сердцах многих слушателей; некоторые же из тех, кого не удалось убедить, были обезоружены речью брата Иакова, произнесенной тогда же, хоть и вне программы. В тот самый день Иаков репетировал ее с хазаном, но это осталось их тайной.
\vs p127 2:9 Иаков высказал уверенность, что Иисус помог бы в освобождении своего народа, если бы он (Иаков) был достаточно взрослым, чтобы взять на себя ответственность за семью, и что, если только Иисусу будет позволено остаться «с нами, быть нашим отцом и учителем, то со временем вы получите из семьи Иосифа не только вождя, но и еще пять преданных националистов, ибо не пятеро ли нас, мальчиков, растут и под руководством брата\hyp{}отца готовятся служить своей нации?» Так вполне благополучно завершилась эта очень напряженная и опасная ситуация.
\vs p127 2:10 Кризис на данный момент миновал, но этот случай никогда не был забыт в Назарете. Беспокойство осталось; Иисус не стал вновь всеобщим любимцем; двойственность чувств по отношению к нему никогда не была преодолена до конца. Усиленная другими и последующими событиями, она стала одной из главных причин переселения в дальнейшем Иисуса в Капернаум. А в Назарете так и осталось двойственное отношение к Сыну Человеческому.
\vs p127 2:11 \pc Иаков в этом году окончил школу и стал работать весь день в домашней мастерской. Он научился прекрасно обращаться с инструментами и смог взять на себя изготовление ярм и плугов, Иисус же теперь больше занимался отделкой домов и тонкой столярной работой.
\vs p127 2:12 \pc В этом году Иисус значительно продвинулся в организации своего разума. Постепенно он сопряг свою божественную и человеческую природу и осуществил эту организацию своего интеллекта силой собственных \bibemph{решений,} с помощью только своего Внутреннего Наблюдателя, каковой присутствует в уме каждого нормального смертного во всех мирах, где состоялось пришествие Сына. Пока что в жизни молодого человека не произошло ничего сверхъестественного, если не считать визита посланца от старшего брата Иммануила, который однажды явился ему ночью в Иерусалиме.
\usection{3. Восемнадцатый год (12 г. н.э.)}
\vs p127 3:1 В течение этого года семья рассталась со всей своей собственностью, кроме дома и огорода. Последняя недвижимость в Капернауме (не считая доли еще в одной), уже ранее заложенная, была продана. Полученные деньги ушли на налоги, на покупку новых инструментов для Иакова и на выплату за старое семейное предприятие --- мастерскую и магазин припасов --- около караванной стоянки, которое Иисус предполагал снова выкупить, поскольку Иаков стал достаточно взрослым, чтобы вместо него работать в домашней мастерской и помогать матери по дому. Материальная нужда временно отступила, и Иисус решил взять Иакова в Иерусалим на Пасху. Они отправились на день раньше других, чтобы быть вдвоем, и пошли через Самарию. Они шли пешком, и по дороге Иисус рассказывал Иакову об исторических местах, которые они проходили, так же, как отец просвещал его самого в подобном путешествии пять лет назад.
\vs p127 3:2 Проходя через Самарию, братья видели много странных вещей. По пути они обсудили многие свои проблемы, личные, семейные и национальные. Иаков был очень религиозным юношей, и хотя он не во всем согласен был с матерью относительно того малого, что знал о планах, касающихся жизненной задачи Иисуса, горел желанием поскорее вырасти и взять на себя ответственность за семью, чтобы Иисус смог приступить к своей миссии. Он был очень признателен Иисусу за то, что тот взял его на Пасху в Иерусалим, и они говорили между собой о будущем более подробно, чем когда\hyp{}либо раньше.
\vs p127 3:3 На пути через Самарию Иисус много размышлял, особенно в Бетеле и когда они пили из колодца Иакова. Они с братом обсуждали традиции, связанные с Авраамом, Исааком, Иаковом. Стремясь смягчить возможное потрясение, какое пережил сам при первом посещении храма, Иисус приложил много стараний, чтобы подготовить брата к тому, что ему предстояло увидеть в Иерусалиме. Но Иаков оказался не столь чувствителен, как был Иисус к некоторым из этих зрелищ. Он отметил формальную, бездушную манеру отправления служб некоторыми священниками, но в целом искренне наслаждался своим пребыванием в Иерусалиме.
\vs p127 3:4 На пасхальный ужин Иисус взял Иакова в Вифанию. Симон к тому времени уже покоился со своими предками, и Иисус председательствовал за столом как глава пасхальной семьи, принеся пасхального ягненка из храма.
\vs p127 3:5 После пасхального ужина Мария села побеседовать с Иаковом, в то время как Марфа и Лазарь до глубокой ночи проговорили с Иисусом. На следующий день они посетили храмовую службу, и Иаков был принят в число граждан Израиля. В то утро, когда они остановились на гребне Масличной горы, чтобы посмотреть на храм, Иаков выражал свой восторг, в то время как Иисус взирал на Иерусалим в молчании. Иакову было непонятно поведение брата. Вечером они опять вернулись в Вифанию и должны были отправиться домой на следующий день, но Иаков настоял на том, чтобы еще раз посетить храм, объяснив, что хочет слышать учителей. Это была правда, но в глубине сердца он втайне хотел услышать, как Иисус участвует в диспутах, как об этом рассказывала мать. Соответственно, они пошли в храм и слушали диспуты, но Иисус не задал ни одного вопроса. Пробуждающемуся разуму человека и Бога все это казалось настолько пустым и незначительным --- он мог лишь пожалеть участников. Иаков был разочарован его молчанием. На расспросы брата Иисус лишь коротко ответил: «Мой час еще не настал».
\vs p127 3:6 На другой день они отправились домой через Иерихон и Иорданскую долину. Иисус по дороге рассказывал о многих вещах, в том числе о том, как он путешествовал здесь в тринадцать лет.
\vs p127 3:7 \pc После возвращения в Назарет Иисус начал работать в старой семейной мастерской у караванной стоянки, и его очень обрадовала возможность каждый день видеть множество разных людей со всей округи и прилегающих областей. Иисус по\hyp{}настоящему любил людей --- самый обычный простой народ. Каждый месяц он делал выплаты за мастерскую и, с помощью Иакова, продолжал обеспечивать семью.
\vs p127 3:8 Несколько раз в году по субботам, когда не было приезжих гостей для проведения службы, Иисус продолжал читать по субботам Писание в синагоге и много раз при этом комментировал прочитанное, но обычно он так подбирал тексты, что комментарий не требовался. Он умел находить такой порядок чтения различных отрывков, что они разъясняли друг друга. Каждую субботу, если погода позволяла, после службы он брал братьев и сестер на загородные прогулки.
\vs p127 3:9 В то время хазан открыл молодежный клуб для философских диспутов, которые происходили в домах его членов и нередко в его собственном доме, и Иисус стал видным членом клуба. Это позволило ему отчасти восстановить свой авторитет среди местного населения, потерянный в результате недавних националистических разногласий.
\vs p127 3:10 Он продолжал поддерживать дружеские связи, хотя и ограниченные. Среди молодых мужчин и женщин Назарета у него было много сердечных приятелей и верных почитателей.
\vs p127 3:11 \pc В сентябре назаретскую семью навестили Елизавета с Иоанном. Иоанн, после того как его отец умер, предполагал вернуться в Иудейские горы, чтобы заняться сельским хозяйством и разведением овец, в том случае, если Иисус не посоветует ему остаться в Назарете, зарабатывая на жизнь плотницким ремеслом или иной работой. Гости не знали, что назаретская семья практически нищая. Чем больше Мария с Елизаветой беседовали о своих сыновьях, тем больше приходили к убеждению, что молодым людям хорошо бы вместе работать и больше видеться друг с другом.
\vs p127 3:12 Иисус и Иоанн тоже много говорили между собой; они обсудили многие очень интимные и личные вопросы. Когда эта встреча подошла к концу, они решили больше не видеться друг с другом до тех пор, пока не соединятся в общественном служении, когда «небесный Отец призовет» их к работе. Увиденное в Назарете привело Иоанна к твердому убеждению, что ему следует вернуться домой и работать, чтобы содержать свою мать. Он убедился, что ему предстоит участвовать в миссии Иисуса, но видел, что Иисусу еще много лет нужно поддерживать свою семью; поэтому он с удовольствием вернулся домой, чтобы вести хозяйство на своей маленькой ферме и заботиться о матери. И больше Иисус и Иоанн не видели друг друга до того дня, когда Сын Человеческий пришел на берег Иордана, чтобы принять крещение.
\vs p127 3:13 \pc В субботу 3 декабря этого года после полудня смерть второй раз поразила назаретский дом. Маленький Амос, самый младший из братьев, умер, неделю проболев с высокой температурой. После того, как Мария оправилась от этого горя, во время которого старший сын был ее единственной поддержкой, она наконец и в полной мере признала Иисуса настоящим главой семьи; и он воистину был достойным главой.
\vs p127 3:14 В течение четырех лет уровень их жизни постоянно снижался; год за годом семью все более и более сжимали тиски нужды. Конец этого года был одним из самых тяжелых периодов всей их изнурительной борьбы за выживание. Иаков еще не начал много зарабатывать, а расходы на похороны в добавление ко всему остальному совсем истощили их. Но Иисус лишь говорил своей обеспокоенной и скорбящей матери: «Мария, матушка, печаль не поможет нам; мы все делаем все, что в наших силах, но материнская улыбка, может быть, придаст нам силы делать еще лучше. Изо дня в день надежда на лучшее будущее поддерживает нас в стремлении справиться с нашими трудностями». Его стойкий и практический оптимизм был поистине заразителен: все дети жили в атмосфере ожидания лучших времен и лучших обстоятельств. Это полное надежды мужество во многом способствовало формированию сильных и благородных характеров, сложившихся, несмотря на гнет бедности.
\vs p127 3:15 Иисус обладал способностью эффективно мобилизовывать все силы ума, души и тела для решения насущной задачи. Он внутренне сосредотачивался на одной проблеме, которую хотел решить, и это, вкупе с его неиссякаемым \bibemph{терпением,} позволяло ему спокойно переносить испытания трудной жизни смертного --- жить так, как если бы он «видел Того, кто невидим».
\usection{4. Девятнадцатый год (13 г. н.э.)}
\vs p127 4:1 К этому времени Иисусу и Марии стало гораздо проще ладить друг с другом. Она воспринимала его больше как отца для своих детей, чем как сына. Каждый день был насыщен безотлагательными практическими проблемами. Иисус и Мария теперь реже говорили о жизненной миссии Иисуса, потому что чем дальше, тем более общие мысли их были заняты обеспечением и воспитанием семьи --- четырех мальчиков и трех девочек.
\vs p127 4:2 К началу этого года Иисус добился того, что Мария полностью согласилась с его принципом воспитания детей, основанном на позитивном предписании поступать хорошо, в отличие от традиционного еврейского принципа --- запрещать поступать дурно. И дома, и всегда в своей общественной деятельности учителя Иисус неизменно использовал \bibemph{позитивную} форму наставлений. Всегда и всюду он говорил: «Ты должен сделать то, тебе следует сделать это». Никогда не прибегал он к негативным формам поучения, происходящим от древних табу. Вместо того, чтобы запретами привлекать внимание к злу, Иисус привлекал внимание к добру, требуя его исполнения. В его доме час молитвы был временем, когда можно было обсуждать абсолютно все, относящееся к благоденствию семьи.
\vs p127 4:3 Иисус так рано начинал приучать своих братьев и сестер к мудрой дисциплине, что никогда не требовалось никаких наказаний, кроме минимальных, чтобы добиться их искреннего и мгновенного послушания. Единственным исключением являлся Иуда, которого Иисус в различных обстоятельствах находил необходимым наказывать за нарушение правил дома. Трижды, когда Иуду считали разумным наказать за сознательные и признанные им самим нарушения семейных правил поведения, наказание ему устанавливали единодушным решением старших детей и, прежде чем исполнить, получали на него согласие самого Иуды.
\vs p127 4:4 Иисус, будучи очень методичен и систематичен во всех своих действиях, при этом принимая то или иное решение всегда сохранял живительную гибкость толкований и индивидуальность приспособления, что заставляло всех детей глубоко ощущать дух справедливости, которым руководствовался их отец\hyp{}брат. Он никогда не занимался дисциплиной своих братьев и сестер по собственному произволу, и такая неизменная справедливость и в то же время индивидуальный подход снискали ему горячую любовь всей семьи.
\vs p127 4:5 Иаков и Симон росли, стараясь в отношениях со сверстниками следовать поведению Иисуса и умиротворять своих воинственных и порой яростных товарищей убеждением и непротивлением, и им это неплохо удавалось; в противоположность, Иосиф и Иуда, хотя дома соглашались с наставлениями Иисуса, при нападениях товарищей спешили защитить себя действием; в особенности Иуда был виновен в нарушении духа этих наставлений. Впрочем, непротивление не было \bibemph{правилом} для семьи. За невыполнение личных учений не наказывали.
\vs p127 4:6 Как правило, все дети, особенно девочки, советовались с Иисусом о своих детских проблемах и доверяли ему, как доверяли бы любящему отцу.
\vs p127 4:7 Иаков рос спокойным и уравновешенным юношей, но он не имел такого духовного настроя, как Иисус. Он был гораздо лучшим учеником, чем Иосиф, который, являясь добросовестным работником, был еще менее духовно настроен. Иосиф был просто трудяга и по интеллектуальному уровню ниже других детей. У Симона были всегда самые лучшие побуждения, но слишком много фантазий. Он медлил с обустройством в жизни и был причиной серьезного беспокойства Иисуса и Марии. Но он всегда был хорошим, с добрыми намерениями мальчиком. Иуда был бунтарь. У него были высочайшие идеалы, но неуравновешенный характер. Он был еще более решителен и энергичен, чем его мать, но гораздо менее, чем она, обладал чувством меры и такта.
\vs p127 4:8 Мириам была уравновешенная и рассудительная дочка, остро чувствующая возвышенное и духовное. Марфа была медлительным в мыслях и поступках, но очень надежным и способным ребенком. Малышка Руфь была словно солнышко в доме; легкомысленная болтушка, но очень сердечная. Она буквально боготворила своего старшего брата\hyp{}отца. Но она не была избалована. Она была очаровательным ребенком, хотя и не такая хорошенькая, как Мириам, первая красавица в семье, а может быть, и в городе.
\vs p127 4:9 \pc Со временем Иисус многое сделал для того, чтобы смягчить и изменить семейные воззрения и обычаи, связанные с соблюдением Субботы и со многими другими сторонами религиозной жизни, и со всеми переменами Мария искренне соглашалась. К этому времени Иисус уже стал бесспорным главой семьи.
\vs p127 4:10 В этом году Иуда пошел в школу, и, чтобы оплатить расходы, Иисусу пришлось продать свою арфу. Так исчезло последнее удовольствие, скрашивающее его досуг. Он очень любил играть на арфе, когда бывал утомлен умственно, душевно и физически, но теперь успокаивал себя мыслью, что, по крайней мере, арфа не достанется сборщикам налогов.
\usection{5. Ребекка, дочь Эзры}
\vs p127 5:1 Бедность не повредила общественному положению Иисуса в городе. Он был одним из выдающихся молодых людей Назарета, большинство молодых женщин были о нем очень высокого мнения. Поскольку Иисус был прекрасно физически и интеллектуально развит и, учитывая его репутацию духовного лидера, не удивительно, что Ребекка, старшая дочь Эзры, богатого назаретского купца и торговца, поняла, что постепенно все больше и больше влюбляется в сына Иосифа. Первым человеком, которому она доверилась, рассказав о своем чувстве, была Мириам, сестра Иисуса, а Мириам поведала все матери. Мария пришла в смятение. Неужели она потеряет сына, ставшего незаменимым главой семьи? Когда придет конец несчастьям? Что еще может произойти? Затем она задумалась о том, как повлияет женитьба на дальнейший жизненный путь Иисуса; она, хотя и не часто, но все же вспоминала порой, что Иисус --- «дитя обетованное». После того, как они с Мириам обсудили ситуацию, они решили попытаться остановить дело до того, как Иисус о нем узнает, пойти к Ребекке, открыть ей все и честно рассказать ей о своей вере в то, что Иисус --- сын предназначения; что ему суждено стать великим религиозным лидером, возможно --- Мессией.
\vs p127 5:2 Ребекка внимательно слушала; она была взволнована рассказом и еще более, чем прежде, укрепилась в намерении связать судьбу со своим избранником и разделить его стезю духовного лидера. Она убедила (себя), что такому мужчине тем более нужна преданная и энергичная жена. Она истолковала старания Марии отговорить ее как естественную реакцию на угрозу потерять главу и единственную опору семьи; но зная, что отец одобряет ее симпатию к сыну плотника, она справедливо рассчитывала, что он с радостью обеспечит семью доходом, полностью компенсирующим потерю заработков Иисуса. Когда отец согласился на этот план, последовали новые переговоры Ребекки с Марией и Мириам, и, не сумев добиться их поддержки, она осмелилась обратиться непосредственно к Иисусу. Она сделала это с помощью отца, пригласившего Иисуса к ним домой на празднование семнадцатого дня рождения Ребекки.
\vs p127 5:3 Иисус выслушал все внимательно и благожелательно сначала из уст отца Ребекки, затем от нее самой. Он благожелательно объяснил им, что никакие деньги не могут снять с него обязательство лично заботиться о семье своего отца, «выполнить самый священный человеческий долг --- верности собственной плоти и крови». Отец Ребекки был глубоко тронут словами Иисуса о семейной преданности и не стал продолжать разговор. Марии, своей жене, он сказал лишь одно: «Он не может быть нам сыном; он слишком благороден для нас».
\vs p127 5:4 Затем произошел тот знаменательный разговор с Ребеккой. До сих пор Иисус в общении почти не делал различий между мальчиками и девочками, молодыми мужчинами и молодыми женщинами. Его голова была слишком занята насущными проблемами в практических земных делах, с одной стороны, и поглощена размышлениями о предстоящем пути, связанном с «работой в том, что принадлежит Отцу», --- с другой, чтобы когда\hyp{}либо серьезно думать о том, чтобы придать личной любви завершенную форму человеческого брака. Теперь ему пришлось встретиться лицом к лицу с еще одной из тех проблем, которые должен встретить и разрешить в своей жизни каждый обычный человек. Воистину, он прошел все те же «испытания, что и вы».
\vs p127 5:5 Внимательно выслушав Ребекку, Иисус искренне поблагодарил ее за высказанное ею восхищение, добавив: «Это будет ободрять и поддерживать меня всю мою жизнь». Потом он объяснил, что не вправе входить в отношения с женщиной, иные, чем чисто братские и дружеские. Он дал понять Ребекке, что его первый и высший долг --- вырастить детей его отца, и пока он его не выполнит, он не может думать о женитьбе, и затем он добавил: «Если я --- сын предназначения, я не должен принимать на себя пожизненные обязательства до тех пор, пока мое предназначение не явит себя».
\vs p127 5:6 Сердце Ребекки было разбито. Она была безутешна и настойчиво просила отца покинуть Назарет, пока в конце концов он не согласился перебраться в Сефорис. В последующие годы многочисленным мужчинам, искавшим ее руки, она давала один и тот же ответ. Единственной целью ее жизни было дождаться часа, когда этот человек, для нее величайший из всех, кто когда\hyp{}либо жил, явит себя как учитель живой истины. И она преданно следовала за ним в течение всех богатых событиями лет его общественного служения, присутствовала (незамеченная Иисусом) при его триумфальном въезде в Иерусалим; и она стояла «среди других женщин» рядом с Марией в тот роковой и трагический послеполуденный час, когда Сын Человеческий был распят на кресте, --- он для нее, так же, как и для бесчисленных небесных миров, «прекраснейший и величайший, лучший из лучших».
\usection{6. Двадцатый год (14 г. н.э.)}
\vs p127 6:1 Историю любви Ребекки к Иисусу по секрету рассказывали в Назарете и позже в Капернауме, поэтому в последующие годы, хотя многие женщины, как и мужчины, любили его, ему не пришлось отвергать личного предложения беззаветной любви и преданности еще какой\hyp{}нибудь хорошей женщины. Любовь людей к Иисусу с этой поры больше носила характер преклонения и восхищенного почитания. И мужчины, и женщины любили его преданно таким, как он есть, без малейшей тени самоудовлетворения или стремления любовного обладания. Но многие годы во всех рассказах о человеческой личности Иисуса присутствовала история о беззаветной любви Ребекки.
\vs p127 6:2 Мириам, которая знала от начала до конца историю с Ребеккой и видела, как ее брат отказался даже от любви столь прекрасной девушки (не осознавая, что будущий судьбоносный жизненный путь Иисуса был фактором в его решении), с тех пор идеализировала Иисуса и любила его трогательной и глубокой любовью равно как отца и как брата.
\vs p127 6:3 \pc Иисус испытывал странное желание быть в Иерусалиме на Пасху, хотя семья едва ли могла позволить себе такие расходы. Мать, зная о недавних его переживаниях с Ребеккой, мудро убеждала его совершить это путешествие. Не вполне сознавая это, на самом деле Иисус больше всего хотел воспользоваться возможностью поговорить с Лазарем, увидеться с Марфой и Марией. После своей семьи он любил этих троих больше всех на свете.
\vs p127 6:4 На этот раз он шел в Иерусалим через Мегиддо, Антипатриду и Лидду, отчасти повторяя тот маршрут, которым его везли в Назарет при возвращении из Египта. Он провел четыре дня в этом путешествии на Пасху и все это время много думал об исторических событиях, происходивших в Мегиддо и его окрестностях, на этом международном поле битвы Палестины.
\vs p127 6:5 Иисус прошел через Иерусалим, остановившись лишь один раз, чтобы взглянуть на храм и собиравшиеся толпы посетителей. У него было странное и все возрастающее внутреннее неприятие этого построенного Иродом храма с его священством, назначаемым светскими властями. Больше всего он хотел видеть Лазаря, Марфу и Марию. Лазарь был ровесником Иисуса и ныне главой семьи: к этому времени мать его также покоилась рядом с предками. Марфа была почти на год старше Иисуса; Мария --- двумя годами младше. Иисус был кумиром всех троих.
\vs p127 6:6 В это посещение произошла одна из периодически повторявшихся вспышек негодования Иисуса против традиции --- выражение его возмущения существующими ритуалами, которые, по его мнению, представляли в ложном свете его Отца Небесного. Не зная о прибытии Иисуса, Лазарь договорился с друзьями, жившими в соседней деревне ниже по Иерихонской дороге, праздновать Пасху у них. Иисус же предложил встретить праздник там, где они были, в доме Лазаря. «Но у нас нет пасхального ягненка», --- ответил Лазарь. Тогда Иисус стал долго и убедительно рассуждать о том, что Отца на небесах на самом деле не интересуют подобные ребяческие и бессмысленные ритуалы. После торжественной и горячей молитвы они встали, и Иисус сказал: «Пусть темные и наивные умы людей моего народа служат своему Богу так, как указал Моисей; это лучше для них, но мы, видевшие свет жизни, да не приблизимся отныне к нашему Отцу через тьму смерти. Будем свободны, ибо знаем истину о вечной любви Отца».
\vs p127 6:7 Тем вечером в сумерках они вчетвером сели за первую среди благочестивых евреев пасхальную трапезу без пасхального ягненка. На этот пасхальный пир приготовлены были пресные лепешки и вино, и эти символы, которые Иисус назвал «хлеб жизни» и «вода жизни», он сам подавал своим товарищам, и они ели торжественно принимая только что провозглашенное учением. Для Иисуса с этих пор стало обыкновением выполнять этот священный ритуал всегда, когда он бывал в Вифании. Вернувшись домой, он рассказал обо всем матери. Она сначала была поражена, но постепенно смогла понять его точку зрения; тем не менее она испытала огромное облегчение, когда Иисус уверил ее, что не собирается вводить это новое понимание Пасхи в их семье. Дома с детьми он из года в год продолжал устраивать пасхальную трапезу «согласно закону Моисея».
\vs p127 6:8 \pc В этом году у Марии был долгий разговор с Иисусом о женитьбе. Она прямо спросила, женился ли бы он, если бы был свободен от ответственности перед семьей. Иисус объяснил, что мало думал об этом, поскольку непосредственные его обязанности исключают эту возможность. Он выразил сомнение в том, что когда\hyp{}либо вступит в брак; он сказал, что со всеми подобными вещами следует ждать до наступления «моего часа», когда «должна начаться работа моего Отца». Уже решив про себя, что ему не суждено стать отцом детей во плоти, он очень мало думал о человеческом браке.
\vs p127 6:9 В этом году он снова начал без устали трудиться над дальнейшим сплетением своих божественной и человеческой природы в простую и эффективную \bibemph{человеческую индивидуальность.} И его моральные устои и духовное восприятие продолжали развиваться.
\vs p127 6:10 Хотя вся их собственность в Назарете (кроме их дома) была утрачена, в этом году они получили немного денег от продажи доли своей недвижимости в Капернауме. Это было последнее, что у них оставалось из собственности Иосифа. Сделка была заключена со строителем лодок по имени Зеведей.
\vs p127 6:11 В этом году Иосиф окончил синагогальную школу и был готов начать работу за небольшим верстаком в домашней мастерской. Хотя наследство, оставленное отцом, истощилось, однако, поскольку уже трое из них стали постоянными работниками, появилась надежда наконец выбраться из нужды.
\vs p127 6:12 \pc Иисус быстро становится мужчиной, не просто молодым человеком, но зрелым мужем. Он хорошо научился нести ответственность. Он знает, как жить, невзирая на разочарования. Он сохраняет мужество, когда его планы расстраиваются и замыслы временно разрушаются. Он научился быть честным и справедливым даже перед лицом несправедливости. Он учится тому, как совмещать свои идеалы духовной жизни с практическими требованиями земного существования. Он учится ставить себе высшую и дальнюю идеалистическую цель и, в то же время ревностно трудиться над решением ближайших насущных жизненных задач. Он постепенно овладевает искусством соразмерять свои устремления с повседневными требованиями человеческой жизни. Он уже почти овладел искусством, как силой духовного побуждения направлять механизм материальных достижений. Он постепенно учится тому, как жить небесной жизнью и в то же время продолжать свое земное существование. Выполняя в своей земной семье роль отца в руководстве детьми и их воспитании, он все более и более вверяет себя предельному руководству своего небесного Отца. Он обретает опыт, как искусно вырывать победу из самых когтей поражения; он обучается тому, как превращать временные препятствия в триумфы вечности.
\vs p127 6:13 \pc Итак, идут годы, этот молодой человек из Назарета постигает на собственном опыте жизнь смертных, обитающих в мирах времени и пространства. Он живет полной, насыщенной, типичной жизнью на Урантии. Он оставил этот мир, обладая зрелым опытом, который приобретают его создания в течение их короткой и напряженной первой жизни, жизни во плоти. И весь этот человеческий опыт является вечным достоянием Владыки Вселенной. Он наш отзывчивый брат, благожелательный друг, опытный властитель и милосердный отец.
\vs p127 6:14 Будучи ребенком, он приобрел колоссальные знания; в юности он классифицировал, осмыслил, проработал эту информацию; ныне, став взрослым человеком этого мира, он начинает организовывать эти интеллектуальные сокровища в систему, чтобы принять их в последующем учительстве, служении и деятельности на благо своих смертных собратьев в этом и всех других обитаемых мирах во всей вселенной Небадон.
\vs p127 6:15 Придя в этот мир младенцем, он прожил в нем детство, прошел последовательно периоды отрочества и юной зрелости. Теперь он стоит на пороге полной зрелости, обогатившийся опытом человеческой жизни, достигший глубокого понимания человеческой природы и полный сочувствия к ее слабостям. Он прекрасно владеет божественным искусством откровения его Небесного Отца смертным созданиям всех возрастов и уровней развития.
\vs p127 6:16 И теперь, как взрослый человек --- зрелый обитатель этого мира, --- он готовится продолжать свою верховную миссию: открыть Бога людям и вести людей к Богу.
