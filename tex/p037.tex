\upaper{37}{Личности локальной вселенной}
\author{Блестящая Вечерняя Звезда}
\vs p037 0:1 Во главе всех личностей в Небадоне стоит Сын\hyp{}Творец, Сын\hyp{}Мастер Михаил, отец и владыка вселенной. Равноправна с ним по божественности и дополняет по творческим качествам Дух\hyp{}Мать локальной вселенной, Божественная Служительница Спасограда. И эти творцы в самом буквальном смысле являются Отцом\hyp{}Сыном и Духом\hyp{}Матерью всех исконных созданий Небадона.
\vs p037 0:2 В предшествующих текстах рассматривались сотворенные чины сыновства; в последующих повествованиях будут описаны духи\hyp{}служители и восходящие чины сыновства. Этот текст посвящен, главным образом, группе, занимающей промежуточное положение --- Вселенским Помощникам, но в нем также будут кратко рассмотрены некоторые из высших духов, постоянно пребывающих в Небадоне, и некоторые чины имеющие постоянное гражданство в локальной вселенной.
\usection{1. Вселенские Помощники}
\vs p037 1:1 В этих текстах Вселенские Помощники включают следующие семь чинов (многие из уникальных чинов, которых обычно относят к этой категории, здесь не представлены):
\vs p037 1:2 \ublistelem{1.}\bibnobreakspace Яркие и Утренние Звезды.
\vs p037 1:3 \ublistelem{2.}\bibnobreakspace Блестящие Вечерние Звезды.
\vs p037 1:4 \ublistelem{3.}\bibnobreakspace Архангелы.
\vs p037 1:5 \ublistelem{4.}\bibnobreakspace Всевышние Помощники.
\vs p037 1:6 \ublistelem{5.}\bibnobreakspace Высокие Уполномоченные.
\vs p037 1:7 \ublistelem{6.}\bibnobreakspace Небесные Надзиратели.
\vs p037 1:8 \ublistelem{7.}\bibnobreakspace Учителя Миров\hyp{}Обителей.
\vs p037 1:9 \pc В каждой локальной вселенной всего один представитель первого чина Вселенских Помощников --- Ярких и Утренних Звезд, и он родился первым из всех созданий, являющихся исконными жителями данной локальной вселенной. Яркая и Утренняя Звезда нашей вселенной известна как Гавриил Спасоградский. Он --- главный распорядитель Небадона, личный представитель Сына\hyp{}Владыки, выступающий также и от имени его творческой союзницы.
\vs p037 1:10 В ранние времена существования Небадона Гавриил действовал только лишь с Михаилом и Творческим Духом. С развитием вселенной и приумножением административных проблем ему был предоставлен штат личных нераскрытых помощников, и со временем эта группа разрослась в результате сотворения Небадонского отряда Вечерних Звезд.
\usection{2. Блестящие Вечерние Звезды}
\vs p037 2:1 Эти блестящие создания были спланированы Мелхиседеками и затем созданы Сыном\hyp{}творцом и Творческим Духом. Они служат во многих качествах, но главным образом как связные Гавриила, главного распорядителя локальной вселенной. Одно или несколько из этих существ представляют его в столицах всех созвездий и систем в Небадоне.
\vs p037 2:2 Гавриил как главный распорядитель по своей должности является председателем или наблюдателем на большинстве конклавов, и нередко бывает, что их одновременно собирается до тысячи. В этих случаях Гавриила представляют Блестящие Вечерние Звезды; он не может быть одновременно в двух местах, и эти сверхангелы восполняют это ограничение. Они выполняют аналогичную службу и для отряда Сынов\hyp{}Учителей Троицы.
\vs p037 2:3 Хотя Гавриил лично занят административными обязанностями, но через Блестящие Вечерние Звезды он причастен ко всем другим аспектам вселенской жизни и дел. Они всегда сопровождают его в планетарных путешествиях и нередко отправляются на отдельные планеты исполнять особые миссии в качестве его личных представителей. При выполнении таких поручений их иногда называли «ангелами Господними». Они часто отправляются на Уверсу представлять Яркую и Утреннюю Звезду в судах и собраниях Древних Дней, но редко выходят за пределы Орвонтона.
\vs p037 2:4 \pc Блестящие Вечерние Звезды --- это уникальный двоякий чин, в который входят и те, кто от роду обладает этим высоким положением, и те, кто достиг его служением. Небадонский отряд этих сверхангелов насчитывает сейчас 13\,641 звезду. Из них 4\,832 были сотворены таковыми, а 8\,809 являются восходящими духами, достигшими этой цели возвышенного служения. Многие из этих восходящих Вечерних Звезд начали свой вселенский путь как серафимы; другие совершали восхождение с нераскрытых уровней живых созданий. Как цель для достижения этот высокий отряд никогда не закрыт для кандидатов на восхождение до тех пор, пока соответствующая вселенная не установлена в свете и жизни.
\vs p037 2:5 Оба типа Блестящих Вечерних Звезд легко видимы для моронтийных личностей и отдельных сверхчеловеческих типов материальных существ. Сотворенные существа этого интересного и разностороннего чина обладают духовной силой, которая может проявляться независимо от их личного присутствия.
\vs p037 2:6 \pc Глава этих сверхангелов --- Гавалия, первый представитель этого чина, родившийся в Небадоне. Со времени возвращения Христа\hyp{}Михаила из его триумфального пришествия на Урантию Гавалия назначен осуществлять служение восходящим смертным, и последние тысяча девятьсот урантийских лет его сподвижник Галантия имеет в Иерусеме свой центр, где он проводит примерно половину всего времени. Галантия --- первый из восходящих сверхангелов, достигший этого высокого статуса.
\vs p037 2:7 Не существует групповой или коллективной организации Блестящих Вечерних Звезд, кроме обычного соединения в пары при выполнении многих заданий. Им не очень часто поручают миссии, связанные с восходящим путем смертных, но когда они получают такие задания, то никогда не действуют в одиночку. В таких случаях они всегда действуют вдвоем: одна --- сотворенная Вечерняя Звезда, другая --- ставшая таковой в результате восхождения.
\vs p037 2:8 Одна из высоких обязанностей Вечерних Звезд --- сопровождать Сынов\hyp{}Авоналов Пришествия в ходе выполнения их миссий, точно так же, как Гавриил сопровождал Михаила во время его пришествия на Урантию. Двое из сопровождающих сверхангелов являются главными личностями в таких миссиях и служат сокомандирами архангелов и всех прочих, назначаемых на эти должности. И именно старший из этих сверхангелов\hyp{}командиров в некий знаменательный момент вечности повелит Сыну\hyp{}Авоналу Пришествия: «Занимайся делами брата своего».
\vs p037 2:9 Подобные же пары этих сверхангелов назначаются в планетарные отряды Сынов\hyp{}Учителей Троицы, которые действуют с целью установления эпохи послепришествия --- зарождающейся духовной эпохи обитаемого мира. При выполнении таких заданий Вечерние Звезды служат связными между смертными мира сего и невидимым отрядом Сынов\hyp{}Учителей.
\vs p037 2:10 \pc \bibemph{Миры Вечерних Звезд.} Шестая группа, состоящая из семи миров Спасограда и подчиненных им сорока двух спутников поручена управлению Блестящих Вечерних Звезд. Семь первичных миров возглавляют сотворенные чины этих сверхангелов, в то время как подчиненные им спутники управляются восходящими Вечерними Звездами.
\vs p037 2:11 Спутники первых трех миров предназначены для школ Сынов\hyp{}Учителей и Вечерних Звезд, посвященных духовным личностям локальной вселенной. На спутниках следующих трех миров аналогичные совместные школы заняты обучением восходящих смертных. Спутники седьмого мира отведены для триединых совещаний Сынов\hyp{}Учителей, Вечерних Звезд и финалитов. В последнее время эти сверхангелы тесно взаимодействовали с Отрядом Финалитов в локальной вселенной и долго были связаны с Сынами\hyp{}Учителями. Существует необыкновенно тесная и важная связь между Вечерними Звездами и Вестниками Гравитации, направленными в рабочие группы финалитов. Сам седьмой первичный мир зарезервирован для нераскрытых дел, касающихся будущих взаимоотношений, которые установятся между Сынами\hyp{}Учителями, финалитами и Вечерними Звездами вслед за завершением появления сверхвселенского выражения личности Бога Верховного.
\usection{3. Архангелы}
\vs p037 3:1 Архангелы являются потомками Сына\hyp{}Творца и Духа\hyp{}Матери Вселенной. Это высший тип высоких духовных существ, создаваемых в локальной вселенной в больших количествах, и во время последней регистрации в Небадоне их было почти восемьсот тысяч.
\vs p037 3:2 Архангелы --- это одна из немногих групп личностей локальной вселенной, на которые обычно не распространяется юрисдикция Гавриила. Они совершенно не занимаются повседневным административным управлением вселенной, их деятельность связана с продолжением существования созданий и содействует восходящему продвижению смертных времени и пространства. Хотя архангелы, как правило, не подчинены Яркой и Утренней Звезде, иногда они действуют с его санкции. Они сотрудничают также и с другими Вселенскими Помощниками, в частности, с Вечерними Звездами, как это видно на примере определенных событий, описанных в разделе о трансплантации жизни в вашем мире.
\vs p037 3:3 \pc Отрядом архангелов Небадона управляет первородный этого чина, а с относительно недавнего времени на Урантии располагается региональный центр архангелов. Именно этот необычный факт сразу же привлекает внимание приезжих из\hyp{}за пределов Небадона учащихся. Уже из первых наблюдений внутривселенских дел им становится ясно, что многими видами деятельности Блестящих Вечерних Звезд, связанными с восхождением, управляют из столицы локальной системы --- Сатании. При дальнейшем наблюдении они обнаруживают, что определенными видами деятельности архангелов управляют из маленького и, видимо, незначительного обитаемого мира, называемого Урантией. И тогда происходит откровение пришествия Михаила на Урантию, и у них сразу же пробуждается интерес к вам и вашей скромной планете.
\vs p037 3:4 Можете ли вы постичь значение того факта, что ваша скромная и смятенная планета стала региональным центром вселенского руководства и управления определенными видами деятельности архангелов, связанными с Райской системой восхождения? Это, несомненно, предвещает будущее средоточие других связанных с восхождением видов деятельности в том мире, куда совершил пришествие Михаил, и придает огромную и торжественную важность личному обещанию Учителя: «Я приду снова».
\vs p037 3:5 \pc Вообще, архангелы назначаются на службу и служение чину сыновства Авоналов, но только после того, как пройдут обширное предварительное обучение по всем аспектам деятельности разнообразных духов\hyp{}служителей. Сотенный отряд сопровождает каждое Райское пришествие Сына в обитаемый мир, будучи назначен к нему на время продолжения этого пришествия. Если бы Сын\hyp{}Повелитель временно стал правителем планеты, эти архангелы действовали бы как главные управляющие всех небесных существ на этой планете.
\vs p037 3:6 Два старших архангела всегда назначаются личными помощниками Райского Авонала во всех планетарных миссиях, связанных и с судебными действиями, и с миссиями повеления, и с воплощениями в связи с пришествием. Когда Райский Сын закончил суд мира сего и мертвые призваны к ответу (так называемое воскресение), тогда, действительно, истинно сказано, что серафимы\hyp{}хранительницы упокоившихся личностей откликаются на «глас архангела». По окончании диспенсации присутствующим архангелом проводится поверка. Это архангел воскресения, называемый иногда «архангелом Михаила».
\vs p037 3:7 \pc \bibemph{Миры архангелов.} Седьмая группа образующих кольцо миров Спасограда вместе со связанными с ними спутниками отведена архангелам. Сфера номер один и все шесть подчиненных ей спутников заняты протоколистами, ведущими записи о личностях. Этот огромный отряд протоколистов занимается непосредственным ведением записей о каждом смертном, живущем во времени, с момента его рождения, на протяжении вселенского пути и до тех пор, когда индивидуум или перейдет из Спасограда под власть сверхвселенной, или будет «вычеркнут из протоколов существования» указом Древних Дней.
\vs p037 3:8 Именно в этих мирах сведения о личности и подтверждающие свидетельства систематизируются, подшиваются в дело и сохраняются на протяжении всего времени, которое проходит между смертью человека и часом реперсонализации, воскресения после смерти.
\usection{4. Всевышние Помощники}
\vs p037 4:1 Всевышние Помощники --- это группа существ\hyp{}добровольцев, родившихся за пределами локальной вселенной и временно назначенных в локальную вселенную как представители и наблюдатели центральной вселенной и сверхвселенной. Их численность постоянно изменяется, но всегда исчисляется многими миллионами.
\vs p037 4:2 Таким образом, время от времени мы извлекаем пользу из служения и помощи таких имеющих райское происхождение существ, как Совершенствователи Мудрости, Божественные Советники, Вселенские Цензоры, Вдохновленные Духи Троицы, Тринитизированные Сыны, Одиночные Вестники, супернафимы, секонафимы, терциафимы и другие милосердные служители, пребывающие с нами с целью помочь нашим исконным личностям в стремлении привести весь Небадон в более полную гармонию с идеями Орвонтона и идеалами Рая.
\vs p037 4:3 Любые из этих существ могут добровольно служить в Небадоне, и потому формально они вне нашей юрисдикции, но когда такие личности сверхвселенной и центральной вселенной действуют по заданию, они не совсем свободны от правил той локальной вселенной, в которой пребывают, хотя и продолжают оставаться представителями более высоких вселенных и действовать в соответствии с инструкциями, определяющими их миссию в нашей сфере. Их общий центр находится в спасоградском секторе Объединяющего Дней, и они действуют в Небадоне под общим руководством этого посланца Райской Троицы. Когда эти личности служат не в прикомандированных к кому\hyp{}либо группах, то обычно являются самоуправляющимися, но когда они служат по чьей\hyp{}либо просьбе, то часто добровольно поступают под полный контроль руководящих управителей тех сфер, в которые они назначены действовать.
\vs p037 4:4 Всевышние Помощники занимают различные посты в локальных вселенных и в созвездиях, но не входят непосредственно в правительства систем или планет. Однако они могут функционировать повсюду в локальной вселенной и назначаться для выполнения любого вида деятельности в Небадоне --- административной, распорядительной, образовательной и других.
\vs p037 4:5 Большая часть этого отряда привлечена в помощь Райским личностям Небадона --- Объединяющему Дней, Сыну\hyp{}Творцу, Верным Дней, Сынам\hyp{}Повелителям и Сынам\hyp{}Учителям Троицы. Время от времени при ведении дел локальной вселенной оказывается целесообразным скрывать какие\hyp{}то детали практически от всех исконных жителей данной локальной вселенной. Отдельные продвинутые планы и сложные постановления также лучше и полнее воспринимаются более зрелым и прозорливым отрядом Всевышних Помощников, который именно в таких ситуациях (как и во многих других) приносит огромную пользу правителям и администраторам вселенной.
\usection{5. Высокие Уполномоченные}
\vs p037 5:1 Высокие Уполномоченные --- это слившиеся с Духом восходящие смертные; они не слиты с Настройщиками. У вас есть вполне отчетливое понимание вселенского пути восхождения смертных кандидатов на слияние с Настройщиками, которое со времени пришествия Христа\hyp{}Михаила в перспективе и является высоким предназначением для всех смертных Урантии. Но это не исключительное предназначение всех смертных в эпохи, предшествующие пришествию, в мирах, подобных вашему; существует и другой тип миров, в обитателях которых никогда постоянно не пребывают Настройщики Мысли. Такие смертные никогда навсегда не сливаются воедино с Таинственным Наблюдателем Райского пришествия; тем не менее Настройщики пребывают в них временно, служа проводниками и паттернами в течение жизни во плоти. На протяжении этого временного пребывания они способствуют развитию бессмертной души точно так же, как и тех существ, с которыми они надеются слиться, но когда путь смертного пройден, они навеки покидают создания, с которыми были временно соединены.
\vs p037 5:2 Продолжающие существование души этого чина достигают бессмертия путем слияния навеки с индивидуализированным фрагментом духа Духа\hyp{}Матери локальной вселенной. Они составляют немногочисленную группу, по крайней мере, в Небадоне. В мирах\hyp{}обителях вы встретитесь и будете общаться с этими смертными, слившимися с Духом, так как они вместе с вами совершают восхождение по Райскому пути до Спасограда, где их восхождение прекращается. Некоторые из них смогут впоследствии совершить восхождение до более высоких вселенских уровней, но большинство навсегда останутся на службе в локальной вселенной; как классу им не предначертано достичь Рая.
\vs p037 5:3 Не слившись с Настройщиками, они никогда не становятся финалитами, но со временем зачисляются в Отряд Совершенства локальной вселенной. В душе они повиновались велению Отца: «Будьте совершенны».
\vs p037 5:4 \pc По достижении Небадонского Отряда Совершенства слившиеся с Духом восходящие могут быть назначены Вселенскими Помощниками, и это только один из открытых для них путей продолжения практического роста. Таким образом они становятся кандидатами на поручение им высокой службы --- объяснения точек зрения эволюционирующих созданий материальных миров небесным властям локальной вселенной.
\vs p037 5:5 Высокие Уполномоченные начинают свою службу на планетах как уполномоченные по расам. В этом качестве они объясняют точки зрения и описывают потребности различных человеческих рас. Они в высшей степени преданы благоденствию человеческих рас, от имени которых и выступают, всегда стремясь добиться для них милосердия, правосудия и справедливого обхождения во всех взаимоотношениях с другими народами. Уполномоченные по расам действуют в бесконечном ряду планетарных кризисов и отчетливо представляют целые группы борющихся с трудностями смертных.
\vs p037 5:6 После продолжительного опыта решения проблем в обитаемых мирах эти уполномоченные по расам продвигаются на более высокие уровни функционирования, в конечном счете достигая статуса Высоких Уполномоченных локальной вселенной. Последняя регистрация зафиксировала в Небадоне чуть больше полутора миллиардов этих Высоких Уполномоченных. Они не являются финалитами, но эти восходящие существа обладают богатым опытом и принесли своей родной сфере большую пользу.
\vs p037 5:7 Мы неизменно обнаруживаем этих уполномоченных во всех судах правосудия, от низших до высших. Они не участвуют в судебных процедурах, но в качестве товарищей судей консультируют председательствующих магистратов по вопросам прошлой жизни, окружающей обстановки и свойств характера тех, в отношении кого выносится судебное решение.
\vs p037 5:8 Высокие Уполномоченные придаются различным сонмам вестников пространства и всегда --- духам\hyp{}служителям времени. Их встретишь на различных вселенских ассамблеях, и эти же самые уполномоченные, прекрасно знающие смертных, всегда прикомандировываются к миссиям Сынов Бога в миры пространства.
\vs p037 5:9 Всегда, когда для осуществления справедливости и правосудия требуется понимание того, как предполагаемая политика или методика подействует на эволюционные расы времени, эти уполномоченные находятся рядом и готовы представить свои рекомендации; они всегда присутствуют, чтобы выступить за тех, кто не смог присутствовать и говорить сам.
\vs p037 5:10 \pc \bibemph{Миры слившихся с Духом смертных.} Восьмая группа из семи первичных миров и подчиненные им спутники в спасоградском контуре являются исключительно владениями слившихся с Духом смертных Небадона. Восходящие смертные, слившиеся с Настройщиками, не имеют отношения к этим мирам, кроме тех случаев, когда они приятно и с пользой проводят там время в качестве приглашенных гостей слившихся с Духом обитателей.
\vs p037 5:11 Эти миры являются постоянным местом обитания слившихся с Духом продолжающих существование --- за исключением тех немногих, кто достиг Уверсы и Рая. Такое умышленное ограничение восхождения смертных идет на пользу локальным вселенным, обеспечивая сохранение прошедшего эволюцию постоянного населения, возрастающий опыт которого в будущем усилит стабильность и диверсификацию управления локальной вселенной. Эти существа могут не достигнуть Рая, но в разрешении проблем Небадона они постигают основанную на опыте мудрость, а это намного превосходит все то, что достигают движущиеся дальше восходящие. И эти продолжающие существование души остаются уникальным сочетанием человеческого и божественного, становясь все более способными объединять эти две сильно обособленные точки зрения и с постоянно растущей мудростью представлять двуединую точку зрения.
\usection{6. Небесные Надзиратели}
\vs p037 6:1 Образовательная система Небадона управляется совместно Сынами\hyp{}Учителями Троицы и обучающим отрядом Мелхиседеков, но значительную часть работы по ее поддержанию и построению исполняют Небесные Надзиратели. Эти существа составляют отряд, в который включены все типы индивидуумов, связанных с системой образования и обучения восходящих смертных. В Небадоне их свыше трех миллионов, и все они --- добровольцы, обладающие благодаря своему опыту достаточной квалификацией, чтобы служить всему Небадону советчиками по образованию. Из своего центра в спасоградских мирах Мелхиседеков эти надзиратели передвигаются по локальной вселенной в качестве инспекторов по небадонской методике преподавания, предназначенной осуществлять обучение разума и духовное образование восходящих созданий.
\vs p037 6:2 Обучение разума и образование духа начинается в мирах, из которых происходят люди, и далее --- в мирах\hyp{}обителях систем и на других связанных с Иерусемом сферах продвижения, на семидесяти прикрепленных к Эдентии сферах, готовящих к коллективной деятельности, и на четырехстах девяноста сферах духовного прогресса, окружающих Спасоград. В самом вселенском центре находятся многочисленные Мелхиседекские школы, колледжи Сынов Вселенной, университеты серафимов и школы Сынов\hyp{}Учителей и Объединяющего Дней. Предпринимаются все возможные меры, чтобы подготовить разные личности вселенной к все более высокой службе и более совершенному функционированию. Вся вселенная --- это одна огромная школа.
\vs p037 6:3 \pc Методы, применяемые во многих высших школах, не укладываются в рамки человеческих представлений об искусстве обучения истине, но основной принцип всей образовательной системы таков: характер обретается через просвещенный опыт. Учителя дают просвещение; положение во вселенной и статус восходящего предоставляют возможность получать опыт; мудрое использование того и другого укрепляют характер.
\vs p037 6:4 По сути, образовательная система Небадона предусматривает поручение какой\hyp{}нибудь задачи, а затем дает возможность получать наставления относительно идеального и божественного способа ее исполнения наилучшим образом. Дается определенная задача, которую надо выполнить, и одновременно предоставляются учителя, способные научить, как лучше всего выполнить эту задачу. Божественный план образования обеспечивает тесную связь практической деятельности и обучения. Мы учим, как лучше всего выполнять то, что мы велим вам делать.
\vs p037 6:5 \pc Цель всего этого обучения и опыта --- подготовить к приему в более высокие и более духовные учебные сферы сверхвселенной. Продвижение вперед внутри каждой данной сферы происходит индивидуально, но переход от одной фазы к другой обычно осуществляется в составе учебной группы.
\vs p037 6:6 Продвижение в вечности не сводится лишь к духовному развитию. Интеллектуальные приобретения тоже являются частью вселенского образования. Опыт разума расширяется в равной степени с расширением духовных горизонтов. Разуму и духу предоставляются одинаковые возможности для обучения и продвижения вперед. К тому же при всем этом великолепном обучении разума и духа ты теперь навсегда свободен от препятствий, чинимых смертной плотью. Тебе уже больше не придется выбирать между духовной и материальной сущностями своей натуры, находящимися друг с другом в постоянном противоречии. Наконец\hyp{}то ты способен испытывать исключительно притяжение прославленного разума, задолго до этого освободившегося от примитивной животной склонности к материальному.
\vs p037 6:7 \pc Прежде чем покинуть вселенную Небадон, большинство смертных Урантии получит возможность послужить то или иное время в качестве членов Небадонского отряда Небесных Надзирателей.
\usection{7. Учителя Миров\hyp{}Обителей}
\vs p037 7:1 Учителя Миров\hyp{}Обителей набираются из числа прославленных херувимов. Как и большинство других преподавателей в Небадоне, они назначаются Мелхиседеками. Они принимают участие в большинстве образовательных предприятий моронтийной жизни, и их численность выходит за пределы понимания человеческого разума.
\vs p037 7:2 Учителя Миров\hyp{}Обителей как уровень, достигаемый херувимами и сановимами, будут рассмотрены далее в следующем тексте, а как учителя, играющие важную роль в моронтийной жизни, будут подробнее обсуждаться в тексте с соответствующим названием.
\usection{8. Высшие духовные чины назначения}
\vs p037 8:1 Кроме центров мощи и физических контролеров, некоторые из духовных существ более высокого происхождения из семьи Бесконечного Духа имеют постоянное назначение в локальной вселенной. Из высших духовных чинов семьи Бесконечного Духа такое назначение получили:
\vs p037 8:2 \pc \bibemph{Одиночные Вестники,} когда они функционально прикреплены к администрации локальной вселенной, оказывают нам неоценимую помощь в усилиях преодолеть препятствия времени и пространства. Когда же они не имеют такого назначения, мы, обитатели локальных вселенных, не имеем над ними абсолютно никакой власти, но даже и тогда эти уникальные существа всегда готовы помочь нам в решении проблем и исполнении наших задач.
\vs p037 8:3 Андовонтия --- имя третичного \bibemph{Руководителя Контуров Вселенных,} назначенного в нашей локальной вселенной. Он занимается только духовными и моронтийными контурами, а не теми, что подведомственны управителям мощи. Именно он, когда Калигастия предал планету, изолировал Урантию на время испытаний, связанных с бунтом Люцифера. Посылая приветствия смертным Урантии, он провидчески выражает уверенность, что когда\hyp{}нибудь вы будете восстановлены в контурах вселенных, которыми он руководит.
\vs p037 8:4 \bibemph{Управитель Переписи} Небадона, Салсатия, имеет свой центр в спасоградском секторе Гавриила. Он автоматически бывает осведомлен о рождении и смерти воли и регистрирует точное на каждый данный момент число существ, обладающих волей, функционирующих в локальной вселенной. Он действует в тесной связи с протоколистами личностей, пребывающими в протокольных мирах архангелов.
\vs p037 8:5 \bibemph{Инспектор\hyp{}Сподвижник} пребывает в Спасограде. Он --- личный представитель Верховного Распорядителя Орвонтона. Его сподвижники --- \bibemph{Назначенные Стражи} локальной системы --- также являются представителями Верховного Распорядителя Орвонтона.
\vs p037 8:6 \bibemph{Вселенские Примирители ---} передвижные суды вселенных со временем и пространством, действующие повсюду --- в эволюционных мирах, в каждой секции локальной вселенной и далее за ее пределами. Эти судьи регистрируются на Уверсе; не записано точное число действующих в Небадоне, но, по моей оценке, в нашей локальной вселенной около ста миллионов примирительных комиссий.
\vs p037 8:7 У нас, в соответствии с нашей квотой, около полумиллиарда \bibemph{Технических Советчиков,} юридических разумов вселенной. Эти существа представляют собой живые и доступные для пользования, основанные на опыте правовые библиотеки всего пространства.
\vs p037 8:8 У нас в Небадоне семьдесят пять \bibemph{Небесных Протоколистов ---} восходящих серафимов. Это старшие, или руководящие протоколисты. Число продвигающихся вперед учащихся этого чина, которые проходят подготовку, равно почти четырем миллиардам.
\vs p037 8:9 Служение семидесяти миллиардов \bibemph{Моронтийных Компаньонов} в Небадоне описано в тех повествованиях, где рассматриваются переходные планеты пилигримов времени.
\vs p037 8:10 \pc Каждая вселенная имеет свой собственный отряд ангелов, являющихся ее исконными жителями; тем не менее бывают случаи, когда очень полезно получать помощь от тех высших духов, которые берут свое начало за пределами локальной вселенной. Некоторые редкие и уникальные службы исполняются супернафимами; сегодняшний глава серафимов Урантии является первичным Райским супернафимом. Повсюду, где действует персонал сверхвселенной, можно встретить отражательных секонафимов, а огромное множество терциафимов временно несут службу в качестве Всевышних Помощников.
\usection{9. Постоянные граждане локальной вселенной}
\vs p037 9:1 Как и в случае со сверхвселенными и центральной вселенной, в локальной вселенной есть свои чины постоянного гражданства. К их числу относятся следующие типы сотворенных существ:
\vs p037 9:2 \ublistelem{1.}\bibnobreakspace Сусации.
\vs p037 9:3 \ublistelem{2.}\bibnobreakspace Унивитации.
\vs p037 9:4 \ublistelem{3.}\bibnobreakspace Материальные Сыны.
\vs p037 9:5 \ublistelem{4.}\bibnobreakspace Срединные создания.
\vs p037 9:6 \pc Эти уроженцы локальной вселенной вместе со слившимися с Духом восходящими смертными и со спиронгами (которые относятся к другой категории существ) составляют относительно постоянную группу граждан. Эти чины существ, вообще говоря, не являются ни восходящими, ни нисходящими. Все они --- создания, развивающиеся с ростом опыта, но их растущий опыт продолжает быть доступным для обитателей вселенной их уровня происхождения. Хотя в отношении адамических Сынов и срединных созданий это не вполне отвечает истине, применительно к этим чинам это относительно справедливо.
\vs p037 9:7 \pc \bibemph{Сусации.} Эти изумительные существа пребывают и функционируют в качестве постоянных граждан в Спасограде, центре этой локальной вселенной. Это блестящие потомки Сына\hyp{}Творца и Творческого Духа, и они тесно связаны с восходящими гражданами локальной вселенной --- слившимися с Духом смертными из Небадонского Отряда Совершенства.
\vs p037 9:8 \pc \bibemph{Унивитации.} Каждое из ста скоплений архитектурных сфер, представляющих центр созвездий, непрерывно пользуется служением пребывающего там чина существ, известных как унивитации. Эти дети Сына\hyp{}Творца и Творческого Духа составляют постоянное население миров\hyp{}центров созвездий. Это невоспроизводящиеся существа, занимающие положение примерно посредине между полуматериальным статусом Материальных Сынов, живущих в центрах систем, и уже определенно духовным уровнем слившихся с Духом смертных и сусаций Спасограда; но унивитации не являются моронтийными существами. Для восходящих смертных во время пересечения ими сфер созвездий они выполняют то, что уроженцы Хавоны делают для духов\hyp{}пилигримов, проходящих через центральную вселенную.
\vs p037 9:9 \pc \bibemph{Материальные Сыны Бога.} Когда творческая связь между Сыном\hyp{}Творцом и вселенским представителем Бесконечного Духа --- Духом\hyp{}Матерью Вселенной завершила свой цикл, когда больше не ожидается совместного потомства, тогда Сын\hyp{}Творец персонализирует в двойственной форме свою последнюю концепцию существа, таким образом окончательно подтверждая свое собственное и изначальное двойственное происхождение. Самостоятельно он создает тогда прекрасных и великолепных Сынов и Дочерей материального чина вселенского сыновства. Таково происхождение изначальных Адама и Евы каждой локальной системы Небадона. Они представляют собой воспроизводящийся чин сыновства, будучи сотворены имеющими мужской и женский пол. Их потомство --- это относительно постоянные граждане столицы системы, хотя некоторых и назначают Планетарными Адамами.
\vs p037 9:10 В ходе планетарной миссии материальным Сыну и Дочери поручается положить начало адамической расе данного мира, которой суждено в конечном счете слиться со смертными обитателями этой планеты. Планетарные Адамы являются одновременно и нисходящими, и восходящими Сынами, но обычно мы причисляем их к восходящим.
\vs p037 9:11 \pc \bibemph{Срединные создания.} В ранние периоды существования большинства обитаемых миров туда назначаются некие сверхчеловеческие, но материализованные существа, но обычно они удаляются по прибытии Планетарных Адамов. Действия таких существ и усилия Материальных Сынов, направленные на усовершенствование эволюционных рас, часто приводят к появлению ограниченного числа созданий, которых трудно отнести к какой\hyp{}либо категории. Эти уникальные существа нередко занимают положение посредине между Материальными Сынами и эволюционными созданиями; отсюда их название --- срединные создания. Эти срединники являются относительно постоянными гражданами эволюционных миров. Это единственная группа разумных существ, которые непрерывно присутствуют на планете с первых же дней прибытия Планетарного Принца и вплоть до очень отдаленного периода установления планеты в свете и жизни. На Урантии срединные служители, в действительности, и являются подлинными хранителями планеты; практически, они и есть граждане Урантии. Смертные на самом деле являются физическими и материальными обитателями эволюционного мира, но вы все так недолговечны; на планете, где родились, вы живете такое короткое время. Вы рождаетесь, живете, умираете и переходите в другие миры эволюционного продвижения. Даже сверхчеловеческие существа, которые служат на планете небесными служителями, назначены временно; немногие из них бывают надолго прикреплены к какой\hyp{}либо данной планете. Срединники же обеспечивают непрерывность планетарного управления на фоне вечно меняющихся небесных служений и постоянно перемещающихся смертных обитателей. На протяжении всех этих непрекращающихся изменений и перемещений срединные создания остаются на планете, непрерывно осуществляя свою деятельность.
\vs p037 9:12 \pc Подобным же образом все административные подразделения локальных вселенных и сверхвселенных имеют свое более или менее постоянное население, обитателей со статусом граждан. Как на Урантии есть срединники, так и в Иерусеме, столице вашей системы, есть Материальные Сыны и Дочери; в Эдентии, центре вашего созвездия --- унивитации, а в Спасограде --- граждане двоякой природы: сотворенные сусации и эволюционировавшие слившиеся с Духом смертные. В мирах, являющихся административными центрами малых и больших секторов сверхвселенных, нет постоянных граждан. Но центральные сферы Уверсы непрерывно ощущают заботу изумительной группы существ --- \bibemph{абандонтеров ---} творений нераскрытых действующих сил Древних Дней и семи Отражательных Духов, обитающих в столице Орвонтона. Эти постоянные граждане Уверсы в настоящее время занимаются управлением повседневными делами своего мира под непосредственным руководством Уверсского отряда слившихся с Сыном смертных. Даже Хавона имеет своих исконных существ, а центральный Остров Света и Жизни является родным домом различных групп Граждан Рая.
\usection{10. Другие группы существ локальной вселенной}
\vs p037 10:1 Помимо чинов серафимов и смертных, которые будут рассмотрены в последующих текстах, существует много других существ, имеющих отношение к обслуживанию и совершенствованию такой гигантской структуры, как вселенная Небадона, в которой уже сейчас имеется более трех миллионов обитаемых миров, а в перспективе --- десять миллионов. Различные типы жизни в Небадоне слишком многочисленны, чтобы всех их перечислять в этом тексте, но можно упомянуть два необычных чина, повсеместно действующих в 647\,591 архитектурной сфере локальной вселенной.
\vs p037 10:2 \pc \bibemph{Спиронги ---} духовные потомки Яркой и Утренней Звезды и Отца\hyp{}Мелхиседека. Их личностное существование бесконечно, но они не являются эволюционными или восходящими существами. Не связаны они функционально и с эволюционной системой восхождения. Это духовные помощники локальной вселенной, выполняющие повседневные духовные задачи в Небадоне.
\vs p037 10:3 \pc \bibemph{Спорнагии.} Центральные архитектурные миры локальной вселенной --- это реальные миры, физические творения. Существует много работы, связанной с их физическим содержанием, и здесь нам помогает группа физических созданий, называемых спорнагиями. Их предназначение --- заботиться о материальных аспектах этих центральных миров, от Иерусема до Спасограда, и об их совершенствовании. Спорнагии --- ни духи, ни личности; это животный чин существ, но если бы вы могли их увидеть, то согласились бы, что это, по всей видимости, совершенные животные.
\vs p037 10:4 \pc Различные \bibemph{гостящие колонии} располагаются в Спасограде и в других местах. Особую пользу приносит нам служение небесных ремесленников в созвездиях и помощь руководителей восстановления, которые действуют, главным образом, в столицах локальных систем.
\vs p037 10:5 Во вселенной всегда имеется назначенный на службу отряд восходящих смертных, в который входят прославленные срединные создания. После того, как эти восходящие достигли Спасограда, их используют в бесконечно разнообразных видах деятельности, связанных с ведением дел во вселенной. С каждого достигнутого уровня эти продвигающиеся вперед смертные стремятся назад и вниз, чтобы протянуть руку помощи своим собратьям, следующим за ними в движении вверх. При поступлении заявок такие смертные, временно пребывающие в Спасограде, назначаются практически во все отряды небесных личностей в качестве помощников, учащихся, наблюдателей и учителей.
\vs p037 10:6 Есть еще и другие типы разумных живых существ, связанных с управлением локальной вселенной, но план данного повествования не предусматривает дальнейшее откровение этих чинов созданий. Здесь уже было достаточно рассказано о жизни и управлении этой вселенной для того, чтобы дать человеческому разуму возможность постичь реальность и великолепие продолжения существования в посмертии. В процессе продвижения вперед приобретенный опыт будет все больше и больше раскрывать вам эти интересные и прелестные существа. Это повествование не более, чем краткий очерк природы и деятельности многообразных личностей, которые заполняют вселенные пространства, осуществляя управление этими творениями как огромными учебными заведениями, школами, в которых пилигримы времени продвигаются вперед от жизни к жизни и из мира в мир до тех пор, пока не будут с любовью отправлены за пределы своей родной вселенной в высшую образовательную систему сверхвселенной, а оттуда --- далее --- в миры духовного обучения Хавоны и в конце концов в Рай и к высокому предназначению финалитов --- вечному назначению исполнять миссии, которые еще не раскрыты вселенным со временем и пространством.
\vsetoff
\vs p037 10:7 [Продиктовано Блестящей Вечерней Звездой Небадона, Номер 1\,146 из Сотворенного Отряда.]
