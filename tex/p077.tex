\upaper{77}{Срединные создания}
\author{Архангел}
\vs p077 0:1 В большинстве обитаемых миров Небадона обретается одна или несколько групп уникальных существ, уровень жизнедеятельности которых находится как бы между смертными мира сего и ангельскими чинами, поэтому они называются \bibemph{срединными} созданиями. В свое время их появление казалось случайным, но они распространились так широко и оказались столь неоценимыми помощниками, что мы давно уже считаем их одним из необходимых чинов нашего общего планетарного служения.
\vs p077 0:2 На Урантии действуют два различных чина: срединники первого рода, или старший отряд, который появился давно, во времена Даламатии, и срединники второго рода, или младший отряд, который восходит ко дням Адама.
\usection{1. Срединники первого рода}
\vs p077 1:1 На Урантии срединники первого рода --- результат уникального соединения материального и духовного. Нам известно о существовании подобных существ и в других мирах и в других системах, но они возникли там иным образом.
\vs p077 1:2 Следует всегда помнить, что последовательные пришествия Сынов Бога на эволюционирующую планету приводят к заметным переменам в духовной сфере мира, и временами они настолько изменяют взаимодействия материальных и духовных сил на планете, что создаются ситуации, осмыслить которые по\hyp{}настоящему нелегко. Статус ста членов телесного штата Калигастии --- наглядный пример такого необычайного соединения: будучи идущими путем восхождения моронтийными гражданами Иерусема, они являлись сверхматериальными созданиями, не обладающими правом воспроизводства. Будучи нисходящими на планету служителями, на Урантии они представляли собой материальные создания, разделенные по половому признаку и способные производить материальное потомство (что некоторые из них и делали позднее). Но мы не можем удовлетворительно объяснить, каким образом эти сто созданий могли исполнять роль родителей на сверхматериальном уровне --- а именно это и произошло. В результате сверхматериальной (неполовой) связи мужчины и женщины --- членов телесного штата --- появился первый новорожденный срединник первого рода.
\vs p077 1:3 Немедленно выяснилось, что существо этого чина, являющегося промежуточным между смертным и ангельским уровнями, может быть весьма полезным для выполнения поручений центра управления Принца, и каждая пара телесного штата получила соответствующее разрешение произвести на свет подобное существо. В результате были созданы первые пятьдесят срединников.
\vs p077 1:4 После года наблюдения за работой этой необычной группы Планетарный Принц разрешил воспроизводство срединников без ограничений. Этот план осуществлялся до тех пор, пока существовала способность к продолжению рода, и, таким образом, возник первый отряд численностью 50 тысяч срединных созданий.
\vs p077 1:5 Между появлениями на свет каждого из срединников проходило полгода, и у каждой пары, родившей тысячу таких существ, воспроизводство прекращалось навсегда. И не существует никакого приемлемого объяснения, почему эта способность утрачивалась после появления тысячного потомка. Но все дальнейшие попытки неизменно заканчивались неудачей.
\vs p077 1:6 \pc Эти создания образовали разведывательный отряд администрации Принца. Они были вездесущими, наблюдая и изучая человеческие расы и оказывая другие неоценимые услуги Принцу и его штату в работе по воздействию на человеческое общество в местах, удаленных от центра управления планеты.
\vs p077 1:7 Такое положение существовало до трагических дней планетарного бунта, в который были втянуты больше четырех пятых срединников первого рода. Оставшиеся верными поступили на службу к Мелхиседекам\hyp{}исполнителям, действуя под официальным командованием Вана вплоть до пришествия Адама.
\usection{2. Раса нодитов}
\vs p077 2:1 Это был рассказ о происхождении, природе и действиях срединных созданий Урантии. Но сходство между двумя чинами срединников --- первого и второго рода --- вынуждает в этом месте прервать повествование о срединниках первого рода и проследить линию потомков мятежных членов телесного штата Принца Калигастии от дней планетарного бунта до времени Адама. Это именно та наследственная линия, которая в первые дни второго сада произвела на свет половину предков срединных созданий второго рода.
\vs p077 2:2 \pc Физические члены штата Принца были разделены по половому признаку для того, чтобы стало возможным их участие в плане производства потомства, обладающего характерными чертами их особого чина, соединенными с качествами рода, выбранного из племен Андона. Все это было осуществлено в предвиденье последующего появления Адама. Носители Жизни замыслили новый тип смертного, представляющего собой плод брачного союза соединенных потомков штата Принца с первым поколением детей Адама и Евы. Таким образом, был задуман план создания нового чина планетарных существ, которые, как они надеялись, могли бы стать правителями\hyp{}учителями человеческого общества. Такие создания предназначались осуществлять общественное, но не гражданское управление. Но поскольку этот план потерпел почти полную неудачу, мы никогда уже не узнаем, каких добрых, благородных правителей, и какой несравненной культуры Урантия в результате лишилась. Ибо позднее, когда после бунта телесный штат стал воспроизводится, они уже утратили свою связь с жизненными токами системы.
\vs p077 2:3 Эпоха, следовавшая после бунта на Урантии, была свидетелем многих необычных событий. Великая цивилизация --- культура Даламатии --- рассыпалась в прах. «В те дни на земле были Сыны Неффалима (нодиты), и когда эти сыны богов вошли к дочерям человеческим и те понесли от них, их дети были „мужами прежних времен“, „мужами знатными“». Хотя едва ли «сыны богов» --- штат Принца и его первые потомки --- так воспринимались эволюционирующими смертными в те далекие дни, даже их рост оказался преувеличенным легендой. Отсюда и пошел миф, существующий почти у всех народов, о том, как боги сошли на землю и там с дочерьми человеческими породили древнее племя героев. А затем, содержащиеся в этой легенде события стали впоследствии путать с более поздним смешением рас с адамитами, во втором саду.
\vs p077 2:4 Так как сто телесных членов штата Принца несли в себе зародышевую плазму андонитских человеческих линий, естественно было ожидать, что, если они займутся половым размножением, то их потомство будет всецело похоже на детей других андонитских родителей. Но когда шестьдесят мятежников из штата Принца, последователи Нода, действительно занялись половым размножением, оказалось, что их дети почти во всех отношениях превосходят и андонитов, и сангикские народы. Это неожиданное превосходство было характерно не только для физических и интеллектуальных способностей, но и для духовных.
\vs p077 2:5 Эти мутантные черты, проявившиеся в первом поколении нодитов, были следствием некоторых изменений, произошедших в конфигурации и химическом составе наследственных факторов зародышевой плазмы андонитов. Эти изменения были вызваны мощными контурами жизнеобеспечения системы Сатании в телах членов штата. Эти контуры жизни привели к тому, что хромосомы особого урантийского паттерна реорганизовались в большей степени в соответствии с паттернами стандартизированной сатанийской специализации проявления жизни, предписанного для Небадона. Техника такой метаморфозы зародышевой плазмы под воздействием жизненных токов системы схожа с теми процедурами, которые ученые Урантии используют для изменения зародышевой плазмы растений и животных с помощью рентгеновских лучей.
\vs p077 2:6 Таким образом, народы нодитов возникли в результате этих необычных и неожиданных модификаций, происходящих в жизненной плазме, которая была перенесена хирургами Авалона из тел андонитов\hyp{}доноров в телесных членов штата.
\vs p077 2:7 \pc Напомним, что сто доноров жизненной плазмы андонитов стали, в свою очередь, обладателями органических добавок из дерева жизни, так что жизненные токи Сатании вошли в их тела. Сорок четыре модифицированных андонита, которые вслед за штатом приняли участие в бунте, тоже вступили между собой в супружеские отношения, что также сказалось на улучшении наследственных черт народа нодитов.
\vs p077 2:8 Эти две группы, включающие 104 существа, которые несли в себе модифицированную зародышевую плазму андонитов, и стали предками нодитов, восьмой расы, появившейся на Урантии. И эта новая форма человеческой жизни на Урантии представляет самостоятельную линию дальнейшей разработки первоначального плана использования этой планеты в качестве сферы модификации жизни --- хотя и давшую непредвиденные результаты развития.
\vs p077 2:9 \pc Чистокровные нодиты были великолепной расой, но постепенно смешиваясь с эволюционирующими народами земли, они вскоре растеряли свои лучшие признаки. Спустя десять тысяч лет после бунта они опустились до такой степени, что средняя продолжительность их жизни была уже ненамного больше, чем у эволюционирующих народов.
\vs p077 2:10 Когда археологи раскапывают глиняные таблицы с записями шумерских потомков нодитов, они обнаруживают списки шумерских царей, правивших несколько тысяч лет тому назад. И чем к более далекому прошлому относятся записи, тем больше увеличивается время правления отдельных царей от приблизительно двадцати пяти или тридцати лет до ста пятидесяти лет и более. Это увеличение времени правления более древних царей означает, что некоторые из первых правителей нодитов (непосредственных потомков штата Принца) на самом деле жили дольше, чем их более поздние потомки; это указывает также на попытку показать, что династии не прерывались вплоть до времен Даламатии.
\vs p077 2:11 Записи о таких долгожителях обуславливаются еще и тем, что в обозначении периодов времени месяцы путали с годами. То же самое можно видеть в библейской родословной Авраама и в древних летописях китайцев. Ошибочное употребление месяца или сезона, содержащего двадцать восемь дней, вместо введенного позже года, содержащего более трехсот пятидесяти дней, объясняет легенды о столь большой продолжительности человеческой жизни. Существуют записи о человеке, который прожил свыше девятисот «лет». На самом деле этот период равнялся неполным семидесяти годам, и такую жизнь веками считали очень долгой; впоследствии о таком возрасте говорили: «шестьдесят лет и десять».
\vs p077 2:12 После дней Адама еще долго исчисляли время месяцами, содержащими двадцать восемь дней. Но когда около семи тысяч лет тому назад египтяне предприняли реформу календаря, они ввели в употребление год, состоящий из 365 дней, и сделали это с большой точностью.
\usection{3. Вавилонская башня}
\vs p077 3:1 После того, как Даламатия ушла под воду, нодиты двинулись на северо\hyp{}восток и вскоре основали новый город Дилмун, ставший их расовым и культурным центром. Спустя приблизительно пятьдесят тысяч лет после смерти Нода, уже после того, как потомство штата Принца стало чересчур многочисленным, чтобы находить себе пропитание в землях, непосредственно окружающих их новый город Дилмун, и после того как они установили связи и породнились с соседними андонитскими и сангикскими племенами, их вожди пришли к выводу, что необходимо что\hyp{}то предпринять, чтобы сохранить мир между расами. Соответственно, был созван совет племен, где после длительного обсуждения одобрили план Вавилота, потомка Нода.
\vs p077 3:2 Вавилот предложил воздвигнуть в центре территории, занимаемой ими в то время, величественный храм прославления рас. Этот храм должен был иметь башню, подобной которой мир еще никогда не видел. Ей надлежало стать монументальным памятником их уходящему величию. Многие хотели, чтобы этот монумент был воздвигнут в Дилмуне, но были и другие, которые, помня предание о затоплении Даламатии, их первой столицы, считали, что такое великое сооружение должно быть расположено на безопасном расстоянии от моря.
\vs p077 3:3 Вавилот планировал, что новые здания станут ядром будущего центра культуры и цивилизации нодитов. Его мнение, в конце концов, одержало верх, и строительство было начато в соответствии с его планами. Новый город должен был быть назван \bibemph{Вавилотом} по имени архитектора и строителя башни. Позднее это место стало известно как Вавилод и, наконец, как Вавилон.
\vs p077 3:4 Но у нодитов все еще не было единого мнения о планах и целях этого предприятия. И их вождей не все удовлетворяло и в планах строительства зданий, и в планах их использования после окончания строительства. После четырех с половиной лет работы произошло бурное обсуждение цели и мотива сооружения башни. Спор был столь ожесточенным, что все работы были остановлены. Разносчики пищи распространили вести о раздорах, и многочисленные племена стали собираться на строительном участке. Относительно цели строительства башни были высказаны три различные точки зрения:
\vs p077 3:5 \ublistelem{1.}\bibnobreakspace Большинство, почти половина, хотели видеть в башне памятник истории и расового превосходства нодитов. Они полагали, что это должно быть огромное и внушительное сооружение, которое вызовет восхищение всех будущих поколений.
\vs p077 3:6 \ublistelem{2.}\bibnobreakspace Другая, немного меньшая, группа хотела, чтобы башня служила напоминанием о культуре Дилмуна. Они предвидели, что Вавилот станет крупнейшим центром торговли, искусства и производства.
\vs p077 3:7 \ublistelem{3.}\bibnobreakspace Наименьшая и наименее значительная часть собравшихся придерживалась взгляда, что сооружение башни дает возможность попытаться искупить вину за преступление своих прародителей, принявших участие в бунте Калигастии. Они считали, что башня должна быть посвящена почитанию Отца Всего Сущего, что само предназначение нового города --- занять место Даламатии и стать культурным и религиозным центром для окружающих варваров.
\vs p077 3:8 \pc Предложение религиозной группы нодитов было немедленно провалено. Большинство отвергло мнение, что их предки повинны в бунте; они были глубоко возмущены тем, что их раса отмечена таким клеймом позора. Отбросив одну из трех точек зрения в этом споре и не сумев путем переговоров прийти к согласию относительно двух других, они передрались. Члены религиозной группировки, сторонники непротивления, бежали к себе домой на юг, а их соотечественники воевали друг с другом до тех пор, пока почти все не погибли.
\vs p077 3:9 \pc Около двенадцати тысяч лет назад была предпринята вторая попытка построить Вавилонскую башню. Смешанные расы андитов (нодитов и адамитов) стали сооружать новый храм на развалинах первой постройки, но эта махина не обладала достаточной прочностью и рухнула под тяжестью собственного гигантского веса. В течение долгого времени эта местность была известна как страна Вавилонская.
\usection{4. Центры цивилизации нодитов}
\vs p077 4:1 Прямым результатом междоусобного конфликта, вспыхнувшего из\hyp{}за Вавилонской башни, было рассеяние нодитов. В этой междоусобной войне погибло значительное число чистокровных нодитов и во многих отношениях именно война была причиной их неспособности построить великую доадамическую цивилизацию. Начиная с этого времени культура нодитов, в продолжение ста двадцати тысяч лет приходила в упадок, до тех пор, пока не возродилась благодаря адамическому влиянию. Но и во времена Адама нодиты еще оставались развитым народом. Многие из их смешанных потомков входили в число строителей Сада, а некоторые нодиты стали командирами отрядов Вана. Были представители этой расы и среди светлых умов в служебном персонале Адама.
\vs p077 4:2 Три из четырех великих центров нодитов были основаны сразу после Вавилотского конфликта:
\vs p077 4:3 \ublistelem{1.}\bibnobreakspace \bibemph{Западные, или сирийские нодиты.} Немногие уцелевшие из тех, кто ратовал за постройку памятника национальным или расовым традициям, ушли на север, чтобы, объединившись с андонитами, основать более поздние центры нодитов на северо\hyp{}западе Месопотамии. Эта отделившаяся группа нодитов была наиболее многочисленной, и она внесла большой вклад в возникшее позже ассирийское племя.
\vs p077 4:4 \ublistelem{2.}\bibnobreakspace \bibemph{Восточные, или эламские нодиты.} Большое число тех, кто выступал за приоритет культуры и торговли, мигрировали в восточном направлении, к Эламу. Там они слились со смешанными сангикскими племенами. Эламиты, жившие тридцать\hyp{}сорок тысяч лет назад, в подавляющем большинстве стали по своей природе такими же, как и сангикские народы, хотя все\hyp{}таки продолжали сохранять цивилизацию, более развитую, чем у окружающих их варваров.
\vs p077 4:5 После основания второго сада постепенно привыкли говорить об этих соседних поселениях нодитов как о «земле Нод». В течение долгого периода относительного мира между этой группой нодитов и адамитами, эти расы в значительной степени ассимилировались, так как у Сынов Бога (адамитов) все больше и больше входило в обычай жениться на дочерях человеческих (нодитах).
\vs p077 4:6 \ublistelem{3.}\bibnobreakspace \bibemph{Центральные, или дошумерские, нодиты.} В устье рек Тигр и Евфрат небольшая группа сохраняла свою расовую чистоту. Народ этот существовал в течение тысяч лет и, в конце концов, стал предками тех нодитов, смешавшихся с адамитами, от которых значительно позже произошли шумерские народы.
\vs p077 4:7 Все это и объясняет, почему на исторической сцене в Месопотамии так внезапно и таинственно возникли шумеры. Исследователи никогда не смогут проследить генеалогию этих племен назад ко времени возникновения шумеров, которые появились двести тысяч лет назад почти сразу после затопления Даламатии. Не оставив нигде в мире и намека на свое происхождение, эти древние племена вдруг обозначились на горизонте цивилизации. Они уже обладали превосходной и зрелой культурой, строили храмы, знали искусство обработки металлов, земледелие, животноводство, гончарное дело, ткачество, имели торговое законодательство, гражданские кодексы, религиозные обряды и древнюю письменность. Задолго до начала исторической эры они утратили алфавит Даламатии, усвоив необычную систему письма, созданную в Дилмуне. Шумерский язык, хотя фактически и потерян для мира, не принадлежал к семитским; он имел много общего с так называемыми арийскими языками.
\vs p077 4:8 Подробные записи, оставленные шумерами, описывают место замечательного поселения, расположенного в Персидском заливе неподалеку от древнего города Дилмуна. Египтяне называли этот город древней славы Дилматом, а жившие позже адамизированные шумеры путали первый и второй город нодитов с Даламатией и все их называли одним именем: Дилмун. И археологи уже нашли древнешумерские глиняные таблички, в которых говорится об этом земном рае, «где Боги впервые благословили человечество, дав ему пример цивилизованной и культурной жизни». И эти таблички, описывающие Дилмун --- человеческий и Господний рай, сегодня молчаливо покоятся на пыльных полках многих музеев.
\vs p077 4:9 Шумерам было хорошо известно о первом и втором Эдеме, но, несмотря на многочисленные браки с адамитами, они продолжали считать жителей северного сада чуждой им расой. Шумеры гордились более древней культурой нодитов, а это привело их к тому, что они предпочли великолепие и райские традиции города Дилмуна дальнейшей перспективе славы.
\vs p077 4:10 \ublistelem{4.}\bibnobreakspace \bibemph{Северные нодиты и амадониты --- ваниты.} Эта народность возникла до Вавилотского конфликта. Именно северные нодиты были потомками тех, кто отказался признать главенство Нода в пользу Вана и Амадона.
\vs p077 4:11 \pc Некоторые из первых сподвижников Вана поселились впоследствии на берегах озера, которое все еще носит его имя, и в этой местности сформировались их традиции. Арарат стал их священной горой, имевшей для позднейших ванитов почти такое же значение, как Синай для иудеев. Десять тысяч лет назад ваниты --- предки ассирийцев, учили, что семь их моральных заповедей были даны Вану Богом на горе Арарат. Они твердо верили, что, когда Ван и его товарищ Амадон высоко на горе были погружены в молитву, они были унесены с планеты.
\vs p077 4:12 Гора Арарат была священной горой северной Месопотамии, и, так как многие из ваших обычаев тех давних времен были связаны с вавилонской историей о потопе, не удивительно, что и гора Арарат, и ее окрестности были позднее вплетены в иудейскую историю о Ное и всемирном потопе.
\vs p077 4:13 Около 35\,000 года до н.э. Адам\hyp{}сын посетил одно из самых восточных поселений древних ванитов, чтобы основать свой центр цивилизации.
\usection{5. Адам\hyp{}сын и Ратта}
\vs p077 5:1 Обрисовав прошлое нодитов, предков срединников второго рода, необходимо поведать теперь о других их предках --- адамитах, ибо срединники второго рода были потомками Адама\hyp{}сына, первого новорожденного фиолетовой расы на Урантии.
\vs p077 5:2 \pc Адам\hyp{}сын был среди тех детей Адама и Евы, которые предпочли остаться на земле со своими отцом и матерью. Этот старший сын Адама часто слышал от Вана и Амадона рассказ об их доме в горах на севере, и вскоре после основания второго сада решил отправиться на поиски этой страны его юношеской мечты.
\vs p077 5:3 В то время Адаму\hyp{}сыну было 120 лет и еще во времена первого сада у него было тридцать два чистокровных ребенка. Сам он хотел остаться со своими родителями и помогать им в строительстве второго сада, и был страшно расстроен потерей своей семьи: супруги и детей, ибо они решили идти в Эдентию вместе с другими адамическими детьми, чтобы стать под опеку Всевышних.
\vs p077 5:4 Адам\hyp{}сын не оставил бы своих родителей на Урантии, он не собирался избегать ни трудностей, ни опасностей, но его не удовлетворяло сообщество второго сада. Он делал много полезного для обороны и строительства, но все\hyp{}таки решил при первой же возможности отправиться на север. Хотя его уход был благопристойно обставлен, Адам и Ева очень горевали, что теряют своего старшего сына, что он уходит в чужой и враждебный мир, откуда, они боялись, он уже никогда не вернется.
\vs p077 5:5 Двадцать семь спутников ушли с Адамом\hyp{}сыном на север, чтобы найти этих людей его детских грез. И немного более чем через три года отряд Адама\hyp{}сына действительно обнаружил цель своего рискованного предприятия. Среди этих людей он нашел удивительно прекрасную женщину двадцати одного года, которая утверждала, что является последним чистокровным потомком штата Принца. Эта женщина, Ратта, говорила, что все ее предки были потомками двух человек из падшего штата Принца. Она была последней в своем роду, у нее не было живых братьев или сестер. Она уже почти решила не выходить замуж, уже собиралась так и умереть бездетной, но полюбила величественного Адама\hyp{}сына. И когда она узнала историю Эдема, о том, как в действительности сбылись предсказания Вана и Амадона, когда она слушала рассказ о разрушении Сада, ее занимала одна\hyp{}единственная мысль --- выйти замуж за этого сына и наследника Адама. А скоро такая же мысль овладела Адамом\hyp{}сыном. Не прошло и трех месяцев, как они поженились.
\vs p077 5:6 \pc У Адама\hyp{}сына и Ратты родились шестьдесят семь детей. Они положили начало замечательному роду правителей мира, но сделали и нечто большее. Надо помнить, что они оба были, по существу, сверхлюдьми. Каждый четвертый родившийся у них ребенок был необычным. Часто он бывал невидим. Никогда еще в мировой истории не случалось ничего подобного. Ратта была очень обеспокоена --- даже стала суеверной, --- но Адам хорошо знал о существовании срединников первого рода и заключил, что что\hyp{}то похожее происходит и у него на глазах. Когда родился второй ребенок со странностями поведения, Адам решил их поженить, поскольку один был мужского, а другой --- женского пола, и этот брак дал начало чину срединников второго рода. В течение ста лет, прежде чем этот феномен утратился, их появилось на свет около двух тысяч.
\vs p077 5:7 \pc Адам\hyp{}сын прожил 396 лет. Много раз он возвращался навестить своего отца и мать. Каждые семь лет он и Ратта путешествовали на юг, во второй сад, и в это время срединники оповещали о благосостоянии его народа. В течение жизни Адама\hyp{}сына они сослужили большую службу в деле строительства нового и независимого мирового центра правды и благочестия.
\vs p077 5:8 Таким образом, под началом Адама\hyp{}сына и Ратты был отряд великолепных помощников, которые в течение всей их долгой жизни трудились, помогая в продвижении высшей истины и распространении высших стандартов духовного, интеллектуального и физического бытия. И последующие эпохи упадка цивилизации так никогда и не смогли полностью сокрыть результаты этих попыток улучшения мира.
\vs p077 5:9 \pc После Адама\hyp{}сына и Ратты их потомки поддерживали высокую культуру на протяжении почти семи тысяч лет. Позднее они смешались с соседними нодитами и андонитами и также были причислены к «могучим мужам прежних времен». И некоторые из достижений этой эпохи продолжали существовать, став латентной составляющей культурного потенциала, который позднее развился в Европейскую цивилизацию.
\vs p077 5:10 Этот центр цивилизации находился в районе, расположенном на востоке от южного побережья Каспийского моря, неподалеку от Копетдага. Невысоко в предгорьях Туркестана еще существуют остатки того, что было когда\hyp{}то центром цивилизации потомков Адама\hyp{}сына, центром фиолетовой расы. На этих нагорьях, образующих узкий древний плодородный пояс, лежащий у подножья горной цепи Копета, последовательно существовали в различные периоды четыре самостоятельные культуры, созданные, соответственно, четырьмя различными группами потомков Адама\hyp{}сына. Вторая из этих групп была той самой, что переселилась на запад --- в Грецию и на острова Средиземноморья. Остатки потомков Адама\hyp{}сына переселились на север и на запад и вошли в Европу вместе со смешанным племенем последней волны андитов, мигрирующим из Месопотамии, кроме того, они входили в число андито\hyp{}арийских завоевателей Индии.
\usection{6. Срединники второго рода}
\vs p077 6:1 В то время, как срединники первого рода имели почти сверхчеловеческое происхождение, чин второго рода являлся плодом скрещивания чистокровных адамитов с очеловеченными потомками тех, кто был предком старшего отряда срединников.
\vs p077 6:2 Среди детей Адама\hyp{}сына было всего шестнадцать избранных стать прародителями срединников второго рода. Эти удивительные дети составили две равные группы соответственно полу, и каждая пара, сочетая половой и неполовой контакт, была способна рождать срединника второго рода каждые семьдесят дней. Подобное никогда не было возможно на земле ни до, ни после этого времени.
\vs p077 6:3 Эти шестнадцать детей жили и умерли, если не принимать во внимание некоторые их особенности, как смертные мира сего. Но их дети, питаемые электрической энергией, продолжали жить и жить, не подвластные ограничениям смертной плоти.
\vs p077 6:4 Каждая из восьми пар произвела 248 срединников и, таким образом, на свет появился вторичный отряд исходной численностью 1\,984 срединника. Существует восемь подгрупп срединников второго рода. Они обозначаются как А\hyp{}Б\hyp{}В первый, второй, третий и так далее. И затем, Г\hyp{}Д\hyp{}Е первый, второй и так далее.
\vs p077 6:5 \pc После срыва Адама срединники первого рода возвратились на службу к Мелхиседекам\hyp{}исполнителям, а срединники второго рода были приданы центру Адама\hyp{}сына до его смерти. Тридцать три из них, бывшие руководителями своего подразделения, к моменту смерти Адама\hyp{}сына хотели перевести весь чин на службу к Мелхиседекам, пытаясь связаться с отрядами первого рода. Но не сумев этого добиться, они покинули своих товарищей и все перешли на службу к планетарным исполнителям.
\vs p077 6:6 После смерти Адама\hyp{}сына оставшиеся на Урантии срединники второго рода стали странным образованием, неорганизованным и никому не подчиняющимся. С этого времени и до дней Мелхиседека Махивенты они вели беспорядочную и сумбурную жизнь. Они были частично взяты под контроль этим Мелхиседеком, но все еще производили много вреда вплоть до времени Христа\hyp{}Михаила. И в течение его пребывания на земле все они сделали окончательный выбор своей будущей судьбы, причем лояльное большинство поступило тогда на службу под начало срединников первого рода.
\usection{7. Срединники\hyp{}мятежники}
\vs p077 7:1 Во времена бунта Люцифера большинство срединников первого рода пошло дорогой греха. Когда после планетарного бунта был подведен итог разорению, оказалось, что (в числе иных потерь) из исходного числа 50\,000, 40\,119 срединников примкнули к мятежу Калигастии.
\vs p077 7:2 Исходное число срединников второго рода равнялось 1\,984, из них 873 не присоединились к правлению Михаила и по планетарному приговору были должным образом интернированы из Урантии в день Пятидесятницы. Никто не может предсказать будущее этих падших созданий.
\vs p077 7:3 Обе группы мятежных срединников содержатся теперь под стражей, ожидая окончательного приговора суда по делам бунта в системе. Но до наступления нынешней планетарной диспенсации они успели совершить на земле множество странных поступков.
\vs p077 7:4 Эти нелояльные срединники при определенных обстоятельствах могли стать видимыми смертным, особенно это было свойственно сподвижникам Вельзевула, вождя изменников среди срединников второго рода. Но этих необычных созданий не следует путать с некоторыми восставшими херувимами и серафимами, которые также существовали на земле до времени смерти и воскресения Христа. Некоторые древние авторы называют этих мятежных срединников злыми духами или демонами, а изменников\hyp{}серафимов --- ангелами зла.
\vs p077 7:5 Ни в одном мире злые духи не могут овладеть разумом смертного после пришествия Райского Сына. Но до дней Христа\hyp{}Михаила на Урантии --- до повсеместного прихода Настройщиков Мысли и излияния духа Учителя на всю живую плоть --- мятежные срединники действительно могли оказывать влияние на некоторые умы низших смертных и в какой\hyp{}то степени контролировать их поступки. Это происходило почти таким же образом, каким действуют лояльные срединные существа, выполняя функции эффективных контактных хранителей человеческих умов урантийского резервного отряда предназначения в те отрезки времени, когда Настройщик во время контакта со сверхчеловеческими разумными существами, в сущности, отсоединен от человека.
\vs p077 7:6 И это не просто риторический оборот, когда летопись пишет: «И они приводили к Нему всевозможных больных, тех, кто были одержимы дьяволами, и тех, кто были безумными». Иисус знал и понимал различие между помешательством и демонической одержимостью, хотя те, кто жил в его время, очень часто путали эти два состояния.
\vs p077 7:7 Но даже и до Пятидесятницы ни один мятежный дух не мог подчинить себе разум нормального человека, а с того дня даже слабый рассудок низших смертных был освобожден от таких влияний. Мнимое изгнание бесов после прибытия Духа Истины стало результатом того, что проявление демонической одержимости было спутано с симптомами таких болезней, как истерия, помешательство и слабоумие. Но просто из того, что пришествие Михаила навсегда освободило человеческий разум на Урантии от возможности попасть во власть демонов, совсем не следует, что такое событие не было реальностью в прошлые века.
\vs p077 7:8 По приказу Всевышних Эдентии в настоящее время вся группа мятежных срединников содержится в заключении. Они больше не бродят по этому миру, склоняя ко злу. Безотносительно к присутствию Настройщиков Мысли излияние Духа Истины на всю живую плоть навсегда исключило возможность для вероломных духов всех видов и званий вторгаться даже в самые слабые человеческие умы. После дня Пятидесятницы такое явление, как одержимость демонами, никогда не может случиться.
\usection{8. Объединенные срединники}
\vs p077 8:1 На последнем суде этого мира, когда Михаил переместил дремлющих в посмертии того времени, срединные создания были оставлены, чтобы помогать в духовной и полудуховной работе на планете. Теперь они действуют как единый отряд, включающий оба чина и насчитывающий 10\,992 срединника. \bibemph{Объединенными Срединниками Урантии} в настоящее время по очереди управляет старший член каждого чина. Такой порядок был введен после их слияния в одну группу вскоре после дня Пятидесятницы.
\vs p077 8:2 Члены более старого, или первого рода обычно имеют числа в качестве имен, им часто дают такие имена, как 1\hyp{}2\hyp{}3 первый, 4\hyp{}5\hyp{}6 первый и так далее. На Урантии адамические срединники обозначаются буквами, чтобы их можно было отличить от числовых обозначений срединников первого рода.
\vs p077 8:3 Оба чина являются нематериальными существами в том, что касается питания и получения энергии, но у них множество человеческих черт, они способны испытывать радость, понимать ваш юмор и ваши богослужения. Приставленные к смертным, они включаются в дух человеческой работы, отдыха и игры. Но срединники не спят и не обладают способностью к произведению потомства. В группе второго рода в определенном смысле различаются мужские и женские линии, о них часто говорят «он» или «она». Они нередко работают такими парами, вместе.
\vs p077 8:4 Срединники --- не люди, но они и не ангелы, однако срединники второго рода по своей природе ближе к человеку, чем к ангелам; они в какой\hyp{}то мере принадлежат к вашим расам и, следовательно, при общении с человеческими существами очень понятливы и отзывчивы, они неоценимы для серафимов в их работе для и с различными человеческими расами. Оба чина необходимы серафимам, которые являются личными хранителями смертных.
\vs p077 8:5 \pc Объединенные Срединники Урантии организованы для служения с планетарными серафимами в соответствии со своими врожденными талантами и приобретенным мастерством в следующие группы:
\vs p077 8:6 \ublistelem{1.}\bibnobreakspace \bibemph{Срединные вестники.} У членов этой группы есть имена; это небольшой отряд, и в эволюционирующем мире он оказывает большую помощь службе быстрой и надежной личной связи.
\vs p077 8:7 \ublistelem{2.}\bibnobreakspace \bibemph{Планетарные стражи.} Срединники --- это хранители, часовые космических миров. Они выполняют важные обязанности наблюдателей за всеми бесчисленными способами и видами связи, которые имеют значение для сверхъестественных существ мира сего. Они патрулируют невидимое царство духа на планете.
\vs p077 8:8 \ublistelem{3.}\bibnobreakspace \bibemph{Устанавливающие связь.} Срединные создания всегда используются для установления связи со смертными созданиями материального мира, например, для связи с субъектом, через которого передаются эти сообщения. Они --- необходимые посредники в установлении подобных связей между духовным и материальным уровнями.
\vs p077 8:9 \ublistelem{4.}\bibnobreakspace \bibemph{Помощники прогресса} Это наиболее духовные из срединных созданий, и они выступают в качестве помощников серафимов различных чинов, которые действуют в специальных группах на планете.
\vs p077 8:10 \pc Срединники сильно различаются по своим способностям устанавливать контакт с серафимами, которые по уровню выше их, и со своими человеческими родственниками, находящимися на более низком уровне. Для срединников первого рода, например, чрезвычайно трудно установить непосредственный контакт с материальными объектами. Они гораздо ближе к ангельскому типу существования, и обычно их назначают работать и помогать представителям духовных сил, пребывающим на планете. Они выполняют функции проводников и сопровождающих для небесных посетителей и временно проживающих студентов, в то время как создания второго рода почти исключительно предназначены помогать материальным существам мира сего.
\vs p077 8:11 1\,111 лояльных срединников второго рода принимают на земле участие в важных миссиях. По сравнению со своими партнерами первого рода они, несомненно, материальны. Они лишь не видимы зрению смертных и обладают необыкновенной способностью устанавливать по желанию физический контакт с тем, что люди называют «материальными вещами». Эти необычные создания определенно имеют некую власть в пространстве и времени над вещами и над животными мира сего.
\vs p077 8:12 Множество, скорее, материальных явлений, обычно приписываемых ангелам, было сделано срединными созданиями второго рода. Когда первые проповедники Евангелия Иисуса были брошены в темницу невежественными религиозными вождями тех времен, тот «ангел Господень», который «ночью отворил двери темницы и выпустил их», был и на самом деле ангелом. Освобождение же Петра после убийства Иродом Иакова совершил именно срединник, а это было приписано ангелу.
\vs p077 8:13 Сегодня их основная работа --- действовать как невидимые помощники, обеспечивающие персональную связь тех мужчин и женщин, которые образуют планетарный резервный отряд предназначения. Именно эта группа второго рода при умелой поддержке некоторых членов группы первого рода осуществила координацию личностей и событий на Урантии, что, в конце концов, побудило планетарных небесных руководителей положить начало тем петициям, которые закончились выдачей мандатов, сделавших возможным ряд откровений, частью которых является настоящий текст. Но следует пояснить, что срединные создания никогда не принимали участие в жалких спектаклях, имевших место под общим названием «спиритуализм». В настоящее время срединники на Урантии --- а все они занимают почетное положение --- не связаны с феноменами так называемых «медиумов»; обычно они не позволяют людям быть свидетелями их физической деятельности, временами необходимой, или же свидетелями других контактов с материальным миром, воспринимаемых с помощью человеческих органов чувств.
\usection{9. Постоянные граждане Урантии}
\vs p077 9:1 Срединники могут рассматриваться как первая группа постоянных обитателей, находящихся на различных мирах во всех вселенных, в противоположность эволюционирующим существам, идущим путем восхождения, таким, как смертные создания или ангелы. Такие постоянные граждане встречаются в различных точках восхождения к Раю.
\vs p077 9:2 В противоположность различным чинам небесных существ, которые назначены \bibemph{служить} на планетах, срединники \bibemph{живут} в обитаемых мирах. Серафимы приходят и уходят, но срединники остаются и будут оставаться, хотя они и являются помощниками уроженцев планеты, но в то же время обеспечивают режим преемственности, который гармонизирует и связывает сменяющиеся администрации серафимов.
\vs p077 9:3 Как истинные граждане Урантии, срединники проявляют искреннюю заинтересованность в судьбе земного шара. Они представляют собой серьезную ассоциацию, настойчиво работающую во имя прогресса родной планеты. Об их решительности говорит девиз их чина: «За что Объединенные Срединники взялись, то Объединенные Срединники сделают».
\vs p077 9:4 Хотя способность пересекать энергетические контуры предоставляет любому срединнику возможность покинуть планету, каждый из них дал обещание не покидать ее до тех пор, пока власти вселенной их не отпустят. Срединники остаются на планете до времени ее установления в свете и жизни. За исключением 1\hyp{}2\hyp{}3 первого, ни один лояльный срединник не покидал Урантию.
\vs p077 9:5 1\hyp{}2\hyp{}3 первый, старейшина первичного чина, был освобожден от непосредственных планетарных обязанностей вскоре после дня Пятидесятницы. Этот благородный срединник твердо оставался с Ваном и Амадоном в трагические дни планетарного бунта, и его бесстрашное руководство способствовало уменьшению потерь среди его чина. В настоящее время он служит в Иерусеме в качестве члена совета двадцати четырех, причем после дня Пятидесятницы он уже был однажды генерал\hyp{}губернатором Урантии.
\vs p077 9:6 \pc Срединники ограничены планетой, но они, так же как и смертные, которые разговаривают с путешественниками, приехавшими издалека, и, таким образом, узнают об отдаленных местах планеты, говорят с небесными путешественниками, чтобы знать о дальних уголках вселенной. Они уже хорошо осведомлены об этой системе и вселенной, даже об Орвонтоне и родственных ему творениях, --- таким образом они готовятся получить гражданство на более высоком уровне существования.
\vs p077 9:7 Хотя срединники приходят в мир совершенно развитыми, у них нет периода роста или взросления, они никогда не перестают набираться опыта и мудрости. Как и смертные, они являются эволюционирующими созданиями, а их культура --- истинно эволюционное приобретение. В урантийском отряде срединников многие обладают высоким умом и могучим духом.
\vs p077 9:8 В более широком смысле, цивилизация Урантии есть плод совместных усилий смертных и срединников Урантии, и это правда, несмотря на существующую разницу двух уровней культуры, разницу, которая не может быть устранена, пока не наступят времена света и жизни.
\vs p077 9:9 Срединное создание, являясь плодом бессмертных граждан планеты, имеет относительный иммунитет к тем временным превратностям судьбы, которые со всех сторон обступают человеческую цивилизацию. Поколения людей забывают, отряд срединников помнит, и эта память --- сокровищница традиций вашего обитаемого мира. Так культура планеты остается всегда пребывающей на планете, и при надлежащих обстоятельствах эти сокровища памяти прошлого становятся доступными точно так же, как история жизни и учение Иисуса были переданы срединниками Урантии своим сородичам во плоти.
\vs p077 9:10 Срединники --- искусные служители, устраняющие разрыв между материальными и духовными делами Урантии, который возник со смертью Адама и Евы. Более того, они ваши старшие собратья, товарищи в долгой борьбе за приобретение установленного статуса света и жизни на Урантии. Объединенные Срединники --- отряд, прошедший испытание бунтом, и они будут неизменно играть свою роль в планетарной эволюции до тех пор, пока этот мир не придет к вековой цели, пока не наступит тот далекий день, когда мир в действительности станет царить на земле и добрая воля воистину будет в сердцах людей.
\vs p077 9:11 По ценной работе, выполненной этими срединниками, мы заключаем, что они поистине необходимая часть духовного хозяйства этого мира. А там, где бунт не повредил делам планеты, они оказывают серафимам еще большую помощь.
\vs p077 9:12 \pc Вся организация высоких духов, ангельского воинства и срединных собратьев с энтузиазмом способствует продвижению Райского плана прогрессивного восхождения и достижения совершенства эволюционирующих смертных. Одно из возвышенных предприятий вселенной --- величественный план продолжения существования, по которому Бог нисходит до человека, а затем, благодаря возвышенному партнерству, человек возносится до Бога и далее к вечности служения и божественности достижения, --- одинаков для смертного и для срединника.
\vsetoff
\vs p077 9:13 [Представлено Архангелом Небадона.]
