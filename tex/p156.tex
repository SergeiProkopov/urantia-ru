\upaper{156}{Пребывание в Тире и Сидоне}
\author{Комиссия срединников}
\vs p156 0:1 В пятницу днем, 10 июня, Иисус и его сподвижники прибыли в окрестности Сидона, где Иисус остановился в доме у состоятельной женщины, бывшей пациентки вифсаидского лазарета в те времена, когда Учитель пользовался величайшей популярностью в народе. Евангелисты и апостолы поселились по соседству, у ее друзей, и в субботний день они отдыхали в этом приятном месте. Они провели почти две с половиной недели в Сидоне и его окрестностях прежде, чем собрались посетить города на севере побережья.
\vs p156 0:2 Был чрезвычайно тихий июньский субботний день и евангелисты, и апостолы --- все размышляли о беседе Учителя о религии, которую они слышали на пути в Сидон. Все они были способны в той или иной степени оценить значимость того, что он им поведал, но никто из них не осознал всю важность его учения в полной мере.
\usection{1. Сирийская женщина}
\vs p156 1:1 Недалеко от дома Каруски, где поселился Учитель, жила сирийская женщина, которая много слышала об Иисусе как о великом целителе и учителе, и в этот субботний день она пришла к нему, приведя с собой свою маленькую дочь. Подросток, примерно двенадцати лет, страдал тяжелым нервным расстройством, сопровождающимся судорогами и другими неприятными проявлениями.
\vs p156 1:2 Иисус велел своим сподвижникам никому не рассказывать о его присутствии в доме Каруски, объяснив, что хочет отдохнуть. Хотя они и выполняли это указание, но служанка Каруски сходила в дом к этой сирийской женщине, Норане, и сообщила, что Иисус живет в доме у ее хозяйки, и убедила несчастную мать привести больную дочь для лечения. Мать, конечно, верила, что ее ребенок одержим демоном, нечистым духом.
\vs p156 1:3 Когда Норана с дочерью пришла, близнецы Алфеевы объяснили через переводчика, что Учитель отдыхает и его нельзя беспокоить; на что Норана ответила, что она и ребенок останутся тут до тех пор, пока Учитель не закончит свой отдых. Петр тоже пытался урезонить ее и убедить пойти домой. Он объяснял, что Иисус устал оттого, что много учил и исцелял, и что он пришел в Финикию, чтобы на время обрести покой и отдых. Но все было бесполезно; Норана никак не уходила. На просьбы Петра она только отвечала: «Я не уйду, пока не увижу Учителя. Я знаю, что он может изгнать демона из моего ребенка, и не уйду, пока этот целитель не посмотрит мою дочь.»
\vs p156 1:4 Потом Фома попытался отправить назад эту женщину, но тоже потерпел неудачу. Ему она сказала: «Я верю, что ваш Учитель может изгнать этого демона, который мучает моего ребенка. Я слышала о его могущественных деяниях в Галилее, и я верую в него. Что случилось с вами, его учениками, что вы отправляете прочь тех, кто приходит за помощью к Учителю?» И когда она это сказала, Фома отступил.
\vs p156 1:5 Затем Симон Зилот принялся увещевать Норану. Симон сказал: «Женщина, ты --- нееврейка и говоришь по\hyp{}гречески. Не следует ожидать, что Учитель возьмет хлеб, предназначенный детям избранной семьи, и бросит его собакам». Но Норана не стала обижаться на выпад Симона. Она только ответила: «Да, учитель, я понимаю твои слова. Я всего лишь собака в глазах евреев, но в том, что касается вашего Учителя, я верующая собака. Я непременно хочу, чтобы он посмотрел мою дочь, потому что убеждена, что если он лишь взглянет на нее, то вылечит ее. И даже ты, мой добрый человек, не осмелишься лишить собаку права хватать те крошки, которые случайно падают с детского стола».
\vs p156 1:6 Как раз в это время у девочки в присутствии их всех началась сильная судорога, и мать закричала: «Вот, теперь вы видите, что мой ребенок одержим злым духом. Если наша беда не производит впечатления на вас, то она тронет вашего Учителя, который, как мне говорили, любит всех людей и решается лечить даже неевреев, когда они веруют. Вы не достойны быть его учениками. Я не уйду, пока мой ребенок не будет исцелен».
\vs p156 1:7 Иисус через открытое окно слышал все эти разговоры и теперь, к их большому удивлению, вышел из дому и сказал «О, женщина, велика твоя вера, так велика, что я не могу отказать тебе в том, чего ты желаешь; иди своей дорогой с миром. Твоя дочь уже выздоровела». И маленькая девочка с того часа была здорова. Когда Норана с дочерью уходили, Иисус попросил их никому не говорить об этом случае; и хотя его соратники исполнили эту просьбу, мать и ребенок непрестанно рассказывали об исцелении маленькой девочки по всем окрестностям, даже в Сидоне, и дело дошло до того, что через несколько дней Иисус счел за лучшее сменить место жительства.
\vs p156 1:8 \pc На следующий день, когда Иисус учил своих апостолов, разъяснял излечение дочери сирийской женщины, он сказал: «И так это было всюду и всегда; вы сами видите, как неевреи способны проявлять спасительную веру в учение евангелия царства небесного. Истинно, истинно говорю я вам, что царство Отца будет занято неевреями, если дети Авраама не проявят достаточно веры, чтобы войти в него».
\usection{2. Учение в Сидоне}
\vs p156 2:1 Входя в Сидон, Иисус и его сподвижники прошли по мосту --- первому в жизни многих из них. Когда они шли по нему, Иисус в числе прочего сказал: «Этот мир --- только мост; вы можете пройти по нему, но не следует думать о том, чтобы построить на нем жилище.»
\vs p156 2:2 \pc Когда двадцать четыре сподвижника начали свою деятельность в Сидоне, Иисус отправился жить в дом Юсты и ее матери, Берники, находившийся к северу от города. Каждое утро Иисус учил в доме у Юсты, после чего все двадцать четыре отправлялись в Сидон, чтобы в течение дня и вечера учить и проповедовать.
\vs p156 2:3 Апостолы и евангелисты очень радовались тому, как неевреи Сидона принимали их весть; во время их краткой остановки к царству примкнуло много людей. Все усилия по привлечению душ в этот почти шестинедельный период пребывания в Финикии были очень плодотворными, но впоследствии еврейские авторы евангелий были склонны умалчивать о том, как тепло восприняли эти неевреи учения Иисуса в то самое время, когда многие его соплеменники принимали его враждебно.
\vs p156 2:4 Во многих отношениях эти верующие неевреи осмыслили учения Иисуса более глубоко, чем евреи. Многие из греко\hyp{}говорящих сирофиникийцев осознали не только то, что Иисус подобен Богу, но и то, что Бог подобен Иисусу. Эти так называемые язычники хорошо поняли учение Учителя о единообразии законов этого мира и всей вселенной. Они восприняли учение о том, что Бог не делает различий между людьми, расами и нациями; что у Отца Всего Сущего нет фаворитов; что Вселенная полностью и всегда подчиняется законам и неизменно надежна. Эти неевреи не боялись Иисуса; они смело принимали его послание. Хотя во все века люди, были способны понять Иисуса, они боялись это делать.
\vs p156 2:5 \pc Иисус объяснил двадцати четырем, что спасся бегством из Галилеи вовсе не потому, что ему недостало смелости противостоять своим врагам. Сподвижники поняли, что он был еще не готов к открытому столкновению с общепринятой религией и не стремился стать мучеником. Именно во время одной из бесед в доме Юсты Учитель впервые сказал своим ученикам, что «даже если небо и земля исчезнут, мои слова истины останутся».
\vs p156 2:6 \pc Темой наставлений Иисуса во время пребывания в Сидоне было духовное развитие. Он сказал им, что нельзя стоять на месте; надо или возрастать в праведности, или же возвращаться назад к злу и греху. Он советовал им «забыть о том, что осталось в прошлом, двигаясь вперед к восприятию более возвышенных реалий царства». Он просил их не довольствоваться состоянием детства в евангелии, но стремиться к достижению полноценного божественного сыновства в общении с духом и в братстве верующих.
\vs p156 2:7 Сказал Иисус: «Мои ученики должны не только перестать делать зло, но и научиться делать добро; вы должны не только очиститься от всякого сознательного греха, но должны перестать даже испытывать чувство вины. Если вы исповедуетесь в своих грехах, они прощаются; поэтому ваша совесть должна быть чиста».
\vs p156 2:8 Иисусу очень нравилось тонкое чувство юмора, которое проявили эти неевреи. Именно чувство юмора, продемонстрированное Нораной, сирийской женщиной, и ее великая и стойкая вера так тронули сердце Учителя и воззвали к его милосердию. Иисус чрезвычайно сожалел, что его народу --- евреям --- так недоставало юмора. Однажды он сказал Фоме: «Мой народ воспринимает себя слишком серьезно; они почти лишены способности понимать юмор. Тягостная религия фарисеев никогда не могла бы укорениться среди людей с чувством юмора. Им также не хватает последовательности: они отцеживают комара, а глотают верблюда».
\usection{3. Путь на север по побережью}
\vs p156 3:1 Во вторник 28 июня Учитель и его сподвижники покинули Сидон и отправились на север по побережью в Порфирион и Хелдую. Неевреи хорошо принимали их, многие примкнули к царству в эту неделю учения и проповедования. Апостолы проповедовали в Порфирионе, а евангелисты учили в Хелдуе. Пока двадцать четыре были заняты такой работой, Иисус покинул их на три\hyp{}четыре дня, чтобы отправиться в приморский город Бейрут, где навестил верующего сирийца по имени Малах, который посетил за год до этого Вифсаиду.
\vs p156 3:2 В среду 6 июля все вернулись в Сидон и оставались в доме у Юсты до воскресного утра, когда отправились на юг вдоль берега через Сарепту в Тир, куда и прибыли в понедельник 11 июля. К этому времени апостолы и евангелисты начали привыкать к работе среди так называемых неевреев, которые в действительности происходили в основном из ранее существовавших ханаанских племен, имевших очень древние семитские корни. Все эти народы говорили на греческом языке. Для апостолов и евангелистов было чрезвычайно удивительно видеть рвение, с которым эти неевреи слушали евангелие, и замечать, с какой охотой многие из них уверовали.
\usection{4. В Тире}
\vs p156 4:1 С 11 до 24 июля они учили в Тире. Каждый из апостолов брал с собой одного из евангелистов, и таким образом по двое они учили и проповедовали по всему Тиру и его окрестностям. Многоязычное население этого оживленного морского порта охотно слушало их, и многие крестились, вступая в открытое братство царства. Иисус остановился в доме еврея по имени Иосиф, верующего, который жил в трех\hyp{}четырех милях к югу от Тира, недалеко от гробницы Хирама --- царя города\hyp{}государства Тир во времена Давида и Соломона.
\vs p156 4:2 Ежедневно в течение этих двух недель апостолы и евангелисты приходили в Тир через дамбу Александра, чтобы проводить небольшие встречи, и каждый вечер большинство из них возвращались в лагерь в дом Иосифа к югу от города. Каждый день верующие выходили из города, чтобы поговорить с Иисусом там, где он жил. Учитель проповедовал в Тире только один раз, днем 20 июля, когда он учил верующих о любви Отца ко всему человечеству и миссии Сына, цель которой --- явить Отца всем человеческим расам. Неевреи проявили такой интерес к евангелию царства, что по этому случаю ему были открыты двери храма Мелкарфа, и следует сказать, что в последующие годы прямо на территории этого древнего храма была построена христианская церковь.
\vs p156 4:3 Многие ведущие производители тирского пурпура, краски, которая прославила Тир и Сидон на весь мир и которой так успешно торговали по всему миру и вследствие этого сильно обогатились, уверовали в царство. Когда вскоре после этого численность морских моллюсков, служивших исходным материалом для этой краски, стала уменьшаться, эти ремесленники отправились на поиски новых мест их обитания. И таким образом, повсюду вплоть до отдаленных уголков земли, они несли с собой весть об отцовстве Бога и братстве людей --- евангелие царства.
\usection{5. Учение Иисуса в Тире}
\vs p156 5:1 Днем в эту среду в своей речи Иисус сначала рассказал своим последователям о белой лилии, которая высоко возносит свою чистую и белоснежную головку к солнечному свету, тогда как корни ее находятся внизу, в иле и навозе темной почвы. «Подобно этому, --- сказал он, --- и смертный человек, несмотря на то, что происхождение и бытие его человеческой природы уходит корнями в животную почву, может через веру возвысить свою духовную природу к солнечному свету небесной истины и действительно выносить возвышенный плод духа».
\vs p156 5:2 В этой же проповеди Иисус в первый и последний раз обратился к притче, связанной с его собственной профессией --- плотничеством. Во время его наставления «Стройте добрый фундамент для развития возвышенного характера, обладающего духовной одаренностью», он сказал: «Для того, чтобы принести плоды духа, вы должны родиться в духе. Вы должны учиться у духа и быть ведомыми духом, если хотите жить исполненной духа жизнью среди своих братьев. Но не совершайте ошибку глупого плотника, который попусту тратит драгоценное время, ровно обтесывая, измеряя и отшлифовывая свое изъеденное личинками и гнилое внутри дерево, а затем, потратив таким образом столько труда на гнилое бревно, вынужден отбросить его как негодное для укладки в фундамент здания, которое он хотел сделать устойчивым перед натиском времени и бурь. Пусть каждый человек убедится, что интеллектуальный и моральный фундамент его характера таков, что он должным образом выдержит возрастающую и возвышающую надстройку духовной природы, которая должна преобразить ум смертного человека, а затем совокупно с таким преобразованным умом обеспечить развитие души, которой уготовано бессмертие. Ваша духовная природа --- совместно созданная душа --- это нечто живое и растущее, но ум и моральные качества конкретного человека --- это почва, из которой должны произрастать эти высшие проявления человеческого развития и божественного предназначения. Почва для развивающейся души --- человеческая и материальная, но предназначение этого совместного создания ума и духа --- духовное и божественное».
\vs p156 5:3 Вечером того же дня Нафанаил спросил Иисуса: «Учитель, почему мы молим Бога не вводить нас в искушение, когда мы хорошо знаем из того, что ты открыл нам об Отце, что он никогда не делает таких вещей?» Иисус ответил Нафанаилу:
\vs p156 5:4 «Ничего странного, что ты задаешь такие вопросы, поскольку видно, что ты начинаешь узнавать Отца так, как я знаю его, а не так, как его смутно представляли себе древние еврейские пророки. Ты хорошо знаешь, как наши предки склонны были видеть Бога почти во всем, что происходило. Они искали руку Бога во всех природных явлениях и в каждом необычном эпизоде человеческого опыта. Они связывали Бога и с добром, и со злом. Они думали, что он смягчил сердце Моисея и ожесточил сердце фараона. Когда человек испытывал сильную потребность что\hyp{}то сделать, хорошее или плохое, было принято объяснять эти необычные эмоции словами: „Господь говорил со мной и сказал: сделай то\hyp{}то и так\hyp{}то, пойди туда\hyp{}то“. Естественно, поскольку люди так часто и яростно сталкивались с искушением, для наших предков стало привычным верить, что Бог вводил их в него, чтобы испытать, наказать или укрепить. Но вы теперь действительно все понимаете лучше. Вы знаете, что всех людей слишком часто вводит в искушение их собственный эгоизм и побуждения их животной природы. Когда вы испытываете такого рода искушения, я советую вам, что если вы честно и искренне отдаете себе отчет в них, то должны разумно перенаправить энергии духа, ума и тела, ищущие выхода, в другие, более высокие русла и на более возвышенные цели. Таким образом вы можете преобразовать свои искушения в высочайшие типы служения, возвышающего смертных, и при этом практически полностью избежать опустошающего и ослабляющего противоборства животной и духовной природы.
\vs p156 5:5 Но позвольте мне предостеречь вас от неразумных попыток преодолеть искушение, заменяя одно желание другим, предположительно более высоким, просто усилием человеческой воли. Если вы действительно хотите победить искушения более низкого характера, вы должны достичь такого духовного уровня, где в вас действительно и истинно развивается настоящий интерес и любовь к тому высшему и более идеалистическому образу действий, которым ваш ум стремится заменить эти низшие и менее идеалистические привычки поведения, рассматриваемые вами как искушения. Таким образом через духовное преображение вы освободитесь и не будете все время мучиться, тщетно стараясь подавить человеческие страсти. Старое и низшее забудется в любви к новому и высшему. Красота всегда побеждает уродство в сердцах тех, кого озаряет любовь к истине. Бьющая ключом энергия новой и искренней духовной любви несет в себе мощную силу. И вновь говорю я вам: не давайте злу победить себя, но побеждайте зло добром».
\vs p156 5:6 До поздней ночи апостолы и евангелисты продолжали задавать вопросы, и многочисленные ответы на них на современном языке можно изложить следующим образом:
\vs p156 5:7 Сильное честолюбие, умные суждения и зрелая мудрость --- основы материального успеха. Лидерство зависит от природных способностей, благоразумия, силы воли и целеустремленности. Духовное предназначение зависит от веры, любви и преданности истине --- жажды праведности --- всепоглощающей страсти найти Бога и быть подобным ему.
\vs p156 5:8 Пусть вас не обескураживает сознание того, что вы --- люди. Человеческая природа может тяготеть к злу, но она не является изначально греховной. Пусть вас не удручает то, что вам не удается полностью забыть что\hyp{}то из вашего горького опыта. Ошибки, которые вам не удается забыть с течением времени, забудутся в вечности. Облегчите груз, лежащий на вашей душе, быстро обретя видение отдаленной перспективы вашего предназначения, вселенского продолжения вашего пути.
\vs p156 5:9 Не совершайте ошибку, оценивая достоинства души по несовершенствам ума или потребностям тела. Не судите о душе и о ее предназначении, на основании одного лишь неудачного эпизода человеческой жизни. Ваше духовное предназначение обусловлено только вашими духовными устремлениями и целями.
\vs p156 5:10 Религия --- это исключительно духовный опыт развития бессмертной души знающего Бога человека, но моральная сила и духовная энергия --- это мощные силы, которые могут быть использованы при разрешении сложных общественных ситуаций и решении нелегких экономических проблем. Эти моральные и духовные дарования делают все уровни человеческой жизни более богатыми и значимыми.
\vs p156 5:11 Вам суждено прожить ограниченную и скудную жизнь, если вы научились любить только тех, кто любит вас. Человеческая любовь может быть действительно взаимной, но божественная любовь не иссякает даже когда не встречает взаимности, к которой стремится. Чем меньше любви в природе какого\hyp{}либо создания, тем больше его потребность в любви, и тем больше божественная любовь стремится удовлетворить такую потребность. Любовь никогда не бывает корыстной и не может быть себялюбивой. Божественная любовь не может быть самодостаточной; она должна дароваться бескорыстно.
\vs p156 5:12 Верующие в царство должны иметь непререкаемую веру, веру от всей души, в несомненное торжество праведности. Строители царства должны не сомневаться в истинности евангелия вечного спасения. Верующие должны учиться все больше уходить от суеты жизни --- избегать треволнений материального существования --- когда они очищают душу, вдохновляют ум и обновляют дух через общение с богом путем почитания его.
\vs p156 5:13 Людей, знающих Бога, не обескураживают неудачи и не удручают разочарования. Верующие не подвержены унынию, проистекающему от чисто материальных невзгод; живущих в духе не приводят в смятение перипетии материального мира. Кандидаты на обретение вечной жизни применяют придающий силы конструктивный подход, позволяющий преодолевать все превратности и невзгоды земной жизни. Верующему с каждым прожитым днем становится \bibemph{легче} поступать правильно.
\vs p156 5:14 Духовная жизнь сильно повышает истинное самоуважение. Но самоуважение --- это не самолюбование. Самоуважение всегда сочетается с любовью и служением ближним. Нельзя уважать себя больше, чем любишь своего соседа; одно есть мера способности другого.
\vs p156 5:15 Со временем каждый истинно верующий становится более искусным в умении вызывать в своих ближних любовь к вечной истине. Стали ли вы сегодня более изобретательны в деле открытия добродетели человечеству, чем были вчера? Стали ли вы в этом году лучшими поборниками праведности, чем были в прошлом году? Стали ли вы искуснее в умении приводить изголодавшиеся души в духовное царство?
\vs p156 5:16 Достаточно ли высоки ваши идеалы, чтобы обеспечить вечное спасение, в то время как ваши идеи достаточно практичны, чтобы сделать вас полезными гражданами, способными действовать на земле в сотрудничестве со своими смертными собратьями? В духе --- ваше гражданство на небе; во плоти вы все еще граждане земных царств. Отдайте кесарям материальное, а Богу --- духовное.
\vs p156 5:17 Мерой духовной способности развивающейся души является ваша вера в истину и любовь к человеку, но мерой силы вашего человеческого характера является ваше умение противостоять собственному чувству обиды и ваша способность сопротивляться тягостным чувствам, вызываемым сильным горем. Поражение --- это то зеркало, в котором вы можете правдиво увидеть свою подлинную личность.
\vs p156 5:18 По мере того, как вы становитесь старше и опытнее в делах царства, становитесь ли вы тактичнее в общении с трудными людьми и терпимее в жизни с упрямыми собратьями? Такт --- это точка опоры социального рычага, а терпимость --- отличительный признак великой души. Если вы обладаете этими редкими и очаровательными дарами, то с течением времени станете умнее и искуснее в своих достойных похвалы стараниях не допускать всевозможных общественных недоразумений. Такие мудрые души способны избегать значительной части тех неприятностей, которые несомненно выпадают на долю всех тех, кому трудно эмоционально приспособиться, тех, кто отказывается взрослеть, отказывается достойно стареть.
\vs p156 5:19 Избегайте нечестности и несправедливости во всех своих усилиях проповедовать истину и возвещать евангелие. Не стремитесь к незаслуженному признанию и не желайте незаслуженной симпатии. Свободно принимайте любовь божественную и человеческую, независимо от ваших заслуг, и свободно любите в ответ. Но что касается всего, что связано с честью и славой, стремитесь только к тому, что по справедливости принадлежит вам.
\vs p156 5:20 Осознавший Бога смертный уверен в спасении; он не боится жизни; он честен и несгибаем. Он умеет стойко переносить неизбежные страдания; он не жалуется, сталкиваясь с неотвратимыми тяготами.
\vs p156 5:21 Истинный верующий не прекращает делать добрые дела, даже если сталкивается с препятствиями. Трудности разжигают пыл поборника истины, а препятствия лишь стимулируют усилия неустрашимого строителя царства.
\vs p156 5:22 \pc И еще многому учил их Иисус прежде, чем они собрались покинуть Тир.
\vs p156 5:23 Прежде чем отправиться из Тира обратно к Галилейскому озеру, Иисус созвал своих сподвижников и велел двенадцати евангелистам возвращаться другим маршрутом нежели он сам с двенадцатью апостолами. После того, как здесь евангелисты расстались с Иисусом, они уже никогда не общались с ним так тесно.
\usection{6. Возвращение из Финикии}
\vs p156 6:1 Около полудня в воскресенье 24 июля Иисус и двенадцать апостолов покинули дом Иосифа, расположенный к югу от Тира, и отправились по побережью на юг в Птолемаиду. Здесь они остановились на день, обратившись со словами утешения к проживавшей в этих местах группе верующих. Вечером 25 июля Петр произнес перед ними проповедь.
\vs p156 6:2 Во вторник они покинули Птолемаиду и отправились по Тивериадской дороге на восток вглубь станы к окрестностям Иотапаты. В среду они остановились в Иотапате и наставляли верующих в делах царства. В четверг они покинули Иотапату и отправились на север по тропе, идущей от Назарета к горе Ливан, мимо Рамы к селению Завулон. В пятницу в Раме они встречались с людьми и оставались там и в субботу. В воскресенье 31 июля они добрались до Завулона, вечером провели встречу с людьми и отбыли на следующее утро.
\vs p156 6:3 Покинув Завулон, они проследовали до пересечения с дорогой Магдала --- Сидон, последний расположен возле Гишалы, и оттуда направились в Геннисарет, к западному берегу Галилейского озера южнее Капернаума, где они договорились встретиться с Давидом Зеведеем и где намеревались держать совет о том, что дальше делать на поприще проповеди евангелия царства.
\vs p156 6:4 Во время короткого совещания с Давидом они узнали, что многие лидеры собрались на противоположном берегу озера возле Хересы, и в тот же самый вечер на лодке переплыли на тот берег. День они спокойно отдыхали в горах, а на следующий отправились в расположенный неподалеку парк, где Учитель некогда накормил пять тысяч человек. Здесь они три дня отдыхали и ежедневно проводили встречи, на которых присутствовали около пятидесяти мужчин и женщин, это все, что осталось от некогда многочисленной группы верующих, проживавших в Капернауме и его окрестностях.
\vs p156 6:5 \pc Пока Иисус отсутствовал в Капернауме и Галилее, находясь в Финикии, его враги посчитали, что со всем движением уже покончено, и пришли к выводу, что поспешный уход Иисуса означает, что он настолько сильно напуган, что, похоже, уже никогда не вернется и не причинит им больше беспокойств. Всякое активное противодействие его учению почти утихло. Верующие снова стали открыто проводить собрания, и происходило постепенное, но действенное сплочение испытанных и верных людей, ибо от верующих в евангелие совсем недавно отступились очень многие.
\vs p156 6:6 Филипп, брат Ирода, отчасти уверовал в Иисуса и дал разрешение Учителю свободно жить и трудиться в его владениях.
\vs p156 6:7 Указ закрыть синагоги всего еврейства для учений Иисуса и всех его последователей возымел неблагоприятные последствия для книжников и фарисеев. Сразу же после того, как Иисус удалился и перестал быть объектом споров, произошло волнение среди всего еврейского народа; возникло недовольство фарисеями и руководителями Синедриона в Иерусалиме. Многие из управителей синагог начали потихоньку открывать свои синагоги для Авенира и его сподвижников, утверждая, что эти учителя были последователями Иоанна, а не учениками Иисуса.
\vs p156 6:8 Даже Ирод Антипа переменил свое отношение, и когда узнал, что Иисус находится на другой стороне озера на территории его брата Филиппа, то послал ему известие, в котором говорилось, что хотя он и подписал ордер на его арест в Галилее, но не отдаст приказ о его аресте в Перее, давая тем самым понять, что Иисус за пределами Галилеи не будет иметь неприятностей, и Ирод сообщил об этом же в своем послании евреям Иерусалима.
\vs p156 6:9 Такова была ситуация перед первым августа 29 года н.э., когда Учитель вернулся из финикийской миссии и начал реорганизацию своих рассеянных, прошедших через испытания и истощенных рядов свих последователей в преддверии богатого событиями последнего года его миссии на земле.
\vs p156 6:10 Вопросы, вокруг которых пойдет борьба, были уже ясно очерчены к моменту, когда Учитель и его сподвижники были готовы начать провозглашение новой религии, религии духа живого Бога, живущего в умах людей.
