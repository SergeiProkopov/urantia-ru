\upaper{4}{Отношение Бога к вселенной}
\author{Божественный Советник}
\vs p004 0:1 Отец всего сущего имеет вечный замысел, касающийся материальных, интеллектуальных и духовных явлений вселенной вселенных, который он выполняет на протяжении всего времени. Бог сотворил вселенные по своей собственной свободной и полновластной воле, и сотворил он их в соответствии со своим всемудрейшим и вечным замыслом. Сомнительно, что кто\hyp{}либо, кроме Райских Божеств и их высочайших сподвижников, действительно много знает о вечном замысле Бога. Даже возвышенные граждане Рая придерживаются очень разных мнений относительно характера вечного замысла Божеств.
\vs p004 0:2 Легко заключить, что замысел сотворения совершенной центральной вселенной Хавоны имел целью просто удовлетворение божественной природы. Хавона может служить паттерном творения для всех прочих вселенных и школой совершенства для странников во времени, движущихся по пути к Раю; однако такое божественное творение должно существовать, в первую очередь, для удовольствия и удовлетворения совершенных и бесконечных Творцов.
\vs p004 0:3 Изумительный план совершенствования эволюционных смертных, а после достижения ими Рая и Отряда Финалитов --- обеспечения дальнейшей подготовки к некой нераскрытой будущей деятельности является в настоящее время одной из главных забот семи сверхвселенных и их многочисленных подразделений; но эта программа восхождения, предназначенная для одухотворения и обучения смертных, живущих во времени и пространстве, никоим образом не единственное, чем заняты разумы вселенной. В действительности, существует множество других захватывающих занятий, на которые уходит время и энергия небесных сонмов.
\usection{1. Позиция Отца по отношению к вселенной}
\vs p004 1:1 Веками обитатели Урантии неправильно понимали Божественное провидение. Существует провидение в вашем мире, состоящее в божественных действиях, но это не то наивное, произвольное и материальное служение, каким его представляли себе многие смертные. Божественное провидение заключается во взаимосвязанной деятельности небесных существ и божественных духов, которые в соответствии с космическим законом непрестанно трудятся во славу Бога и для духовного прогресса его вселенских детей.
\vs p004 1:2 Разве вы не можете продвинуться в своем представлении об отношении Бога к человеку до такого уровня, чтобы осознать, что девиз вселенной --- \bibemph{прогресс} ? На протяжении долгих веков человечество боролось, чтобы достичь своего нынешнего положения. На протяжении всех этих тысячелетий Провидение разрабатывало план прогрессивной эволюции. Эти две мысли не противостоят друг другу на практике, а только лишь в ошибочных концепциях человека. Божественное провидение никогда не находится в оппозиции к истинному человеческому прогрессу, будь то мирской или духовный. Провидение всегда находится в соответствии с неизменной и совершенной природой верховного Законодателя.
\vs p004 1:3 «Бог истин» и «все его заповеди справедливы». «Его истинность утверждена на самих небесах». «На веки, Господи, твое слово утверждено на небесах. Твоя истинность --- всем поколениям; ты поставил землю, и она стоит». «Он --- истинный Создатель.»
\vs p004 1:4 Нет ограничений для сил и личностей, которых Отец может использовать для утверждения своего замысла и поддержки своих созданий. «Вечный Бог --- наше прибежище и неустанный покровитель». «Живущий под кровом Всевышнего будет пребывать под сенью Всемогущего». «Смотри, хранящий нас не воздремлет и не заснет». «Мы знаем, что все содействует ко благу любящим Бога», «ибо глаза Господа обращены к праведным, и его уши открыты для их молитв».
\vs p004 1:5 Бог поддерживает «все словом своей силы». А когда рождаются новые миры, он «посылает своих Сынов, и они создаются». Бог не только творит, но и «сохраняет их всех». Бог постоянно поддерживает все материальные вещи и все духовные существа. Вселенные вечно стабильны. Среди кажущейся нестабильности есть стабильность. Среди энергетических переворотов и физических катаклизмов в звездных мирах есть основополагающие порядок и надежность.
\vs p004 1:6 Отец Всего Сущего не устраняется от управления вселенными; он --- не бездеятельное Божество. Если бы Бог перестал поддерживать все сотворенное, то немедленно наступил бы крах вселенной. Если бы не Бог, то не было бы никакой \bibemph{реальности.} В данный конкретный момент, как и в отдаленные эпохи прошлого и в вечном будущем, Бог продолжает поддерживать, Божественное покровительство простирается над всем кругом вечности. Вселенная не заводится, подобно часам, которые долго ходят, а затем останавливаются; все постоянно обновляется. Отец непрерывно изливает энергию, свет и жизнь. Работа Бога --- является не только духовной, но и буквальной. «Он простирает север над пустотой и опирает землю ни на что».
\vs p004 1:7 \pc Существо моего чина в состоянии открыть предельную гармонию и обнаружить далеко идущую и глубокую скоординированность обычных дел, связанных с управлением вселенной. Многое из того, что кажется смертному разуму разрозненным и случайным, в моем понимании представляется планомерным и структурированным. Но во вселенных происходит очень много такого, чего я до конца не понимаю. Я долгое время изучал признанные силы, энергии, разумы, моронтии, дух и личности локальных вселенных и сверхвселенных, и я более или менее знаком с ними. Я имею общее представление о том, как действуют эти факторы и личности, и я близко знаком с работой облеченных полномочиями духовных разумов великой вселенной. Несмотря на мое знание явлений вселенных, я постоянно сталкиваюсь с космическими реакциями, которые не могу полностью постичь. Я непрерывно встречаюсь с, по\hyp{}видимому, случайными стеченьями взаимосвязей сил, энергий, интеллектов и духов, которые не могу удовлетворительным образом объяснить.
\vs p004 1:8 Я совершенно правомочен прослеживать и анализировать действия всех феноменов, являющихся прямым результатом функционирования Отца Всего Сущего, Вечного Сына, Бесконечного Духа и, в значительной степени, Острова Рая. Мое недоумение вызвано столкновением с тем, что проявляется как действие таинственных и в равной степени великих трех Абсолютов потенциальности. Эти Абсолюты, похоже, заменяют собой материю, превосходят разум и дополняют дух. Меня постоянно приводит в смущение и часто в недоумение моя неспособность понять те сложные процессы, которые я связываю с присутствием и действиями Неограниченного Абсолюта, Божественного Абсолюта и Вселенского Абсолюта.
\vs p004 1:9 Эти Абсолюты повсюду во вселенной должны быть не полностью раскрытыми присутствиями, которые, как явление потенции пространства и функционирование других сверхпредельностей, лишают возможности физиков, философов или даже приверженцев религии с уверенностью предвидеть, как истоки силы, понятий или духа будут реагировать на требования, предъявляемые в сложной реальной ситуации, связанной с верховными настройками и предельными ценностями.
\vs p004 1:10 \pc Во вселенных времени и пространства существует также органическое единство, которое, похоже, лежит в основе всей структуры космических событий. Это живое присутствие эволюционирующего Верховного Существа, эта Имманентность Проектируемого Незавершенного, то и дело необъяснимым образом проявляется в том, что представляется поразительной случайной согласованностью, казалось бы, несвязанных между собой вселенских событий. Это, должно быть, действие Провидения --- сферы Верховного Существа и Носителя Объединенных Действий.
\vs p004 1:11 Я склонен верить, что именно этот обширный и, в целом, не опознаваемый контроль за координацией и взаимосвязью всех фаз и форм вселенской деятельности приводит к тому, что такая разнообразная и, казалось бы, безнадежно запутанная смесь физических, умственных, нравственных и духовных феноменов так безошибочно работает во славу Бога и на благо людей и ангелов.
\vs p004 1:12 Но в более широком смысле кажущиеся «случайности» космоса, несомненно, являются частью конечной драмы пространственно\hyp{}временного начинания Бесконечного в процессе его вечного управления Абсолютами.
\usection{2. Бог и природа}
\vs p004 2:1 Природа --- это, в узком смысле, физическое облачение Бога. Поведение, или деятельность Бога ограничиваются и предварительно модифицируются экспериментальными планами и эволюционными паттернами локальной вселенной, созвездия, системы или планеты. Во всей обширной главной вселенной Бог действует в соответствии с четко определенным, неизменным, непреложным законом; но он модифицирует паттерны своей деятельности таким образом, чтобы способствовать согласованному и сбалансированному руководству каждой вселенной, созвездием, системой, планетой и личностью в соответствии с локальными целями, замыслами и планами конечных проектов эволюционного развертывания.
\vs p004 2:2 Поэтому природа, как понимает ее смертный человек, представляет собой основополагающий фундамент и существенный фон неизменного Божества и его непреложных законов, модифицируемых, колеблющихся и переживающих потрясения в соответствии с действием локальных планов, замыслов, схем и условий, которым положили начало и которые выполняют силы и личности локальной вселенной, созвездия, системы и планеты. Например: с тех пор, как в Небадоне были утверждены законы Бога, они модифицируются планами, установленными Сыном\hyp{}Творцом и Творческим Духом этой локальной вселенной; и вдобавок ко всему этому, на действие этих законов в дальнейшем оказывали влияние ошибки, срывы и мятежи некоторых существ, постоянно проживающих на вашей планете и принадлежащих к вашей непосредственной планетарной системе Сатании.
\vs p004 2:3 \pc Природа является пространственно\hyp{}временным результатом двух космических факторов: во\hyp{}первых, неизменности, совершенства и высоконравственности Райского Божества и, во\hyp{}вторых, экспериментальных планов, промахов исполнителей, мятежных проступков, незавершенности развития и несовершенства мудрости обитающих вне рая творений, от высших до низших. Поэтому природа протягивает единую, неизменную, величественную и чудесную нить совершенства от круга вечности; но в каждой вселенной, на каждой планете и в каждой индивидуальной жизни эта природа модифицируется, ограничивается и, возможно, портится проступками, ошибками и неверностью творений эволюционных систем и вселенных; и поэтому природа всегда должна быть переменчивой и к тому же причудливой, хотя и стабильной в своей основе, и меняющейся в соответствии с образом действия локальной вселенной.
\vs p004 2:4 Природа --- это совершенство Рая, деленное на несовершенство, зло и греховность незавершенных вселенных. Эта дробь выражает, таким образом, совершенное и неполное, вечное и временное. Продолжающаяся эволюция видоизменяет природу, увеличивая содержание Райского совершенства и уменьшая содержание зла, ошибок и дисгармонии относительной реальности.
\vs p004 2:5 \pc Бог лично не присутствует в природе или в каких\hyp{}либо силах природы, ибо явление природы --- это наложение несовершенств прогрессивной эволюции и иногда последствий мятежного бунта на Райскую основу вселенского закона Бога. Ясно, что в таком мире, как Урантия, природа никогда не может быть адекватным выражением, подлинным образом, верным изображением всемудрейшего и бесконечного Бога.
\vs p004 2:6 Природа в вашем мире --- это ограничение законов совершенства эволюционными планами локальной вселенной. Какое заблуждение --- поклоняться природе потому, что отчасти, в каком\hyp{}то смысле, она наполнена Богом; потому, что она является фазой вселенской и потому божественной силы! Природа также есть проявление незаконченности, незавершенности, несовершенства развития, роста и прогресса вселенского эксперимента космической эволюции.
\vs p004 2:7 Видимые недостатки природного мира не свидетельствуют о каких\hyp{}либо таких же соответствующих недостатках в характере Бога. Такие наблюдаемые несовершенства представляют собой просто неизбежные паузы при показе вечно движущейся киноленты --- экранизации бесконечности. Именно эти самые несовершенные перерывы в непрерывности совершенства дают возможность конечному разуму материального человека уловить мимолетную картину божественной реальности во времени и пространстве. Материальные проявления божественности представляются эволюционному разуму человека несовершенными только потому, что смертный человек упорно воспринимает явления природы примитивно, с человеческой точки зрения, без помощи моронтийной моты или же откровения, ее компенсационного заменителя в мирах времени.
\vs p004 2:8 И природа искажается, ее прекрасное лицо обезображивается, ее черты увядают в результате бунта, дурного поведения, дурных мыслей мириад творений, которые являются частью природы, но способствуют ее искажению во времени. Нет, природа --- это не Бог. Природа --- не объект поклонения.
\usection{3. Неизменный характер Бога}
\vs p004 3:1 Слишком долго человек думал о Боге как о подобном себе. Бог не испытывает, никогда не испытывал и никогда не будет испытывать ревности по отношению к человеку или какому\hyp{}либо другому существу во вселенной вселенных. При том, что Сын\hyp{}Творец намеревался сделать человека шедевром планетарного творения, правителем всей земли, вид того, как он пребывает во власти своих собственных низменных страстей, как преклоняется перед идолами --- деревом, камнем, золотом и эгоистическими амбициями --- эти отталкивающие картины вызывают у Бога и его Сынов ревностную заботу \bibemph{о} человеке, но никогда не вызывают ревность по отношению к нему.
\vs p004 3:2 Вечный Бог не способен на гнев и ярость в том смысле, как их переживает и как эти реакции понимает человек. Это низкие и презренные эмоции; едва ли они заслуживают того, чтобы называться человеческими, а уж тем более --- божественными; и такие чувства абсолютно чужды совершенной природе и милосердному характеру Отца Всего Сущего.
\vs p004 3:3 \pc В большой, очень большой степени трудность понимания Бога смертными Урантии объясняется далеко идущими последствиями бунта Люцифера и измены Калигастии. В мирах, которые не подверглись изоляции из\hyp{}за греха, эволюционные расы гораздо лучше могут сформулировать представления об Отце Всего Сущего; они меньше страдают от смешения, искажения и извращения понятий.
\vs p004 3:4 \pc Бог не раскаивается ни в чем, что он когда\hyp{}либо сделал, делает сейчас или когда\hyp{}либо сделает. Он --- всемудрейший, равно как и всемогущий. Человеческая мудрость вырастает из проб и ошибок человеческого опыта; мудрость Бога состоит в неограниченном совершенстве его бесконечного вселенского понимания, и это божественное предвидение эффективно руководит созидательной свободной волей.
\vs p004 3:5 Отец Всего Сущего никогда не делает ничего такого, что впоследствии станет причиной печали или сожаления, но обладающие волей творения, спроектированные и созданные его личностями\hyp{}Творцами во внешних вселенных, своим неудачным выбором вызывают иногда чувство божественной печали у личностей, являющихся их Творцами\hyp{}родителями. Но хотя Отец не делает ошибок, не испытывает сожалений и не подвержен печали, ему присуща отцовская любовь, и его сердце, несомненно, бывает опечалено, когда его дети не достигают тех духовных уровней, которых они способны достичь с той помощью, которую так щедро предоставляют вселенские планы духовных достижений и политика восхождения смертных.
\vs p004 3:6 Бесконечная добродетель Отца выходит за пределы понимания конечного разума, пребывающего во времени; поэтому, чтобы эффективно продемонстрировать все фазы относительной добродетели, всегда нужен контраст со сравнительным злом (не грехом). Совершенство божественной добродетели можно увидеть с помощью несовершенной человеческой проницательности только потому, что оно находится в сопоставимой связи с относительным несовершенством во взаимоотношениях времени и материи в движениях пространства.
\vs p004 3:7 Характер Бога бесконечно сверхчеловеческий; поэтому такая природа божественности должна быть персонализирована в божественных Сынах прежде, чем конечный разум человека сможет постичь ее через веру.
\usection{4. Постижение Бога}
\vs p004 4:1 Бог --- единственное стационарное, самодостаточное и неизменное существо во всей вселенной вселенных, нет ничего вне его, за пределами его, у него нет прошлого и будущего. Бог --- это целенаправленная энергия (творческий дух) и абсолютная воля, и они являются самостоятельно существующими и всеобъемлющими.
\vs p004 4:2 Поскольку Бог является самостоятельно существующим, он абсолютно независим. Сама идентичность Бога не подвержена изменениям. «Я, Господь, не изменяюсь». Бог --- неизменен; но пока вы не достигнете Райского состояния, вы не можете даже начать понимать, как Бог может переходить от простоты к сложности, от идентичности к варьированию, от неподвижности к движению, от бесконечности к конечности, от божественного к человеческому и от единства к двуединству и триединству. И, таким образом, Бог может видоизменять проявления своей абсолютности, потому что божественная неизменность не подразумевает неподвижность; Бог обладает волей --- он является волей.
\vs p004 4:3 Бог --- существо с абсолютным самоопределением; его вселенские реакции не имеют ограничений, кроме установленных им же самим, и его добровольные действия обусловлены только теми божественными качествами и совершенными свойствами, которые неотъемлемо присущи его вечной природе. Поэтому Бог относится к вселенной как существо, обладающее окончательной добродетелью и бесконечно\hyp{}созидательной свободной волей.
\vs p004 4:4 Отец Абсолют является творцом центральной и совершенной вселенной и Отцом всех остальных Творцов. Личность, добродетель и многочисленные другие качества общие для Бога, человека и прочих существ, но бесконечность воли свойственна лишь ему одному. Бог ограничен в своих созидательных действиях только чувствами своей вечной природы и велениями своей бесконечной мудрости. Бог лично выбирает только то, что бесконечно совершенно, и отсюда --- божественное совершенство центральной вселенной; и, хотя Сыны\hyp{}Творцы полностью разделяют его божественность, даже фазы его абсолютности, они не всецело ограничены той законченностью мудрости, которая руководит бесконечностью воли Отца. Поэтому в чине сыновства Михаила созидательная свободная воля становится еще более активной, полностью божественной и почти предельной, если не абсолютной. Отец бесконечен и вечен, но отрицание возможности его волевого самоограничения равносильно отрицанию самого этого представления о его волевой абсолютности.
\vs p004 4:5 \pc Абсолютность Бога пронизывает все семь уровней вселенской реальности. И вся эта абсолютная природа подчинена взаимоотношениям Творца с его семьей вселенских творений. Безошибочность, так можно охарактеризовать справедливость Троицы во вселенной вселенных, но во всех его обширных семейных взаимоотношениях с созданиями времени Бог вселенных руководствуется \bibemph{божественным чувством.} В первую и в последнюю очередь --- вечно --- бесконечный Бог является \bibemph{Отцом.} Из всех возможных имен, под которыми он обоснованно может быть известен, мне велено характеризовать Бога всего творения как Отца Всего Сущего.
\vs p004 4:6 Добровольными поступками Бога Отца не управляет сила; не управляет ими и один лишь интеллект; божественная личность определяется как заключающаяся в духе и проявляющая себя вселенным как любовь. Поэтому во всех своих личных отношениях с личностями\hyp{}созданиями вселенных Первоисточник и Центр всегда и последовательно выступает как любящий Отец. Бог --- это Отец в самом высоком смысле слова. Им вечно движет совершенный идеализм божественной любви, и эта нежная натура находит самое полное выражение и величайшее удовлетворение в том, чтобы любить и быть любимым.
\vs p004 4:7 \pc В науке Бог --- это Первопричина; в религии --- вселенский и любящий Отец; в философии --- единственный, кто существует сам по себе, независимо ни от кого другого, но милосердно дарует реальность существования всем вещам и всем другим созданиям. Однако необходимо откровение, чтобы показать, что Первопричина в науке и самостоятельно существующее Единство в философии --- это и есть Бог религии, исполненный милосердия и благости и пообещавший обеспечить вечное существование своим земным чадам.
\vs p004 4:8 Мы жаждем представления о Бесконечном, но почитаем идею\hyp{}знание о Боге, нашу способность везде и всегда постигать личность и божественные особенности нашей наивысшей идеи Божества.
\vs p004 4:9 Сознание торжества человеческой жизни на земле рождается из той веры творения, которая, столкнувшись с ужасной картиной человеческих недостатков, осмеливается всякий раз противостоять любому обстоятельству, неизменно заявляя: Даже если я не могу сделать этого, во мне живет тот, кто может и кто сделает это, часть Отца Абсолюта вселенной вселенных. И это «победа, которая побеждает мир, а именно --- ваша вера.»
\usection{5. Ошибочные представления о Боге}
\vs p004 5:1 Религиозная традиция --- это несовершенно сохраненная запись опыта знающих Бога людей прошлых эпох, но такие записи не заслуживают доверия в качестве руководства для религиозной жизни или источника подлинных сведений об Отце Всего Сущего. Такие древние верования изначально оказывались измененными в силу того, что примитивный человек был мифотворцем.
\vs p004 5:2 Один из величайших источников путаницы на Урантии относительно природы Бога проистекает из того, что в ваших священных книгах не удалось ясно установить различие между личностями Райской Троицы и между Райским Божеством и творцами и руководителями локальной вселенной. На протяжении прошлых диспенсаций частичного понимания ваши священники и пророки не сумели провести ясное разграничение между Планетарными Принцами, Владыками Систем, Отцами Созвездий, Сынами\hyp{}Творцами, Правителями Сверхвселенных, Верховным Существом и Отцом Всего Сущего. Многие сообщения от подчиненных личностей, таких как Носители Жизни и разные другие ангельские чины, представлены в ваших записях как исходящие от самого Бога. Религиозная мысль Урантии до сих пор путает личности, связанные с Божеством, с самим Отцом Всего Сущего, так что все они объединены под одним именем.
\vs p004 5:3 \pc Люди Урантии продолжают страдать от влияния примитивных представлений о Боге. Боги, которые буйствуют во время бури; которые в ярости сотрясают землю и в гневе сражают людей; которые выносят приговоры и выражают неудовольствие во времена голода и наводнений, --- это боги примитивной религии; это не те Боги, которые живут и правят вселенными. Такие представления --- пережиток тех времен, когда люди полагали, что прихоти таких воображаемых богов руководят вселенной и властвуют над ней. Но смертный человек начинает понимать, что он живет в сфере действия относительной законности и порядка --- в той степени, в какой они связаны с административной политикой и руководством Верховных Творцов и Верховных Контролеров.
\vs p004 5:4 \pc Варварская идея умиротворения разгневанного Бога, умилостивления оскорбленного Господа, обретения благосклонности Божества с помощью жертвоприношений и искупления и даже пролития крови олицетворяет религию совершенно незрелую и примитивную, философию, не достойную просвещенной эпохи науки и истины. Такие верования крайне отвратительны для небесных существ и божественных правителей, которые служат и царствуют во вселенных. Для Бога --- оскорбление, если верят, полагают или учат, что для обретения его благосклонности или отвращения воображаемого божественного гнева должна быть пролита невинная кровь.
\vs p004 5:5 Евреи верили, что «без пролития крови не может быть отпущения греха». Они не смогли избавиться от старой и языческой идеи, что Бога нельзя умиротворить иначе, как видом крови, хотя Моисей и сделал явный шаг вперед, запретив человеческие жертвоприношения и заменив их для своих примитивных, по\hyp{}детски наивных последователей\hyp{}кочевников церемониальным жертвоприношением животных.
\vs p004 5:6 Пришествие Райского Сына в ваш мир было неотъемлемой частью ситуации завершения планетарного периода, оно было неизбежным и не вызвано необходимостью завоевать расположение Бога. Это пришествие оказалось также последним личным действием Сына\hyp{}Творца на долгом пути к тому, чтобы заслужить путем собственного опыта владычество над своей вселенной. Какое искажение бесконечной сущности Бога! Это учение о том, что его отеческое сердце, пребывающее в суровой холодности и жестокости, настолько не трогали беды и печали его творений, что его чуткое сострадание не проявлялось до тех пор, пока он не увидел, как его невинный Сын истекает кровью и умирает на кресте Голгофы!
\vs p004 5:7 Но обитатели Урантии должны избавиться от этих древних заблуждений и языческих суеверий относительно природы Отца Всего Сущего. Откровение истины о Боге появляется, и человечеству предначертано узнать Отца Всего Сущего во всей той красоте проявления и прелести черт, которые так великолепно изобразил Сын\hyp{}Творец, живший на Урантии как Сын Человеческий и Сын Божий.
\vsetoff
\vs p004 5:8 [Представлено Божественным Советником Уверсы.]
