\upaper{191}{Явления апостолам и другим лидерам}
\author{Комиссия срединников}
\vs p191 0:1 День, в который произошло воскресение, был ужасным днем в жизни апостолов; десять из них большую часть дня провели в комнате наверху за закрытыми дверями. Апостолы могли бы бежать из Иерусалима, но боялись, что агенты синедриона арестуют их, как только они выйдут из дома в Беф\hyp{}Фаге. Фома в одиночестве размышлял о своих бедах. Для него было бы лучше, если бы он остался со своими собратьями\hyp{}апостолами, он мог бы помочь направить их дискуссии в более разумное русло.
\vs p191 0:2 Весь день Иоанн отстаивал идею о том, что Иисус воскрес из мертвых. Он перечислил не менее пяти различных случаев, когда Учитель утверждал, что воскреснет, и, по крайней мере, три случая, когда тот упоминал о третьем дне. Суждения Иоанна оказали на апостолов значительное влияние, в особенности на его брата Иакова и на Нафанаила. Если бы Иоанн не был самым младшим среди апостолов, он бы повлиял на них еще больше.
\vs p191 0:3 Изоляция апостолов имела прямое отношение к их бедам. Иоанн Марк сообщал им о развитии событий вокруг храма и передавал многочисленные слухи, проникавшие в город, но ему и в голову не пришло собирать новости, обращаясь к различным группам верующих, которым Иисус уже явился. Раньше службу такого рода осуществляли вестники Давида, но они все отсутствовали, выполняя свое последнее задание как вестники воскресения для тех верующих, которые жили далеко от Иерусалима. Впервые за все эти годы апостолы осознали, сколь многим они были обязаны вестникам Давида за ежедневно предоставлявшиеся ими сведения о делах царства.
\vs p191 0:4 Весь этот день Петр со свойственной ему эмоциональностью колебался между верой и сомнениями в воскресении Учителя. Петр не мог забыть, как выглядели погребальные одежды, оставшиеся в гробнице, как будто тело Иисуса просто испарилось из них. «Но, --- рассуждал Петр, --- если он воскрес и может являться женщинам, то почему не являет себя нам, своим апостолам?» Петр печалился, считая, что, возможно, Иисус не приходит к апостолам из\hyp{}за него, Петра, потому что той ночью на дворе Анны он от Иисуса отрекся. А потом ободрял себя известием, которое принесли женщины: «Идите и скажите моим апостолам --- и Петру». Но воодушевиться этим посланием означало, что он должен поверить, что женщины действительно видели и слышали воскресшего Учителя. Таким образом Петр целый день колебался между верой и сомнениями, пока в начале девятого не решился выйти на двор. Из\hyp{}за своего отречения от Учителя Петр решил покинуть апостолов, чтобы не мешать Иисусу прийти к ним.
\vs p191 0:5 Иаков Зеведеев сначала предлагал всем вместе пойти к гробнице и был решительно настроен что\hyp{}нибудь предпринять, чтобы разгадать тайну. Нафанаил же помешал им последовать совету Иакова, и не дал показаться на людях, просто напомнив им предостережение Иисуса не подвергать чрезмерной опасности свои жизни в это время. К полудню Иаков вместе с остальными апостолами настроился на бдительное ожидание. Он мало говорил и был страшно огорчен тем, что Иисус не являлся им; о многих явлениях Учителя к разным группам, и отдельным людям ему известно не было.
\vs p191 0:6 Андрей в этот день много слушал. Он был чрезвычайно озадачен возникшей ситуацией и мучился сомнениями, но по крайне мере испытывал некоторое облегчение, освободившись от ответственности за руководство собратьями\hyp{}апостолами. Он был действительно благодарен тому, что Учитель избавил его от бремени водительства прежде, чем для них наступили эти мрачные времена.
\vs p191 0:7 Не раз в долгие и томительные часы этого трагического дня единственной поддержкой для всех была постоянная помощь Нафанаила, выражавшаяся в характерных для него философских советах. На протяжении всего дня он оказывал существенное влияние на десятерых апостолов. И ни разу не высказался ни в отношении веры, ни в отношении неверия в воскресение Учителя. Однако с течением дня Нафанаил стал все больше склоняться к мысли, что Иисус исполнил свое обещание воскреснуть.
\vs p191 0:8 Симон Зилот был настолько подавлен, что не участвовал в дискуссиях. Большую часть времени он пролежал на тахте в углу комнаты, повернувшись лицом к стене, и за весь день не произнес и полудюжины фраз. Его представление о царстве рухнуло, и он не мог понять, что воскресение Учителя способно существенно изменить ситуацию. Его разочарование было столь личным и таким глубоким, что он не мог от него быстро оправиться даже столкнувшись с таким знаменательным фактом, как воскресение.
\vs p191 0:9 Как ни странно, но обычно неразговорчивый Филипп в этот день после полудня говорил много Почти все утро он, в основном, молчал, но после полудня стал все время задавать другим апостолам вопросы. Эти вопросы часто раздражали Петра, но другие апостолы выслушивали их доброжелательно. Филиппу особенно хотелось узнать: останутся ли на теле Иисуса физические раны от распятия, если он действительно воскрес из могилы.
\vs p191 0:10 Матфей был крайне смущен; он прислушивался к рассуждениям собратьев, но то и дело возвращался к мыслям о их финансовых проблемах, ожидающих их в будущем. Несмотря на предполагаемое воскресение Иисуса, Давид без церемоний передал казну ему, Матфею, ибо Иуды уже не было, как не было и авторитетного лидера. Но прежде чем Матфей собрался серьезно обдумать их доводы относительно воскресения, он лицом к лицу встретился с Учителем.
\vs p191 0:11 Близнецы Алфеевы почти не принимали участия в серьезных разговорах, а были заняты своими привычными обязанностями. Один из них выразил позицию обоих, когда в ответ на вопрос, заданный Филиппом, сказал: «Мы ничего не понимаем про воскресение, но наша мать говорит, что она разговаривала с Учителем, и мы ей верим».
\vs p191 0:12 Фома переживал один из характерных для него приступов отчаянной депрессии. Часть дня он спал, а остальное время бродил по горам. Фома испытывал потребность присоединиться к своим собратьям, но желание побыть одному было сильнее.
\vs p191 0:13 Учитель откладывал первое моронтийное явление апостолам по ряду причин. Во\hyp{}первых, он хотел, чтобы после того, как они услышат о его воскресении, у них было время подумать над тем, что он говорил им о своей смерти и воскресении, когда еще был с ними во плоти. Прежде, чем явить себя всем апостолам, Учитель хотел, чтобы Петр преодолел некие присущие ему недостатки. Во\hyp{}вторых, он желал, чтобы во время его первого явления Фома тоже был вместе со всеми. В это воскресенье ранним утром Иоанн Марк нашел Фому в Беф\hyp{}Фаге в доме Симона и около одиннадцати часов сообщил об этом апостолам. Фома тут же вернулся бы к ним в этот день, если бы Нафанаил или два других апостола пришли за ним. Фома действительно хотел вернуться, но был слишком горд, чтобы столь быстро сделать это самому после того, как ушел от них так, как в прошлый вечер. К следующему дню он впал в такую депрессию, что потребовалась почти неделя, чтобы он решился возвратиться. Апостолы ждали его, а он ждал, когда его братья разыщут его и попросят вернуться к ним. Таким образом, Фомы не было среди своих товарищей до вечера следующей субботы, когда после наступления темноты Петр и Иоанн пришли в Беф\hyp{}Фагу и привели его обратно. По этой же причине апостолы не пошли в Галилею сразу после того, как Иисус явился им в первый раз; без Фомы они идти не хотели.
\usection{1. Явление Петру}
\vs p191 1:1 В этот воскресный вечер около половины девятого часа Иисус явился Симону Петру в саду у дома Марков. Это было его восьмое моронтийное явление. После своего отречения от Учителя Петр жил под постоянным тяжким бременем сомнений и вины. Всю субботу и это воскресенье он боролся со страхом, что, возможно, он уже не апостол. Судьба Иуды приводила его в содрогание, и он даже думал, что он тоже предал своего Учителя. Все время после полудня он думал, что, быть может, его присутствие среди апостолов и не дает Иисусу явиться им --- при условии, конечно, что тот действительно воскрес из мертвых. И именно Петру, который пребывал в таком настроении и таком душевном состоянии, явился Иисус, когда удрученный апостол бродил в цветущем кустарнике.
\vs p191 1:2 Петр вспоминал, как приветливо посмотрел на него Учитель, когда проходил мимо него на крыльце Анны, и мысленно возвращался к тому чудесному посланию, что рано утром принесли ему женщины, пришедшие от пустой гробницы: «Пойдите и скажите моим апостолам --- и Петру». Когда он размышлял об этих знаках прощения, его вера начала брать верх над сомнениями, он остановился и, сжимая кулаки, громко сказал: «Я верю, что он воскрес из мертвых; я пойду и скажу об этом моим братьям». Как только он произнес это, перед ним внезапно возникла фигура человека, который со знакомыми интонациями в голосе заговорил с ним и сказал: «Петр, враг хотел завладеть тобой, но я тебя не отдал. Я знал, что не от сердца было твое отречение от меня; поэтому я простил тебя еще до того, как ты меня попросил; однако теперь ты должен перестать думать о себе и бедах часа сего и приготовиться нести благую весть евангелия тем, кто пребывает во тьме. Ты больше не должен заботиться о том, что можешь получить от царства, но заниматься тем, что можешь дать живущим в страшной духовной нищете. Симон, приготовься к битве нового дня, к борьбе с духовной тьмой и вредными сомнениями, свойственными человеческому разуму».
\vs p191 1:3 Петр и Иисус, в моронтийном состоянии, почти пять минут гуляли по саду, беседуя о прошлом, настоящем и будущем. Затем сказав: «Прощай, Петр, до того, как увижусь с тобой и собратьями твоими», Учитель стал невидим.
\vs p191 1:4 На мгновение Петра ошеломило сознание того, что он говорил с воскресшим Учителем и он может быть уверен в том, что по\hyp{}прежнему является посланцем царства. Он только что слышал, как увенчанный славой Учитель призвал его продолжать проповедь евангелия. И со всем, что переполняло его сердце, вбежал наверх в комнату, где находились апостолы, и с волнением, от которого перехватывало дыхание, воскликнул: «Я видел Учителя; он был в саду. Я говорил с ним, и он меня простил».
\vs p191 1:5 Слова Петра о том, что он видел Иисуса в саду, произвели на апостолов глубокое впечатление, они уже были готовы отказаться от своих сомнений, но Андрей встал и предостерег их от того, чтобы поддаваться чрезмерному влиянию рассказа его брата. Андрей упомянул, что и прежде Петр видел то, чего на самом деле не было. Хотя Андрей и не намекал прямо на видение, имевшее место ночью на Галилейском море, когда Петр утверждал, что видел, как Учитель по воде шел к ним, он сказал достаточно, чтобы было ясно, что имеет в виду именно этот случай. Симон Петр был очень обижен намеками своего брата и тотчас погрузился в удрученное молчание. Близнецам стало очень жаль Петра, и они оба подошли к нему, чтобы выразить свое сочувствие и сказать, что они верят ему, и еще раз подтвердить, что их собственная мать тоже видела Учителя.
\usection{2. Первое явление апостолам}
\vs p191 2:1 В тот вечер сразу после девяти часов отбыли Клеопа и Иаков, а десять апостолов в комнате наверху из боязни ареста заперли двери, и в то время, как близнецы Алфеевы утешали Петра, а Нафанаил увещевал Андрея, неожиданно явился им Учитель в моронтийном облике и сказал: «Мир вам. Почему вы так испугались при моем появлении, как будто увидели духа? Разве не говорил я вам об этом, когда был с вами во плоти? Разве не сказал вам, что первосвященники и правители отдадут меня на смерть, что один из вашего числа предаст меня и что на третий день я воскресну? Чем же вызваны ваши сомнения и все это обсуждение рассказов женщин, Клеопы и Иакова и даже Петра? До каких пор будете сомневаться в моих словах и отказываться верить моим обещаниям? И поверите ли теперь, когда действительно видите меня? Даже сейчас один из вас отсутствует. Когда еще раз соберетесь вместе и после того, как все вы будете уверены, что Сын Человеческий воскрес из могилы, тогда идите в Галилею. Имейте веру в Бога; имейте веру друг в друга; и так войдете в новое служение царства небесного. Я останусь с вами в Иерусалиме, пока вы не подготовитесь идти в Галилею. Мир мой оставляю вам».
\vs p191 2:2 Кончив говорить с ними, моронтийный Иисус мгновенно стал для них невидим. Они же все пали ниц, хваля Бога и благоговея перед исчезнувшим Учителем. Это было девятое моронтийное явление Учителя.
\usection{3. С моронтийными существами}
\vs p191 3:1 Следующий день, понедельник, был целиком проведен с моронтийными существами, которые тогда присутствовали на Урантии. Для участия в опыте по переходу Учителя через моронтийные состояния на Урантию прибыло более миллиона моронтийных руководителей и сподвижников вместе со смертными разных чинов из семи миров\hyp{}обителей Сатании, совершавшими такой же переход. Перешедший в моронтийное состояние Иисус провел с этими великолепными разумными существами сорок дней. Он наставлял их и узнавал от их руководителей о моронтийном пути, совершаемом смертными обитаемых миров Сатании по мере прохождения ими моронтийных сфер системы.
\vs p191 3:2 Около полуночи в этот понедельник моронтийное тело Учителя было подготовлено к переходу на вторую ступень моронтийного восхождения. Когда Иисус явился своим смертным детям на земле в следующий раз, то совершил это как моронтийное существо второй ступени. По мере продвижения Учителя по моронтийному пути моронтийным разумным существам и их преобразующим сподвижникам технически становилось все труднее и труднее делать Учителя видимым для материальных глаз смертных.
\vs p191 3:3 Переход на третью ступень моронтии Иисус совершил в пятницу 14 апреля; на четвертую ступень --- в понедельник 17 апреля; на пятую --- в субботу 22 апреля; на шестую --- в четверг 27 апреля, на седьмую --- во вторник 2 мая; в воскресенье 7 мая стал гражданином Иерусема и в воскресенье 14 мая был заключен в объятия Всевышнего Эдентии.
\vs p191 3:4 Таким образом Михаил из Небадона завершил свое вселенское служение, поскольку, включая его предыдущие пришествия, когда он пребывал в центрах созвездия, а потом служил в центрах сверхвселенной, уже всецело познал жизнь идущих по пути восхождения смертных, живущих во времени и пространстве. Именно этими моронтийными переживаниями Сын\hyp{}Творец Небадона действительно завершил и достойно увенчал седьмое и последнее вселенское пришествие.
\usection{4. Десятое явление (в Филадельфии)}
\vs p191 4:1 Десятое моронтийное явление Иисуса смертным было во вторник 11 апреля в начале девятого в Филадельфии, где он явил себя Авениру, Лазарю и приблизительно ста пятидесяти их сподвижникам, среди которых было не менее пятидесяти членов евангельского отряда, состоявшего из семидесяти человек. Это явление произошло сразу после открытия особого собрания в синагоге, созванного Авениром, чтобы обсудить распятие Иисуса и поступившее позднее сообщение о воскресении, принесенное вестником Давида. Так как воскрешенный Лазарь теперь был членом этой группы верующих, сообщению о том, что Иисус воскрес из мертвых, им было поверить не трудно.
\vs p191 4:2 Авенир и Лазарь, вместе стоявшие на кафедре, как раз открывали собрание в синагоге, когда все верующие увидели внезапно появившуюся фигуру Учителя. Он сделал шаг вперед с места, где появился между Авениром и Лазарем, ни один из которых не видел его, и, приветствуя собравшихся, сказал:
\vs p191 4:3 \pc «Мир вам. Вы все знаете, что у нас есть единый Отец на небе и что нет другого евангелия кроме евангелия, царства --- благой вести о даре вечной жизни, который люди получают благодаря вере. Радуясь своей верности евангелию, молите Отца истины послать в сердца ваши новую и большую любовь к братьям вашим. Вы должны любить всех людей, как любил вас я; вы должны служить всем людям, как я вам служил. С понимающим сочувствием и братской любовью принимайте в братство всех посвятивших себя провозглашению благой вести, кем бы они ни были, евреями или неевреями, греками или римлянами, персами или эфиопами. Иоанн предвозвестил царство; вы проповедовали евангелие в силе; греки уже учат благой вести; и я вскоре пошлю Дух Истины в души всех моих братьев, столь бескорыстно посвятивших свои жизни просвещению своих собратьев, пребывающих в духовной тьме. Вы все --- дети света; поэтому не попадайтесь в тенета, созданные непониманием и порожденные человеческой подозрительностью и нетерпимостью. Если вы одухотворены благодатью веры, дабы любить неверующих, то не должны ли вы в равной степени любить и своих верующих собратьев в разрастающейся семье веры. Помните: по тому, как вы любите друг друга, все люди узнают, что вы мои ученики.
\vs p191 4:4 Идите в мир, возвещая это евангелие об отцовстве Бога и братстве людей всем народам и расам, и всегда будьте мудры выбирая путь передачи благой вести различным расам и племенам человечества. Даром получили это евангелие царства, даром отдавайте благую весть всем народам. Не бойтесь противоборства зла, ибо я с вами всегда, до скончания века. Мир мой оставляю вам».
\vs p191 4:5 \pc И сказав: «Мир мой оставляю вам», стал невидим. Если исключить одно из явлений в Галилее, где Иисуса одновременно видело свыше пятисот верующих, эта группа смертных в Филадельфии была самая многочисленная из тех, кто видел его в одно и то же время.
\vs p191 4:6 Ранним утром следующего дня, пока апостолы еще были в Иерусалиме, ожидая, когда Фома придет в себя, эти верующие Филадельфии пошли, возвещая, что Иисус из Назарета воскрес из мертвых.
\vs p191 4:7 Следующий день, среду, Иисус не отвлекаясь провел в обществе своих моронтийных сподвижников и в течение нескольких послеполуденных часов принимал посещавшие его делегации из миров\hyp{}обителей каждой локальной системы обитаемых сфер всего созвездия Норлатиадека. И все приходившие были рады возможности встретиться со своим Творцом как с одним из принадлежащих к их собственному чину разумных существ вселенной.
\usection{5. Второе явление апостолам}
\vs p191 5:1 Фома провел в одиночестве целую неделю в горах, расположенных вокруг Елеонской горы. В это время он видел лишь тех, кто был в доме Симона, и Иоанна Марка. В субботу 15 апреля около девяти часов два апостола нашли его и отвели на место встречи в дом Марка. На следующий день Фома слушал истории о разных явлениях Учителя, но упорно отказывался верить. Он утверждал, что Петр ввел их в состояние крайнего возбуждения и заставил думать, будто они видели Учителя. Нафанаил попытался урезонить его, но тщетно. Это было упрямство, вызванное нервным напряжением, и в то же время обычной для него недоверчивостью, и это настроение, вкупе с досадой на то, что он от них убежал, вместе создавали ощущение оторванности, которое до конца не понимал даже сам Фома. Он уединился от своих собратьев, он пошел своим путем и теперь, когда опять оказался среди них, невольно стремился не соглашаться с остальными. Он уступал неохотно и не любил сдаваться. Сам того не ведая, он действительно наслаждался уделявшимся ему вниманием и невольно получал удовольствие от старания всех своих собратьев убедить и обратить его. Целую неделю он скучал по ним и теперь ему очень нравились их настойчивые заботы.
\vs p191 5:2 Чуть позже шести часов они вкушали от вечерней трапезы, во время которой Петр сидел с одной стороны от Фомы, а Нафанаил --- с другой, и сомневающийся апостол вдруг сказал: «Я не поверю, пока своими глазами не увижу Учителя и не вложу перста моего в раны от гвоздей». Когда они так ужинали при крепко закрытых и запертых дверях, Учитель в моронтийном состоянии неожиданно появился в центре изгиба стола, прямо напротив Фомы и сказал:
\vs p191 5:3 «Мир вам. Я задержался на целую неделю, дабы снова явиться вам, когда вы будете все вместе, чтобы еще раз услышать поручение идти в мир и всюду проповедовать это евангелие царства. Я вновь говорю вам: Как Отец послал меня в мир, так и я посылаю вас. Как я открывал Отца, так и вы будете открывать божественную любовь, открывать не просто словами, но своей повседневной жизнью. Я посылаю вас любить не души людей, но \bibemph{любить людей.} Вы должны не просто возвещать радости небесные, но и своим повседневным поведением являть сии духовные реалии божественной жизни, ибо вы через веру уже имеете жизнь вечную как дар Бога. Когда у вас есть вера, когда сила небесная, Дух Истины снизошел на вас, вы не станете прятать свет здесь, за закрытыми дверями, но откроете любовь и милосердие Бога всему человечеству. Сейчас от страха вы бежите событий, переживание которых вам неприятно, но, приняв крещение Духа Истины, смело и радостно пойдете навстречу новому опыту, провозглашая благую весть о вечной жизни в царстве Бога. Вы можете ненадолго задержаться здесь и в Галилее, пока не придете в себя от потрясения, вызванного переходом от ложной уверенности, опирающейся на приверженности традициям, к новому порядку, который создает уверенность, построенная на фактах, истине и вере в верховные реальности живого опыта. Миссия, с которой вы отправитесь в мир, основана на том, что я жил среди вас раскрывающей Бога жизнью; на истине, что вы и все остальные люди --- сыновья Бога; и миссия эта будет заключаться в жизни, которой вы будете жить среди людей, --- в действительном и живом опыте любви к людям и служении им, как любил вас и служил вам я. Пусть вера явит миру свет ваш; пусть откровение истины откроет глаза, ослепленные традицией; пусть ваше служение любви до основания разрушит предрассудки, порожденные невежеством. И так понимающим сочувствием и бескорыстной преданностью сблизившись со своими собратьями\hyp{}людьми, вы поведете их к спасительному познанию любви Отца. Евреи превознесли добродетель; греки возвысили красоту; индусы проповедуют преданность; далекие аскеты учат благоговению; римляне требуют верности; я же от моих учеников требую жизни --- жизни, отданной служению любви вашим братьям во плоти».
\vs p191 5:4 Кончив говорить, Учитель посмотрел Фоме в лицо и сказал: «А ты, Фома, говоривший, что не поверишь, пока не увидишь меня и не вложишь перст свой в раны от гвоздей на руках моих, ныне увидел меня и услышал мои слова; и хоть ты не видишь ран от гвоздей на моих руках, так как я воскрес в теле, которое вы тоже будете иметь, когда покинете этот мир, что сейчас скажешь ты своим братьям? Ты признаешь истину, ибо в сердце своем ты начал верить уже тогда, когда столь решительно заявлял о своем неверии. Твои сомнения, Фома, всегда упрямо заявляют о себе как раз перед тем, как разрушиться. Фома, я повелеваю тебе быть не неверующим, но верующим и знаю, что ты будешь верить, верить всем сердцем».
\vs p191 5:5 Услышав эти слова, Фома упал на колени перед моронтийным Учителем и воскликнул: «Верю! Господь мой и Учитель мой!» Тогда Иисус сказал Фоме: «Фома, ты поверил, потому что, действительно, видел и слышал меня. Блаженны те, кто в грядущие века уверует, не видя глазами плоти и не слыша смертным ухом».
\vs p191 5:6 Затем, когда фигура Учителя приблизилась к главе стола, он обратился ко всем апостолам и сказал: «А теперь идите все в Галилею, где я вскоре явлюсь вам». И, сказав это, стал невидим.
\vs p191 5:7 \pc Теперь одиннадцать апостолов окончательно убедились, что Иисус воскрес из мертвых, и на следующий день рано утром, еще до рассвета, отправились в Галилею.
\usection{6. Явление в Александрии}
\vs p191 6:1 Пока одиннадцать апостолов приближались к концу своего пути --- к Галилее, 18 апреля во вторник вечером около половины девятого Иисус явился Родану и почти восьмидесяти другим верующим в Александрии. Это было двенадцатое явление Учителя в моронтийном облике. Иисус явился этим грекам и евреям в момент, когда вестник Давида заканчивал рассказ о распятии. Вестник был пятым бегуном в эстафете между Иерусалимом и Александрией, он прибежал в Александрию спустя несколько часов пополудни; когда же он доставил свое послание Родану, решено было собрать верующих, чтобы те услышали это трагическое сообщение от самого вестника. Около восьми часов вестник по имени Натан из Бусириса предстал перед собравшимися и подробно рассказал все, что сказал ему предыдущий бегун. Свой трогательный рассказ Натан закончил такими словами: «Однако Давид, посылающий нам это известие, сообщает, что, предсказывая свою смерть, Учитель объявил, что он воскреснет». Когда Натан еще говорил, моронтийный Учитель явился там видимый всеми. И когда Натан сел, Иисус сказал:
\vs p191 6:2 «Мир вам. То, что, посылая в мир, поручил мне основать Отец мой, не принадлежит ни расе, ни нации, ни особой группе учителей и проповедников. Евангелие царства принадлежит и еврею, и нееврею; богатым и бедным; свободным и рабам; мужчинам и женщинам; даже детям малым. Вы все должны возвещать это евангелие любви и истины жизнями, которыми живете во плоти. Любите друг друга новой и удивительной любовью, как я любил вас. Служите человечеству с новой и поразительной преданностью, как я вам служил. Когда же люди увидят, что вы их так любите, и узрят, что вы столь ревностно служите им, то поймут, что вы стали верующими причастниками царства небесного, и последуют за Духом Истины, который увидят в жизнях ваших, и найдут вечное спасение.
\vs p191 6:3 Как Отец послал меня в сей мир, так и я ныне посылаю вас. Вы все призваны нести благую весть тем, кто пребывает во тьме. Евангелие царства принадлежит всем верующим в него и не будет отдано в ведение простых священников. Вскоре Дух Истины снизойдет на вас и наставит вас во всякой истине. Идите же в мир и всюду проповедуйте это евангелие и, вот, я с вами всегда, даже до скончания века».
\vs p191 6:4 Сказав это, учитель стал невидим. Всю ту ночь эти верующие не расходились, вспоминая о пережитом ими как верующими царства и слушая пространные речи Родана и его соратников. И все они поверили, что Иисус воскрес из мертвых. Вообразите же удивление посланного Давидом вестника воскресения, прибывшего на второй день после этого, когда они в ответ на его сообщение сказали: «Да, мы знаем, ибо видели его. Он явился нам позавчера».
