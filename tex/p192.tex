\upaper{192}{Явления в Галилее}
\author{Комиссия срединников}
\vs p192 0:1 К тому моменту, как апостолы, покинув Иерусалим, ушли в Галилею, еврейские управители заметно успокоились. Поскольку Иисус являлся только своей семье верующих в царство, а апостолы прятались и не проповедовали публично, правители евреев посчитали, что евангельское движение все\hyp{}таки действительно подавлено. Конечно, их смущало усиливающееся распространение слухов о том, что Иисус воскрес из мертвых, но они рассчитывали, что подкупленные стражи будут успешно противодействовать всем таким рассказам, повторяя историю о том, как несколько последователей Иисуса унесли его тело.
\vs p192 0:2 С этого времени и до тех пор, пока поднимавшаяся волна преследований не разметала апостолов в разные стороны, общепризнанным главой апостольского отряда был Петр. Подобных полномочий Иисус ему никогда не давал, а его собратья\hyp{}апостолы никогда формально не избирали его на столь ответственный пост; конечно, Петр принял его и удерживал за собой с общего согласия, и еще потому, что был их главным проповедником. Отныне публичная проповедь стала основным делом апостолов. После их возвращения из Галилеи их казначеем стал Матиас, которого они избрали на место Иуды.
\vs p192 0:3 В течение недели, проведенной ими в Иерусалиме, Мария, мать Иисуса, большую часть времени провела с верующими женщинами, остановившимися в доме Иосифа Аримафейского.
\vs p192 0:4 Рано утром в понедельник, когда апостолы отправились в Галилею, Иоанн Марк тоже пошел с ними. Он следовал за апостолами до тех пор, пока те не вышли за пределы города и не миновали Вифанию, и тогда открыто присоединился к ним, чувствуя уверенность, что они не отправят его обратно.
\vs p192 0:5 По пути в Галилею апостолы несколько раз останавливались, чтобы рассказать историю о своем воскресшем Учителе, поэтому пришли в Вифсаиду лишь поздней ночью в среду. В четверг же, когда они проснулись и приготовились завтракать, был уже полдень.
\usection{1. Явление у озера}
\vs p192 1:1 В пятницу 21 апреля около шести часов утра Учитель в моронтийном облике совершил свое тринадцатое, а в Галилее первое явление десяти апостолам, когда их лодка приблизилась к берегу рядом с местом, где обычно причаливали в Вифсаиде.
\vs p192 1:2 После того, как апостолы в доме Зеведея провели в ожидании послеполуденные часы и начало вечера четверга, Симон Петр предложил им порыбачить. Когда же Петр предложил отправиться ловить рыбу, все апостолы решили присоединиться к нему. Всю ночь они промучались с сетями, но ничего не поймали. Оставшись без улова, они не печалились, ибо были переполнены интересными впечатлениями, которые нужно было обсудить, например, все то, что столь недавно произошло в Иерусалиме. Однако на рассвете они решили вернуться в Вифсаиду. Приблизившись к берегу, апостолы увидели, что у лодочного причала кто\hyp{}то стоит у костра. Сначала они подумали, что это Иоанн Марк, который пришел встречать их с уловом, но, подплыв к берегу еще ближе, поняли, что ошиблись --- человек был намного выше Иоанна. Никто из них даже не догадался, что человек на берегу --- это Учитель. Они не совсем понимали, почему Иисус хотел встретиться с ними в местах, где началось их общение, на открытом воздухе на лоне природы, вдали от замкнутого пространства Иерусалима и связанных с ним трагических ощущений страха, предательства и смерти. Он сказал им, что если они пойдут в Галилею, то он встретит их там, и готовился выполнить это обещание.
\vs p192 1:3 Когда апостолы бросили якорь и приготовились пересесть в маленькую лодку, чтобы сойти на берег, человек на берегу обратился к ним: «Друзья, поймали ли вы что\hyp{}нибудь?» И когда те ответили: «Нет», заговорил снова. «Закиньте сети по правую сторону лодки и поймаете рыбу». Хотя апостолы не знали, что руководит ими Иисус, дружно забросили сеть, как им велели, и сеть тотчас наполнилась, так что они едва смогли ее вытянуть. Иоанн Зеведеев был весьма сообразительный и, увидев наполненную сеть, понял, что с ними говорит Учитель. Когда мысль эта пришла ему в голову, он, наклонившись, прошептал Петру: «Это Учитель». Петр всегда был человеком импульсивным и пылким; поэтому, когда Иоанн прошептал ему это, он стремительно встал и бросился в воду, чтобы быстрее доплыть до Учителя. Его сотоварищи подоспели следом за ним, причалив к берегу в небольшой лодке и таща за собой сеть с рыбой.
\vs p192 1:4 Иоанн Марк уже не спал и, видя, что апостолы сходят на берег с наполненной сетью, побежал вниз по берегу поздороваться с ними; увидев же одиннадцать человек вместо десяти, он предположил, что неизвестный --- это воскресший Иисус, и в то время, как десять изумленных апостолов стояли молча, юноша бросился к Учителю и, упав на колени к его ногам, сказал: «Господь мой и Учитель мой». Тогда Иисус заговорил, но не так, как в Иерусалиме, когда приветствовал их словами: «Мир вам», но обратился к Иоанну Марку в обычном тоне: «Ну что ж, Иоанн, я рад видеть тебя снова в беспечальной Галилее, где мы можем хорошо поговорить. Оставайся и завтракай с нами, Иоанн».
\vs p192 1:5 Когда Иисус беседовал с молодым человеком, десять апостолов пребывали в таком изумлении и потрясении, что забыли вытянуть на берег сеть с рыбой. Тогда Иисус сказал: «Принесите вашу рыбу и приготовьте на завтрак. У нас уже разложен огонь и есть много хлеба».
\vs p192 1:6 Когда Иоанн Марк поклонился Учителю, Петр при виде углей, тлеющих на берегу, был на мгновение потрясен; эта картина так живо напомнила ему о костре в ту полночь на дворе Анны, когда он отрекся от Учителя; однако Петр собрался и, упав на колени у ног Учителя, воскликнул: «Господь мой и Учитель мой!»
\vs p192 1:7 Затем Петр присоединился к своим товарищам, вытягивающим сеть. Вытащив же свой улов, апостолы его сосчитали, и оказалось, что было 153 большие рыбы. И вновь была допущена ошибка, ибо это событие было названо еще одним чудесным уловом рыбы. На самом деле никакого чуда в нем не было. Это было всего лишь проявлением предвидения Учителя. Иисус знал, где была рыба и указал апостолам, куда забросить сеть.
\vs p192 1:8 Обращаясь к ним, Иисус сказал: «Теперь придите все и завтракайте. Даже близнецы, и те должны сесть, пока я с вами; Иоанн Марк приготовит рыбу». Иоанн Марк принес семь крупных рыб, которых Учитель положил на огонь, и когда те испеклись, юноша подал их десяти апостолам. Затем Иисус преломил хлеб и дал его Иоанну, а тот, в свою очередь, передал его голодным апостолам. Когда же все они получили пищу, Иисус велел Иоанну Марку сесть, и сам подал юноше рыбу и хлеб. Пока же они ели, Иисус беседовал с ними и говорил о многом случившемся с ними в Галилее и у этого озера.
\vs p192 1:9 \pc Это был третий раз, когда Иисус явил себя сразу всем апостолам. Вначале, когда Иисус обратился к ним и спросил, есть ли у них рыба, они не подозревали, кто он, потому что для рыбаков с Галилейского моря было обычным делом, если к ним, когда они высаживались на берег, вот так же обращались торговцы рыбой из Тарихеи, которые, как правило, были готовы купить свежий улов для тех, кто рыбу вялил.
\vs p192 1:10 \pc Иисус общался с десятью апостолами и Иоанном Марком более часа, а потом, прогуливаясь вдоль берега, по очереди говорил уже только с двумя --- но каждая из пар была уже не та, что вначале он посылал вместе учить. Все одиннадцать апостолов вместе пришли из Иерусалима, но с приближением к Галилее Симон Зилот все больше и больше впадал в уныние, так что, когда они прибыли в Вифсаиду, оставил своих собратьев и вернулся в свой дом.
\vs p192 1:11 Этим утром, перед тем, как проститься с ними, Иисус велел, чтобы двое из апостолов вызвались пойти к Симону Зилоту и привели его обратно в тот же день. И Петр с Андреем сделали это.
\usection{2. Общение с парами апостолов}
\vs p192 2:1 Когда все позавтракали, Иисус подозвал Петра и Иоанна, показав им, что они должны прогуляться с ним вдоль берега, пока остальные оставались у костра. И когда они пошли с ним, Иисус спросил Иоанна: «Любишь ли ты меня, Иоанн?» Когда же Иоанн ответил: «Да, Учитель, всем сердцем моим», Учитель сказал: «Тогда, Иоанн, избавься от своей нетерпимости и научись любить людей, как я любил вас. Посвяти свою жизнь тому, чтобы доказать, что любовь --- самая великая вещь на свете. Именно любовь Бога побуждает людей искать спасение. Любовь есть источник всякой духовной добродетели, она --- сущность истины и красоты».
\vs p192 2:2 Затем Иисус обратился к Петру и спросил: «Любишь ли ты меня, Петр?» Петр ответил: «Господи, ты знаешь, что я люблю тебя всею душою». Тогда Иисус сказал: «Если любишь меня, Петр, паси агнцев моих. Не забывай служить слабым, бедным и молодым. Проповедуй евангелие без страха и беспристрастно; всегда помни, что Бог не взирает на лица. Служи людям, как я служил тебе; прощай смертных, как прощал тебя я. Пусть опыт научит тебя ценности размышления и силе рассудительности».
\vs p192 2:3 Когда же они отошли немного дальше, Учитель обратился к Петру и спросил: «Действительно ли ты любишь меня, Петр?» Тогда Симон сказал: «Да, Господи, ты знаешь, что я люблю тебя». И Иисус снова сказал: «Тогда хорошо заботься об овцах моих. Будь добрым и истинным пастырем стаду. Не предавай их веру в тебя. Не дай врагу застать тебя врасплох. Всегда будь начеку --- бодрствуй и молись».
\vs p192 2:4 Когда же они сделали еще несколько шагов, Иисус обратился к Петру и спросил в третий раз: «Истинно ли ты любишь меня, Петр?» И тогда Петр, несколько опечаленный кажущимся недоверием к нему Учителя, с большим чувством сказал: «Господи, ты все знаешь, и поэтому знаешь, что я действительно и истинно люблю тебя». Тогда Иисус сказал: «Паси овец моих. Не бросай стадо. Будь примером и вдохновением для всех своих собратьев\hyp{}пастырей. Люби стадо, как я любил тебя, и посвяти себя его благоденствию, как я посвятил свою жизнь твоему благоденствию. И следуй за мною до самого конца».
\vs p192 2:5 Петр воспринял последнее высказывание буквально --- что он должен продолжать следовать за Иисусом --- и, обращаясь к нему, показал на Иоанна и спросил: «Если я и дальше пойду за тобой, что делать этому человеку?» Тогда, видя, что Петр неправильно понял его слова, Иисус сказал: «Петр, не заботься о том, что делать твоим братьям. Если я захочу, чтобы Иоанн жил после того, как умрешь ты, даже до времени, когда я вернусь, что тебе до того? Заботься только о том, чтобы самому следовать за мной».
\vs p192 2:6 \pc И пронеслось это слово между собратьями и было истолковано как утверждение Иисуса о том, что Иоанн не умрет прежде, чем Учитель вернется, чтобы, как думали и надеялись многие, установить царство в силе и славе. И именно это толкование сказанного Иисусом во многом подвигло Симона Зилота возвратиться к служению и продолжить свои труды.
\vs p192 2:7 \pc Когда они вернулись к остальным, Иисус пошел гулять и беседовать уже с Андреем и Иаковом. Когда они отошли на небольшое расстояние, Иисус спросил Андрея: «Доверяешь ли ты мне, Андрей?» Услышав подобный вопрос Иисуса, бывший глава апостолов остановился и ответил: «Да, Учитель, я, несомненно, доверяю тебе, и ты знаешь об этом». Тогда Иисус сказал: «Если ты доверяешь мне, Андрей, доверяй больше и братьям твоим --- даже Петру. Однажды я доверил тебе руководство твоими братьями. Теперь же ты должен доверять другим, ибо я оставляю вас и иду к Отцу. Когда братья твои будут рассеяны вследствие суровых преследований, будь внимательным и мудрым советником Иакову, моему брату во плоти, когда на него наложат тяжкое бремя, к которому он будет совершенно не готов. И тогда продолжай доверять, ибо я тебя не подведу. Когда же твоя жизнь на земле закончится, ты придешь ко мне».
\vs p192 2:8 Затем, обращаясь к Иакову, Иисус спросил: «Доверяешь ли ты мне, Иаков?» И Иаков, конечно, ответил: «Да, Учитель, я доверяю тебе всем сердцем моим». Тогда Иисус сказал: «Если будешь больше доверять мне, Иаков, то будешь менее нетерпим к своим братьям. Если будешь доверять мне, это поможет тебе быть добрым к братству верующих. Научись рассчитывать последствия своих высказываний и поступков. Помни: что посеешь, то и пожнешь. Молись о спокойствии духа и воспитывай в себе терпение. Эти добродетели в сочетании с живой верой укрепят тебя, когда настанет час пить жертвенную чашу. Но никогда не тревожься; когда жизнь твоя на земле закончится, ты тоже придешь и будешь со мной».
\vs p192 2:9 \pc Затем Иисус беседовал с Фомой и Нафанаилом. Фоме он сказал: «Служишь ли ты мне, Фома?» Фома ответил: «Да, Господи, служу тебе сейчас и буду служить всегда». Тогда Иисус сказал: «Если хочешь служить мне, служи братьям моим во плоти, как я служил тебе. И не уставай свершать добро, но действуй упорно как посвященный Богом этому служению любви. Когда же закончишь служить со мной на земле, будешь служить со мной в славе. Фома, ты должен перестать сомневаться и возрасти в вере и в познании истины. Верь в Бога, как дитя, но перестань поступать так по\hyp{}детски. Будь смел; будь силен в вере и могуч в царстве Бога».
\vs p192 2:10 Затем Учитель сказал Нафанаилу: «Служишь ли ты мне, Нафанаил?» И апостол ответил: «Да, Учитель, со всею любовью». Тогда Иисус сказал: «Поэтому, если ты служишь мне всем сердцем, то с неустанной любовью посвящаешь себя благоденствию братьев моих на земле. В совете твоем должно быть больше дружелюбия, а в философии твоей --- больше любви. Служи людям, как я тебе служил. Оберегай людей, как и я ограждал тебя. Будь не столь критичен; ожидай меньшего от некоторых людей и тебя постигнет меньшее разочарование. Когда же труды твои здесь, на земле, завершатся, будешь служить со мной на небе».
\vs p192 2:11 \pc После этого Учитель беседовал с Матфеем и Филиппом. Филиппа он спросил: «Послушен ли ты мне, Филипп?» Филипп ответил: «Да, Господи, я буду послушен тебе всю свою жизнь». Тогда Иисус сказал: «Если хочешь быть послушным мне, иди в земли неевреев и возвещай евангелие. Пророки говорили вам, что послушание лучше жертвы. Благодаря вере ты стал знающим Бога сыном царства. Есть лишь один закон, которому должно следовать --- это заповедь идти, возвещая евангелие царства. Перестань бояться людей; не бойся проповедовать благую весть о вечной жизни своим собратьям, что томятся во тьме и алчут света истины. Филипп, ты более не будешь заниматься деньгами и обеспечением. Теперь ты свободен и можешь проповедовать благую весть так же, как твои братья, я же пойду перед тобою и буду с тобой до самого конца».
\vs p192 2:12 Затем, говоря с Матфеем, Учитель сказал: «По сердцу ли тебе, Матфей, следовать мне?» Матфей ответил: «Да, Господи, я целиком посвящаю себя исполнению твоей воли». Тогда Учитель сказал: «Матфей, если ты хочешь следовать мне, иди и учи все народы этому евангелию царства. Ты больше не будешь служить своим братьям, удовлетворяя потребные для жизни материальные нужды; впредь ты тоже должен возвещать благую весть духовного спасения. Отныне думай лишь об исполнении своей миссии --- проповедовать евангелие царства Отца. Как я исполнял волю Отца на земле, так и ты будешь исполнять божественное назначение. Помни, что и евреи, и неевреи --- твои братья. Возвещая спасительные истины евангелия царства небесного, никого не бойся. И куда я иду, туда вскоре придешь и ты».
\vs p192 2:13 \pc Затем Иисус гулял и беседовал с близнецами Алфеевыми, Иаковом и Иудой, и, обращаясь к ним обоим, спросил: «Верите ли вы в меня, Иаков и Иуда?» И когда оба ответили: «Да, Учитель, верим», он сказал: «Вскоре я покину вас. Вы видите, что я уже покинул вас во плоти. В облике же этом пробуду лишь короткое время перед тем, как идти к Отцу моему. Вы верите в меня --- вы мои апостолы, и будете ими всегда. Продолжайте верить и помнить ваше общение со мной, когда я уйду, и после того, как вы, быть может, вернетесь к труду, которым занимались прежде, чем вы ушли со мной. Никогда не позволяйте перемене в мирском труде вашем влиять на вашу преданность. Верьте в Бога до конца ваших дней на земле. Никогда не забывайте, что для верующего сына Бога всякий честный труд священен. Какое бы дело ни делал сын Бога, оно не может быть заурядным. Поэтому отныне делайте ваше дело, как делали бы его для Бога. Когда же ваша жизнь на земле закончится, у меня есть иные и лучшие миры, где вы будете так же трудиться для меня. И во всяких трудах ваших, и в этом мире и в иных мирах, я буду трудиться с вами, и дух мой пребудет в вас».
\vs p192 2:14 \pc Было около десяти часов, когда Иисус вернулся после беседы с близнецами Алфеевыми; покидая апостолов, он сказал: «Прощайте до полудня завтрашнего дня, когда я встречу всех вас на горе вашего посвящения». И сказав это, стал для них невидим.
\usection{3. На горе посвящения}
\vs p192 3:1 В полдень в субботу 22 апреля одиннадцать апостолов собрались на встречу, назначенную им на горе близ Капернаума, и Иисус явился им. Эта встреча произошла на той же горе, где Учитель отметил их как своих апостолов и как посланцев царства Отца на земле. И было это четырнадцатым моронтийным явлением Учителя.
\vs p192 3:2 На сей раз одиннадцать апостолов встали на колени вокруг Учителя и услышали, как он повторил напутствия, и увидели, как он вновь посвятил их так же, как тогда, когда они впервые были отмечены для особого труда царства. И, кроме молитвы Учителя, все это было для них как бы напоминанием об их предыдущем посвящении служению Отцу. Теперь, когда Учитель --- Иисус, перешедший в моронтийное состояние, --- молился, то делал это величественным голосом и со словами, исполненными такой силы, какой апостолы раньше никогда не слышали. Теперь их Учитель говорил с правителями вселенных как обладающий в своей вселенной всей силой и властью. И эти одиннадцать человек никогда не забывали этого опыта моронтийного, повторного, посвящения исполнению ранее данных обетов посланничества. Проведя на горе со своими посланцами только один час, Учитель ласково простился с ними и стал невидим.
\vs p192 3:3 \pc И никто не видел Иисуса целую неделю. Апостолы действительно не представляли, что делать, и не знали, ушел ли Учитель к Отцу. И в этом состоянии неопределенности оставались в Вифсаиде. Они не решались рыбачить, боясь, что Учитель придет к ним, а они его не застанут. А Иисус всю эту неделю провел с моронтийными созданиями на земле и был занят переходом через состояния моронтии, который он совершал на этой планете.
\usection{4. Встреча на берегу озера}
\vs p192 4:1 Весть о явлениях Иисуса распространялась по всей Галилее, и каждый день все больше верующих приходило в дом Зеведея, чтобы спросить о воскресении Учителя и узнать истину о происшедших явлениях его, о которых до них дошла молва. В начале недели Петр передал известие, что всем надо собраться на берегу моря в следующую субботу в три часа дня.
\vs p192 4:2 Поэтому в субботу 29 апреля в три часа более пятисот верующих из окрестностей Капернаума собралось в Вифсаиде, чтобы услышать первую после воскресения публичную проповедь Петра. Апостол был в ударе; и когда закончил свою потрясающую речь, лишь немногие из его слушателей все еще сомневались, что Учитель воскрес из мертвых.
\vs p192 4:3 Завершая свою проповедь, Петр сказал: «Мы утверждаем, что Иисус из Назарета не умер; мы заявляем, что он восстал из гробницы; мы возвещаем, что видели его и говорили с ним». Когда же он закончил эту декларацию веры, рядом с ними перед всеми собравшимися явился Учитель в моронтийном облике и хорошо знакомым голосом сказал: «Мир вам; мир мой оставляю вам». Явившись же так и сказав это, стал невидим. Это было пятнадцатое моронтийное явление воскресшего Учителя.
\vs p192 4:4 \pc От определенных слов, сказанных одиннадцати апостолам во время их общения с Учителем на горе посвящения, у них возникло впечатление, что Учитель вскоре публично явится группе верующих галилеян и что после того, как он совершит это, они должны будут вернуться в Иерусалим. Поэтому на следующее утро, в воскресенье 30 апреля одиннадцать апостолов покинули Вифсаиду и пошли в Иерусалим. Идя вниз по течению реки Иордан, они много учили и проповедовали, так что в дом Марков в Иерусалиме пришли только поздно вечером в среду 3 мая.
\vs p192 4:5 \pc Для Иоанна Марка это возвращение домой было печальным. Всего за несколько часов до прихода домой Иоанна его отец Илия Марк скоропостижно умер от кровоизлияния в мозг. Хотя мысль о несомненности воскрешения сильно утешала апостолов в их горе, они вместе с тем истинно скорбели об утрате своего доброго друга, который был их неколебимым сторонником даже в период великих бед и разочарований. Иоанн Марк сделал все, что мог, чтобы утешить свою мать, и, говоря от ее имени, предложил апостолам по\hyp{}прежнему останавливаться в ее доме. И до дня Пятидесятницы комната наверху стала пристанищем апостолов.
\vs p192 4:6 \pc Апостолы намеренно вошли в Иерусалим после наступления темноты, чтобы еврейские власти не увидели их. Не появлялись они на людях и на похоронах Илии Марка. И весь следующий день тихо оставались в уединении в комнате наверху --- свидетельнице стольких событий.
\vs p192 4:7 В этой комнате наверху в четверг ночью у апостолов произошла чудесная встреча, и все, кроме Фомы, Симона Зилота и близнецов Алфеевых, торжественно поклялись идти публично проповедовать новое евангелие о воскресшем Господе. Таким образом, первые шаги в замене евангелия Царства --- евангелия о сыновстве по отношению к Богу и братстве по отношению к человеку --- провозглашением воскресения Иисуса уже были сделаны. Нафанаил возражал против подобного изменения в сути их публичного учения, но не мог устоять перед красноречием Петра и не сумел преодолеть воодушевление учеников, особенно верующих женщин.
\vs p192 4:8 Итак, под энергичным руководством Петра и до вознесения Учителя к Отцу его действовавшие из лучших побуждений представители начали еле заметный процесс постепенного и конкретного преобразования \bibemph{религии Иисуса} в новую и измененную форму \bibemph{религии об Иисусе.}
