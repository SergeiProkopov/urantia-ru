\upaper{74}{Адам и Ева}
\author{Солония}
\vs p074 0:1 Адам и Ева прибыли на Урантию за 37\,848 лет до 1934 года н.э. Была середина лета, Сад находился в полном расцвете. В самый полдень, без всякого объявления, два серафима перемещения, сопровождаемые персоналом Иерусема, которому было поручено доставить реализаторов биологического подъема, тихо опустились на поверхность вращающейся планеты неподалеку от храма Отца Всего Сущего. Вся работа по рематериализации тел Адама и Евы была проведена на территории, прилегающей к этой только что созданной святыни. С момента их прибытия прошло десять дней, прежде чем они были воссозданы в двойственной человеческой форме для представления в качестве новых правителей мира. Они одновременно пришли в сознание. Материальные Сыны и Дочери всегда осуществляют свое служение вместе. Во все времена и во всех местах их служения существенным является то, что они никогда не разлучаются. Они задуманы для работы в паре; редко они выполняют свои функции поодиночке.
\usection{1. Адам и Ева в Иерусеме}
\vs p074 1:1 Урантийские Адам и Ева состояли в старшем отряде Материальных Сынов Иерусема под общим номером 14\,311. Они принадлежали к третьему физическому выпуску и были ростом немногим более восьми футов.
\vs p074 1:2 В то время, когда Адам был избран, чтобы отправиться на Урантию, он трудился вместе со своей супругой в контрольно\hyp{}испытательных физических лабораториях Иерусема. Более пятнадцати тысяч лет они работали директорами отдела экспериментальной энергии, применяемой в модификации живых форм. Задолго до этого они были учителями в школах гражданства для новоприбывших в Иерусем. Все это надо принимать во внимание, изучая их последующее поведение на Урантии.
\vs p074 1:3 Когда официально был объявлен призыв добровольцев для выполнения рискованного предприятия --- адамической миссии на Урантии, весь старший отряд Материальных Сынов и Дочерей изъявил желание стать добровольцами. С одобрения Ланафоржа и Всевышних Эдентии Мелхиседеки\hyp{}экзаменаторы выбрали, в конце концов, Адама и Еву, которые впоследствии и начали действовать как реализаторы биологического подъема на Урантии.
\vs p074 1:4 Во время восстания Люцифера Адам и Ева оставались верными Михаилу, однако эта пара была вызвана, чтобы предстать перед Владыкой Системы и всеми членами его кабинета, подвергнуться проверке и получить инструкции. Был предоставлен полный отчет обо всех событиях на Урантии; Адам и Ева были подробно проинструктированы относительно того, как они должны действовать, принимая на себя обязанности управлять такой раздираемой противоречиями планетой. Они были приведены к двойной присяге верности --- Всевышним Эдентии и Михаилу Спасоградскому. Им было настоятельно рекомендовано считать себя находящимися в подчинении урантийского отряда Мелхиседеков\hyp{}исполнителей до тех пор, пока этот руководящий орган не сочтет нужным передать им правление в мире их назначения.
\vs p074 1:5 \pc Эта иерусемская пара оставила в столице Сатании и в других местах сто детей --- пятьдесят сыновей и пятьдесят дочерей --- изумительных созданий, которые избежали опасностей свойственных развитию, и во время отъезда их родителей на Урантию всем им были доверены посты вселенского масштаба. И все они присутствовали в прекрасном храме Материальных Сынов, принимая участие в прощальных торжествах, посвященных последним церемониям, связанным с принятием миссии пришествия. Эти дети сопровождали своих родителей до центра дематериализации их чина и были последними, кто попрощался с ними и пожелал божественной скорости, когда те заснули погрузившись в бессознательное состояние личности которое предшествует подготовке к серафическому перемещению. Какое\hyp{}то время дети провели вместе на семейной встрече, радуясь, что скоро их родителям предстоит стать руководителями, а в действительности --- единственными правителями планеты 606 в системе Сатании.
\vs p074 1:6 Таким образом, Адам и Ева покинули Иерусем под звуки приветствий и добрых пожеланий его граждан. Они отправились к своим новым обязанностям, должным образом снаряженные и полностью проинструктированные относительно всех обязанностей и опасностей, которые могут встретиться на Урантии.
\usection{2. Прибытие Адама и Евы}
\vs p074 2:1 Адам и Ева заснули в Иерусеме, а, когда проснулись в храме Отца на Урантии среди огромной толпы собравшихся их приветствовать, то оказались лицом к лицу с двумя существами, о которых уже много слышали, --- Ваном и его верным товарищем Амадоном. Эти два героя, боровшиеся против бунта Калигастии, были первыми, кто приветствовал их в новом доме\hyp{}саде.
\vs p074 2:2 Языком Эдема был андонитский диалект, на котором говорил Амадон. Ван и Амадон значительно усовершенствовали этот язык, создав новый алфавит из двадцати четырех букв; они надеялись увидеть то время, когда он станет языком Урантии, по мере того как культура Эдема распространится по всему миру. Адам и Ева полностью овладели этим человеческим диалектом еще до того, как покинули Иерусем, так что этот сын Андона услышал, как благородный правитель его мира обращается к нему на его же собственном языке.
\vs p074 2:3 И в этот день великое возбуждение и радость царили во всем Эдеме: гонцы бежали, спеша изо всех сил туда, где находились почтовые голуби, собранные из разных концов земли, и кричали: «Выпускайте птиц! Пусть они несут весть, что обетованный Сын пришел!» Сотни поселений верующих из года в год старательно поставляли домашних голубей именно для такого случая.
\vs p074 2:4 \pc Как только весть о прибытии Адама распространилась, тысячи людей из соседних племен приняли учение Вана и Амадона; в то же время пилигримы месяц за месяцем стекались в Эдем, чтобы приветствовать Адама и Еву и воздать почести своему невидимому Отцу.
\vs p074 2:5 \pc Вскоре после пробуждения Адам и Ева были препровождены на официальный прием, устроенный на большом холме к северу от храма. Этот естественный холм расширили и подготовили для введения во власть новых правителей мира. Здесь в полдень комиссия Урантии по приему приветствовала Сына и Дочь системы Сатании. Амадон был председателем этой состоявшей из двенадцати членов комиссии, в которую входили представители каждой из шести сангикских рас, действующий глава срединников, Аннан --- верная дочь нодитов и их представитель, Ной --- сын архитектора и строителя Сада, исполнитель планов своего покойного отца и два постоянно пребывающих Носителя Жизни.
\vs p074 2:6 Затем Адаму и Еве были вручены полномочия опекать планету, это было сделано старшим Мелхиседеком, главой исполнительного совета на Урантии. Материальные Сын и Дочь принесли присягу верности Всевышним Норлатиадека и Михаилу из Небадона и были провозглашены правителями Урантии. Последнее было сделано Ваном, тем самым он сложил с себя номинальные полномочия, которыми обладал свыше ста пятидесяти тысяч лет по воле Мелхиседеков\hyp{}исполнителей.
\vs p074 2:7 И по случаю официального введения Адама и Евы в правление миром они были облачены в царские одежды. Не все искусства Даламатии были утрачены в этом мире, ткачество все еще существовало во времена Эдема.
\vs p074 2:8 Затем прозвучало послание архангелов, и далеко разносящийся голос Гавриила объявил о второй судной поверке Урантии и воскресении спящих в посмертии второй диспенсации милости и прощения на 606\hyp{}й планете Сатании. Диспенсация Принца закончилась, и третья планетарная эпоха, время Адама началась в обстановке подлинного величия; новые правители Урантии вступили в свое правление, казалось бы, в благоприятных условиях, несмотря на всемирную смуту, вызванную отсутствием сотрудничества с их предшественником по власти на планете.
\usection{3. Адам и Ева знакомятся с планетой}
\vs p074 3:1 И теперь, после официального введения во власть, Адам и Ева болезненно ощутили свою планетарную изоляцию. Молчали знакомые им средства вещания, отсутствовали все контуры межпланетных коммуникаций. Их иерусемские коллеги отправились к мирам, развивающимся нормальным эволюционным путем, где прочно обосновались Планетарный Принц и его опытный персонал, готовый их принять и сотрудничать с ними в начальный период их деятельности на этих мирах. Но мятеж на Урантии все изменил. Конечно, здесь присутствовал Планетарный Принц, но он, хотя и был практически лишен своей способности творить зло, все\hyp{}таки мог чинить миссии Адама и Евы препятствия и сделать ее до некоторой степени рискованной. А Сын и Дочь Иерусема, озабоченные и разочарованные, прогуливались той ночью, освещенные светом полной луны, обсуждая планы на следующий день.
\vs p074 3:2 Так закончился первый день Адама и Евы на изолированной Урантии, планете, переживающей смутное время из\hyp{}за предательства Калигастии; они гуляли и разговаривали до глубокой ночи, их первой ночи на земле, --- и им было очень одиноко.
\vs p074 3:3 \pc Второй день Адама на Урантии прошел в совещаниях с планетарными исполнителями и консультативным советом. От Мелхиседеков и их сподвижников Адам и Ева узнали многочисленные подробности о бунте Калигастии и о его последствиях для мирового развития. В общем, это была грустная история --- длинное повествование о плохом управлении земными делами. Они узнали все о полном провале плана Калигастии ускорить процесс социальной эволюции. И они пришли к ясному пониманию того, сколь безумно было пытаться достичь планетарного прогресса вне связи с божественным планом продвижения. Так закончился грустный, но насыщенный день, их второй день на Урантии.
\vs p074 3:4 \pc Третий день был посвящен осмотру Сада. С фандоров, больших пассажирских птиц, на которых они летели по воздуху, Адам и Ева осмотрели обширные пространства Сада, этого самого прекрасного места на земле. День закончился грандиозным банкетом в честь тех, кто своим трудом создал этот сад эдентийской красоты и величия. И снова, поздней ночью своего третьего дня Сын и его супруга гуляли по Саду и говорили о необъятности своих проблем.
\vs p074 3:5 \pc На четвертый день Адам и Ева обратились к собранию Сада. С инаугурационного холма они говорили с народом о своих планах восстановления мира и рассказали, как они постараются поднять социальную культуру Урантии с того низкого уровня, на котором она оказалась в результате греха и бунта. Это был великий день, и завершился он праздником для совета мужчин и женщин, которые были избраны на посты в новой администрации мировых дел. Заметьте: в этой группе были как мужчины, так и женщины, такое случилось на земле впервые со времени Даламатии! Это было удивительное новшество --- видеть Еву, женщину, разделяющую с мужчиной честь и ответственность решений в мировых делах. И так закончился четвертый день на земле.
\vs p074 3:6 \pc Пятый день был занят организацией временного правительства --- администрации, которая должна была функционировать до тех пор, пока Мелхиседеки\hyp{}исполнители не покинут Урантию.
\vs p074 3:7 \pc Шестой день был посвящен приему многочисленных представителей племен и осмотру животных. Целый день Адама и Еву водили вдоль стен, расположенных на востоке Эдема, и они наблюдали жизнь животных на планете, постепенно приобретая более ясное понимание того, что необходимо сделать, чтобы привести в порядок этот мир хаоса, заселенный таким разнообразием живых существ.
\vs p074 3:8 Сопровождавшие Адама в этом походе были в высшей степени удивлены, видя, как глубоко он понимает характер и назначения тысяч и тысяч животных, показываемых ему. Ему стоило лишь взглянуть на животное, чтобы квалифицировать его характер и поведение. С первого же взгляда Адам мог описать признаки, определяющие происхождение, характер и функции всех материальных существ. Те, кто сопровождал его в этом инспекционном обходе, не знали, что новый правитель мира --- один из лучших специалистов по анатомии во всей Сатании; в равной степени знатоком была и Ева. Адам изумлял своих спутников, описывая множество живых существ, которых из\hyp{}за их малости невозможно увидеть простым глазом.
\vs p074 3:9 Когда истек шестой день их пребывания на земле, Адам и Ева первый раз отдыхали в своем новом доме «на востоке Эдема». Первые шесть дней на Урантии были чрезвычайно насыщенными, и они с большим удовольствием предвкушали, как проведут целый день, отдыхая от всех дел.
\vs p074 3:10 Но обстоятельства распорядились иначе. События только что прошедшего дня, когда Адам так умно и так подробно говорил о жизни животных Урантии, его прекрасная инаугурационная речь и очаровательная манера говорить в такой степени завоевали умы и сердца обитателей Сада, что они не только всем сердцем готовы были признать только что прибывших Сына и Дочь Иерусема своими правителями, но большинство было готово пасть ниц и поклоняться им, как богам.
\usection{4. Первые беспорядки}
\vs p074 4:1 В ночь, которая наступила после шестого дня, когда Адам и Ева спали, произошли странные события около храма Отца в центральном секторе Эдема. Там, в мягком свете луны, сотни восторженных и возбужденных мужчин и женщин часами слушали страстные призывы своих вождей. У них были добрые намерения, но они просто не могли еще воспринять простоту братского и демократического поведения своих новых правителей. И задолго до рассвета новые временные управляющие мировых дел пришли практически к единодушному мнению, что Адам и его супруга все же слишком скромны и непритязательны. Они решили, что в облике человека на землю спустилось Божество, что Адам и Ева действительно являются богами или же столь близки к богам, что заслуживают благоговейного почитания.
\vs p074 4:2 Поразительные события первых шести дней пребывания Адама и Евы на земле превзошли все их ожидания и произвели неизгладимое впечатление на неподготовленное сознание людей, пусть даже и самых лучших в мире. Их умы были в смятении. Они были одержимы мыслью привести благородную пару в полдень к храму Отца, чтобы каждый мог выразить им свое почтительное поклонение и пасть ниц в смиренном повиновении. И обитатели Сада были во всем этом совершенно искренны.
\vs p074 4:3 Ван протестовал. Амадон отсутствовал, так как был начальником почетного караула, который оставался при Адаме и Еве всю ночь. Но доводы Вана были отвергнуты. Ему сказали, что он тоже слишком скромен и слишком непритязателен, что он сам почти бог, иначе как бы он мог так долго жить на земле и как иначе он смог бы осуществить такое грандиозное событие, как пришествие Адама? А так как возбужденные эдемиты собирались и его вести на холм для поклонения, Ван выбрался из толпы и, обладая способностью связываться со срединниками, спешно послал их главу к Адаму.
\vs p074 4:4 Незадолго до рассвета седьмого дня на земле Адам и Ева услышали поразительную новость о намерениях этих, действующих из лучших побуждений, но заблуждающихся смертных. И тогда, как раз в то время, когда пассажирские птицы только мчались, чтобы перевезти их в храм Отца, срединники (которые обладали способностью делать такие вещи) сами доставили туда Адама и Еву. Ранним утром того же седьмого дня, с холма, где проходил недавний прием, Адам выступил с разъяснением чина богосыновства и растолковал земным умам, что поклоняться можно только Отцу и тем, на кого тот указал. Адам прямо заявил, что охотно примет любые почести и знаки уважения, но богопочитание --- никогда!
\vs p074 4:5 Это был знаменательный день, и перед самым полуднем, почти в то же время, когда прибыл серафический вестник, принесший подтверждение Иерусемом введения во власть правителей мира, Адам и Ева, отделившись от толпы, указали на храм Отца и сказали: «Идите теперь к материальному символу невидимого присутствия Отца и поклонитесь тому, кто создал всех нас и кто дает нам жизнь. И пусть этот акт будет залогом того, что вы никогда больше не поддадитесь соблазну почитать кого бы то ни было, кроме Бога». Все они поступили так, как сказал Адам. Материальные Сын и Дочь стояли одни на холме со склоненными головами, в то время как народ пал ниц у храма.
\vs p074 4:6 \pc Это стало началом традиции субботнего дня. В Эдеме в седьмой день всегда происходил полуденный сбор у храма. Долгое время существовал обычай посвящать этот день самосовершенствованию. До полудня занимались физическим развитием, полдень посвящали духовному богопочитанию, время после полудня --- развитию умственных способностей; вечером устраивали общий праздник. В Эдеме это никогда не было законом, но стало обычаем, существовавшим пока адамическая администрация сохраняла власть на земле.
\usection{5. Администрация Адама}
\vs p074 5:1 В течение почти семи лет после прибытия Адама Мелхиседеки\hyp{}исполнители продолжали исполнять свои служебные обязанности, но пришло, наконец, время, когда они полностью передали управление делами мира Адаму и возвратились в Иерусем.
\vs p074 5:2 Прощание с исполнителями заняло целый день, а вечером каждый Мелхиседек передал Адаму и Еве свои прощальные советы и наилучшие пожелания. Адам неоднократно просил своих советников остаться с ним на земле, но такие просьбы всегда отклонялись. Наступило время, когда Материальные Сыны должны были принять всю ответственность по управлению миром на себя. И тогда, в полночь, серафимы перемещения с четырнадцатью пассажирами Сатании покинули планету, направляясь в Иерусем. Перемещение Вана и Амадона произошло одновременно с отбытием двенадцати Мелхиседеков.
\vs p074 5:3 \pc Какое\hyp{}то время на Урантии все шло довольно хорошо и казалось, что Адам, в конце концов, сумеет реализовать некий план постепенного распространения эдемской цивилизации. По совету Мелхиседеков он стал поощрять производственные ремесла, тем самым развивая торговлю с внешним миром. Когда Эдем был разрушен, там было свыше сотни действующих примитивных мануфактур и были налажены широкие торговые связи с соседними племенами.
\vs p074 5:4 В продолжение веков Адама и Еву учили методам улучшения мира, готовя их для особой роли в деле прогресса эволюционирующей цивилизации, но теперь они столкнулись с острыми проблемами, такими как установление законности и порядка в мире дикарей, варваров и полуцивилизованных народов. За исключением лучших из лучших населения, собранных в Саду, всего лишь несколько групп в разных местах земли были готовы принять адамическую культуру.
\vs p074 5:5 Адам предпринимал героические усилия для создания мирового правительства, но каждый его шаг встречал ожесточенное сопротивление. Адам уже привел в действие систему коллективного управления по всему Эдему и объединил на федеративных началах все группы в эдемскую лигу. Но неприятности, серьезные неприятности, начались тогда, когда он вышел за пределы Сада и попытался реализовать свои идеи в отдаленных племенах. Как только соратники Адама начали работать за пределами Сада, они встретили открытое и хорошо скоординированное сопротивление Калигастии и Далигастии. Падший Принц был низложен как правитель мира, но он не был изгнан с планеты. Он все еще находился на земле и был способен, по крайней мере, в какой\hyp{}то степени, оказывать сопротивление всем планам Адама по восстановлению человеческого общества. Адам пытался предостеречь людей от козней Калигастии, но это было очень трудно, поскольку его главный враг был невидим для глаз смертных.
\vs p074 5:6 Даже среди эдемитов были люди с помраченным сознанием, которые склонялись к учению Калигастии о неограниченной личной свободе, и они доставляли Адаму бесконечные неприятности: они всегда нарушали так хорошо составленные планы методичного продвижения вперед и существенного развития общества. В конце концов, Адам был вынужден отказаться от своей программы немедленной социализации. Он вернулся к принципам организации, использовавшимся Ваном: эдемиты делились на группы по сто человек во главе с капитаном, а те --- на группы в десять человек во главе с лейтенантом.
\vs p074 5:7 Адам и Ева прибыли, чтобы вместо монархии ввести представительное народное правление, но в масштабах всей земли они не нашли формы правления, соответствующей этому названию. Адам на время оставил все попытки создать представительное правительство, но перед крахом эдемского правления ему удалось сформировать почти сто самостоятельных торговых и социальных центров, которыми от его имени управляли сильные личности. Основы большинства таких центров были заложены в прошлые времена Ваном и Амадоном.
\vs p074 5:8 Именно со времен Адама вошло в практику отправлять послов от одного племени к другому. Это было большим шагом вперед в развитии форм правления.
\usection{6. Домашняя жизнь Адама и Евы}
\vs p074 6:1 Парк при доме семьи Адама занимал чуть больше пяти квадратных миль. Земля, непосредственно прилегающая к этому участку, предназначалась для содержания более трехсот тысяч потомков по прямой линии. Однако успели построить только первую очередь проектируемых зданий: еще до того, как численность семьи Адама превзошла эти первоначальные приготовления, весь эдемский план был нарушен, а Сад --- всеми покинут.
\vs p074 6:2 \pc Адам\hyp{}сын был первым новорожденным фиолетовой расы на Урантии, за ним родились сестра и Ева\hyp{}сын, второй сын Адама и Евы. До отъезда Мелхиседеков у Евы было пять детей --- три сына и две дочери. Следующие двое были близнецами. До срыва Ева родила шестьдесят три ребенка --- тридцать две девочки и тридцать одного мальчика. Когда Адам и Ева покинули Сад, четыре поколения семьи, насчитывали 1647 потомков по прямой линии. После ухода из Сада у них было еще сорок два собственных ребенка и двое детей смешанного происхождения --- от смертных жителей земли. В это число не входят дети адамического происхождения от связей с нодитами и эволюционирующими расами.
\vs p074 6:3 Адамические дети, после того как в возрасте одного года они переставали сосать материнскую грудь, не употребляли молока животных. В распоряжении Евы было молоко самых разнообразных орехов, соки многих фруктов. Хорошо зная состав и калорийность этих пищевых продуктов, она составляла необходимые смеси, которыми кормила детей до тех пор, пока у них не прорежутся зубы.
\vs p074 6:4 За пределами непосредственно адамического сектора Эдема приготовление пищи практиковалось повсеместно, но домашние Адама пищу не варили. Они считали, что продукты --- фрукты, орехи и злаки --- готовы к употреблению, как только созреют. Они ели раз в день, вскоре после полудня. Кроме того, Адам и Ева получали «свет и энергию» непосредственно от неких пространственных эманаций в сочетании с употреблением плодов дерева жизни.
\vs p074 6:5 \pc Тела Адама и Евы излучали мерцающий свет, но они всегда носили одежду в соответствии с обычаем своих сподвижников. Днем они едва прикрывали тело, к ночи надевали вечерние одежды. Происхождение традиционного нимба вокруг голов благочестивых и святых людей восходит ко времени Адама и Евы. Так как излучение света от их тел в значительной степени скрывалась одеждой, было видно только сияние, исходящее от их голов. Потомки Адам\hyp{}сына стали всегда таким образом отображать свое представление о личностях, которых считали выдающимися по своему духовному развитию.
\vs p074 6:6 Адам и Ева могли общаться друг с другом и со своими детьми на расстоянии до пятидесяти миль. Передача мыслей осуществлялась с помощью чувствительных газовых капсул, расположенных в непосредственной близости от мозга. Посредством этого устройства они могли посылать и принимать мысли. Но эта способность мгновенно пропадала, если сознание под воздействием зла приходило в расстройство и разрушалось.
\vs p074 6:7 \pc До шестнадцати лет адамические дети ходили в школу, причем старшие обучали младших. У младших групп уроки менялись каждые тридцать минут, у старших --- через час. И, несомненно, было необычным для Урантии видеть, как дети Адама и Евы играют, получая удовольствие и радость от самого процесса игры. Игры и шутки современных рас в значительной степени имеют адамическое происхождение. Все адамиты отличались большой музыкальностью и тонким чувством юмора.
\vs p074 6:8 Помолвки обычно происходили в восемнадцать лет. В это время молодые люди поступали на двухгодичные курсы, где проходили подготовку к тому, чтобы принять на себя супружеские обязанности. В двадцать лет они имели право вступить в брак, и после заключения брака начинали заниматься делом своей жизни или проходили для этого специальную подготовку.
\vs p074 6:9 Впоследствии у отдельных народов в царских родах, происходящих, по общему мнению, от богов, разрешались браки между братом и сестрой, это берет свое начало от обычаев адамических потомков --- сочетаться браком друг с другом, ибо они вынуждены были поступать так по необходимости. Брачные церемонии первого и второго поколений Сада всегда проводились Адамом и Евой.
\usection{7. Жизнь в Саду}
\vs p074 7:1 За исключением четырех лет, когда дети Адама учились в западных школах, они жили и работали «на востоке Эдема». В соответствии с методами школ Иерусема, до шестнадцати лет уделялось внимание развитию интеллектуальных способностей. От шестнадцати до двадцати лет они обучались в школах Урантии на другом краю Сада, выполняя при этом и обязанности учителей младших классов.
\vs p074 7:2 Конечной целью системы образования западных школ была \bibemph{социализация.} Перерывы между занятиями до полудня посвящались растениеводству и земледелию, а после полудня --- играм\hyp{}соревнованиям. Вечера проводили в общении друг с другом, поощрялась дружба между отдельными людьми. Религиозное и половое воспитание считалось домашним делом, это было обязанностью родителей.
\vs p074 7:3 В этих школах преподавание включало обучение:
\vs p074 7:4 \ublistelem{1.}\bibnobreakspace Здоровью и уходу за телом.
\vs p074 7:5 \ublistelem{2.}\bibnobreakspace Золотому правилу, стандарту социального общения.
\vs p074 7:6 \ublistelem{3.}\bibnobreakspace Отношению прав человека к правам групп и обязательствам перед обществом.
\vs p074 7:7 \ublistelem{4.}\bibnobreakspace Истории и культуре различных рас земли.
\vs p074 7:8 \ublistelem{5.}\bibnobreakspace Методам расширения и совершенствования мировой торговли.
\vs p074 7:9 \ublistelem{6.}\bibnobreakspace Согласованию противоречащих друг другу обязанностей и эмоций.
\vs p074 7:10 \ublistelem{7.}\bibnobreakspace Развитию игр, юмора и состязаний вместо физических драк.
\vs p074 7:11 \pc Школы, как и любые другие заведения Сада, были всегда открыты для посетителей. Безоружные посетители на короткий срок получали свободный доступ в Эдем. Для временного проживания в Эдеме урантиец должен был быть «усыновлен». Он получал наставления о замысле и цели адамического пришествия, подтверждал свое намерение присоединиться к этой миссии, а затем заявлял о своей верности мирской власти Адама и духовному господству Отца Всего Сущего.
\vs p074 7:12 \pc Законы Сада, основанные на более древних кодексах Даламатии, были сведены в единый свод законов, состоящий из семи глав:
\vs p074 7:13 \ublistelem{1.}\bibnobreakspace Законы здравоохранения и санитарии.
\vs p074 7:14 \ublistelem{2.}\bibnobreakspace Социальные правила Сада.
\vs p074 7:15 \ublistelem{3.}\bibnobreakspace Торгово\hyp{}промышленный кодекс.
\vs p074 7:16 \ublistelem{4.}\bibnobreakspace Законы честной игры и конкуренции.
\vs p074 7:17 \ublistelem{5.}\bibnobreakspace Законы домашней жизни.
\vs p074 7:18 \ublistelem{6.}\bibnobreakspace Гражданские кодексы золотого правила.
\vs p074 7:19 \ublistelem{7.}\bibnobreakspace Семь заповедей верховного морального кодекса.
\vs p074 7:20 \pc Закон морали Эдема практически не отличался от семи заповедей Даламатии. Но адамитам преподали множество дополнительных аргументов в пользу этих заповедей. Например --- запрет на убийство: неизменное пребывание Настройщика Мысли было представлено как дополнительный аргумент в пользу тезиса, что человеческая жизнь священна. Адамиты учили, что «кто бы ни пролил кровь человека, человеком будет пролита его кровь, ибо по своему подобию сотворил Бог человека».
\vs p074 7:21 В Эдеме полдень был временем публичного богопочитания, закат --- временем богопочитания в семьях. Адам делал все возможное, чтобы отучить от привычки произносить заученные тексты, он учил, что действенной может быть только абсолютно индивидуальная молитва, что это должно быть «стремление души». Но эдемиты продолжали использовать молитвы и ритуалы, восходящие к временам Даламатии. Адам также стремился заменить кровавые жертвоприношения приношениями плодов земли, но мало преуспел в этом ко времени разрушения Сада.
\vs p074 7:22 \pc Адам пытался привить народам понятие о равенстве полов. То, что Ева трудилась бок о бок со своим супругом, производило глубокое впечатление на жителей Сада. Адам прямо заявлял, что женщина наравне с мужчиной дает те жизненные силы, которые, объединяясь, создают новое живое существо. До той поры люди полагали, что вся сила, определяющая рождение потомства, сосредоточена в «чреслах отца». Они считали, что мать нужна лишь для питания еще не родившегося ребенка и для кормления новорожденного.
\vs p074 7:23 Адам учил своих современников всему тому, что они были способны воспринять, а этого было в общем\hyp{}то не так много. Тем не менее более разумные народы земли страстно предвкушали то время, когда им будет позволено вступить в смешанные браки с превосходящими их детьми фиолетовой расы. И как бы изменился мир Урантии, если бы этот грандиозный план подъема рас был осуществлен! Ведь принесла же прекрасные плоды лишь капля крови этой привнесенной расы, которую случайно получили эти эволюционирующие народы.
\vs p074 7:24 Так трудился Адам для благополучия и духовного подъема мира, ставшего его пристанищем. Но вывести на правильный путь эти разнородные и смешанные народы было трудной задачей.
\usection{8. Легенда о творении}
\vs p074 8:1 Рассказ о сотворении Урантии в шесть дней основан на предании, что Адам и Ева потратили ровно шесть дней на свой первоначальный осмотр Сада. Это обстоятельство как бы наложило священную санкцию на период времени, равный неделе, который, впрочем, первоначально был введен даламатийцами. То, что Адам потратил шесть дней на осмотр Сада и выработку первоначальных планов организации, не было предопределено заранее --- он просто работал день за днем. Выбор седьмого дня для богопочитания целиком объясняется обстоятельствами, изложенными выше.
\vs p074 8:2 На самом деле, легенда о сотворении мира в шесть дней была придумана много позже, более тридцати тысяч лет спустя. Одна деталь в изложении, а именно, внезапное появление солнца и луны, могла иметь своим истоком предание, что мир однажды внезапно возник в пространстве из плотного облака мельчайших частиц материи, которое долго скрывало и солнце, и луну.
\vs p074 8:3 История о создании Евы из ребра Адама представляет собой искаженное наложение двух событий --- адамического пришествия и небесной хирургической операции, связанной с обменом живыми субстанциями, который имел место во время прихода телесного штата Планетарного Принца более чем четыреста пятьдесят тысяч лет до адамического пришествия.
\vs p074 8:4 \pc Предания о том что, Адам и Ева обрели материальный облик, созданный для каждого из них во время их прибытия на Урантию, сильно повлияло на население Урантии. Вера в то, что человек был создан из глины, в восточном полушарии была почти повсеместной. Можно проследить распространение этого предания от Филиппинских островов и далее --- вокруг земного шара --- к Африке. Многие народы, вместо прежних верований в последовательное творение, т.е. в эволюцию, приняли этот рассказ о происхождении человека из глины как результат некоего особого акта творения.
\vs p074 8:5 Вдали от влияний Даламатии и Эдема человечеству была присуща вера в постепенное развитие человеческой расы. Факт эволюции не является новейшим открытием; древние люди осознавали медленный и эволюционный характер человеческого прогресса. Об этом имели ясное представление древние греки, несмотря на их близость к Месопотамии. Хотя понятие эволюции стало страшно запутанным у различных рас земли, тем не менее многие примитивные племена верили и учили тому, что они являются потомками различных животных. У примитивных народов был обычай выбирать для своих «тотемов» животных, которые, предполагалось, были их предками. Некоторые племена индейцев Северной Америки верили, что они происходят от бобров и койотов. Некоторые африканские племена утверждают, что являются потомками гиены, племя малайцев считает, что происходит от лемура, а народности Новой Гвинеи думают, что от попугая.
\vs p074 8:6 Вавилоняне, находясь в непосредственном контакте с остатками цивилизации адамитов, расширили и приукрасили историю о сотворении человека, они учили, что человек ведет свою родословную непосредственно от богов. Они придерживались представления о благородном происхождении расы, которое, конечно, было несовместимо с концепцией создания человека из глины.
\vs p074 8:7 \pc В Ветхом Завете рассказ о сотворении относится ко времени, значительно более позднему, чем время Моисея, тот никогда не учил евреев подобной искаженной истории. Но он представил израильтянам простое и сжатое изложение фактов творения, надеясь таким образом усилить свой призыв почитать Создателя, Творца Всего Сущего, которого называл Господом, Богом Израиля.
\vs p074 8:8 В своих первых проповедях Моисей поступал очень мудро, не пытаясь проникнуть во времена до Адама, а, поскольку Моисей был верховным учителем у евреев, легенды об Адаме стали тесно связывать с легендами о творении. То, что более ранняя традиция признавала доадамическую цивилизацию, ясно видно из того факта, что позднейшие редакторы, стремясь исключить все ссылки на присутствие человека во времена до Адама, позабыли вычеркнуть свидетельствующее об этом упоминание об уходе Каина «в страну Нод», где тот нашел себе жену.
\vs p074 8:9 Иудеи еще долгое время после того, как пришли в Палестину, не имели письменности как таковой. Они переняли алфавит у своих соседей --- филистимлян, политических беженцев с Крита, где существовала высокоразвитая цивилизация. Иудеи почти не имели письменности приблизительно до 900 года до н.э., и, вследствие отсутствия письменности столь долгое время, у них было в ходу несколько различных историй о творении, но после вавилонского пленения они стали все больше склоняться к тому, чтобы принять модифицированную месопотамскую версию.
\vs p074 8:10 Иудейская традиция выкристаллизовалась вокруг фигуры Моисея, а поскольку тот стремился проследить линию предков от Авраама к Адаму, евреи посчитали, что Адам и был первым человеком. Иегова был творцом, а так как Адам предполагался первым человеком, Иегова должен был создать мир прежде, чем создать Адама. Тогда\hyp{}то предание о шести днях Адама и было вплетено в рассказ. В результате, спустя тысячу лет после пребывания Моисея на земле, легенда о шести днях творения и была записана, а затем и приписана ему.
\vs p074 8:11 Когда иудейские священники возвратились в Иерусалим, они уже закончили написание своего толкования начала вещей. Вскоре они заявили, что это повествование есть только что обнаруженная история творения, написанная Моисеем. Но евреи --- их современники, жившие около 500 г. до н.э., не считали эти рукописи божественным откровением; они относились к ним в значительной степени так же, как народы в более поздние времена относились к мифологическим историям.
\vs p074 8:12 Эти фальшивые документы, считающиеся учением Моисея, привлекли внимание грека Птолемея, царя Египта, который поручил их перевести на греческий группе из семидесяти ученых для своей новой библиотеки в Александрии. И, таким образом, этот рассказ вошел в число тех текстов, которые впоследствии составили часть более позднего собрания «священных писаний» иудейской и христианской религий. И благодаря их идентификации с этими теологическими системами, такие представления в течение длительного времени существенно влияли на философию многих восточных народов.
\vs p074 8:13 Учителя христианства продолжали верить в мгновенный акт творения человеческого рода, и это напрямую привело к возникновению гипотезы о существовавшем когда\hyp{}то золотом веке утопического счастья, и к теории о грехопадении человека или сверхчеловека, который был ответственен за состояние общества, не соответствующее этой утопии. Такой взгляд на жизнь и место человека во вселенной является, по меньшей мере, обескураживающим, так как он предполагает скорее веру в регресс, чем в прогрессивное развитие. Он предполагает также существование мстительного Божества, которое дает волю своему гневу на человеческий род в отместку за ошибки прошлых правителей планеты.
\vs p074 8:14 \pc «Золотой век» --- это миф, но существование Эдема --- это факт, и цивилизация Сада действительно была низвергнута. Адам и Ева трудились в Саду сто семнадцать лет, когда, вследствие нетерпения Евы и ошибочных суждений Адама, они позволили себе отойти от предначертанного пути, что тотчас же навлекло на них несчастье и роковым образом повлияло на процесс развития всей Урантии.
\vsetoff
\vs p074 8:15 [Представлено Солонией, серафическим «голосом в Саду».]
